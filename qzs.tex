\documentclass[11pt,a4paper]{memoir}
\def\asydir{\jobname}
\usepackage{thesis-preamble}
\EndPreamble
\begin{document}

\chapter{\QZS/ design}
\chaplabel{qzs}

\referpaper{The work presented in \secref{qzs} is based on material that has been published as a journal paper~\cite{robertson2009-jsv}.}

\section{Introduction to \qzs/ springs}

In a conventional mass--spring system, the static deflection increases as the
stiffness of the support is reduced, and a lower limit on the stiffness is
imposed by constraints on the allowable displacement. Consider the linear
isolator with force balance described by
\begin{math}
  \mass \gravity = \stiffness \disp
\end{math},
and a natural frequency of
\begin{math}
  \natfreq = \sqrt{\stiffness/\mass}
\end{math}.
The displacement required to support this mass can be written
\begin{math}
  \disp = \gravity / \natfreq^2
\end{math}.

This inverse-quadratic relationship imposes a heavy penalty on the static
displacement required to support loads at low frequency (note that this
expression is independent of the mass of the object being supported). While a
resonance at \SI{5}{Hz} requires a static displacement of approximately
\SI{1}{cm}, a \SI{1}{Hz} resonance requires a displacement of almost
\SI{25}{cm}. A different approach is desired for achieving a load bearing
relationship that can exhibit a low \emph{dynamic} stiffness without the need
for an associated large static deflection.

An example of a system with such behaviour is that of a cubic force \vs\
displacement characteristic; for a dynamic force characteristic of, say,
$\func{f}{x}=f_0+x^3$, localised zero stiffness occurs at zero deflection
(\ie, $\func{f'}{0}=0$), which is termed `\qzs/'. The `quasi' qualifier is
included for two reasons: because the zero stiffness is only valid at a single
displacement; and because the system is usually infeasible to implement in a
physical system. An investigation into the literature yields a variety of such
devices, as previously discussed in \secref{qzs-explore}.

This chapter will investigate the quasi-static behaviour of two of these \qzs/ systems.
The first, in \secref{qzszks}, is an `inclined spring' device that has been studied previously in the literature; here, it is studied from a fresh perspective in order to contrast its behaviour with the second \qzs/ system examined in \secref{qzs}: an arrangement of attracting and repelling permanent magnets that allows much simpler tuning of the system to adapt to dynamic load bearing.


\section{Inclined springs}
\seclabel{qzszks}

The spring arrangement shown in \figref{zks} is a well studied planar \qzs/ device~\cite{molyneux1957,alabuzhev1989,carrella2007-jsv,carrella2009-jsv}.
It consists of a load bearing vertical spring in parallel with a pair of inclined springs that behave in a buckling regime.
Generally, analyses of this device have only considered its stiffness properties in a single \dof/, in the direction of the primary load bearing.

This section consists of an analysis of the quasi-static behaviour of this inclined spring system and re-formulates the force and stiffness characteristics in both vertical and horizontal directions, describing in some detail the approach by which low stiffness in both directions can be achieved.
Low stiffness in the vertical direction has been previously documented (as cited previously) due to the negative vertical stiffness of a pair of horizontal springs in compression.
Low stiffness in the horizontal direction is newly analysed here, which is achieved due to the negative stiffness in the horizontal direction of the load-bearing vertical spring.

A potential drawback of the planar low-stiffness system is observed, which is that large compressions are required from the springs in order to operate at \qzs/ in both directions simultaneously.

\subsection{Geometry}

\Figref{zks-dia} shows the planar inclined spring system without load (\ie,
with undeflected springs) and \figref{zks-qzs} shows the same system after
deflection to the position which has the potential of achieving `\qzs/', which
is the position of maximum compression of the inclined springs. The overall
stiffness of the system must be tuned to support the mass of the load at this
position.

\begin{figure}
\begin{wide}
  \subbottom[Inclined springs in their unloaded, uncompressed state, corresponding
to a vertical displacement of $\ZKSdisp=\ZKSheight$.\figlabel{zks-dia}]{\asyinclude{\jobname/zks-schematic}}
\hfil
  \subbottom[Inclined springs at a position of maximum negative stiffness,
corresponding to a vertical displacement of $\ZKSdisp=0$.\figlabel{zks-qzs}]{\asyinclude{\jobname/zks-schematic-qzs}}
\end{wide}
\caption{Negative stiffness inclined springs in parallel with a positive stiffness spring.}
\figlabel{zks}
\end{figure}

At the unloaded state shown in \figref{zks-dia}, all springs are considered to
be in their uncompressed state; with inclined spring lengths
$\ZKSlength=\sqrt{\ZKSwidth^2+\ZKSheight^2}$ and vertical spring length
$\ZKSvlength=\ZKSlengthratio\ZKSlength$, where $\ZKSlengthratio$ is denoted the `length ratio' between the vertical and inclined springs.
The inclined springs each have stiffness~$\ZKSstiffincl$ and the vertical spring has stiffness
$\ZKSstiffvert=\ZKSkratio\ZKSstiffincl$, with $\ZKSkratio$ denoted the `stiffness ratio' between the vertical and inclined springs.
The stiffness and deflection properties of the springs are summarised in \tabref{spring-properties}.

\begin{table}
  \caption{Properties of the springs in the \qzs/ inclined spring system
defining stiffness ratio $\ZKSkratio$ and length ratio $\ZKSlengthratio$.}
  \tablabel{spring-properties}
  \begin{tabular}{@{}lcc@{}}
    \toprule
      Spring & Stiffness & Undeflected length \\
    \cmidrule(r){1-1}\cmidrule(l){2-3}
      Inclined & $\ZKSstiffincl$ & $\ZKSlength = \sqrt{\ZKSwidth^2+\ZKSheight^2}$ \\
      Vertical & $\ZKSstiffvert = \ZKSkratio\ZKSstiffincl$
        & $\ZKSvlength = \ZKSlengthratio\ZKSlength$ \\
    \bottomrule
  \end{tabular}
\end{table}

The position of maximum compression of the inclined spring, shown in
\figref{zks-qzs}, defines the displacement origin of the system, where
$\ZKSdisp$ is the displacement in the load bearing direction, and $\ZKSshift$
is the displacement in the non--load bearing direction (this is used later for
the derivation of the horizontal stiffness of the system).

The deflected lengths of the springs from vertical displacement $\ZKSdisp$ and horizontal displacement $\ZKSshift$ are $\ZKSlengthcomp$ for the inclined spring and $\ZKSvlengthcomp$ for the vertical spring.
The compressed length of the inclined spring on the left is
\begin{dmath}[label=ZKSlengthcomp]
\ZKSlengthcomp =
  \sqrt{\gp{\ZKSwidth+\ZKSshift}^2+\ZKSdisp^2}
\end{dmath},
and the vertical spring length is
\begin{dmath}[label=ZKSvlengthcomp]
\ZKSvlengthcomp =
  \sqrt{\ZKSshift^2+\gp{\ZKSvlength-\ZKSheight+\ZKSdisp}^2}
\end{dmath};
note that $\ZKSlengthcomp{0,\ZKSheight}=\ZKSlength$ and
$\ZKSvlengthcomp{0,\ZKSheight}=\ZKSvlength$.

The geometry that has been chosen uses linear springs that are all arranged to
be undeflected in the unloaded state of the device.
The use of pretension and nonlinear springs \parencite{kovacic2008} is not explored in this study.

\subsection{Vertical forces}

The forces on the mass are calculated by analysing the components due to each spring individually.
The force due to the inclined spring
(on the left of \figref{zks-dia}), in the direction of the spring, is given by
\begin{dmath}
\ZKSforceincl\fn{\ZKSshift,\ZKSdisp} = \gp{\ZKSlength-\ZKSlengthcomp} \ZKSstiffincl
  = \gp{\sqrt{\ZKSwidth^2+\ZKSheight^2}-
        \sqrt{\gp{\ZKSwidth+\ZKSshift}^2}+\ZKSdisp^2} \ZKSstiffincl
\end{dmath}.
Assuming only vertical displacement ($\ZKSshift=0$),
the vertical component of this inclined spring force is
\begin{dmath}[label=ZKSforceinclV,compact]
\ZKSforceinclV\fn{\ZKSshift,\ZKSdisp} = \ZKSforceincl\fn{0,\ZKSdisp} \frac{\ZKSdisp}{\ZKSlengthcomp{0,2}}
% from LaTeXPrint[Fv, x, 0]; bring stiffincl to the front:
=  \ZKSdisp \ZKSstiffincl
  \gp{
    \frac {\sqrt{\ZKSwidth^2+\ZKSheight^2}}
          {\sqrt{\ZKSwidth^2+\ZKSdisp^2}  } -1
  }
\end{dmath}.
It is convenient to normalise this result by representing the lengths and
displacements as ratios of the uncompressed height of the inclined springs.
With the \coordinate/ substitutions $\ZKSnorm=\ZKSdisp/\ZKSheight$ and
$\ZKSratio=\ZKSwidth/\ZKSheight$, the inclined spring force in the vertical
direction can be written in non-dimensional form as
\begin{dmath}[label=ZKSforceinclVnorm]
  \frac{\ZKSforceinclV\fn{\ZKSnorm}}{\ZKSheight \ZKSstiffincl} =
    \ZKSnorm \gp{\sqrt{\frac{\ZKSratio^2+1}{\ZKSratio^2+\ZKSnorm^2}}-1}
\end{dmath},
where $\ZKSratio$ is denoted the `geometric ratio' of the device and $\ZKSnorm$ the normalised displacement.

This is a different normalisation than used in the literature~\cite{carrella2007-jsv,carrella2009-jsv}; note that here $\ZKSratio=0$ corresponds to unloaded inclined springs at~\ang{90} (\ie, vertical) before compression, and $\ZKSratio=\infty$ corresponds to unloaded inclined springs at~\ang{0} (\ie, horizontal).
In the \coordinate/ system used here, the displacement origin $\ZKSdisp=0$ corresponds to the position of maximum compression of the inclined springs; \ie, when they are horizontal.

\Figref{ZKSforce-plain} illustrates the force characteristic of
\eqref{ZKSforceinclVnorm} \vs\  normalised displacement for a range of geometric
ratios $\ZKSratio$.
The `snap-through' forces that cause the negative stiffness are especially strong for smaller values of geometric ratio $\ZKSratio$ (\ie, the more vertical the spring angles before deflection in \figref{zks-dia}).

\begin{figure}
\begin{wide}
\subbottom[Vertical force due to inclined springs only using
\eqref{ZKSforceinclV} for a range of geometric ratios
$\ZKSratio$.\figlabel{ZKSforce-plain}]{\hspace*{-1.5cm}\psfragfig{PhD/Simulations/ZKS/fig/ZKSforce-plain}}
\hfil
\subbottom[Normalised vertical force characteristic of the system calculated
with \eqref{ZKSforcetotalVnorm}. Minimum displacements are calculated from
\eqref{ZKSnormMin}; these termination points represent the limiting case of the vertical
spring being compressed to zero length.\figlabel{ZKSforcetotalVnorm}]{\psfragfig{PhD/Simulations/ZKS/fig/ZKSforce-total}}
\end{wide}
\lofcaption{Vertical forces due to the inclined springs.}{ $\ZKSratioQZS$ is the
value of $\ZKSratio$ for which \qzs/ is achieved at $\ZKSnorm=0$, calculated
from \eqref{ZKSratioQZS}.}
\figlabel{ZKSforce-both}
\end{figure}

The total vertical force produced by the system, $\ZKSforcetotalV\fn{\ZKSshift,\ZKSdisp}$, is
calculated by combining \eqref{ZKSforceinclV} for each inclined spring with
the force due to the vertical spring:
\begin{dmath}[label=ZKSforcetotalV]
\ZKSforcetotalV\fn{\ZKSshift,\ZKSdisp}
  = 2\ZKSforceinclV\fn{\ZKSshift,\ZKSdisp} + \ZKSforcevertV\fn{\ZKSshift,\ZKSdisp}
\end{dmath}.
For vertical displacements, the force due to the vertical spring is given by
\begin{dmath}[label=ZKSforcevertV]
\ZKSforcevertV\fn{\ZKSshift,\ZKSdisp}
  = \gp{\ZKSheight-\ZKSdisp} \ZKSstiffvert
\end{dmath},
and the total force in the vertical direction can be nondimensionally
represented by
\begin{dmath}[label=ZKSforcetotalVnorm]
\frac{\ZKSforcetotalV\fn{\ZKSshift,\ZKSdisp}}{\ZKSheight \ZKSstiffincl} = -\ZKSnorm \ZKSkratio+\ZKSkratio+2
\ZKSnorm \gp{\sqrt{\frac{\ZKSratio^2+1}{\ZKSratio^2+\ZKSnorm^2}}-1}
\end{dmath},
recalling that $\ZKSkratio=\ZKSstiffvert/\ZKSstiffincl$ is the stiffness ratio between the vertical and inclined springs. This equation is depicted in
\figref{ZKSforcetotalVnorm} for a unity stiffness ratio ($\ZKSkratio=1$), where it can be seen that by
selecting the geometric ratio $\ZKSratio$ appropriately it is possible to generate a local region
of low stiffness at displacement $\ZKSnorm=0$, approaching the \qzs/ condition under ideal
circumstances.
The calculation for $\ZKSratioQZS$, the value of the geometric ratio $\ZKSratio$ for which \qzs/ is achieved, will be shown later in \eqref{ZKSratioQZS}.

The force curves in \figref{ZKSforcetotalVnorm} terminate at a certain point
in the negative displacement region, which corresponds to the maximum possible
compression of the vertical spring, given by the condition
$\ZKSvlengthcomp{0,\ZKSdispMin}=0$. In other words, the spring has been
compressed to zero length (which would be troublesome to achieve in practice).
This condition can be solved for $\ZKSdispMin$ and subsequently normalised for
the equivalent $\ZKSnormMin$, which are given by
\begin{dseries}[label=ZKSnormMin]
\begin{math}
  \ZKSdispMin = \ZKSheight - \ZKSvlength
\end{math},
\begin{math}
  \ZKSnormMin = 1-\ZKSlengthratio\sqrt{\ZKSratio^2+1}
\end{math}.
\end{dseries}


\subsection{Vertical stiffnesses}

\begin{figure}
\begin{wide}
\subbottom
  [\figlabel{ZKSstiffness-vert} Normalised vertical stiffness of the system calculated with \eqref{ZKSstiffnessV}.]
  {\null\hspace*{-1cm}\psfragfig{PhD/Simulations/ZKS/fig/ZKSstiffness-total}}%
\hfil
\subbottom
  [
  Normalised horizontal stiffness of the system calculated with \eqref{ZKSstiffnessH}.
  \figlabel{ZKSstiffness-horiz}
  ]{\psfragfig{PhD/Simulations/ZKS/fig/ZKSstiffnessH-total}}
\end{wide}
\lofcaption
 {Vertical and horizontal stiffness characteristics for a range of geometric ratios $\ZKSratio$ at $\ZKSkratio=1$.}
 { Plots are labelled with their ratio to $\ZKSratioQZS$, which is calculated for a length ratio of $\ZKSlengthratio=1$.}
\figlabel{ZKSstiffness-both}
\end{figure}

\begin{figure}
\begin{wide}
  \subbottom
    [\figlabel{ZKSoff}
The stable and unstable equilibrium points for $\ZKSoff\in\{-0.1, 0, 0.1\}$.
The rest position will move from the unstable point to the stable point of equilibrium.]
    {\mbox{\null\hspace*{-1cm}\psfragfig{PhD/Simulations/ZKS/fig/ZKSoffqzs}}}\hfil
  \subbottom
    [\figlabel{ZKS-ratioErr}
The stiffness at equilibrium as $\ZKSoff$ varies; as the stiffness becomes
negative, the stiffness shown corresponds to the stable point of equilibrium shown in the figure adjacent.]
    {\psfragfig{PhD/Simulations/ZKS/fig/ZKSratioErr}}
\end{wide}
\caption{Force and stiffness of the inclined spring system near \qzs/, showing
the effect of unstable equilibrium.}
\end{figure}

The vertical stiffness characteristic, $\ZKSstiffnessV$, of the system is
calculated by differentiating the vertical force, \eqref{ZKSforcetotalVnorm}, with respect to vertical displacement~$\ZKSdisp$:
\begin{dgroup*}
\begin{dmath}
  \ZKSstiffnessV = -\D{}{\ZKSdisp} \ZKSforcetotalV\fn{\ZKSshift,\ZKSdisp}
\end{dmath},
\begin{dmath}[label=ZKSstiffnessV]
% edited to move the gamma^2 in front of the sqrt
  \frac{\ZKSstiffnessV}{\ZKSstiffincl} = -2\ZKSratio^2 \sqrt{\frac{\ZKSratio^2+1}{\gp{\ZKSratio^2+\ZKSnorm^2}^3}}+\ZKSkratio+2
\end{dmath}.
\end{dgroup*}
Graphs of the normalised vertical stiffness ${\ZKSstiffnessV}/{\ZKSstiffincl}$ \vs\ normalised displacement $\ZKSnorm$ are shown in \figref{ZKSstiffness-vert} together with the associated horizontal stiffness (\figref{ZKSstiffness-horiz}), which will be analysed in \secref{qzs-horiz}.
The parameter selection required to achieve a \qzs/
condition in the vertical direction can be found by solving
\eqref{ZKSstiffnessV} for $\ZKSstiffnessV=0$ at $\ZKSnorm=0$. This results in
the relation
\begin{dmath}[label=ZKSratioQZS]
  \ZKSratioQZS =
    \frac{2}{\sqrt{\ZKSkratio^2+4 \ZKSkratio}}
\end{dmath}
which is used as the reference value of the geometric ratio $\ZKSratio$ for the results shown in \figref{ZKSforce-both,ZKSstiffness-both}.

Achieving exactly \qzs/ with this spring is not feasible in practice as the
stiffness characteristic becomes negative for $\ZKSratio<\ZKSratioQZS$, as
shown in \figref{ZKSstiffness-vert}.
This is important as the geometric ratio $\ZKSratio$ will have some uncertainty in its value due to environmental conditions such as temperature and physical imperfections such as creep.
The deviation of $\ZKSratio$ from $\ZKSratioQZS$, $\ZKSoff$, can be defined by
\begin{dmath}
  \ZKSratio = \gp{1+\ZKSoff}\ZKSratioQZS
\end{dmath}.
\Figref{ZKSoff} shows the total vertical force, $\ZKSforcetotalV\fn{\ZKSshift,\ZKSdisp}$, of the system for
$\ZKSoff\in\{-0.1, 0, 0.1\}$. It can be seen that negative values of $\ZKSoff$
(\ie, a geometric ratio less than that for \qzs/)
correspond to negative stiffness at normalised displacement $\ZKSnorm=0$.
A system in this condition is in a position of unstable equilibrium, and will move towards and remain at the position of stable equilibrium indicated in the figure rather than the design point at $\ZKSnorm=0$.

\Figref{ZKS-ratioErr} plots the stiffness at this deviated equilibrium point as $\ZKSoff$ varies;
in the unstable zone, the system will move to the equilibrium point shown in
\figref{ZKSoff} away from $\ZKSnorm=0$.
\note{With sufficient excitation the system will `snap though' from one equilibrium position to another with a resulting displacement profile that is comparatively large given the excitation amplitude; this mechanism has been proposed as a useful phenomenon for energy harvesting purposes \cite{ramlan2009-nd}.}
It can be seen that the stiffnesses in
the stable region for $\ZKSoff>0$ are smaller than the stiffnesses in the
equilibrium region for $\ZKSoff<0$. This highlights the importance of never
breaching the $\ZKSoff<0$ instability condition. Therefore, a chosen value for
the geometric ratio $\ZKSratio$ will approach $\ZKSratioQZS$ but always be slightly greater in
order to retain stability of the equilibrium position.

\subsection{Horizontal stiffness characteristic}
\seclabel{qzs-horiz}

Now that the vertical stiffness characteristics of the system have been analysed and a condition
derived to achieve \qzs/ in that direction, the same approach will be taken
for the horizontal behaviour.

In order to calculate the horizontal stiffness of the system, the force from
the vertical spring needs to be represented in terms of both vertical and
horizontal displacements. This force, aligned in the direction of the nominally-vertical spring,
is

\begin{dmath}[label=ZKSforcevertlong]
% edited to insert \ZKSlength
\ZKSforcevert\fn{\ZKSshift,\ZKSdisp} =
  \gp{\ZKSlengthratio\ZKSlength-\sqrt{\ZKSshift^2+
  \gp{-\ZKSheight+\ZKSdisp+\ZKSlengthratio\ZKSlength}^2}} \ZKSstiffvert
\end{dmath},
recalling that $\ZKSshift$ is the displacement of the mass in the horizontal
direction. Substituting $\ZKSshift=0$ into \eqref{ZKSforcevertlong} yields the
previous \eqref{ZKSforcevertV}. The horizontal component of this force is
\begin{dmath}[label=ZKSforcevertH]
\ZKSforcevertH\fn{\ZKSshift,\ZKSdisp} = \ZKSforcevert\fn{\ZKSshift,\ZKSdisp}\frac{\ZKSshift}{\ZKSvlengthcomp}
\end{dmath}.
Similarly, the horizontal component of the force from the inclined spring on
the left (referring to \figref{zks-dia}) is given by
\begin{dmath}[label=ZKSforceinclH]
\ZKSforceinclH\fn{\ZKSshift,\ZKSdisp}\bigg|_{\text{left}} = \ZKSforceincl\fn{\ZKSshift,\ZKSdisp} \frac{\ZKSwidth+\ZKSshift}{\ZKSlengthcomp}
\end{dmath},
and the horizontal component of the force from the inclined spring on the
right is
\begin{dmath}
\ZKSforceinclH\fn{\ZKSshift,\ZKSdisp}\bigg|_{\text{right}} = -\ZKSforceinclH\fn{-\ZKSshift,\ZKSdisp}
\end{dmath}.
The stiffness characteristic in the horizontal direction, $\ZKSstiffnessH$,
is derived in a similar fashion to the vertical stiffness. The total force in
the horizontal direction is
\begin{dmath}
\ZKSforcetotalH\fn{\ZKSshift,\ZKSdisp}=\ZKSforceinclH\fn{\ZKSshift,\ZKSdisp}-\ZKSforceinclH\fn{-\ZKSshift,\ZKSdisp}+\ZKSforcevertH\fn{\ZKSshift,\ZKSdisp}
\end{dmath},
using \eqref{ZKSforcevertH,ZKSforceinclH}. Differentiating with respect to
horizontal displacement $\ZKSshift$ and evaluating at $\ZKSshift=0$ gives the horizontal stiffness
characteristic as the vertical displacement varies,
\begin{dmath}[label=ZKSstiffnessH]
% edited to move elements in front of the sqrts
\frac{\ZKSstiffnessH}{\ZKSstiffincl} = -2 \ZKSnorm^2 \sqrt{\frac{\ZKSratio^2+1}{\gp{\ZKSratio^2+\ZKSnorm^2}^3}} +\frac{\ZKSkratio \gp{\ZKSnorm-1}}{\ZKSlengthratio\sqrt{\ZKSratio^2+1} +\ZKSnorm-1}+2
\end{dmath}.
This equation has been previously graphed together with the vertical stiffness in \figref[vref]{ZKSstiffness-both}.
In these figures, it can be seen that while the vertical stiffness is zero at
normalised displacement $\ZKSnorm=0$ and geometric ratio $\ZKSratio=\ZKSratioQZS$ (which is as derived), the horizontal stiffness exhibits
separate behaviour, and can even be negative (\ie, unstable) for values of
$\ZKSratio$ lower than around $1.25\ZKSratioQZS$.

Since the vertical stiffness and horizontal stiffness are independent, further analysis into the behaviour of the horizontal stiffness at the
vertical \qzs/ condition is warranted.
Substituting the \qzs/ condition of
\eqref{ZKSratioQZS} into \eqref{ZKSstiffnessH} at displacement $\ZKSnorm=0$,
gives the normalised horizontal stiffness as a function of stiffness ratio
$\ZKSkratio$:

\begin{dmath}[label=ZKSstiffnessHnorm]
% edited to eliminate nested fraction
  \frac{\ZKSstiffnessH}{\ZKSstiffincl}\bigg|_{\text{V.\,QZS}} =
   2-\ZKSkratio\gp{\frac{\gp{\ZKSkratio+2} \ZKSlengthratio}{\sqrt{\ZKSkratio \gp{\ZKSkratio+4}}}-1}^{-1}
\end{dmath}.

This equation is depicted in \figref{ZKSstiffnessKnorm}; it can be seen that
the horizontal stiffness of the spring may be chosen by varying both
the spring stiffness ratio $\ZKSkratio$ and the spring length ratio $\ZKSlengthratio$.
Since the length ratio $\ZKSlengthratio$ is not found in \eqref{ZKSstiffnessV}, the horizontal and
vertical stiffnesses may be tuned independently in order to achieve \qzs/ in both simultaneously.

\begin{figure}
\psfragfig{PhD/Simulations/ZKS/fig/ZKSkratio-effect}
\caption
[Horizontal stiffness characteristic at the vertical \qzs/ condition.]
{Horizontal stiffness characteristic at the vertical \qzs/ condition
for varying stiffness ratio $\ZKSkratio$ and length ratio $\ZKSlengthratio$,
calculated with \eqref{ZKSstiffnessHnorm}.}
\figlabel{ZKSstiffnessKnorm}
\end{figure}

To obtain \qzs/ in the horizontal direction, \eqref{ZKSstiffnessHnorm} is
solved at $\ZKSstiffnessH=0$, showing a relationship between $\ZKSkratio$ and
$\ZKSlengthratio$ when the \qzs/ condition is achieved in both the vertical
and the horizontal directions:
\begin{dseries}[label=ZKSQZS]
\begin{math}
\ZKSkratioQZS\fn{\ZKSlengthratio} =
  2 \gp{\sqrt{\ZKSlengthratio^2+1}-1}
\end{math}, or\quad
\begin{math}
\ZKSlengthratioQZS\fn{\ZKSkratio} =
    \frac{1}{2} \sqrt{\ZKSkratio \gp{\ZKSkratio+4}}
\end{math}
\end{dseries}
As a consequence, increasing $\ZKSlengthratio$ (say, in order to reduce the compression of the
vertical spring) results in an increasing value of the vertical spring
stiffness in order to remain at \qzs/.

Using $\ZKSkratioQZS$ from \eqref{ZKSQZS} in
the stiffness equations
\eqref*{ZKSstiffnessV,ZKSstiffnessH} allows the stiffness characteristics of
the system in the two directions to be compared when both have \qzs/
simultaneously. Considering the vertical stiffnesses first in
\figref{ZKSstiffnessV-QZS}, it can be seen that increasing the length ratio
$\ZKSlengthratio$ increases the vertical stiffness gradient, which is an
important parameter to be kept small in order to mitigate possible nonlinear dynamic
effects that may arise due to a large rate of change of stiffness over
displacement.

\Figref{ZKSstiffnessH-QZS} illustrates that the \qzs/ condition is
always marginally unstable in the horizontal direction since negative displacement will result in negative stiffness.
In practice this requires
that the system be tuned slightly away from the \qzs/ condition in the horizontal direction after
accommodating for the maximum disturbance displacement of the isolator. It is
possible to do this without compromising the \qzs/ condition in the vertical
direction because the spring length ratio $\ZKSlengthratio$ does not affect
the vertical stiffness.

As an example, \figref{ZKSstiffness-NotQZS} shows the horizontal stiffness for
a stiffness ratio detuned by five percent below that required for \qzs/ (\ie, $\ZKSkratio=0.95\ZKSkratioQZS$). In comparison with \figref{ZKSstiffnessH-QZS},
the spring has a stable displacement range of approximately
$\ZKSnorm=\pm0.025$. Provided that the spring length ratio $\ZKSlengthratio$ is large enough, the
horizontal stiffness at $\ZKSnorm=0$ is still significantly reduced.

Therefore, there is a direct compromise between the nonlinearity of the
stiffness in the vertical direction (which increases with $\ZKSlengthratio$)
and the amount of stiffness reduction in the horizontal direction (which
decreases with $\ZKSlengthratio$).

\begin{figure}[p]
\begin{wide}
\hspace*{-1cm}
\subbottom[\figlabel{ZKSstiffnessV-QZS}
Normalised vertical stiffness of the system.]
  {\psfragfig{PhD/Simulations/ZKS/fig/ZKSstiffnessV-QZS}}
\hfil
\subbottom[\figlabel{ZKSstiffnessH-QZS}
Normalised horizontal stiffness of the system;
negative displacement will result in negative stiffness.]
  {\psfragfig{PhD/Simulations/ZKS/fig/ZKSstiffnessH-QZS}}
\end{wide}
\caption
[Vertical and horizontal stiffness characteristics at
\qzs/ in both directions.]
{Vertical and horizontal stiffness characteristics at
\qzs/ in both directions, for a range of spring length ratios, $\ZKSlengthratio$.}
\figlabel{ZKSstiffness-bothQZS}
\end{figure}

\begin{figure}[p]
  \psfragfig{PhD/Simulations/ZKS/fig/ZKSstiffnessH-NotQZS}
  \caption
  [Normalised horizontal stiffness with parameters to achieve a small range of
displacement with positive stiffness.]
  {Normalised horizontal stiffness of the system at
$\ZKSkratio=0.95\ZKSkratioQZS$ in order to obtain a small range of
displacement around $\ZKSnorm=0$ with positive stiffness (compare
with \figref{ZKSstiffnessH-QZS} which has negative stiffness for
$\ZKSnorm<0$). The vertical \qzs/ condition is unaffected.}
  \figlabel{ZKSstiffness-NotQZS}
\end{figure}

\subsection{Spring compression}

One factor that has not been addressed with this particular design is the
amount of spring compression required at the equilibrium position of \qzs/. The
total allowable compression will be limited by the shape and properties of the
springs themselves. By adjusting the design parameters of the inclined
spring system, the amount of compression in each spring at equilibrium can be
selected.

The amount of spring compression can be analysed with a metric here called the `compression ratio' related to the uncompressed spring length.
For the inclined and vertical springs, respectively, the compression ratios $\ZKScompressI$ and $\ZKScompressV$ are given by
\begin{dseries}[label=ZKScompress]
\begin{math}
\ZKScompressI = 1 - \frac{\ZKSlengthcomp}{\ZKSlength}
\end{math},
\begin{math}
\ZKScompressV = 1 - \frac{\ZKSvlengthcomp}{\ZKSvlength}
\end{math}.
\end{dseries}
This metric for the compression ratio was chosen to be zero for a spring in its uncompressed position and unity if it is compressed to its full length (which is the theoretical limit for springs in compression).

The compression ratios were evaluated at the \qzs/ condition in both directions; \ie,
$(\ZKSshift,\ZKSdisp)=(0,0)$, $\ZKSratio=\ZKSratioQZS$, and
$\ZKSlengthratio=\ZKSlengthratioQZS$ (see \eqref{ZKSratioQZS,ZKSQZS}),
yielding
\begin{dseries}[label=ZKScompressQZS]
\begin{math}
{\ZKScompressI\Big|}\ZKSQZS =
  \frac{\ZKSkratio}{\ZKSkratio+2}
\end{math},
\begin{math}
{\ZKScompressV\Big|}\ZKSQZS =
  \frac{2}{\ZKSkratio+2}
\end{math}.
\end{dseries}

The compression ratios of the vertical and inclined springs at \qzs/ are
shown in \figref{ZKScompress} as functions of varying spring stiffness ratio
$\ZKSkratio$. (Recall that \qzs/ is achieved by adjusting $\ZKSlengthratio$
for each specified value of $\ZKSkratio$ with the relationship shown in
\eqref{ZKSQZS}.) These results show that a large compression (greater
than fifty percent) in at least one of the springs is required to achieve
\qzs/.

\begin{figure}
\psfragfig{PhD/Simulations/ZKS/fig/ZKScompress}
\lofcaption{Compression relationship of the springs for a range of spring
stiffness ratios, at \qzs/ both horizontally and vertically.}
{
  A compression of 100\% implies a change in displacement of the entire spring length, which is difficult to realise in practice.
}
\figlabel{ZKScompress}
\end{figure}

\subsection{The general applicability of the inclined springs system}

In order to adapt this system to withstand time-varying load conditions, both
the anchor positions of the inclined and the vertical springs must be adjusted
in order to tune for, first, the required load bearing, and second, the amount
of negative stiffness required to achieve \qzs/ in the vertical direction.

Dynamically changing the system for \qzs/ in the horizontal direction requires
that either the uncompressed spring lengths or the spring stiffnesses be
adjustable during operation.
Spring lengths could be adjusted by mounting the springs on ball-screws or a similarly stiff actuator.
Springs stiffnesses could be varied, \eg, with \magnetorh/ fluid springs, or using geometric effects with leaf springs in a four-bar linkage~\cite{choi2011-ietr}.

Note that the principles discussed for horizontal stiffness can be extended to a three-dimensional system, most easily with a rotationally-symmetric structure with horizontally aligned springs in each plane.
The main conclusion from the analysis in this section is that \qzs/ in both vertical and horizontal directions can be achieved with this spring arrangement, which is not generally considered in analyses of this type.
However, the tuning required to maintain \qzs/ for this spring arrangement is difficult to achieve in practice.
In the next section, a magnetic system is examined in detail that exploits a similar negative stiffness principle but can be more easily adjusted online.

\section{Magnetic \qzs/}
\seclabel{qzs}

This section examines a different system that can also exhibit localised zero stiffness.
Introduced in \secref{qzs-basics}, this system consists of a pair of fixed magnets that supports a mass against gravity by respectively repelling the mass from below and attracting it from above, as shown in \figref{schem}.
This arrangement of magnets has seen some previous attention~\cite{nijsse2001,robertson2006-activeconf,robertson2007-icsv}, in which the force \vs\  displacement characteristic can be approximated by a quadratic polynomial for small gaps between the magnets.

\begin{figure}
  \asyinclude{\jobname/zerok-comb}
  \caption
  [Schematic of a magnetic spring with \qzs/.]
  {Schematic of a magnetic spring with \qzs/ at
 $\ndispZ=0$ to isolate displacement $\dispMass$ from ground disturbance vibration $\dispBase$.
 Large arrows indicate direction of polarisation of the magnets.
 In this section, cubical magnets are used with side length $\mdim$,
 distance $2\mdim\ngap$ between the centres of the fixed magnets,
 and displacement of the floating magnet from the zero stiffness
 position $\mdim\ndispZ$. Rest position at positive $\ndispZ$ (upwards)
 corresponds to unstable equilibrium; negative $\ndispZ$ (downwards) is stable.}
  \figlabel{schem}
\end{figure}

A `cubic-like' force curve such as discussed in \secref{qzszks} can be more
useful because it creates a stable inflection point with localised zero
stiffness;
% \note{Bear in mind, however, that only with careful tuning can the system \emph{approach} a local \qzs/ condition. This is because the configuration exists in only a marginally stable position itself, but (assuming no change in system parameters) this is in reference to the static forces not the dynamic ones, which is the big difference.}
in contrast, a
quadratic-type spring is marginally stable at its \qzs/ position and must be
operated away from this point to avoid moving into the unstable region.
For example, the inclined spring system can be tuned such that its force characteristic is very close to the cubic-like \qzs/ regime (\figref{ZKSoff}) and displacements around the equilibrium point are guaranteed to be stable.
On the other hand, for the \qzs/ magnetic spring, as soon as the mass displaces closer to the top magnet than the bottom the system is now unstable regardless of the tuning of the magnet locations.
Nonetheless, the magnet arrangement is worth studying as a mechanism to
achieve \emph{low} stiffness, since this device reduces the stiffness in all
three translational degrees of freedom in a mechanically simple way compared
to the inclined spring system studied in the previous section.

As well as its use to design low frequency isolation mounts, this idea of
non-contact magnetic forces can have particular application in support
structures where a reduction in stiffness is desired to mitigate a vibration
problem that has been discovered after its construction. The noncontact nature
of the force between the magnets allows their easy attachment to an existing
structure.

The structure of this section is as follows. The exact equation for the
forces between two cubical magnets is presented in \secref{qzs-mag}, which is
used to calculate the exact and approximate force \vs\  displacement
profiles of the \qzs/ magnetic spring. In \secref{cons}, the
criteria that govern the behaviour of a vibration isolation device
are applied to the magnetic spring and suitable ranges for the
design parameters are found to achieve the design goals.
Finally, \secref{nonlin} analyses the dynamic behaviour of the spring
to ensure that the isolation capabilities are not compromised by the
nonlinear characteristics of the magnetic spring.


\subsection{Magnet forces}
\seclabel{qzs-mag}

For this study, the simplified expressions of \textcite{akoun1984} for cubical magnets with a vertical offset are used, as shown in \eqref[vref]{magforce}.
The distance between the magnet centres $\cdz$ with respect to the size of the magnets (side length $\mdim$) is expressed as a normalised length $\ndisp$:
\begin{dmath}
  \ndisp = \cdz/\mdim .
\end{dmath}
After some manipulation of the original equation given these simplifying
assumptions, the force $\magforce\fn{\mdim,\ndisp}$ on the second magnet in
attraction (\ie, for magnets with polarisation in the same direction)
can be shown to be directly proportional to the facing area of the magnets,
$\mdim^2$, for a fixed normalised displacement, $\ndisp$, between the magnets:
\begin{dmath}[label=force]
  \magforce\fn{\mdim,\ndisp} = \mdim^2 \nforce\fn{\ndisp} .
\end{dmath}
For magnets in repulsion, the expression has opposite sign. The derived
expression for the normalised force $\nforce\fn{\ndisp}$ was given in
\eqref{nforce} in \secref[vref]{cube-forces}.

The stiffness between two cubical magnets can be calculated by differentiating
the force expression in \eqref{force} with respect to vertical displacement
$\cdz=\mdim\ndisp$ and can be shown to be proportional to the magnet size $\mdim$:
\begin{dmath}[label=stiffness,compact]
  \magstiffness\fn{\mdim,\ndisp} = \frac{\partial}{\partial\cdz} \magforce\fn{\mdim,\ndisp}
             = \frac1\mdim\frac{\partial}{\partial\ndisp} \mdim^2\nforce\fn{\ndisp}
             = \mdim\nstiffness\fn{\ndisp} .
\end{dmath}
The derived expression for the normalised stiffness $\nstiffness\fn{\ndisp}$
is given in \eqref[vref]{nstiffness}.

A `\qzs/' magnetic spring consists of an attracting magnetic pair above a repelling magnet pair as shown in \figref{schem}, originally introduced in \secref{qzs-basics}.
Parameter $\mdim\ngap$ is the gap between the centres
of the magnet pairs at \qzs/, and $\dispMass=\mdim\ndispZ$ is the
static displacement of the floating mass about the centre line between
the magnets.
The parameters $\ngap$ and $\ndispZ$ are referred to
as normalised magnet gap and normalised magnet displacement, respectively.
The force due to the lower magnet in repulsion is
\begin{align}
\frepl &= -\magforce\fn{\mdim,\mdim\ngap+\mdim\ndispZ} \\
       &= -\mdim^2\nforce\fn{\ngap+\ndispZ},
\end{align}
and the force due to the upper magnet in attraction is
\begin{align}
\fattr &= \magforce\fn{\mdim,-\mdim\ngap+\mdim\ndispZ} \\
       &= \mdim^2\nforce\fn{-\ngap+\ndispZ}.
\end{align}
The total force on the floating magnet, $\magforceZ$, is a superposition
of $\frepl$ and $\fattr$, yielding
\begin{align}
  \magforceZ &= \frepl+\fattr \\
          &= \mdim^2\gp{-\nforce\fn{ \ngap+\ndispZ}
                     +\nforce\fn{-\ngap+\ndispZ}} \\
          &\eqdef \mdim^2 \nforceZ . \eqlabel{nforceZ}
\end{align}
The stiffness of the system can be similarly expressed as
\begin{dmath}
  \magstiffnessZ = \mdim\nstiffnessZ
\end{dmath},
where
\begin{dmath}[label=nstiffnessZ]
  \nstiffnessZ = -\nstiffness\fn{ \ngap+\ndispZ}
                  +\nstiffness\fn{-\ngap+\ndispZ}
\end{dmath}.
The force $\nforceZ$ and stiffness $\nstiffnessZ$ of the magnetic spring are
readily calculated for values of normalised displacement and gap, $\ndispZ$
and $\ngap$, from \eqref{nforce,nstiffness}.
However, these
expressions are too complex for use in any calculation where they must be
inverted (say, finding a value of $\ngap$ for which a certain $\nstiffnessZ$
holds). It is therefore necessary to obtain a simpler model of $\nforceZ$.

Previously \cite{nijsse2001,robertson2006-activeconf}, $\nforceZ$ has been modelled as a quadratic polynomial with coefficients $\qqO$, $\qqII$ that vary with magnet gap:
\begin{dmath}[label=quad]
\nforceZ \approx \qqII \,\ndispZ^2 + \qqO  .
\end{dmath}
Over small displacement ranges this approximation yields adequate results, but
the resulting model is dependent on the gap between the magnets and local to
the displacement range used to fit the model.

A more accurate result (which is also accurate over larger displacement
ranges) can be achieved with a quartic polynomial approximation,
\begin{dmath}[label=quartic]
\nforceZ \approx \qqIV \,\ndispZ^4 + \qqII \,\ndispZ^2 + \qqO ,
\end{dmath}
but the same criticisms hold as for the quadratic polynomial model: the
resulting model is not general; furthermore, if the model approximation is to
also represent changes in magnet gap $\ngap$, the functions $\qqO$,
$\qqII$, and $\qqIV$ require high-order polynomials to represent the
variation sufficiently well.

Due to the complexity of the expression, a simpler approximation of
the normalised force $\nforce\fn{\ndisp}$ can be found by numerically fitting the
constant coefficients $\ccA$, $\ccB$, and possibly $\nn$ in the empirical
approximation for the forces between two magnets
\begin{dmath}[label=nnfit]
  \nforce\fn{\ndisp} \approx \frac{\ccA}{\gp{\ccB+\ndisp}^\nn}.
\end{dmath}
\textcite{xu1993} used the more complicated approximation
\begin{dmath}
  \nforce\fn{\ndisp} \approx \ccD\gp{\frac{\ccA}{\ccB+\ndisp}}^\nn+\ccC
\end{dmath}
although in this case the additional complexity does not justify
the slight increase in accuracy this expression may offer.

\textcite{bonisoli2007-mssp,bonisoli2007-mrc} used $\nn=3$ in their work, and
\textcite{piombo2003} recommended either $\nn=2$ or $\nn=4$. The values for
$\nn$ they selected were based on the best fit curves of the forces for the
specific geometry of their magnets in each case.
As these are empirical equations, however, $\nn$ is not restricted to integer values and does not have to be chosen \emph{a priori} to the curve fitting.

An unbounded least squares fit is performed with \eqref{nnfit} varying all three
parameters $\ccA$, $\ccB$, and $\nn$ over the range $2\le\ndispZ\le5$
to achieve an approximation of \eqref{nforce}. Over the displacement
range used to fit the model, modelling errors of less than $1\%$ are
achieved. (The remanence magnetisation of each magnet,
$\remanence$, is taken as unity.) Including displacements outside of this range
(especially $1<\ndispZ\le2$) diminishes the accuracy of the fit, but is less
relevant for this work: displacements $1<\ndispZ\le2$ generate high
stiffnesses, and displacements $\ndispZ\ge5$ have low supporting forces. Both
of these properties are undesirable for a vibration isolator in terms of
resonance frequency and adequate load bearing, respectively.

\Figref{nnfit} shows the approximations found with \eqref{nnfit} for set
values $\nn=3$ and $\nn=4$ (only coefficients $\ccA$ and $\ccB$ are varied);
in the third curve, $\nn$ has also been allowed to range for the least-squared
fit. \Tabref{nnfit} displays the values for all three calculated parameters in
each case. The curves for $\nn=2$ and $\nn=5$ produce much greater errors than
those shown in \figref{nnfit} and have been omitted for clarity.

\begin{figure}
  \psfragfig{PhD/Writings/QZS/fig/nnfit}
  \caption
  [Modelling errors of \eqref{nnfit} compared to the exact.]
  {Modelling errors of \eqref{nnfit} compared to the exact \eqref{nforce}, for parameters shown in \tabref{nnfit}.}
  \figlabel{nnfit}
\end{figure}

\begin{table}
  \lofcaption{
    Best fit parameters for \eqref{nnfit}.
  }{
    Fixed integer values of $\nn$ were chosen
    for the first two cases, and the latter
    value best fits the model by varying all
    three parameters.
    Note that these are unitless parameters.
  }
  \tablabel{nnfit}
  \begin{tabular}{@{}cc@{\qquad}c@{}}
    \toprule
    $\nn$ & $\ccA$ & $\ccB$ \\
    \midrule
    \num{3} & \num{6.580e4} & \makebox[0pt][r]{$-$}\num{0.5796} \\
    \num{4} & \num{4.071e5} & \num{0.0607} \\
    \num{4.197} & \num{6.028e5} & \num{0.1883} \\
    \bottomrule
  \end{tabular}
\end{table}

An empirical approximation of \eqref{force} for the force between two coaxial cubical magnets  is thus given by
\begin{equation}
\magforce\fn{\mdim,x} \approx \pm \mdim^2 \remanence^2 \frac{\SI{6.028e5}{}}{\gp{0.1883+x/\mdim}^{4.197}}
\end{equation}
with force $\magforce\fn{\mdim,x}$ in Newtons, magnet size $\mdim$ and centre displacement
$x$ both in metres, and remanence magnetisation $\remanence$ in Tesla, where the expression is
positive for magnets in repulsion and negative for magnets in attraction.

\Figref{nnforceZ} shows the model of
the magnetic system using \eqref{nnfit} to calculate the force due to the
repelling and attracting magnets separately:
\begin{dmath}[label=nnfitz]
  \nforceZ \approx \ccA\gp{\ccB+\ngap+\ndispZ}^{-\nn}
                  +\ccA\gp{\ccB+\ngap-\ndispZ}^{-\nn} ,
\end{dmath}
where $\ccA$, $\ccB$, and $\nn$ are the best-fit parameters previously
discussed. This solution is both simpler in form and more accurate than the
polynomial models of \eqref{quad,quartic}. It is also much easier to invert
numerically than the exact \eqref{nforceZ}, although note that \eqref{nnfitz}
still cannot be algebraically inverted; this is not an issue for the work to
be discussed in the following sections. The normalised stiffness can be
approximated by differentiating \eqref{nnfitz} with respect to $\ndispZ$ (as
shown previously in \eqref{stiffness}):
\begin{dmath}[label=nnfitzk]
  \nstiffnessZ \approx {\nn\ccA}{\gp{\ccB+\ngap+\ndispZ}^{-\nn-1}}
                 + {\nn\ccA}{\gp{\ccB+\ngap-\ndispZ}^{-\nn-1}}.
\end{dmath}

\begin{figure}
  {%
    \let\labelsize\footnotesize
    \def\LBL#1{\colorbox{white}{#1}}%
    \psfragfig{PhD/Writings/QZS/fig/nnfit-z}%
  }%
  \caption
  [Normalised force \vs\  displacement curves of a \qzs/ magnetic system.]
  {Normalised force $\nforceZ$ \vs\  displacement $\ndispZ$
    curves of a \qzs/ magnetic system for a range of
    normalised gap $\ngap$ (the stiffness is zero at $\ndispZ=0$).
    Single points correspond to the exact
    solution; solid lines correspond to the approximation given by
    \eqref{nnfit}.}
  \figlabel{nnforceZ}
\end{figure}

In this section, a model of the magnet forces was presented that is accurate
over large relative displacements. Despite the fact that the vibration
disturbance will occur in small magnitudes relative to the size of the magnet
dimension, it is important to model the magnet forces accurately over a
large displacement range so that the design evaluations in the next section
may be applied over variations in magnet gap.


\subsection{Design criteria}
\seclabel{cons}

In the previous section, a magnetic system was introduced with an exact
expression and various approximations for calculating the force \vs\  displacement
characteristics for a range of system designs. It is proposed that this system is
suitable for a vibration isolation platform due to the possibility of low
inherent stiffness of the design. However, the stiffness is dependent on the
load that is being supported, and the magnetic arrangement must be designed
for this purpose. In this section, constraints are imposed on the system
parameters to satisfy these criteria and a design principle developed.

The normalised equilibrium position of the system $\heq{}$ can be found by
equating the magnet force at equilibrium $\feq{}$ with the load due to
gravity $\mass\gravity$ and inverting numerically:
\begin{dmath}[label=eq,compact]
\feq = \nforceZ{1,\heq{}} = \mass \gravity/\mdim^2 .
\end{dmath}
A positive displacement of the mass ($\ndispZ\ge0$) is unstable, and so
\eqref{eq} must be solved such that $\heq{}<0$. \Figref{heq-map}
shows the equilibrium position $\heq{}$ varying over $\mdim$ and $\ngap$ for a
system of mass $\mass=\SI{0.5}{kg}$.

\begin{figure}
  \psfragfig{PhD/Writings/QZS/fig/heq-map}
  \caption
  [Map of the normalised equilibrium position over varying
    magnet size and normalised magnet gap.]
  {Map of the normalised equilibrium position $\heq{}$ over varying
    magnet size $\mdim$ and normalised magnet gap $\ngap$ for a system
    with mass $\mass=\SI{0.5}{kg}$.}
  \figlabel{heq-map}
\end{figure}

The normalised equilibrium position $\heq{}$ of a mass can be approximated
from the polynomial fits of the force curve by equating, \eg, \eqref{quartic}
with the force due to gravity and solving for $\ndispZ$. However, the more
accurate approximation \eqref{nnfitz} cannot be algebraically rearranged to
solve for $\ndispZ$; hence, solutions based on that equation must be obtained
numerically.

The magnet size and gap must be chosen based on $\heq{}$ to ensure that the
spring is of sufficient stiffness to support a desired load but not so strong
that the supported mass cannot be in equilibrium in the stable region of the
spring. A static deflection, or `strong', criterion is defined to achieve
these constraints:
\begin{dmath}[compact]
\nforceZ{1,0} < \mass \gravity/\mdim^2 < \nforceZ{1,-\ngap+1}, \eqlabel{strong}
\end{dmath}
which is shown graphically in the vertical axis of \figref{heq}.  This criterion is
depicted in \figref{cons-strong} as a region over the parameters
$\mdim$ and $\ngap$ for three different mass loads.
The shape and location of the regions
show that for a fixed magnet gap, larger magnets are
required to support larger loads.

\begin{figure}
  \asyinclude{\jobname/heq-combined}
  \caption
  [Normalised force \vs\ normalised displacement curve illustrating the bounds of allowable load and the displacement range.]
  {Normalised force \vs\  normalised displacement curve
 illustrating the bounds of both allowable load that can be bourne
 and the displacement range of the magnet; see \eqref{strong,saturate,stable}
 respectively for the mathematical representations. $-\ngap+1$ is
 the position where the floating magnet is touching the upper face
 of the lower fixed magnet, $2\dmax/\mdim$ is the displacement bound due to the external excitation, and $-\dmin$ is the closest allowable
 position to the \qzs/ position.}
  \figlabel{heq}
\end{figure}

\begin{figure}
  \centerline{
    {\psfragfig{PhD/Writings/QZS/fig/cons-strongN1}}
    {\psfragfig{PhD/Writings/QZS/fig/cons-strongN2}}
    {\psfragfig{PhD/Writings/QZS/fig/cons-strongN3}}
    {\psfragfig{PhD/Writings/QZS/fig/cons-strongNall}}
    }
  \caption
  [Regions of $\mdim$ and $\ngap$ satisfying
    the static deflection criterion.]
  {
    Regions of $\mdim$ and $\ngap$ satisfying
    the static deflection criterion
    of \eqref{strong} for a range of masses.
    Darker sections denote overlap of the
    regions in the overlay plot.}
  \figlabel{cons-strong}
\end{figure}

Assuming that linearisation is appropriate to model the vibration
response of the system, the equivalent linear stiffness at equilibrium,
$\keq{}$, must satisfy the resonance frequency criterion:
$\keq{}=\magstiffnessZ{1,2,\heq{}}\le\freqdesign^2\mass$, where $\freqdesign$ is the
maximum allowable resonance frequency. From \eqref{nnfitzk}, this
stiffness, or `soft', requirement can be formulated as
\begin{dmath}
\gp{\ccB+\ngap+\heq{}}^{-\nn-1}
                + {\gp{\ccB+\ngap-\heq{}}^{-\nn-1}}
   \le \frac{\freqdesign^2 \mass}{\mdim\nn\ccA} . \eqlabel{soft}
\end{dmath}
This criterion is shown for a range of desired nominal stiffnesses in
\figref{cons-soft}. As the resonance frequency decreases, larger
magnets are required to support the load with a small force gradient.

\begin{figure}
  \centerline{
    {\psfragfig{PhD/Writings/QZS/fig/cons-softN1}}
    {\psfragfig{PhD/Writings/QZS/fig/cons-softN2}}
    {\psfragfig{PhD/Writings/QZS/fig/cons-softN3}}
    {\psfragfig{PhD/Writings/QZS/fig/cons-softNall}}
    }
\caption[Regions of $\mdim$ and $\ngap$ satisfying the resonance frequency
  criterion.]{Regions of $\mdim$ and $\ngap$ satisfying the resonance frequency
  criterion of \eqref{soft} for a range of allowable resonance frequencies and a
  mass $\mass=\SI{0.5}{kg}$. Darker sections denote overlap of the
  regions in the overlay plot.}
  \figlabel{cons-soft}
\end{figure}

Finally, the amplitude of the input vibration must be smaller than the
physical displacement bounds of the system. This can be visualised on a normalised
force/displacement plot as shown in \figref{heq}. For a maximum
(absolute) displacement, $\dmax$, of the mass,
\begin{align}
\heq{} - \dmax/\mdim & \ge -\ngap+1  , \eqlabel{saturate} \\
\heq{} + \dmax/\mdim & < -\dmin , \eqlabel{stable}
\end{align}
where $\mdim \dmin$ is the tolerance of the closest allowable distance to
the (marginally stable) \qzs/ position at which the system can operate.
\Eqref{saturate} is a maximum displacement (`saturatation') criterion to ensure that the lower
displacement bound lies above the face of the fixed lower magnet.
\Eqref{stable} is a `stability' criterion to provide a buffer region to ensure
that the moving magnet is not perturbed past the \qzs/ position into the
unstable zone.

These latter constraints impose only small limits on the design of the
system. The maximum displacement criterion, shown in \figref{cons-saturate} as
a region over the parameters $\mdim$ and $\ngap$ for a range of
$\dmax$, limits the lower size of the magnet.

\begin{figure}
  \centerline{
    {\psfragfig{PhD/Writings/QZS/fig/cons-saturateN1}}
    {\psfragfig{PhD/Writings/QZS/fig/cons-saturateN2}}
    {\psfragfig{PhD/Writings/QZS/fig/cons-saturateN3}}
    {\psfragfig{PhD/Writings/QZS/fig/cons-saturateNall}}
    }
  \caption[Regions of $\mdim$ and $\ndispZ$ satisfying the
    maximum displacement criterion.]{Regions of $\mdim$ and $\ndispZ$ satisfying the
    maximum displacement criterion
    of \eqref{saturate} for a range of disturbance
    displacements $\dmax$ and a mass $\mass=\SI{0.5}{kg}$. Darker
    sections denote overlap of the regions in the overlay plot.}
  \figlabel{cons-saturate}
\end{figure}

Note that the maximum displacement of the spring will not be symmetric
with input displacement due to the softening spring stiffness. Without
loss of generality, however, it is possible to represent the maximum displacement
in the stability and maximum displacement criteria (\eqref{saturate,stable})
with the same symbol.

The stability criterion is governed by two parameters, $\dmax$ and
$\dmin$, which are varied in \figref{cons-stable-1,cons-stable-2}
respectively. Again, this criterion only has a small effect on the
constraint region, but it is an important effect. Because the stability
criterion limits the minimum distance between the \qzs/ position and the
equilibrium position, this prevents the spring from reaching the very
low equilibrium stiffnesses found just below the \qzs/ position.

\begin{figure}
  \centerline{{\psfragfig{PhD/Writings/QZS/fig/cons-stableN1-1}}
         {\psfragfig{PhD/Writings/QZS/fig/cons-stableN2-1}}
         {\psfragfig{PhD/Writings/QZS/fig/cons-stableN3-1}}
         {\psfragfig{PhD/Writings/QZS/fig/cons-stableNall-1}}}
\caption[Regions of $\mdim$ and $\ngap$ satisfying the
  the stability criterion.]{Regions of $\mdim$ and $\ngap$ satisfying the
  the stability criterion of \eqref{stable} for a range of normalised displacement
  tolerances $\dmin$ and a maximum disturbance displacement of
  $\dmax=\SI{5}{mm}$. Darker sections denote overlap of the regions in the overlay plot.}
  \figlabel{cons-stable-1}
\end{figure}

\begin{figure}
  \centerline{{\psfragfig{PhD/Writings/QZS/fig/cons-stableN1-2}}
         {\psfragfig{PhD/Writings/QZS/fig/cons-stableN2-2}}
         {\psfragfig{PhD/Writings/QZS/fig/cons-stableN3-2}}
         {\psfragfig{PhD/Writings/QZS/fig/cons-stableNall-2}}}
\caption[Region of $\mdim$, $\ngap$ satisfying the stability criterion.]{Region of $\mdim$, $\ngap$ satisfying the stability criterion
  of \eqref{stable} for a range of disturbance displacements $\dmax$
  and a normalised displacement tolerance $\dmin=\num{0.1}$. Darker
  sections denote overlap of the regions in the overlay plot.}
  \figlabel{cons-stable-2}
\end{figure}

The introduced criteria may be simultaneously satisfied for some specified
values of mass, resonance frequency, and displacement range ($\mass$, $\freq$,
and $\dmax$) by varying the free parameters magnet size $\mdim$ and magnet gap
$\mdim\ngap$.
A summary of the criteria introduced previously is shown in \tabref{crit-summary}.
The shared region of the criteria previously introduced
(\eqrangeref{strong}{stable}) are shown in \figref{cons-all} for various
combinations of supported mass and resonance frequency. This figure presents a
complete design map that shows how the techniques presented in this paper can
be used for speculative design work and optimisation.

\begin{table}
\caption{Summary of design criteria for \qzs/ magnetic design.}
\tablabel{crit-summary}
\begin{tabular}{lll}
\toprule
Criterion & Equation & Figure \\
\midrule
Strong   & \eqref*{strong}   & \figref*{cons-strong} \\
Soft     & \eqref*{soft}     & \figref*{cons-soft}  \\
Saturation & \eqref*{saturate} & \figref*{cons-saturate}  \\
Stability& \eqref*{stable}   & \figref*{cons-stable-1,cons-stable-2}  \\
\bottomrule
\end{tabular}
\end{table}

\begin{figure}
    \centering
    \gridVI
      {\psfragfig{PhD/Writings/QZS/fig/cons-allN1}}
      {\psfragfig{PhD/Writings/QZS/fig/cons-allN2}}
      {\psfragfig{PhD/Writings/QZS/fig/cons-allN3}}
      {\psfragfig{PhD/Writings/QZS/fig/cons-allN4}}
      {\psfragfig{PhD/Writings/QZS/fig/cons-allN5}}
      {\psfragfig{PhD/Writings/QZS/fig/cons-allN6}}%
    \\[1.5em]
    \psfragfig{PhD/Writings/QZS/fig/cons-allNall}
\caption[Regions of $\mdim$ and $\ngap$ satisfying all design criteria.]{Regions of $\mdim$ and $\ngap$ satisfying all design criteria,
  demonstrating the effects of changing the mass $\mass$ and the
  resonance frequency $\freq$ of the desired system. Darker sections
  denote overlap of the regions in the overlay plot.}
  \figlabel{cons-all}
\end{figure}

\subsubsection{Measure of stiffness reduction}
\seclabel{kratio}

In the previous section, ranges for magnet size and magnet gap were
shown to achieve certain design criteria. It is not impossible, however,
that this process could yield a design that behaves acceptably with
only a single repulsive magnet: the magnet gap could be so large as to
render the effect of the negative stiffness negligible.

The total stiffness characteristic of the magnetic system is given
by \eqref{nstiffnessZ} as the sum of the stiffnesses due to repulsive
and attractive magnets respectively. At equilibrium, this can be written as
\begin{dmath}
  \keq = \krepl{1,2,\heq{}} + \kattr{1,2,\heq{}} = \krepl{}\cdot\gp{1-\QZSkratio{}}
\end{dmath},
where
\begin{dmath}
  \QZSkratio = \Abs{\frac{\kattr{}}{\krepl{}}}
\end{dmath}.
The variable $\QZSkratio$ can be considered as the ratio of `stiffness
reduction' achieved by the presence of the attractive magnet. For
$\QZSkratio{}=0$, the upper magnet is providing no negative stiffness
to the system; for $\QZSkratio{}=1$, the equilibrium position is at
quasi-zero stiffness and the system is marginally stable.

\Figref{kratio} illustrates the variation of $\QZSkratio{}$ over a
particular design region; this plot shows that designs achieved with
larger magnet gaps have little influence from the stiffness reducing
effect of the attractive magnet. As $\QZSkratio{}$ tends towards one, the
resonance frequency drops dramatically as the equilibrium position
approaches the \qzs/ position (compare with \figref{heq-map}).

\begin{figure}
\centering
\psfragfig{PhD/Writings/QZS/fig/kratio}
\caption
[Contours of stiffness reduction due to the negative stiffness of the attractive magnet.]
{
  Contours of stiffness reduction $\QZSkratio{}$ due to the negative
  stiffness of the attractive magnet,
  shown inside the region of $\mdim$,
  $\ngap$ satisfying all design criteria for parameters $\mass=\SI{1}{kg}$,
  $\freq=\SI{5}{Hz}$, $\dmax=\SI{1}{mm}$, $\dmin=0.1$.}
\figlabel{kratio}
\end{figure}

\subsection{Nonlinear behaviour}
\seclabel{nonlin}

In \secref{cons}, constraints were imposed on the design of the
magnetic system such that a certain resonance frequency was achieved
for a given mass loading. The slope of the force \vs\  displacement
curve at equilibrium was used as the basis for the resonance frequency
criterion. However, as the mass is perturbed from equilibrium the
stiffness of the spring changes. When designing for vibration
isolation, it is important to ensure that this nonlinearity does not
produce a significant effect in the response of the system.

As a proxy for the nonlinearity of the system, the change in spring stiffness over its maximum peak-to-peak
displacements can be compared to its nominal stiffness at equilibrium.
This measure will be refered to here as the nonlinearity ratio $\nk$, given by
\begin{dmath}[label=nl]
\nk = \frac{\Delta\magstiffnessZ{}}{2\keq{}} ,
\end{dmath}
where
\begin{align}
\Delta\magstiffnessZ{} &= \magstiffnessZ{1,2,\heq{}+\dmax/\mdim}
                      -\magstiffnessZ{1,2,\heq{}-\dmax/\mdim} \\
\keq{} &= \magstiffnessZ{1,2,\heq{}} .
\end{align}
For the quadratic approximation of $\nforceZ$ given by \eqref{quad}, $\nk$ is the ratio
between the maximum displacement and the equilibrium position:
\begin{dmath}[label=nk-approx]
\nk \approx \frac{\dmax}{\mdim\heq{}} ,
\end{dmath}
and since $\dmax/\mdim<\Abs{\heq{}}$ for stability (recall
\eqref{stable}), it follows that the degree of
nonlinearity is directly related to the amplitude of disturbance
vibration. \Figref{nl-contours} demonstrates the manner in which $\nk$
varies over an allowed region of design parameters, with $\nstiffnessZ$
given by the exact expression of \eqref{nstiffnessZ}.

The nonlinearity ratio increases both as magnet size and normalised magnet gap
decrease.  Comparing \figref{nl-contours} to \figref{heq-map,kratio}, it
can also be seen that the nonlinearity ratio increases the closer the equilibrium
displacement becomes to the \qzs/ position, and the greater the effect of
the negative stiffness from the attractive magnet.

\begin{figure}
  \psfragfig{PhD/Writings/QZS/fig/nl-contourss}
\caption[Contours of nonlinearity ratio.]{Contours of nonlinearity ratio $\nk$ shown inside the region of $\mdim$,
  $\ngap$ satisfying all criteria for parameters $\mass=\SI{1}{kg}$,
  $\freqdesign=\SI{5}{Hz}$, $\dmax=\SI{1}{mm}$, $\dmin=0.1$.
  A closeup (of the box) is shown in \figref{nl-contours-zoom}.}
  \figlabel{nl-contours}
\end{figure}

\begin{figure}
  \psfragfig{PhD/Writings/QZS/fig/nl-contours-zoom}
  \caption{Zoom of \figref{nl-contours} to show the behaviour of the
 nonlinearity ratio in the lower limits of a magnetic design.}
  \figlabel{nl-contours-zoom}
\end{figure}

To analyse these nonlinear effects on the vibration response of the magnetic
spring, the system shown in \figref{schem} is simulated with the dynamics
\begin{dmath}[label=response]
  \mass \accMass + \damping \gp{\velMass-\velBase} - \magforceZ{1,2,\gp{\dispMass-\dispBase}/\mdim} + \mass \gravity = 0
\end{dmath}
where $\dispMass$ is the displacement of the isolated mass, $\dispBase$ is the
displacement of the external disturbance, and $\damping$ is the viscous damping coefficient. The system is excited
tonally at the damped natural frequency of the linearised system $\resfreq$ in order to depict the steady state response with
the greatest magnitude and therefore greatest nonlinearity ratio.
The excitation has amplitude $\ampExcite$, and is given as a function of time $\Time$ by
\begin{dmath}[label=y]
  \dispBase\fn{\Time} = \ampExcite \sin\fn{\natfreq \Time}
\end{dmath}.

The system parameters for the simulation are $\mass=\SI{0.5}{kg}$,
$\mdim=\SI{20}{mm}$, $\ngap=3$, excitation frequency $\natfreq=\SI{3.58}{Hz}$,
and with these parameters the equilibrium position $\heq{}$ is \SI{10.4}{mm} below the \qzs/
position. While the exact amount of damping present in the system will be
highly model-dependent, the damping ratio $\dampingratio=0.5\damping/\sqrt{\keq{}\mass}$ is chosen as 5\% based on the
small damping forces due to eddy current and air resistance effects
\cite{fung2005,bonisoli2006,nijsse2001}.

\Figref{nl} shows the steady state response of \eqref{response} for a range of
peak-to-peak ground disturbance amplitudes in increments of \SI{0.2}{mm} up to $\ampExcite=\SI{1}{mm}$.
Because the excitation is at resonance, the output displacement is greater
than the input displacement. The phase plot of the response becomes
increasingly skewed as the amplitude of vibration, and hence the nonlinearity ratio,
increases; \tabref{nl} summarises the nonlinearities calculated using
\eqref{nl}. Physically, this is interpreted as the spring being perturbed
further into the stiffer region as the mass is moved closer to the repulsive
lower magnet, which results in stronger forces as the normalised displacement
increases. Conversely, as the mass moves upwards, closer to the \qzs/
position, the stiffness and force both decrease.

The results shown in \figref{nl} indicate that for small magnitude
disturbances the phase plot is very similar to a linear system.
As the amplitude increases, the
nonlinear system response increases at a slower rate than the linear
spring. This slowdown is due to a
shifting of the resonance peak as the nonlinearity ratio increases, as
will be seen later in this section.

\begin{figure}
  \psfragfig{PhD/Writings/QZS/fig/nl}
\caption[Phase plot of the magnetic system at steady
  state resonance.]{Phase plot of the magnetic system at steady
  state resonance for a range of input disturbance amplitudes
  shown in \tabref{nl}.
  The light line is the equivalent linear response.}
  \figlabel{nl}
\end{figure}

\begin{table}
  \caption{Nonlinearity values, $\nk$, of the responses shown in \figref{nl}.}
  \tablabel{nl}
  \begin{tabular}{@{}ccc@{}}
    \toprule
    $\ampExcite$, mm & $\Delta \magstiffnessZ{}$, \si{N/m} & $\nk$ \\
    \midrule
    \num{0.2} & \num{68} & \num{0.13} \\
    \num{0.4} & \num{135} & \num{0.27} \\
    \num{0.6} & \num{198} & \num{0.39} \\
    \num{0.8} & \num{253} & \num{0.50} \\
    \num{1.0} & \num{297} & \num{0.59} \\
    \bottomrule
  \end{tabular}
\end{table}

It is important to consider the role of damping in the results shown
in \figref{nl}. A low damping coefficient results in a greater
displacement response at resonance in the low frequency range. This increase in the
displacement response will also increase the nonlinear behaviour of
the spring. However, the advantage of low damping is a very fast
roll-off in vibration attenuation at frequencies above
resonance. Since the damping of the non-contact magnetic spring is
very low, either the bandwidth of excitation must lie above the
resonance frequency or active sky-hook damping must be applied in
order to reduce the strong effect that the resonance has on the
low-frequency response. As previously discussed, sky-hook damping is
especially suitable for this purpose because the isolation region of
the frequency response is not affected.

\subsection{Variance gain of the magnetic system}

The transmissibility $T_{yx}$ of a system characterises response of a system to inputs that are broadband in frequency.
It is usually calculated using a transfer function approach, given by
\begin{math}
  T_{yx} = P_{yx}/P_{yy}
\end{math},
where $P_{yx}$ is the cross power spectral density between the input and output signals and $P_{yy}$ is the power spectral density of the input signal.
This formulation is generally only useful for linear systems as it is specifically designed to reject nonlinear signals that may arise due to nonlinearities in the system.

One metric to evaluate the response of a nonlinear system in the frequency
domain is known as the `variance gain' $V$ \cite{savaresi2007}, which
is calculated as a ratio of the \RMS/
output to input signals:
\begin{dmath}[label=nl-trans]
\varianceGain = \left.\sqrt{
      \tfrac1\period \Int{{\let\maybedot\bar \dispMass}\fn \Time^2}{\Time,0,\period}} \middle/
    \sqrt{
      \tfrac1\period \Int{{\let\maybedot\bar \dispBase}\fn \Time^2}{\Time,0,\period}}
    \right. ,
\end{dmath}
where $\period$ is the time interval over which the variance gain is calculated,
and $\bar \dispMass$ and $\bar \dispBase$ are the mean-zero output and input displacements
of the vibration isolator, respectively.
For linear systems, this expression simplifies to the standard formulation for transmissibility.
For nonlinear systems, \eqref{nl-trans} describes the ratio of output to input energy for
a given excitation.
For the tonal input disturbance of \eqref{y}, the variance
gain at the frequency of excitation $\freq_e$ is
\begin{dmath}[label=nl-trans2]
\varianceGain{\freq_e} = \frac{1}{\ampExcite}\sqrt{\frac{2}{\period} \Int{{\let\maybedot\bar \dispMass}\fn \Time^2 }{\Time,0,\period}} .
\end{dmath}
The variance gain allows the nonlinearity of the system to be visualised on a familiar transmissibility--like plot for a range of resonant frequencies and equilibrium positions.

The variance gain technique is useful for broadly quantifying the behaviour of nonlinear systems.
For measuring the frequency response function of a predominantly linear system that is contaminated by nonlinear signals whose effects should be discarded, other techniques are available \cite[\eg,][]{schoukens2001}.
There are generalisations of the frequency response function for certain classes of nonlinear systems \cite{lang2007,peng2008}, but with nonlinear analysis different techniques focus on examining different behaviours in the response and there is no single approach that is suitable for all cases.
For example, \textcite{peng2008a} compare two methods, one of which captures `jump phenomena', and the other which shows superharmonic behaviour.
But there is not one general method for classes of nonlinear system, nor perhaps can one exist.
For the purposes of this study, the variance gain is useful as it demonstrates the frequency-dependent behaviour in a manner similar to that which would be measured in a linear system under broadband disturbance.

\Figref{variance-amp} shows the variance gain for a magnetic system with
$\mass=\SI{0.5}{kg}$, $\mdim=\SI{20}{mm}$, and $\ngap=2.8$ with excitation
amplitude ranging from \SI{0.1}{mm} to \SI{0.5}{mm}. With these parameters, the
equilibrium position is \SI{1.55}{mm} below the \qzs/ position. As seen
in the figure, the greater the excitation amplitude, the greater the softening
nonlinearity of the spring.
One particular aspect of the nonlinearity which can be seen is the `snap-through' phenomenon at the resonance peak; this is a common feature of such nonlinear systems in general \cite{dangola2006}, shown particularly clearly experimentally by \textcite{bonisoli2007-mrc} for a magnetic spring exhibiting a hardening nonlinearity.

\Tabref{variance-amp} shows some data from the
simulated responses, including the maximum displacement of the spring towards
the \qzs/ position, and the nonlinearity measures $\nk$ for each amplitude of
excitation. The maximum displacement $x_{\text{max}}$ corresponds with the
maximum normalised displacement $\ndispZ_{\text{max}}$ shown in \figref{heq};
as $\ndispZ_{\text{max}}$ tends towards zero, the motion of the magnetic
spring moves closer to the \qzs/ position. Note that the pronounced
nonlinearity seen in \figref{variance-amp} belongs to a system that is
approaching its bounds of stability, reaching \SI{0.24}{mm} below the \qzs/
position in its most extreme displacement.

\begin{figure}\centering
\let\labelsize\footnotesize
\psfragfig{PhD/Writings/QZS/fig/variance-amp-crop}
\caption
[Variance gain at a position close to \qzs/.]
{Variance gain at a position close to \qzs/, for a range of excitation
amplitudes $\ampExcite$, as labelled. The spring softening effect at resonance can
be clearly seen as the amplitude increases. Tabulated values of nonlinearity
and maximum displacement are shown in \tabref{variance-amp}.}
\figlabel{variance-amp}
\end{figure}

\begin{table}
\lofcaption{Nonlinearity values, $\nk$, and maximum displacements, $x_{\text{max}}$,
of the responses shown in \figref{variance-amp}.}{ The equilibrium stiffness is
$\keq{}=\SI{46.6}{N/m}$. The nonlinearity can be seen to increase with excitation
amplitude. In the most extreme case, the magnetic system departs \SI{0.24}{mm}
from the \qzs/ position.}
\tablabel{variance-amp}
\begin{tabular}{@{}cccc@{}}
\toprule
$\ampExcite$, mm & $\Delta \magstiffnessZ{}$, \si{N/m} & $x_{\text{max}}$, mm & $\nk$ \\
\midrule
 \num{0.1} & \num{29.3} & \num{-1.04} & \num{0.31} \\
 \num{0.3} & \num{57.6} & \num{-0.47} & \num{0.62} \\
 \num{0.5} & \num{66.3} & \num{-0.24} & \num{0.71} \\
\bottomrule
\end{tabular}
\end{table}

\Figref{variance-gaps} shows the variance gain of the same system as the magnet gap
$\ngap$ is increased. Results are shown for excitations of both $\ampExcite=\SI{0.1}{mm}$
and $\ampExcite=\SI{0.5}{mm}$. Data for the simulations with $\ampExcite=\SI{0.5}{mm}$, including
nonlinearity measure $\nk$ and maximum displacement $x_{\text{max}}$, are shown in
\tabref{variance-gaps}. Two related features are important to note from \figref{variance-gaps}.
The first is the large decrease in resonance frequency as the magnet gap decreases. The second
is the corresponding increase in nonlinear behaviour as this occurs, seen with both a snap-through resonance and a superharmonic-like behaviour at twice the resonance frequency.
As the equilibrium position moves away from the instability at \qzs/, the variance gain quickly exhibits linear behaviour.

\begin{figure}\centering
\let\labelsize\footnotesize
\psfragfig{PhD/Writings/QZS/fig/variance-gaps}
\caption
[Variance gain of the magnetic system comparing two excitation
amplitudes.]
{Variance gain of the magnetic system comparing two excitation
amplitudes with varying values of magnet gap $\ngap$, as labelled. Low amplitude
is $\ampExcite=\SI{0.1}{mm}$ and high amplitude is $\ampExcite=\SI{0.5}{mm}$; shown as dashed
and solid lines, respectively. The nonlinear effects (snap-through resonance and superharmonic) are only prominent in the case
close to \qzs/, with an excitation amplitude that almost destabilises the system.
Tabulated values of nonlinearity and maximum displacement are shown for the high
amplitude case in \tabref{variance-gaps}.}
\figlabel{variance-gaps}
\end{figure}

\begin{table}
\caption
  [Nonlinearity values and maximum displacements of the high amplitude responses shown in \figref{variance-gaps}.]
  {Nonlinearity values, $\nk$, and maximum displacements, $x_{\text{max}}$,
of the high amplitude responses ($\ampExcite=\SI{0.5}{mm}$) shown in \figref{variance-gaps},
for a range of
magnet gaps, $\ngap$. As the stiffness is decreased by the upper attractive magnet,
greater nonlinearity is seen for smaller magnet gaps, as the equilibrium position
moves closer to the point of \qzs/.}
\tablabel{variance-gaps}
\begin{tabular}{@{}ccccc@{}}
\toprule
$\ngap$ & $\keq{}$, \si{N/m} & $\Delta \magstiffnessZ{}$, \si{N/m} & $x_{\text{max}}$, mm & $\nk$ \\
\midrule
 2.8 & \hphantom{1}66.3 & 46.6 & $-0.24$ & 0.71 \\
 2.85 & 138.8 & 145.8 & $-2.72$ & 0.48 \\
 3.0 & 167.6 & 253.5 & $-7.80$ & 0.33 \\
\bottomrule
\end{tabular}
\end{table}

\subsection{Summary of magnetic \qzs/ design}
\seclabel{end}

This section has analysed a magnetic spring for the purposes of
load bearing with low stiffness. Approximate expressions
were derived for cube-shaped magnets for analysing the behaviour of
this system, which are simple and accurate over a large displacement range, and may be used for cubical magnets of any size.

Four design criteria were imposed on the design in terms of the two variable design
parameters: magnet size and gap between the fixed magnets. A technique for explicitly
mapping these constraints to a map of valid parameters was shown. Allowable stroke
and magnet size availability allow a design to be optimised using this technique
based on the required load bearing and resonance frequency.

The magnetic isolator is weakly nonlinear with a phase plot that is distorted compared to
a linear system; the variance gain shows a resonance peak skewed into the lower
frequencies. These nonlinearities only become apparent at larger vibration amplitudes
at equilibrium positions that are close to the \qzs/ position
(that is, large measures of nonlinearity). Provided the system remains stable,
the nonlinearities are not detrimental to the frequency response of the system.

The vibration isolator described in this section is therefore suitable for
precision applications where low resonance frequencies are required. The
design is scalable in that many such isolators may be used in parallel to
achieve greater load bearing even if only small magnets are available. The
inherent low damping of the system results in good high frequency performance,
but low frequency disturbances will result in large outputs due to the high
resonance peak. This effect could be mitigated by the application of sky-hook
damping to the system.

Further research on the practical development of this system is currently being undertaken at The University of Adelaide, with a particular focus on adaptive load balancing, and control for both stabilisation and vibration attenuation in six degrees of freedom \cite{zhu2011-icmt}.

\section{Conclusion on \qzs/ systems}

This chapter consisted of separate analyses of two \qzs/ systems.
The first was a mechanical design using linear springs, seen in the literature, and analysed in a planar form herein for the first time.
It was shown that \qzs/ was possible to achieve in both horizontal and vertical directions simultaneously with careful tuning of the parameters of the system.
However, were the operating conditions of the system to change, this tuning would be difficult to achieve in practice.

The second analysis considered a simple magnetic \qzs/ arrangement and investigated the parameters necessary to design such a system based on desired load bearing and allowable operating parameters.
A set of criteria was developed and the solution space for one such design investigated.
The nonlinear aspects of using magnetic forces were considered and it was shown that nonlinearities became significant when the spring tended close to its \qzs/ position.

An important point to emphasise for \qzs/ systems is that their ideal operating condition  cannot be achieved in practice.
For the magnetic design with a quadratic-shaped force curve, instability results from displacements that breach the \qzs/ position since the stiffness becomes negative.
For the inclined linear spring design with a cubic-shaped force curve, any slight deviation past the \qzs/ tuning point results in a spring with greater stiffness and a shifted equilibrium position.
In both cases, the operating point of the system must be chosen as close as possible to the \qzs/ position; the more extreme the operating requirements the finer the tolerances on manufacturing and, if necessary, stability measures.

\end{document}
