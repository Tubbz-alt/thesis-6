%!TEX root = thesis.tex

\chapter{\QZS\ design}
\chaplabel{qzs}

\chapterprecis{
  This chapter investigates the nonlinear forces of two different springs
  (one mechanical, one magnetic) to see how `\qzs' can be used to aid
  vibration isolation. The mechanical system with \qzs\ develops theory introduced previously
  by other researchers as a comparison to the magnetic system; horizontal
  forces as well as vertical forces are investigated and some regions of
  instability are identified. In the magnetic case, a set of design criteria
  are developed around load bearing and vibration isolation properties and
  a generalised design pattern is formulated to show how the magnetic system
  can be developed.
}

\section{Introduction to \qzs\ springs}

In a conventional mass--spring system, the static deflection increases as the
stiffness of the support is reduced, and a lower limit on the stiffness is
imposed by constraints on the allowable displacement. Consider the linear
isolator with force balance described by
\begin{math}
  \mass \gravity = \stiffness \disp
\end{math},
and a natural frequency of
\begin{math}
  \natfreq = \sqrt{\stiffness/\mass}
\end{math}.
The displacement required to support this mass can be written
\begin{math}
  \disp = \gravity / \natfreq^2
\end{math}.

This inverse-quadratic relationship imposes a heavy penalty on the static
displacement required to support loads at low frequency (note that this
expression is independent of the mass of the object being supported). While a
resonance at \SI{5}{Hz} requires a static displacement of approximately
\SI{1}{cm}, a \SI{1}{Hz} resonance requires a displacement of almost
\SI{25}{cm}. A different approach is desired for achieving a load bearing
relationship that can exhibit a low \emph{dynamic} stiffness without the need
for an associated large static deflection.

An example of a system with such behaviour is that of a cubic force vs.\
displacement characteristic; for a dynamic force characteristic of, say,
$\func{f}{x}=f_0+x^3$, localised zero stiffness occurs at zero deflection
(\ie, $\func{f'}{0}=0$), which is termed `\qzs'. The `quasi' qualifier is
included for two reasons: because the zero stiffness is only valid at a single
displacement; and because the system is usually infeasible to implement in a
physical system. An investigation into the literature yields a variety of such
devices, as previously discussed in \secref{vibrations-qzs}.

This chapter will investigate two of these \qzs\ systems. The first, in
\secref{qzskzs}, is an `inclined spring' device that has been studied
previously in the literature; here, it is studied from a fresh perspective in
order to contrast its behaviour with the second \qzs\ system shown in
\secref{qzs}: an arrangement of attracting and repelling permanent magnets
that allows much simpler tuning of the system to adapt to dynamic load
bearing.

\section{\QZS\ is not zero stiffness}
\seclabel{qzs-not-zerk}

It has been established that the goal of a `zero stiffness' device is to
reduce the resonance frequency of the system to as low a value as possible. In
the limiting case, if the system is stable and the nominal force of the spring
indeed matches the weight of the mass, then the gradient of the force at the
operating point will equal zero.

However, it is necessary to use a nonlinear spring to achieve this zero
stiffness condition. And the behaviour of a nonlinear oscillator varies
considerably from that of the classic linear spring. Most obviously, the shape
of the frequency response is not independent of the amplitude of the forcing
disturbance.

Consider the stable single degree of freedom system
\begin{dmath}
m \ddot x + b \dot x + \stiffnessDuffing (x+s)^3 = 0, 
\end{dmath}
where $s$ is an induced displacement disturbance. At the operating
position $x=0$, the nonlinear spring stiffness is
$3kx^2|_{x=0}=0$. For a disturbance $s$, the spring is perturbed and
generates a reaction force of $ks^3$ on the mass. The stiffness here
is $3ks^2$; \ie, dependent on the amplitude of disturbance. The
ramifications of this nonlinear force on the vibratory response of the
system are not exactly straightforward.

\textcite{tentor2001} analysed a spring generated by repulsion magnets
which behaved as a Duffing oscillator for large amplitude
vibrations. \fixme{(The Duffing oscillator is well-known in the literature of
nonlinear vibrations. Please reference it more.)} The difference for
his system was a significant linear component in the force equation:
\begin{dmath}
F_{\text{Duffing}} = \stiffnessLinear x + \stiffnessDuffing x^3.
\end{dmath}
The nonlinear dynamics only affected the response of the system when
the nonlinear term dominated over the linear term. For a zero
stiffness spring, $\stiffnessLinear=0$ and the nonlinear dynamics are
more significant.

Rather than perform a nonlinear analysis on the system above (which is
known in the literature as a Duffing oscillator — anything else I
need to add?), it is instructive to examine the power spectra generated
with a range of spring stiffnesses and Gaussian inputs.

\Figref{cubic-resonance-disturb,cubic-resonance-stiffness} show the
square root of the ratio of the power spectrum of $s$ and $x$, where
$s$ is a white noise signal of variance $S$. (The transfer function is
not examined because it removes nonlinear components of the original
signals.) The results are compared with the system
\begin{dmath}
m \ddot x + b \dot x + k_{\text{lin}}(x+s) = 0, 
\end{dmath}
which has a linear stiffness $k_{\text{lin}}=3kS^2$ equivalent to the
stiffness of the nonlinear spring at the variance displacement.

In both nonlinear systems simultions, significant nonlinearities in the
response can be seen. Maximum frequency and amplitude of the response increase
with both greater input disturbance amplitude and greater nonlinear spring
stiffness.

\begin{figure}
  \grf{Simulations/Zero_stiffness/eps/cubic-resonance-disturb}
  \caption{Cubic stiffness response with various amplitudes of
    disturbance in comparison to some approximately similar linear
    systems.}
  \figlabel{cubic-resonance-disturb}
\end{figure}

\begin{figure}
  \grf{Simulations/Zero_stiffness/eps/cubic-resonance-stiffness}
  \caption{Cubic stiffness response with varying values of the
    stiffness coefficient (with $S=1$ to compare with
    \figref{cubic-resonance-disturb}) in comparison to some
    approximately similar linear systems.}
  \figlabel{cubic-resonance-stiffness}
\end{figure}

When compared to the linear response of stiffness equivalent to the maximum
stiffness reached by the white noise input, the linear system response is
\emph{smaller} than the \qzs\ response. This restricts the applications of
using nonlinear springs to achieve low resonance frequencies: only when it
becomes infeasible to decrease the stiffness of a conventional linear system
any further should a nonlinear system be chosen instead.




\section{Inclined springs}
\seclabel{qzskzs}

A well studied \qzs\ device is that of a load bearing vertical spring in
parallel with a pair of inclined springs that behave in a buckling
regime~\cite{molyneux1957,alabuzhev1989,carrella2006,carrella2007,carrella2008}.
Generally the studies of this device have only analysed its stiffness
properties in a single degree of freedom, in the direction of the primary load
bearing.

This section studies this inclined spring system and re-formulates the force
and stiffness characteristics in both vertical and horizontal directions,
describing in some detail the approach by which low stiffness in both
directions can be achieved. Low stiffness in the vertical direction has been
previously documented (as cited above) due to the negative vertical stiffness
of a pair of horizontal springs in compression. Low stiffness in the
horizontal direction is newly analysed here, which is achieved due to the
negative stiffness in the horizontal direction of the load-bearing vertical
spring.

To re-iterate, the vertical spring has negative stiffness horizontally, and
the horizontal springs have negative stiffness vertically.

A drawback of the planar low-stiffness system is noticed, which is that
large compression ratios are required from the compressed springs in order to
operate at \qzs\ in both directions.

\subsection{Geometry}

\Figref{zks-dia} show the planar inclined spring system without load (\ie,
with undeflected springs) and \figref{zks-qzs} shows the same system after
deflection to the position which has the potential of achieving `\qzs', which
is the position of maximum compression of the inclined springs. The overall
stiffness of the system must be tuned to support the mass of the load at this
position.

\begin{figure}
\begin{wide}
\begin{subfigure}
  \asyfig{ZKS/schematic}
  \caption{Inclined springs in their unloaded, uncompressed state, corresponding 
to a vertical displacement of $\ZKSdisp=\ZKSheight$.\figlabel{zks-dia}}
\end{subfigure}
\begin{subfigure}
  \asyfig{ZKS/schematic-qzs}
  \caption{Inclined springs at a position of maximum negative stiffness, 
corresponding to a vertical displacement of $\ZKSdisp=0$.\figlabel{zks-qzs}}
\end{subfigure}
\end{wide}
\caption{Negative stiffness inclined springs in parallel with a positive stiffness spring.}
\end{figure}

At the unloaded state shown in \figref{zks-dia}, all springs are considered to
be in their uncompressed state; with inclined spring lengths
$\ZKSlength=\sqrt{\ZKSheight^2+\ZKSwidth^2}$ and vertical spring length
$\ZKSvlength=\ZKSlengthratio\ZKSlength$. The inclined springs each have
stiffness~$\ZKSstiffincl$ and the vertical spring has stiffness
$\ZKSstiffvert=\ZKSkratio\ZKSstiffincl$. The stiffness and deflection
properties of the springs are summarised in \tabref{spring-properties}.

\begin{table}
  \caption{Properties of the springs in the \qzs\ inclined spring system 
defining stiffness ratio $\ZKSkratio$ and length ratio $\ZKSlengthratio$.}
  \tablabel{spring-properties}
  \begin{tabular}{@{}lcc@{}}
    \toprule
      Spring & Stiffness & Undeflected length \\
    \cmidrule(r){1-1}\cmidrule(l){2-3}
      Inclined & $\ZKSstiffincl$ & $\ZKSlength = \sqrt{\ZKSheight^2+\ZKSwidth^2}$ \\
      Vertical & $\ZKSstiffvert = \ZKSkratio\ZKSstiffincl$ 
        & $\ZKSvlength = \ZKSlengthratio\ZKSlength$ \\
    \bottomrule
  \end{tabular}
\end{table}

The position of maximum compression of the inclined spring, shown in
\figref{zks-qzs}, defines the displacement origin of the system, where
$\ZKSdisp$ is the displacement in the load bearing direction, and $\ZKSshift$
is the displacement in the non--load bearing direction (this is used later for
the derivation of the horizontal stiffness of the system).

The deflected lengths of the springs from displacement $(\ZKSdisp,\ZKSshift)$
are denoted for the inclined spring by $\ZKSlengthcomp$ and for the vertical
spring by $\ZKSvlengthcomp$. The compressed length of the inclined spring (on
the left) is
\begin{dmath}[label=ZKSlengthcomp]
\ZKSlengthcomp =
  \sqrt{x^2+\gp{\ZKSwidth+y}^2}
\end{dmath},
and the vertical spring length is
\begin{dmath}[label=ZKSvlengthcomp]
\ZKSvlengthcomp =
  \sqrt{\ZKSshift^2+\gp{\ZKSdisp-\ZKSheight+\ZKSvlength}^2}
\end{dmath};
note that $\ZKSlengthcomp{\ZKSheight,0}=\ZKSlength$ and
$\ZKSvlengthcomp{\ZKSheight,0}=\ZKSvlength$.

The geometry that has been chosen uses linear springs that are all arranged to
be undeflected in the unloaded state of the device. \textcite{kovacic2008}
explores the effects of including pretension and the use of nonlinear
softening springs.

\subsection{Vertical forces}

The forces on the mass are calculated by looking at the components of the
forces due to each spring individually. The force due to the inclined spring
(on the left of \figref{zks-dia}), in the direction of the spring, is given by

\begin{dmath}
\ZKSforceincl = \gp{\ZKSlength-\ZKSlengthcomp} \ZKSstiffincl
  = \gp{\sqrt{\ZKSheight^2+\ZKSwidth^2}-
        \sqrt{x^2+\gp{\ZKSwidth+y}^2}} \ZKSstiffincl
\end{dmath}.

Assuming only vertical displacement ($\ZKSshift=0$),
the vertical component of this inclined spring force is 
\begin{dmath}[label=ZKSforceinclV,compact]
\ZKSforceinclV = \ZKSforceincl{\ZKSdisp,0} \frac{x}{\ZKSlengthcomp{1,0}}
% from LaTeXPrint[Fv, x, 0]; bring stiffincl to the front:
=  \ZKSdisp \ZKSstiffincl 
  \gp{
    \frac {\sqrt{\ZKSheight^2+\ZKSwidth^2}}
          {\sqrt{\ZKSwidth^2+\ZKSdisp^2}  } -1
  } 
\end{dmath}.

It is convenient to normalise this result by representing the lengths and
displacements as ratios of the uncompressed height of the inclined springs.
\Ie, by making the coördinate substitutions $\ZKSnorm=\ZKSdisp/\ZKSheight$ and
$\ZKSratio=\ZKSwidth/\ZKSheight$, the inclined spring force in the vertical
direction can be written in non-dimensional form as

\begin{dmath}[label=ZKSforceinclVnorm]
  \frac{\ZKSforceinclV{\ZKSnorm}}{\ZKSheight \ZKSstiffincl} = 
    \ZKSnorm \gp{\sqrt{\frac{\ZKSratio^2+1}{\ZKSratio^2+\ZKSnorm^2}}-1}
\end{dmath}.

This is a different normalisation than used in the
literature~\cite{carrella2007}; note that here $\ZKSratio=0$ corresponds to
unloaded inclined springs at \ang{90} (\ie, vertical) before compression, and
$\ZKSratio=\infty$ corresponds to unloaded inclined springs at \ang{0} (\ie,
horizontal). In the coördinate system used here, the displacement origin $x=0$
corresponds to the position of maximum compression of the inclined springs,
\ie, when they are horizontal.

\Figref{ZKSforce-plain} illustrates the characteristic of
\eqref{ZKSforceinclV} vs.\ normalised displacement for a range of geometric
ratios $\ZKSratio$. The `snap-through' forces that cause the negative
stiffness is especially strong the smaller $\ZKSratio$ becomes (\ie, the more
vertical the spring angles before deflection in \figref{zks-dia}).

\begin{figure}
\begin{wide}
\begin{subfigure}
\psfragfig{\phdpath ZKS/fig/ZKSforce-plain}
\caption{Vertical force due to inclined springs only using
\eqref{ZKSforceinclV} for a range of geometric ratios
$\ZKSratio$.\figlabel{ZKSforce-plain}}
\end{subfigure}
\begin{subfigure}
\psfragfig{\phdpath ZKS/fig/ZKSforce-total}
\caption{Normalised vertical force characteristic of the system calculated
with \eqref{ZKSforcetotalVnorm}. Minimum displacements are calculated from
\eqref{ZKSnormMin}; these termination points represent the limiting case of the vertical
spring being compressed to zero length.\figlabel{ZKSforcetotalVnorm}}
\end{subfigure}
\end{wide}
\caption{Vertical forces due to the inclined springs. $\ZKSratioQZS$ is the 
value of $\ZKSratio$ for which \qzs\ is achieved at $\ZKSnorm$, calculated 
from \eqref{ZKSratioQZS}.}
\figlabel{ZKSforce-both}
\end{figure}

The total vertical force produced by the system, $\ZKSforcetotalV$, is
calculated from combining \eqref{ZKSforceinclV} for each inclined spring with
the force due to the vertical spring:

\begin{dmath}[label=ZKSforcetotalV]
\ZKSforcetotalV 
  = 2\ZKSforceinclV + \ZKSforcevertV
\end{dmath}.

For vertical displacements, the force due to the vertical spring is given by

\begin{dmath}[label=ZKSforcevertV]
\ZKSforcevertV 
  = \gp{\ZKSheight-x} \ZKSstiffvert
\end{dmath},

and the total force in the vertical direction can be nondimensionally
represented by

\begin{dmath}[label=ZKSforcetotalVnorm]
\frac{\ZKSforcetotalV}{\ZKSheight \ZKSstiffincl} = -\ZKSnorm \ZKSkratio+\ZKSkratio+2 
\ZKSnorm \gp{\sqrt{\frac{\ZKSratio^2+1}{\ZKSratio^2+\ZKSnorm^2}}-1}
\end{dmath},

where $\ZKSkratio=\ZKSstiffvert/\ZKSstiffincl$. This equation is depicted in
\figref{ZKSforcetotalVnorm} for $\ZKSkratio=1$, where it can be seen that by
selecting $\ZKSratio$ appropriately, it is possible to generate a local region
of low stiffness at $\ZKSnorm=0$, approaching \qzs\ condition under ideal
circumstances. The calculation for $\ZKSratioQZS$, the value of $\ZKSratio$
for which \qzs\ is achieved, will be shown later in \eqref{ZKSratioQZS}.

The force curves in \figref{ZKSforcetotalVnorm} terminate at a certain point
in the negative displacement region, which corresponds to the maximum possible
compression of the vertical spring given by the condition
$\ZKSvlengthcomp{\ZKSdispMin,0}=0$. In other words, the spring has been
compressed to zero length (which would be troublesome to achieve in practice).
This condition can be solved for $\ZKSdispMin$ and subsequently normalised for
the equivalent $\ZKSnormMin$, which are given by

\begin{dseries}[label=ZKSnormMin]
\begin{math}
  \ZKSdispMin = \ZKSheight - \ZKSvlength
\end{math},
\begin{math}
  \ZKSnormMin = 1-\ZKSlengthratio\sqrt{\ZKSratio^2+1}
\end{math}.
\end{dseries}


\subsection{Vertical stiffnesses}

The vertical stiffness characteristic, $\ZKSstiffnessV$, of the system is
calculated by differentiating \eqref{ZKSforcetotalVnorm} with respect to $x$:

\begin{dgroup*}
\begin{dmath}
  \ZKSstiffnessV = -\D{}{\ZKSdisp} \ZKSforcetotalV
\end{dmath},
\begin{dmath}[label=ZKSstiffnessV]
% edited to move the gamma^2 in front of the sqrt
  \frac{\ZKSstiffnessV}{\ZKSstiffincl} = -2\ZKSratio^2 \sqrt{\frac{\ZKSratio^2+1}{\gp{\ZKSratio^2+\ZKSnorm^2}^3}}+\ZKSkratio+2
\end{dmath}.
\end{dgroup*}

\begin{figure}
\begin{wide}
\begin{subfigure}
\psfragfig{\phdpath ZKS/fig/ZKSstiffness-total}

\caption{\figlabel{ZKSstiffness-vert} Normalised vertical stiffness of the
system calculated with \eqref{ZKSstiffnessV}.}

\end{subfigure}
\begin{subfigure}
\psfragfig{\phdpath ZKS/fig/ZKSstiffnessH-total}

\caption{\figlabel{ZKSstiffness-horiz} Normalised horizontal stiffness of the
system calculated with \eqref{ZKSstiffnessH}.}

\end{subfigure}
\end{wide}

\caption{A comparison of vertical and horizontal stiffness characteristics for
a range of geometric ratios $\ZKSratio$ at $\ZKSkratio=1$. $\ZKSratioQZS$ is
calculated at $\ZKSlengthratio=1$.}

\figlabel{ZKSstiffness-both}
\end{figure}

This equation is shown in \figref{ZKSstiffness-vert} together with the
associated horizontal stiffness, which will be calculated in
\secref{qzs-horiz}. The parameter selection required to achieve a \qzs\
condition in the vertical direction can be found by solving
\eqref{ZKSstiffnessV} for $\ZKSstiffnessV=0$ at $\ZKSnorm=0$. This results in
the relation

\begin{dmath}[label=ZKSratioQZS]
  \ZKSratioQZS =
    \frac{2}{\sqrt{\ZKSkratio^2+4 \ZKSkratio}}  
\end{dmath}

which is used as the reference value of $\ZKSratio$ in
\figref{ZKSforce-both,ZKSstiffness-both}.

Achieving exactly \qzs\ with this spring is not feasible in practice as the
stiffness characteristic becomes negative for $\ZKSratio<\ZKSratioQZS$, as
shown in \figref{ZKSstiffness-vert}. Define $\ZKSoff$ as the error of
$\ZKSratioQZS$ as

\begin{dmath}
  \ZKSratio = \gp{1+\ZKSoff}\ZKSratioQZS
\end{dmath}.

\Figref{ZKSoff} shows the vertical force, $\ZKSforcetotalV$, of the system for
$\ZKSoff\in\{-0.1, 0, 0.1\}$. It can be seen that negative values of $\ZKSoff$
correspond to negative stiffness at the equilibrium point $\ZKSnorm=0$. A
system at this equilibrium point will tend towards and remain at the position
of stable equilibrium indicated in the figure.

\Figref{ZKS-ratioErr} plots the stiffness at equilibrium as $\ZKSoff$ varies;
in the unstable zone, the system will move to the equilibrium point shown in
\figref{ZKSoff} away from $\ZKSnorm=0$. It can be seen that the stiffnesses in
the stable region for $\ZKSoff>0$ are smaller than the stiffnesses in the
equilibrium region for $\ZKSoff<0$. This highlights the importance of never
breaching the $\ZKSoff<0$ instability condition. Therefore, a chosen value for
$\ZKSratio$ will approach $\ZKSratioQZS$ but always be slightly greater in
order to retain stability.

\begin{figure}
\begin{wide}
\begin{subfigure}
  \psfragfig{\phdpath ZKS/fig/ZKSoffqzs}
  \caption{\figlabel{ZKSoff}
The stable and unstable equilibrium points for $\ZKSoff\in\{-0.1, 0, 0.1\}$. 
The rest position will move from the unstable point to the stable point of equilibrium.}
\end{subfigure}
\begin{subfigure}
  \psfragfig{\phdpath ZKS/fig/ZKSratioErr}
  \caption{\figlabel{ZKS-ratioErr}
The stiffness at equilibrium as $\ZKSoff$ varies; as the stiffness becomes 
negative, the stiffness shown corresponds to the stable point of equilibrium shown in the figure adjacent.}
\end{subfigure}
\end{wide}
\caption{Force and stiffness of the inclined spring system near \qzs, showing 
the effect of unstable equilibrium.}
\end{figure}

Now that the vertical stiffness of the system has been shown and a condition
derived to achieve \qzs\ in that direction, the same approach will be taken
for the horizontal behaviour.

\subsection{Horizontal stiffness characteristic}
\seclabel{qzs-horiz}

In order to calculate the horizontal stiffness of the system, the force from
the vertical spring needs to be represented in terms of both vertical and
horizontal displacements. This force, in the direction of the vertical spring,
is

\begin{dmath}[label=ZKSforcevertlong]
% edited to insert \ZKSlength
\ZKSforcevert =
  \gp{\ZKSlengthratio\ZKSlength-\sqrt{\ZKSshift^2+
  \gp{-\ZKSheight+\ZKSdisp+\ZKSlengthratio\ZKSlength}^2}} \ZKSstiffvert
\end{dmath},

recalling that $\ZKSshift$ is the displacement of the mass in the horizontal
direction. Substituting $\ZKSshift=0$ into \eqref{ZKSforcevertlong} yields the
previous \eqref{ZKSforcevertV}. The horizontal component of this force is

\begin{dmath}[label=ZKSforcevertH]
\ZKSforcevertH = \ZKSforcevert\frac{\ZKSshift}{\ZKSvlengthcomp}
\end{dmath}.

Similarly, the horizontal component of the force from the inclined spring on
the left (referring to \figref{zks-dia}) is given by

\begin{dmath}[label=ZKSforceinclH]
\ZKSforceinclH = \ZKSforceincl \frac{\ZKSwidth+\ZKSshift}{\ZKSlengthcomp}
\end{dmath},

and the horizontal component of the force from the inclined spring on the
right is

\begin{dmath}
\ZKSforceinclH\bigg|_{\text{right}} = -\ZKSforceinclH{1,-\ZKSshift}
\end{dmath}.

The stiffness characteristic in the horizontal direction, $\ZKSstiffnessH$,
is derived in a similar fashion to the vertical stiffness. The total force in
the horizontal direction is

\begin{dmath}
\ZKSforcetotalH=\ZKSforceinclH-\ZKSforceinclH{\ZKSdisp,-\ZKSshift}+\ZKSforcevertH
\end{dmath},

using \eqref{ZKSforcevertH,ZKSforceinclH}. Differentiating with respect to
$\ZKSshift$ and evaluating at $\ZKSshift=0$ gives the horizontal stiffness
characteristic as the vertical displacement varies,

\begin{dmath}[label=ZKSstiffnessH]
% edited to move elements in front of the sqrts
\frac{\ZKSstiffnessH}{\ZKSstiffincl} = -2 \ZKSnorm^2 \sqrt{\frac{\ZKSratio^2+1}{\gp{\ZKSratio^2+\ZKSnorm^2}^3}} +\frac{\ZKSkratio \gp{\ZKSnorm-1}}{\ZKSlengthratio\sqrt{\ZKSratio^2+1} +\ZKSnorm-1}+2
\end{dmath}.

This equation has been shown previously in \figref{ZKSstiffness-horiz}
together with the vertical stiffness in \eqref{ZKSstiffnessV}. In these
figures, it can be seen that while the vertical stiffness is zero at
$\ZKSnorm=0$ and $\ZKSratio=\ZKSratioQZS$, the horizontal stiffness exhibits
its own behaviour, and can even be negative (\ie, unstable) for values of
$\ZKSratio$ lower than around $1.25\ZKSratioQZS$.

Further analysis into the behaviour of the horizontal stiffness at the
vertical \qzs\ condition is warranted. Substituting the \qzs\ condition of
\eqref{ZKSratioQZS} into \eqref{ZKSstiffnessH} at displacement $\ZKSnorm=0$,
gives the normalised horizontal stiffness as a function of stiffness ratio
$\ZKSkratio$:

\begin{dmath}[label=ZKSstiffnessHnorm]
% edited to eliminate nested fraction
  \frac{\ZKSstiffnessH}{\ZKSstiffincl}\bigg|_{\text{V.\,QZS}} =
   2-\ZKSkratio\gp{\frac{\gp{\ZKSkratio+2} \ZKSlengthratio}{\sqrt{\ZKSkratio \gp{\ZKSkratio+4}}}-1}^{-1}
\end{dmath}.

This equation is depicted in \figref{ZKSstiffnessKnorm}; it can be seen that
the horizontal stiffness of the spring may be chosen by varying both
$\ZKSkratio$ and $\ZKSlengthratio$. Since the spring length ratio
$\ZKSlengthratio$ is not found in \eqref{ZKSstiffnessV}, the horizontal and
vertical stiffnesses may be tuned independently in order to achieve \qzs\ in
each.

\begin{figure}
\psfragfig{\phdpath ZKS/fig/ZKSkratio-effect}

\caption{Horizontal stiffness characteristic at the vertical \qzs\ condition
for varying stiffness ratio $\ZKSkratio$ and length ratio $\ZKSlengthratio$,
calculated with \eqref{ZKSstiffnessHnorm}.}

\figlabel{ZKSstiffnessKnorm}
\end{figure}

To obtain \qzs\ in the horizontal direction, \eqref{ZKSstiffnessHnorm} is
solved at $\ZKSstiffnessH=0$, showing a relationship between $\ZKSkratio$ and
$\ZKSlengthratio$ when the \qzs\ condition is achieved in both the vertical
and the horizontal directions.

\begin{dseries}[label=ZKSQZS]
\begin{math}
\ZKSkratioQZS =
  2 \gp{\sqrt{\ZKSlengthratio^2+1}-1}
\end{math}, or\quad
\begin{math}
\ZKSlengthratioQZS =
    \frac{1}{2} \sqrt{\ZKSkratio \gp{\ZKSkratio+4}}
\end{math}
\end{dseries}

This relationship is illustrated in \figref{ZKS-hk-ratio}, which shows that
increasing $\ZKSlengthratio$ (say, in order to reduce the compression of the
vertical spring) results in an increasing value of the vertical spring
stiffness in order to remain at \qzs.

\begin{figure}
  \psfragfig{\phdpath ZKS/fig/ZKS-hk-ratio}
  \caption{Relationship between $\ZKSkratio$ and $\ZKSlengthratio$ at \qzs\ 
both horizontally and vertically.}
  \figlabel{ZKS-hk-ratio}
\end{figure}

Substituting $\ZKSkratioQZS$ from \eqref{ZKSQZS} into
\eqref{ZKSstiffnessV,ZKSstiffnessH} allows the stiffness characteristics of
the system in the two directions to be compared when both have \qzs\
simultaneously. Considering the vertical stiffnesses first in
\figref{ZKSstiffnessV-QZS}, it can be seen that increasing the length ratio
$\ZKSlengthratio$ increases the vertical stiffness gradient, which is
important to keep small in order to mitigate possible nonlinear dynamic
effects that may arise due to a large rate of change of stiffness over
displacement.

\Figref{ZKSstiffnessH-QZS} illustrates that the \qzs\ condition is
marginal\-ly unstable in the horizontal direction. In practice this requires
that the system be tuned slightly away from the \qzs\ condition after
accommodating for the maximum disturbance displacement of the isolator. It is
possible to do this without compromising the \qzs\ condition in the vertical
direction because the spring length ratio $\ZKSlengthratio$ does not affect
the vertical stiffness.

As an example, \figref{ZKSstiffness-NotQZS} shows the horizontal stiffness for
$\ZKSkratio=0.95\ZKSkratioQZS$. In comparison with \figref{ZKSstiffnessH-QZS},
the spring has a stable displacement range of approximately
$\ZKSnorm=\pm0.025$. Provided that $\ZKSlengthratio$ is large enough, the
horizontal stiffness at $\ZKSnorm=0$ is still significantly reduced.

Therefore, there is a direct compromise between the nonlinearity of the
stiffness in the vertical direction (which increases with $\ZKSlengthratio$)
and the amount of stiffness reduction in the horizontal direction (which
decreases with $\ZKSlengthratio$).

\begin{figure}[p]
\begin{wide}
\begin{subfigure}
  \psfragfig{\phdpath ZKS/fig/ZKSstiffnessV-QZS}
  \caption{\figlabel{ZKSstiffnessV-QZS} 
Normalised vertical stiffness of the system.}
\end{subfigure}
\begin{subfigure}
  \psfragfig{\phdpath ZKS/fig/ZKSstiffnessH-QZS}
  \caption{\figlabel{ZKSstiffnessH-QZS} 
Normalised horizontal stiffness of the system.}
\end{subfigure}
\end{wide}
\caption{A comparison of vertical and horizontal stiffness characteristics at
\qzs\ in both directions, for a range of spring length ratios, $\ZKSlengthratio$.}
\figlabel{ZKSstiffness-bothQZS}
\end{figure}

\begin{figure}[p]
  \psfragfig{\phdpath ZKS/fig/ZKSstiffnessH-NotQZS}
  \caption{Normalised horizontal stiffness of the system at
$\ZKSkratio=0.95\ZKSkratioQZS$ in order to obtain a small range of
displacement around $\ZKSnorm=0$ with positive stiffness (compare
with \figref{ZKSstiffnessH-QZS} which has negative stiffness for
$\ZKSnorm<0$). The vertical \qzs\ condition is unaffected.}
  \figlabel{ZKSstiffness-NotQZS}
\end{figure}

\subsection{Spring compression}

One factor that has not been addressed with this particular design is the
amount of spring compression required at the equilibrium position of \qzs. The
total allowable compression will be limited by the shape and properties of the
springs themselves. By adjusting the design parameters of the inclined
spring system, the amount of compression in each spring at equilibrium can be
selected.

The compression ratio of the inclined and vertical springs, respectively, is
given by

\begin{dseries}[label=ZKScompress]
\begin{math}
\ZKScompressI = 1 - \frac{\ZKSlengthcomp}{\ZKSlength}
\end{math},
\begin{math}
\ZKScompressV = 1 - \frac{\ZKSvlengthcomp}{\ZKSvlength}
\end{math}.
\end{dseries}

These equations are evaluated at the \qzs\ condition in both directions; \ie,
$(x,y)=(0,0)$, $\ZKSratio=\ZKSratioQZS$, and
$\ZKSlengthratio=\ZKSlengthratioQZS$ (see \eqref{ZKSratioQZS,ZKSQZS}),
yielding

\begin{dseries}[label=ZKScompressQZS]
\begin{math}
\ZKSQZS{\ZKScompressI\Big|} =
  \frac{\ZKSkratio}{\ZKSkratio+2}
\end{math},
\begin{math}
\ZKSQZS{\ZKScompressV\Big|} =
  \frac{2}{\ZKSkratio+2}
\end{math}.
\end{dseries}

The compression ratios of the vertical and inclined springs at \qzs\ are
shown in \figref{ZKScompress} as functions of varying spring stiffness ratio
$\ZKSkratio$. (Recall that \qzs\ is achieved by adjusting $\ZKSlengthratio$
for each specified value of $\ZKSkratio$ with the relationship shown in
\eqref{ZKSQZS}.) These results show that a large compression (greater
than fifty percent) in at least one of the springs is required to achieve
\qzs.

\begin{figure}
\psfragfig{\phdpath ZKS/fig/ZKScompress}

\caption{Compression relationship of the springs for a range of spring
stiffness ratios, at \qzs\ both horizontally and vertically. A compression of
100\% implies a change in displacement of the entire spring length, which is
difficult to realise in practice.}

\figlabel{ZKScompress}
\end{figure}

\subsection{The general applicability of the inclined springs system}

In order to adapt this system to withstand time varying load conditions, both
the anchor positions of the inclined and the vertical springs must be adjusted
in order to tune for, first, the required load bearing, and second, the amount
of negative stiffness required to achieve \qzs\ in the vertical direction.

Dynamically changing the system for \qzs\ in the horizontal direction requires
that either the uncompressed spring lengths or the spring stiffnesses be
adjustable during operation. Since the deflection properties of the spring
cannot be adjusted, a time-varying stiffness is required. This could be
achieved, for example, with \magnetorh\ fluid springs.

It is possible that a design that includes pretension in the springs (such
that there is no rest state in which all springs are undeflected, as shown in
\figref{zks-dia}) may be able to adapt to horizontal \qzs\ without variable
stiffness springs.

Note that in a three dimensional structure, while the principles discussed
above still hold, at least three inclined springs are required to ensure
symmetry in the orthogonal horizontal direction.

In the next section, a magnetic system is examined in more detail that
exploits a similar negative stiffness principle.

\section{Magnetic \qzs}
\seclabel{qzs}

This section is heavily based on the publication of this work \parencite{robertson2009}.

\subsection{Introduction}

This paper examines another system that exhibits localised zero
stiffness: a pair of fixed magnets that supports a mass against
gravity by respectively repelling the mass from below and attracting
it from above, as shown in \figref{schem}. This arrangement of magnets
has seen some previous
attention~\cite{nijsse2001,robertson2006,robertson2007}, in which the force
vs.\ displacement characteristic can be approximated by a
quadratic polynomial for small gaps between the magnets.

\begin{figure}
  \asyfig{Systems/zerok-comb}
  \caption{Schematic of a magnetic spring with \qzs\ at
 $\ndispZ=0$ to isolate displacement $x$ from disturbance vibration $y$.
 Large arrows indicate direction of polarisation of the magnets.
 In this paper, cubical magnets are used with side length $\mdim$,
 distance $2\mdim\ngap$ between the centres of the fixed magnets,
 and displacement of the floating magnet from the zero stiffness
 position $\mdim\ndispZ$. Rest position at positive $\ndispZ$ (upwards)
 corresponds to unstable equilibrium; negative $\ndispZ$ (downwards) is stable.}
  \figlabel{schem}
\end{figure}

The cubic force curve is more useful because it creates a stable
inflection point with localised zero stiffness; in contrast, a
quadratic-type spring is marginally stable at its \qzs\
position and cannot be operated about this point. Nonetheless, the
magnet arrangement is worth studying as a mechanism to achieve
\emph{low} stiffness, since this device reduces the stiffness in all
three translational degrees of freedom.
As well as its use to design low frequency isolation mounts, this
idea can have particular application in support structures where
a reduction in stiffness is desired to mitigate a vibration problem
that has been discovered after its construction. The noncontact
nature of the force between the magnets allows their easy attachment
to an existing structure.

The structure of this section is as follows. The exact equation for the
forces between two cubical magnets is presented in \secref{mag} and
used to calculate the exact and approximate force vs.\ displacement
profiles of the \qzs\ magnetic spring. In \secref{cons}, the
criteria that govern the behaviour of a vibration isolation device
are applied to the magnetic spring and suitable ranges for the
design parameters are found to achieve the design goals.
Finally, \secref{nonlin} analyses the dynamic behaviour of the spring
to ensure that the isolation capabilities are not compromised by the
nonlinear characteristics of the magnetic spring.


\subsection{Magnet forces}
\seclabel{mag}

The force that is generated between parallel cuboid magnets (\ie, with
faces orthogonal but not necessarily equal) can be calculated
with the formulation of \textcite{akoun1984}, for magnet
geometries depicted in \figref{akoun}. \textcite{bancel1999} published
an equivalent equation that is algorithmically easier to apply for
complex magnet geometries and multipole arrays.

\begin{figure}
  \asyfig{Magnets/akoun}
  \caption{Geometry for the expression by \textcite{akoun1984} to
  calculate the forces between two parallel cuboid magnets with
  magnetisations in the vertical direction, distance between their centres
  $(\alpha,\beta,\gamma)$, and magnet sizes as shown.}
  \figlabel{akoun}
\end{figure}

For this work, the original expression of \textcite{akoun1984} is
simplified for cubical magnets with a vertical offset; \ie, magnet
sizes $a=b=c=A=B=C$ and only vertical relative displacements, such
that $\alpha=\beta=0$. The distance between the magnet centres with
respect to the size of the magnets is expressed as a normalised
length:
\begin{dmath}
  \ndisp = \gamma/\mdim .
\end{dmath}
After some manipulation of the original equation given these simplifying 
assumptions, the force $\force$ on the second magnet in
attraction (\ie, for magnets with polarisation in the same direction)
can be shown to be directly proportional to the facing area of the magnets, 
$\mdim^2$, for a fixed normalised displacement, $\ndisp$, between the magnets:
\begin{dmath}[label=force]
  \magforce = \mdim^2 \nforce .
\end{dmath}
For magnets in repulsion, the expression has opposite sign. The derived
expression for the normalised force $\nforce$ is given in
\eqref{nforce} in the appendix. The $\mdim^2$ relationship shown in \eqref{force}
is interesting because it is not evident from Akoun~and Yonnet's original
equations that such a simplification (for various subsets of magnet geometries
such as the one considered here) is possible. 

\fixme(That stuff is not in an appendix in the thesis?)

The stiffness between two cubical magnets can be calculated by differentiating
the force expression in \eqref{force} with respect to vertical displacement
$\gamma=\mdim\ndisp$ and can be shown to be proportional to the magnet size
$\mdim$:
\begin{dmath}[label=stiffness]
  \magstiffness = \frac{\partial}{\partial\gamma} \magforce
             = \frac1\mdim\frac{\partial}{\partial\ndisp} \mdim^2\nforce
             = \mdim\nstiffness .
\end{dmath}
The derived expression for the normalised stiffness $\nstiffness$
is given in \eqref{nstiffness} in the appendix. \fixme{again, not appendix?}

A `\qzs' magnetic spring consists of an attracting magnetic
pair above a repelling magnet pair as shown in
\figref{schem}. Parameter $\mdim\ngap$ is the gap between the centres
of the magnet pairs at \qzs, and $x=\mdim\ndispZ$ is the
static displacement of the floating mass about the centre line between
the magnets. The parameters $\ngap$ and $\ndispZ$ are referred to
as normalised magnet gap and normalised magnet displacement, respectively.
The force due to the lower magnet in repulsion is
\begin{align}
\frepl &= -\magforce{\mdim,\mdim\ngap+\mdim\ndispZ} \\
       &= -\mdim^2\nforce{\ngap+\ndispZ}, 
\end{align}
and the force due to the upper magnet in attraction is
\begin{align}
\fattr &= \magforce{\mdim,-\mdim\ngap+\mdim\ndispZ} \\
       &= \mdim^2\nforce{-\ngap+\ndispZ}.
\end{align}
The total force on the floating magnet, $\magforceZ$, is a superposition
of $\frepl$ and $\fattr$, yielding
\begin{align}
  \magforceZ &= \frepl+\fattr \\
          &= \mdim^2\gp{-\nforce{ \ngap+\ndispZ} 
                     +\nforce{-\ngap+\ndispZ}} \\
          &\eqdef \mdim^2 \nforceZ . \eqlabel{nforceZ}
\end{align}
The stiffness of the system can be similarly expressed as
\begin{align}
  \magstiffnessZ &= \mdim\nstiffnessZ , \\
\intertext{where}
  \nstiffnessZ &= -\nstiffness{ \ngap+\ndispZ} 
                  +\nstiffness{-\ngap+\ndispZ} . \eqlabel{nstiffnessZ}
\end{align}
The force $\nforceZ$ and stiffness $\nstiffnessZ$ of the magnetic spring are
readily calculated for values of normalised displacement and gap, $\ndispZ$
and $\ngap$, from the derived expressions in the appendix. However, these
expressions are too complex for use in any calculation where they must be
inverted (say, finding a value of $\ngap$ for which a certain $\nstiffnessZ$
holds). It is therefore necessary to obtain a simpler model of $\nforceZ$.

Previously, $\nforceZ$ has been modelled as a quadratic polynomial
\cite{nijsse2001,robertson2006} with coefficients that vary with magnet gap:
\begin{dmath}[label=quad]
\nforceZ \approx \polyq2 \ndispZ^2 + \polyq0  .
\end{dmath}
Over small displacement ranges this approximation yields adequate results, but
the resulting model is dependent on the gap between the magnets and local to
the displacement range used to fit the model.

A more accurate result (which is also accurate over larger displacement
ranges) can be achieved with a quartic polynomial approximation,
\begin{dmath}[label=quartic]
\nforceZ \approx \polyc4 \ndispZ^4 + \polyc2 \ndispZ^2 + \polyc0 ,
\end{dmath}
but the same criticisms hold as for the quadratic polynomial model: the
resulting model is not general; furthermore, if the model approximation is to
also represent changes in magnet gap $\ngap$, the functions $\polyc0$,
$\polyc2$, and $\polyc4$ require high-order polynomials to represent the
variation sufficiently well.

Due to the complexity of the expression, a simpler approximation of
the normalised force $\nforce$ can be found by numerically fitting the
constant coefficients $\ccA$, $\ccB$, and possibly $\nn$ in the empirical
approximation for the forces between two magnets
\begin{dmath}[label=nnfit]
  \nforce \approx \frac{\ccA}{\gp{\ccB+\ndisp}^\nn}. 
\end{dmath}
\textcite{xu1993} used the more complicated approximation
\begin{dmath}
  \nforce \approx \ccD\gp{\frac{\ccA}{\ccB+\ndisp}}^\nn+\ccC
\end{dmath}
although in this case the additional complexity does not justify 
the slight increase in accuracy this expression may offer.

\textcite{bonisoli2007,bonisoli2007b} used $\nn=3$ in their work, and
\textcite{piombo2003} recommended either $\nn=2$ or $\nn=4$. The values for
$\nn$ they selected were based on the best fit curves of the forces for the
specific geometry of their magnets in each case. However, $\nn$ is not
restricted to integer values and does not have to be chosen \emph{a priori} to
the curve fitting.

A least squares fit is performed with \eqref{nnfit} varying all three
parameters $\ccA$, $\ccB$, and $\nn$ over the range $2\le\ndispZ\le5$
to achieve an approximation of \eqref{nforce}. Over the displacement
range used to fit the model, modelling errors of less than $1\%$ are
achieved. (Without loss of generality, the magnetisation of each magnet,
$J$, is taken as unity.) Including displacements outside of this range
(especially $1<\ndispZ\le2$) diminishes the accuracy of the fit, but is less
relevant for this work: displacements $1<\ndispZ\le2$ generate high
stiffnesses, and displacements $\ndispZ\ge5$ have low supporting forces. Both
of these properties are undesirable for a vibration isolator in terms of
resonance frequency and adequate load bearing, respectively.

\Figref{nnfit} shows the approximations found with \eqref{nnfit} for set
values $\nn=3$ and $\nn=4$ (only coefficients $\ccA$ and $\ccB$ are varied);
in the third curve, $\nn$ has also been allowed to range for the least-squared
fit. \Tabref{nnfit} displays the values for all three calculated parameters in
each case. The curves for $\nn=2$ and $\nn=5$ produce much greater errors than
those shown in \figref{nnfit} and have been omitted for clarity.

\begin{figure}
  \psfragfig{\phdpath QZS/fig/nnfit}
  \caption{Modelling errors of \eqref{nnfit} compared to the
  exact \eqref{nforce}, for parameters shown in \tabref{nnfit}.}
  \figlabel{nnfit}
\end{figure}

\begin{table}
  \begin{tabular}{@{}ccc@{}}
    \toprule
    $\nn$ & $\ccA$ & $\ccB$ \\
    \midrule
    \num{3} & \num{6.580e4} & \makebox[0pt][r]{$-$}\num{0.5796} \\
    \num{4} & \num{4.071e5} & \num{0.0607} \\
    \num{4.197} & \num{6.028e5} & \num{0.1883} \\
    \bottomrule
  \end{tabular}
  \caption{%
    Best fit parameters for \eqref{nnfit}. 
    Fixed integer values of $\nn$ were chosen 
    for the first two cases, and the latter 
    value best fits the model by varying all 
    three parameters. Note that these are 
    are unitless parameters.}
  \tablabel{nnfit}
\end{table}

An empirical approximation of \eqref{force} for the force between two
cubical magnets is thus given by
\begin{equation}
\force{\mdim,x} \approx \pm a^2J^2\frac{\SI{6.028e5}{}}{\gp{0.1883+x}^{4.197}}
\end{equation}
with force $\force{\mdim,x}$ in Newtons, magnet size $\mdim$ and displacement
$x$ both in metres, and magnetisation $J$ in Tesla, where the expression is
positive for magnets in repulsion and negative for magnets in attraction.

\Figref{nnforceZ} shows the model of
the magnetic system using \eqref{nnfit} to calculate the force due to the
repelling and attracting magnets separately:
\begin{dmath}[label=nnfitz]
  \nforceZ \approx \ccA\gp{\ccB+\ngap+\ndispZ}^{-\nn} 
                  +\ccA\gp{\ccB+\ngap-\ndispZ}^{-\nn} ,
\end{dmath}
where $\ccA$, $\ccB$, and $\nn$ are the best-fit parameters previously
discussed. This solution is both simpler in form and more accurate than the
polynomial models of \eqref{quad,quartic}. It is also much easier to invert
numerically than the exact \eqref{nforceZ}, although note that \eqref{nnfitz}
still cannot be algebraically inverted; this is not an issue for the work to
be discussed in the following sections. The normalised stiffness can be
approximated by differentiating \eqref{nnfitz} with respect to $\ndispZ$: (as
shown previously in \eqref{stiffness})
\begin{dmath}[label=nnfitzk]
  \nstiffnessZ \approx {n\ccA}{\gp{\ccB+\ngap+\ndispZ}^{-\nn-1}} 
                 + {n\ccA}{\gp{\ccB+\ngap-\ndispZ}^{-\nn-1}}.
\end{dmath}

\begin{figure}  
  {%
    \let\labelsize\footnotesize
    \def\LBL#1{\colorbox{white}{#1}}%
    \psfragfig{\phdpath QZS/fig/nnfit-z}%
  }%
  \caption{Normalised force $\nforceZ$ vs.\ displacement $\ndispZ$
    curves of a \qzs\ magnetic system for a range of
    normalised gap $\ngap$ (the stiffness is zero at $\ndispZ=0$).
    Single points correspond to the exact
    solution; solid lines correspond to the approximation given by
    \eqref{nnfit}.}
  \figlabel{nnforceZ}
\end{figure}

In this section, a model of the magnet forces was presented that is accurate
over large relative displacements. Despite the fact that the vibration
disturbance will occur in small magnitudes relative to the size of the magnet
dimension, it is important to model the magnet forces accurately over a
large displacement range so that the design evaluations in the next section
may be applied over variations in magnet gap.


\subsection{Design criteria}
\seclabel{cons}

In the previous section, a magnetic system was introduced with an exact
expression and various approximations for calculating the force vs.\ displacement
characteristics for a range of system designs. It is proposed that this system is
suitable for a vibration isolation platform due to the possibility of low
inherent stiffness of the design. However, the stiffness is dependent on the
load that is being supported, and the magnetic arrangement must be designed
for this purpose. In this section, constraints are imposed on the system
parameters to satisfy these criteria and a design principle developed.

The normalised equilibrium position of the system $\heq{}$ can be found by
equating the magnet force at equilibrium $\feq{}$ with the load due to
gravity and inverting numerically:
\begin{dmath}[label=eq]
\feq = \nforceZ{1,\heq{}} = \mass g/\mdim^2 .
\end{dmath}
A positive displacement of the mass ($\ndispZ\ge0$) is unstable, and so
\eqref{eq} must be solved such that $\heq{}<0$. \Figref{heq-map}
shows the equilibrium position $\heq{}$ varying over $\mdim$ and $\ngap$ for a
system of mass $\mass=\SI{0.5}{kg}$.

\begin{figure}  {\psfragfig{\phdpath QZS/fig/heq-map}}
  \caption{Map of the normalised equilibrium position over varying
    magnet size $\mdim$ and normalised magnet gap $\ngap$ for a system
    with mass $\mass=\SI{0.5}{kg}$.}
  \figlabel{heq-map}
\end{figure}

The normalised equilibrium position $\heq{}$ of a mass can be approximated
from the polynomial fits of the force curve by equating, \eg, \eqref{quartic}
with the force due to gravity and solving for $\ndispZ$. However, the more
accurate approximation \eqref{nnfitz} cannot be algebraically rearranged to
solve for $\ndispZ$; hence, solutions based on that equation must be obtained
numerically.

The magnet size and gap must be chosen based on $\heq{}$ to ensure that the
spring is of sufficient stiffness to support a desired load but not so strong
that the supported mass cannot be in equilibrium in the stable region of the
spring. A `static deflection criterion' is defined to achieve
these constraints:
\begin{dmath}
\nforceZ{1,0} < \mass g/\mdim^2 < \nforceZ{1,-\ngap+1}, \eqlabel{strong}
\end{dmath}
which is shown graphically in \figref{heq}.  This criterion is
depicted in \figref{cons-strong} as a region over the parameters
$\mdim$ and $\ngap$ for three mass loads. The shape and location of the regions
show that for a fixed magnet gap, larger magnets are
required to support larger loads.

\begin{figure}
  \asyfig{QZS/heq-combined}
  \caption{Normalised force vs.\ normalised displacement curve
 illustrating the bounds of both allowable load that can be bourne
 and the displacement range of the magnet; see \eqref{strong,stable}
 respectively for the mathematical representations. $-\ngap+1$ is
 the position where the floating magnet is touching the upper face
 of the lower fixed magnet, and $-\epsilon$ is the closest allowable
 position to the \qzs\ position.}
  \figlabel{heq}
\end{figure}

\begin{figure}
  \gridIV
    {\psfragfig{\phdpath QZS/fig/cons-strongN1}}
    {\psfragfig{\phdpath QZS/fig/cons-strongN2}}
    {\psfragfig{\phdpath QZS/fig/cons-strongN3}}
    {\psfragfig{\phdpath QZS/fig/cons-strongNall}}
  \caption{
    Regions of $\mdim$ and $\ngap$ satisfying
    the static deflection criterion
    of \eqref{strong} for a range of masses. 
    Darker sections denote overlap of the
    regions in the overlay plot.}
  \figlabel{cons-strong}
\end{figure}

Assuming that linearisation is appropriate to model the vibration
response of the system, the equivalent linear stiffness at equilibrium,
$\keq{}$, must satisfy the resonance frequency criterion:
$\keq{}=\magstiffnessZ{1,2,\heq{}}\le\freqdesign^2\mass$, where $\freqdesign$ is the
maximum allowable resonance frequency. From \eqref{nnfitzk}, this
requirement can be formulated as
\begin{dmath}
\gp{\ccB+\ngap+\heq{}}^{-\nn-1}
                + {\gp{\ccB+\ngap-\heq{}}^{-\nn-1}}
   \le \frac{\freqdesign^2 \mass}{\mdim\nn\ccA} . \eqlabel{soft}
\end{dmath}
This criterion is shown for a range of desired nominal stiffnesses in
\figref{cons-soft}. As the resonance frequency decreases, larger
magnets are required to support the load with a small force gradient.

\begin{figure}
  \gridIV
    {\psfragfig{\phdpath QZS/fig/cons-softN1}}
    {\psfragfig{\phdpath QZS/fig/cons-softN2}}
    {\psfragfig{\phdpath QZS/fig/cons-softN3}}
    {\psfragfig{\phdpath QZS/fig/cons-softNall}}
\caption{Regions of $\mdim$ and $\ngap$ satisfying the resonance frequency
  criterion of \eqref{soft} for a range of allowable resonance frequencies and a
  mass $\mass=\SI{0.5}{kg}$. Darker sections denote overlap of the
  regions in the overlay plot.}
  \figlabel{cons-soft}
\end{figure}

Finally, the amplitude of the input vibration must be smaller than the
physical bounds of the system. This can be visualised on a normalised
force/displacement plot as shown in \figref{heq}. For a maximum
(absolute) displacement, $\dmax$, of the mass,
\begin{dmath}
\heq{} - \dmax/\mdim \ge -d+1  , \eqlabel{saturate} \\
\heq{} + \dmax/\mdim < -\epsilon , \eqlabel{stable}
\end{dmath}
where $\mdim \epsilon$ is the tolerance of the closest allowable distance to
the (marginally stable) \qzs\ position that the system can operate.
\Eqref{saturate} is a maximum displacement criterion to ensure that the lower
displacement bound lies above the face of the fixed lower magnet.
\Eqref{stable} is a stability criterion to provide a buffer region to ensure
that the moving magnet is not perturbed past the \qzs\ position into the
unstable zone.

These latter constraints impose only small limits on the design of the
system. The maximum displacement criterion, shown in \figref{cons-saturate} as
a region over the parameters $\mdim$ and $\ngap$ for a range of
$\dmax$, limits the lower size of the magnet.

\begin{figure}
  \gridIV
    {\psfragfig{\phdpath QZS/fig/cons-saturateN1}}
    {\psfragfig{\phdpath QZS/fig/cons-saturateN2}}
    {\psfragfig{\phdpath QZS/fig/cons-saturateN3}}
    {\psfragfig{\phdpath QZS/fig/cons-saturateNall}}
  \caption{Regions of $\mdim$ and $\ndispZ$ satisfying the 
    maximum displacement criterion
    of \eqref{saturate} for a range of disturbance
    displacements $\dmax$ and a mass $\mass=\SI{0.5}{kg}$. Darker
    sections denote overlap of the regions in the overlay plot.}
  \figlabel{cons-saturate}
\end{figure}

Note that the maximum displacement of the spring will not be symmetric
with input displacement due to the softening spring stiffness. Without
loss of generality, however, it is possible to represent the maximum displacement
in the stability and maximum displacement criteria (\eqref{saturate,stable})
with the same symbol.

The stability criterion is governed by two parameters, $\dmax$ and
$\epsilon$, which are varied in \figref{cons-stable-1,cons-stable-2}
respectively. Again, this criterion only has a small effect on the
constraint region, but it is an important effect. Because the stability
criterion limits the minimum distance between the \qzs\ position and the
equilibrium position, this prevents the spring from reaching the very
low equilibrium stiffnesses found just below the \qzs\ position.

\begin{figure}
  \gridIV{\psfragfig{\phdpath QZS/fig/cons-stableN1-1}}
         {\psfragfig{\phdpath QZS/fig/cons-stableN2-1}}
         {\psfragfig{\phdpath QZS/fig/cons-stableN3-1}}
         {\psfragfig{\phdpath QZS/fig/cons-stableNall-1}}
\caption{Regions of $\mdim$ and $\ngap$ satisfying the
  the stability criterion of \eqref{stable} for a range of normalised displacement
  tolerances $\epsilon$ and a maximum disturbance displacement of
  $\dmax=\SI{5}{mm}$. Darker sections denote overlap of the regions in the overlay plot.}
  \figlabel{cons-stable-1}
\end{figure}

\begin{figure}
  \gridIV{\psfragfig{\phdpath QZS/fig/cons-stableN1-2}}
         {\psfragfig{\phdpath QZS/fig/cons-stableN2-2}}
         {\psfragfig{\phdpath QZS/fig/cons-stableN3-2}}
         {\psfragfig{\phdpath QZS/fig/cons-stableNall-2}}
\caption{Region of $\mdim$, $\ngap$ satisfying the stability criterion
  of \eqref{stable} for a range of disturbance displacements $\dmax$
  and a normalised displacement tolerance $\epsilon=\num{0.1}$. Darker
  sections denote overlap of the regions in the overlay plot.}
  \figlabel{cons-stable-2}
\end{figure}

The introduced criteria may be simultaneously satisfied for some specified
values of mass, resonance frequency, and displacement range ($\mass$, $\freq$,
and $\dmax$) by varying the free parameters magnet size $\mdim$ and magnet gap
$\mdim\ngap$. The shared region of the criteria previously introduced
(\eqrangeref{strong}{stable}) are shown in \figref{cons-all} for various
combinations of supported mass and resonance frequency. This figure presents a
complete design map that shows how the techniques presented in this paper can
be used for speculative design work and optimisation.

\begin{figure}
  \begin{wide}
    \centering
    \gridVI
      {\psfragfig{\phdpath QZS/fig/cons-allN1}}
      {\psfragfig{\phdpath QZS/fig/cons-allN2}}
      {\psfragfig{\phdpath QZS/fig/cons-allN3}}
      {\psfragfig{\phdpath QZS/fig/cons-allN4}}
      {\psfragfig{\phdpath QZS/fig/cons-allN5}}
      {\psfragfig{\phdpath QZS/fig/cons-allN6}}%
    \\[1.5em]
    \psfragfig{\phdpath QZS/fig/cons-allNall}
  \end{wide}
\caption{Regions of $\mdim$, $\ngap$ satisfying all design criteria,
  demonstrating the effects of changing the mass $\mass$ and the
  resonance frequency $\freq$ of the desired system. Darker sections
  denote overlap of the regions in the overlay plot.}
  \figlabel{cons-all}
\end{figure}

\subsection{Measure of stiffness reduction}
\seclabel{kratio}

In the previous section, ranges for magnet size and magnet gap were
shown to achieve certain design criteria. It is not impossible, however,
that this process could yield a design that behaves acceptably with
only a single repulsive magnet: the magnet gap could be so large as to
render the effect of the negative stiffness negligible.

The total stiffness characteristic of the magnetic system is given
by \eqref{nstiffnessZ} as the sum of the stiffnesses due to repulsive
and attractive magnets respectively. At equilibrium, this can be written as
\begin{align}
  \keq & = \krepl{1,2,\heq{}} + \kattr{1,2,\heq{}} = \krepl{}\gp{1-\QZSkratio{}}, \\
\intertext{where}
  \QZSkratio & = \Abs{\frac{\kattr{}}{\krepl{}}}.
\end{align}
The variable $\QZSkratio$ can be considered as the ratio of `stiffness
reduction' achieved by the presence of the attractive magnet. For
$\QZSkratio{}=0$, the upper magnet is providing no negative stiffness
to the system; for $\QZSkratio{}=1$, the equilibrium position is at
quasi-zero stiffness and the system is marginally stable.

\Figref{kratio} illustrates the variation of $\QZSkratio{}$ over a
particular design region; this plot shows that designs achieved with
larger magnet gaps have little influence from the stiffness reducing
effect of the attractive magnet. As $\QZSkratio{}$ tends towards one, the
resonance frequency drops dramatically as the equilibrium position
approaches the \qzs\ position (compare with \figref{heq-map}).

\begin{figure}
\centering
\psfragfig{\phdpath QZS/fig/kratio}
\caption{%
  Contours of stiffness reduction $\QZSkratio{}$ due to the negative
  stiffness of the attractive magnet,
  shown inside the region of $\mdim$,
  $\ngap$ satisfying all design criteria for parameters $\mass=\SI{1}{kg}$,
  $\freq=\SI{5}{Hz}$, $\dmax=\SI{1}{mm}$, $\epsilon=0.1$.}
\figlabel{kratio}
\end{figure}

\subsection{Nonlinear behaviour}
\seclabel{nonlin}

In \secref{cons}, constraints were imposed on the design of the
magnetic system such that a certain resonance frequency was achieved
for a given mass loading. The slope of the force vs.\ displacement
curve at equilibrium was used as the basis for the resonance frequency
criterion. However, as the mass is perturbed from equilibrium the
stiffness of the spring changes. When designing for vibration
isolation, it is important to ensure that this nonlinearity does not
produce a significant effect in the response of the system.

A measure of the nonlinearity of the system, $\nk$, can be found by
comparing to the nominal stiffness the mean change in stiffness of the
spring at equilibrium over its maximum peak-to-peak
displacements from equilibrium:
\begin{dmath}[label=nl]
\nk = \frac{\Delta\magstiffnessZ{}}{2\keq{}} ,
\end{dmath}
where
\begin{align}
\Delta\magstiffnessZ{} &= \magstiffnessZ{1,2,\heq{}+\dmax/\mdim}
                      -\magstiffnessZ{1,2,\heq{}-\dmax/\mdim} \\
\keq{} &= \magstiffnessZ{1,2,\heq{}} .
\end{align}
For the quadratic approximation of $\nforceZ$, $\nk$ is a ratio
between the maximum displacement and the equilibrium position:
\begin{dmath}[label=nk-approx]
\nk \approx \frac{\dmax}{\mdim\heq{}} ,
\end{dmath}
and since $\dmax/\mdim<\Abs{\heq{}}$ for stability (recall
\eqref{stable}), it follows that the degree of
nonlinearity is directly related to the amplitude of disturbance
vibration. \Figref{nl-contours} demonstrates the manner in which $\nk$
varies over an allowed region of design parameters, with $\nstiffnessZ$
given by the exact expression of \eqref{nstiffnessZ}.

The nonlinearity increases both as magnet size and normalised magnet gap
decrease.  Comparing \figref{nl-contours} to \figref{heq-map,kratio}, it
can also be seen that the nonlinearity increases the closer the equilibrium
displacement becomes to the \qzs\ position, and the greater the effect of
the negative stiffness from the attractive magnet.

\begin{figure}
  \psfragfig{\phdpath QZS/fig/nl-contourss}
\caption{Contours of nonlinearity shown inside the region of $\mdim$,
  $\ngap$ satisfying all criteria for parameters $\mass=\SI{1}{kg}$,
  $\freqdesign=\SI{5}{Hz}$, $\dmax=\SI{1}{mm}$, $\epsilon=0.1$.
  A closeup (of the box) is shown in \figref{nl-contours-zoom}.}
  \figlabel{nl-contours}
\end{figure}

\begin{figure}
  \psfragfig{\phdpath QZS/fig/nl-contours-zoom}
  \caption{Zoom of \figref{nl-contours} to show the behaviour of the
 nonlinearity in the lower limits of a magnetic design.}
  \figlabel{nl-contours-zoom}
\end{figure}

To analyse these nonlinear effects on the vibration response of the magnetic
spring, the system shown in \figref{schem} is simulated with the dynamics
\begin{dmath}[label=response]
  \mass \ddot x + \damping \gp{\dot x-\dot y} - \magforceZ{1,2,\gp{x-y}/\mdim} + \mass g = 0
\end{dmath}
where $x$ is the displacement of the isolated mass and $y$ is the
displacement of the external disturbance. The system is excited
tonally at resonance in order to depict the steady state response with
the greatest magnitude and therefore greatest nonlinearity. The
excitation has amplitude $A_e$, and is given by
\begin{dmath}[label=y]
  y = A_e \sin\fn{\natfreq t} .
\end{dmath}

The system parameters for the simulation are $\mass=\SI{0.5}{kg}$,
$\mdim=\SI{20}{mm}$, $\ngap=3$, excitation frequency $\natfreq=\SI{3.58}{Hz}$,
and equilibrium position $\heq{}$ is \SI{10.4}{mm} below the \qzs\
position. While the exact amount of damping present in the system will be
highly model-dependent, the damping coefficient is chosen as 5\% based on the
small damping forces due to eddy current and air resistance effect
\cite{fung2005,bonisoli2006,nijsse2001}.

\Figref{nl} shows the steady state response of \eqref{response} for a range of
ground disturbance amplitudes in even increments up to $A_e=\SI{1}{mm}$.
Because the excitation is at resonance, the output displacement is greater
than the input displacement. The phase plot of the response becomes
increasingly skewed as the amplitude of vibration, and hence the nonlinearity,
increases; \tabref{nl} summarises the nonlinearities calculated using
\eqref{nl}. Physically, this is interpreted as the spring being perturbed
further into the stiffer region as the mass is moved closer to the repulsive
lower magnet, which results in stronger forces as the normalised displacement
increases. Conversely, as the mass moves upwards, closer to the \qzs\
position, the stiffness and force both decrease.

The results shown in \figref{nl} indicate that for small magnitude
disturbances the phase plot is very similar to a linear system.
As the amplitude increases, the
nonlinear system response increases at a slower rate than the linear
spring. This slowdown is due to a
shifting of the resonance peak as the nonlinearity increases, as
will be seen later in this section.

\begin{figure}
  \psfragfig{\phdpath QZS/fig/nl}
\caption{Phase plot of the magnetic system at steady
  state resonance for a range of input disturbance amplitudes
  shown in \tabref{nl}. The
  light line is the equivalent linear response.}
  \figlabel{nl}
\end{figure}

\begin{table}
  \begin{tabular}{@{}ccc@{}}
    \toprule
    $A_e$, mm & $\Delta \magstiffnessZ{}$, \SI{}{N/m} & $\nk$ \\
    \midrule
    0.2 & 68 & 0.13 \\
    0.4 & 135 & 0.27 \\
    0.6 & 198 & 0.39 \\
    0.8 & 253 & 0.50 \\
    1.0 & 297 & 0.59 \\
    \bottomrule
  \end{tabular}
  \caption{Nonlinearity values, $\nk$, of the responses shown in \figref{nl}.}
  \tablabel{nl}
\end{table}

It is important to consider the role of damping in the results shown
in \figref{nl}. A low damping coefficient results in a greater
resonance response in the low frequency range. This increase in the
displacement response will also increase the nonlinear behaviour of
the spring. However, the advantage of low damping is a very fast
roll-off in vibration attenuation at frequencies above
resonance. Since the damping of the non-contact magnetic spring is
very low, either the bandwidth of excitation must lie above the
resonance frequency or active sky-hook damping must be applied in
order to reduce the strong effect that the resonance has on the
low-frequency response. As previously discussed, sky-hook damping is
especially suitable for this purpose because the isolation region of
the frequency response is not affected.

\subsection{Variance gain of the magnetic system}

One metric to evaluate the response of a nonlinear system in the frequency
domain is known as the `variance gain' $V$ \cite{savaresi2007}, which
is calculated as a ratio of the root-mean-square
output to input signals:
\begin{dmath}[label=nl-trans]
V = \sqrt{\left.\tfrac1T\Int_0^T \bar x\fn t^2{t}
    \middle/
    \tfrac1T\Int_0^T \bar y\fn t^2{t}\right.} ,
\end{dmath}
where $T$ is the time interval over which the variance gain is calculated,
and $\bar x$ and $\bar y$ are the mean-zero output and input displacements
of the vibration isolator, respectively. For linear systems, this expression
simplifies to the standard formulation for transmissibility. For nonlinear
systems, \eqref{nl-trans} describes the ratio of output to input energy for
a given excitation. For the tonal input disturbance of \eqref{y}, the variance
gain at the frequency of excitation is
\begin{dmath}[label=nl-trans2]
V = \frac{1}{A_e}\sqrt{\frac{2}{T} \Int_0^T \bar x\fn t^2 {t}} .
\end{dmath}
The variance gain allows the nonlinearity of the system to be visualised
on a familiar transmissibility--like plot
for a range of resonant frequencies and equilibrium positions.

\Figref{variance-amp} shows the variance gain for a magnetic system with
$\mass=\SI{0.5}{kg}$, $\mdim=\SI{20}{mm}$, and $\ngap=2.8$ with excitation
amplitude ranging from \SI{0.1}{mm} to \SI{0.5}{mm}. With these parameters, the
equilibrium position is \SI{1.55}{mm} below the \qzs\ position. As seen
in the figure, the greater the excitation amplitude, the greater the softening
nonlinearity of the spring.

\Tabref{variance-amp} shows some data from the
simulated responses, including the maximum displacement of the spring towards
the \qzs\ position, and the nonlinearity measures $\nk$ for each amplitude of
excitation. The maximum displacement $x_{\text{max}}$ corresponds with the
maximum normalised displacement $\ndispZ_{\text{max}}$ shown in \figref{heq};
as $\ndispZ_{\text{max}}$ tends towards zero, the motion of the magnetic
spring moves closer to the \qzs\ position. Note that the pronounced
nonlinearity seen in \figref{variance-amp} belongs to a system that is
approaching its bounds of stability, reaching \SI{0.24}{mm} below the \qzs\
position in its most extreme displacement.

\begin{figure}\centering
\let\labelsize\footnotesize
\psfragfig{\phdpath QZS/fig/variance-amp-crop}
\caption{Variance gain at a position close to \qzs, for a range of excitation
amplitudes $A_e$, as labelled. The spring softening effect at resonance can
be clearly seen as the amplitude increases. Tabulated values of nonlinearity
and maximum displacement are shown in \tabref{variance-amp}.}
\figlabel{variance-amp}
\end{figure}

\begin{table}\centering
\begin{tabular}{@{}cccc@{}}
\toprule
$A_e$, mm & $\Delta \magstiffnessZ{}$, \SI{}{N/m} & $x_{\text{max}}$, mm & $\nk$ \\
\midrule
 0.1 & 29.3 & $-1.04$ & 0.31 \\
 0.3 & 57.6 & $-0.47$ & 0.62 \\
 0.5 & 66.3 & $-0.24$ & 0.71 \\
\bottomrule
\end{tabular}
\caption{Nonlinearity values, $\nk$, and maximum displacements, $x_{\text{max}}$,
of the responses shown in \figref{variance-amp}. The equilibrium stiffness is
$\keq{}=\SI{46.6}{N/m}$. The nonlinearity can be seen to increase with excitation
amplitude. In the most extreme case, the magnetic system comes \SI{0.24}{mm}
from the \qzs\ position.}
\tablabel{variance-amp}
\end{table}

\Figref{variance-gaps} shows the variance gain of the same system as the magnet gap
$\ngap$ is increased. Results are shown for excitations of both $A_e=\SI{0.1}{mm}$
and $A_e=\SI{0.5}{mm}$. Data for the simulations with $A_e=\SI{0.5}{mm}$, including
nonlinearity measure $\nk$ and maximum displacement $x_{\text{max}}$, are shown in
\tabref{variance-gaps}. Two related features are important to note from \figref{variance-gaps}.
The first is the large decrease in resonance frequency as the magnet gap decreases. The second
is the corresponding increase in nonlinear behaviour as this occurs. As the equilibrium
position moves away from the instability at \qzs, the variance gain quickly exhibits
linear behaviour.

\begin{figure}\centering
\let\labelsize\footnotesize
\psfragfig{\phdpath QZS/fig/variance-gaps-crop-new}
\caption{Variance gain of the magnetic system, comparing between two excitation
amplitudes with varying values of magnet gap $\ngap$, as labelled. Low amplitude
is $A_e=\SI{0.1}{mm}$ and high amplitude is $A_e=\SI{0.5}{mm}$; shown as dashed
and solid lines, respectively. The nonlinear effect is only prominent in the case
close to \qzs, with an excitation amplitude that almost destabilises the system.
Tabulated values of nonlinearity and maximum displacement are shown for the high
amplitude case in \tabref{variance-gaps}.}
\figlabel{variance-gaps}
\end{figure}

\begin{table}\centering
\begin{tabular}{@{}ccccc@{}}
\toprule
$\ngap$ & $\keq{}$, \SI{}{N/m} & $\Delta \magstiffnessZ{}$, \SI{}{N/m} & $x_{\text{max}}$, mm & $\nk$ \\
\midrule
 2.8 & \hphantom{1}66.3 & 46.6 & $-0.24$ & 0.71 \\
 2.85 & 138.8 & 145.8 & $-2.72$ & 0.48 \\
 3.0 & 167.6 & 253.5 & $-7.80$ & 0.33 \\
\bottomrule
\end{tabular}
\caption{Nonlinearity values, $\nk$, and maximum displacements, $x_{\text{max}}$,
of the high amplitude responses ($A_e=\SI{0.5}{mm}$) shown in \figref{variance-gaps},
for a range of
magnet gaps, $\ngap$. As the stiffness is decreased by the upper attractive magnet,
greater nonlinearity is seen for smaller magnet gaps, as the equilibrium position
moves closer to the point of \qzs.}
\tablabel{variance-gaps}
\end{table}

\subsection{Summary}
\seclabel{end}

This paper has analysed a magnetic spring for the purposes of
load bearing with low stiffness. Exact and approximate expressions
were derived for cube-shaped magnets for analysing the behaviour of
this system. The approximate expression is very simple and accurate
over a large displacement range, and may be used for cubical magnets of
any size.

Four design criteria were imposed on the design in terms of the two variable design
parameters: magnet size and gap between the fixed magnets. A technique for explicitly
mapping these constraints to a map of valid parameters was shown. Allowable stroke
and magnet size availability allow a design to be optimised using this technique
based on the required load bearing and resonance frequency.

The magnetic isolator is weakly nonlinear with a distorted phase plot compared to
a linear system; the variance gain shows a resonance peak skewed into the lower
frequencies. These nonlinearities only become apparent at larger vibration amplitudes
at equilibrium positions that are close to the \qzs\ position
(that is, large measures of nonlinearity). Provided the system remains stable,
the nonlinearities are not detrimental to the frequency response of the system.

The vibration isolator described in this paper is therefore suitable for
precision applications where low resonance frequencies are required. The
design is scalable in that many such isolators may be used in parallel to
achieve greater load bearing even if only small magnets are available. The
inherent low damping of the system results in good high frequency performance,
but low frequency disturbances will result in large outputs due to the high
resonance peak. This effect can be mitigated by the application of sky-hook
damping to the system.
