%!TEX root = thesis.tex

\chapter{Vibration isolation}


\subsection{Modal systems}

The issue of selecting appropriate control gains for velocity feedback control
systems was examined by \textcite{engels2008} for the cases of centralised and
decentralised control devices. In centralised control, a global model of the
system is used when allocating the feedback signals. In decentralised control,
each sensor/actuator pair operates independently to minimize the energy at the
mounting point. Generally centralised control is more difficult in practise
but can give better results. For the simple two-degree of freedom vibrating
system examined by \textcite{engels2008}, the centralised control performed
better although the differences to the decentralised control were small. An
analogous result was shown by \textcite{hoque2006} for a three-axis vibration
platform supported by so-called `infinite stiffness' magnets (see
\secref{infstiff}).



\textcite{yoshioka2001} designed and built a prototype vibration
isolation platform with a load bearing capacity about
\SI{1000}{kg}. Their device used a combination of pneumatic,
voice-coil linear motors, and piezoelectric actuators to control the
first half dozen vibration modes of the platform. Feedback velocity of
the table and feedforward velocity and displacement of the floor were
used as control signals (with integrated accelerometer
measurements). A genetic algorithm was used to generate a set of
control gains for the system, and very good results were achieved.




\section{Higher-level control}

Vibration isolation can be considered exclusively as a control
problem. \textcite{guo2005}, for example, used system identification
and a variety of feedforward/feedback control techqniques to isolate a
multi-degree of freedom structure in a single direction. Adaptive
control in six degrees of freedom for broadband noise has also been
shown by \textcite{duindam2005}.

\textcite{balandin1998} review the field of optimal control as applied
to shock and vibration isolation problems.\footnote{Their comment that the
  \enquote{number of papers is so great that there is little incentive to
  discuss them here} does not bode well for any attempts by me to
  even summarise their review.} 
  They differentiate shock and vibration isolation succintly:
\begin{quote}
  The operating quality of shock isolators is usually described in
  terms of certain characteristics of the transient motion of the body
  being isolated, whereas the quality of vibration isolators is
  determined by the characteristics of steady-state forced
  oscillations.
\end{quote}
That is, optimising for shock will result in minimising, say, the peak
displacement of the mass, whereas optimising for vibration will result
in a low natural frequency (characterised by a minimum achievable
\RMS\ displacement). Since an optimal controller is based around a
cost function that will be dependent both on the vibratory system
itself and the mode of disturbance (transient, broadband noise, \etc),
such control approaches are heavily case-specific and are best used
when a plant is pre-determined and a vibration problem needs to be
considered after the fact. \textcite{bolotnik2001} discusses these
methods in more detail.

I haven't finished reading the paper by \textcite{savaresi2007}, but
it's about a combination skyhook/acceleration-feedback controller that
is really nice. It is in the context of vehicle vibration control for
ride `smoothness'; the premise is that skyhook damping knocks out
resonance peaks (without any other side-effects; \cf\ relative
velocity feedback) while acceleration ($\equiv$ `mass') feedback
decreases the overall transmissibility at higher frequencies.

Earthquake protection systems deal with a form of vibration isolation,
but focus more on transient response than continuous disturbance. The
nature of the problem is important to analyse before control
strategies are applied. \textcite{gavin2007} discuss using negative
stiffness and skyhook damping in the context of isolating machinery
in a building. (These guys also wrote the paper ``Drift-free
integrators'', which I may find useful to look into.)

Skyhook damping is considered an ideal case (I need some references
— \ie, detail — for this) and is implemented with a switching
controller in practice. This is because the abstraction of skyhook
damping is to attach a damper to the `sky' itself; that is, the added
damping is proportional to the absolute velocity of the damped
mass. In practice, however, a velocity sensor can only measure a
\emph{relative} velocity between the mass and the position of the
fixed sensor. I find it interested than observer techniques haven't
been investigated for this purpose. Perhaps I've just missed that
particular research cul-de-sac.

\textcite{kerber2007} modelled a plate-plate coupling through four
contact points in six degrees of freedom system and applied active
vibration isolation in the vertical direction using a range of control
methods. Of those trialed, \Hinf\ control had the best low frequency
response, and \PI\ control (with anti-windup) the best high
frequency response. Output feedback, state feedback with an observer,
and velocity feedback methods all performed worse to various extents.

\textcite{chen2007} used \Hinf\ control to suppress the resonance peak at
around \SI{3}{Hz} of a pneumatic vibration isolation mount. Very good
reductions in transmissibility were achieved; this technique highlights the
appeal of vibration control using an isolator with a damped resonance.

A modern look at output feedback methods is shown by \cite{mottershead2008},
where the dynamics of complex structures is altered via robust pole-placement
techniques that do not require models of the system.

And here's a method using delayed acceleration feedback by
\textcite{chatterjee2008}.


In a recent paper, \textcite{madkour2007} compare a slew of nonlinear
`artifical intelligence' adaptive algorithms to observe the parameters
of a vibrating pinned beam in order to apply feedback cancelation. The
algorithms compared are traditional RLS, a genetic algorithm with a
specific inheritance method, a neural network for estimating a smooth
function, and a combination fuzzy modelling/neural network procedure
for system identification. Control based on the genetic algorithm
performed the best.

Such techniques are appropriate when system identification is required
during operation; \ie, when the plant cannot be measured beforehand or
it slowly changes over time. However, such `artifical intelligence'
algorithms are probably sub-optimal for the observation of such a
dynamics problem when the system can be broadly modelled as a set of
differential equations in the time domain, for which nonlinear
techniques such as backstepping or sliding mode control are directly
suitable. Leave the heuristic approaches to systems that cannot be
modelled.

Note that this statement only applies to the use of such methods in
the control phase of vibration control; genetic algorithms, \eg,
provide a useful technique in the design phase when limited numbers of
sensors and actuators must be placed in `optimal' positions on a
complex structure. \fixme{cite us}




\section{Infinite stiffness}
\seclabel{infstiff}

Both \textcite{nijsse2001} and \textcite{mizuno2003a} (with related
publications) talk about using a combination of a positive and
negative stiffness springs to achieve affects like
\begin{dmath*} 
  k_T = \frac{ k_1 k_2 }{ k_1 + k_2 } = \infty 
  \condition{when $ k_2 = -k_1 $.}
\end{dmath*}

\textcite{xing2005} formalises this idea in their paper, or at least claim to.

A swath of papers have been published by Mizuno over the years 
\cite{mizuno2001,mizuno2003a,mizuno2003b,mizuno2007,mizuno2002}. 
The term `infinite stiffness' is perhaps unfortunate, as it's not true in
any useful sense of the term.

The work was extended by \textcite{hoque2006} for three--degree of freedom
vibration control.

\textcite{mizuno2003c} also present a modified version of the idea, whereby
rather than using magnetic springs in attraction, as in their other work, this
paper looks at using a voice coil actuator with designed negative stiffness. A
controller must be used to obtain this negative stiffness, but it is not clear
to me, at this stage, if their system will have the same sort of problem as
I've shown in my Active 2006 paper with the cancelling dynamics of a positive
stiffness mechanism.

The papers dealing with infinite stiffness are not doing what they
claim. That is, their static behaviour converges to something that
looks like infinite stiffness, but \emph{dynamically} they're the same
as any other structure.

This is because the negative spring in combination must be
stabilised somehow, and this stabilisation removes the nice property
of summing to zero with the other stiffness in the system.

\section{Nonlinear systems}

\textcite{virgin2008} proposes an interesting alternative to the
classical helical spring for vibration isolation support: a `pinched
loop' created by a slender beam with both ends clamped has a wide
range of tuning possibilities and offers isolation in two degrees of
freedom.

Interesting results have been shown using nonlinear springs to attach the
vibration absorber to the structure, such as \textcite{jo2008} who use
repulsive magnetic springs to produce a tuned absorber with a resonance at
double the frequency of the main resonance of the structure. \fixme{Why would
you do this?!}

A similar idea with a different approach is taken by
\textcite{liu2006a}, who used a permanent magnet and an electromagnet
to provide a variable stiffness in the absorber.

\textcite{zhang2008} use a nonlinear damper (specifically of the form viscous
plus cubic with velocity; \ie, $a_1\dot x\fn{t}+a_3\ddot x^3\fn{t}$) to excite
the structure at harmonic frequencies of the resonance. This results in less
energy at the frequency of vibration, although the resonance peak does remain.
While it does not seem likely that this method can compete with the reductions
seen with the approach of a tuned vibration absorber, this nonlinear damping
method does have the advantage that it does not require tuning for a
particular frequency and its effectiveness will not change with a time-varying
resonance frequency.

A very recent and comprehensive review of nonlinear vibration
isolation systems for a broad range of techniques was published by
\textcite{ibrahim2008}. It highlights the importance of nonlinear
analysis in this field: in some cases, better results can achieved
by using nonlinear spring forces to couple to and absorb vibration
energy (\fixme{eg}); in other cases, the behaviour of an isolator
cannot be adequately modelled by using linear systems theory.

Various combinations of magnetic springs create different classes of
approximate nonlinear oscillators, for which standard methods exist to analyse
their behaviour \fixme{can I cite a book here I haven't read?}. The
characteristic property of nonlinear springs is their exhibition of so-called
`jump phenomena' whereby the frequency response curve can take multiple values
at a single frequency, depending on the initial conditions.

\textcite{2006} analysed the behaviour of a nonlinear isolation mount in
detail, developing analytical models for the jump phenomena of a system with
cubic stiffness and quadratic damping. Critical values were illustrated to
avoid the ill effects of the nonlinearities; additional damping had the
general effect of decreasing the adverse nonlinear response.

For the purpose of vibration control, augmenting a linear spring system with
nonlinear magnetic springs alters the behaviour of the natural frequency of
the system to be weakly coupled to the mass of the system.
\textcite{dangola2006} analysed the nonlinear dynamics of a nonlinear system
in which a variation of both stiffness and mass by up to 50\% yields an
increase of stiffness of 6\%. For an equivalent linear system, the natural
frequency variation is ten times greater. This is a very interesting result
for loading elements for which the mass to be supported is largely variable,
in that the frequency response will vary significantly less than for
conventional linear springs.

However, for weakly nonlinear magnetic springs, variation in the mass will
still lead to changes in the resonance frequency. \textcite{todaka2001}
created a mechanical linkage to support two magnets in repulsion such that as
their air gap increased (due to less mass being supported), a horizontal
offset between them was created to lower the linearised operating stiffness.
This allowed a much smaller variation between the $k/m$ ratio than for flush
magnets in repulsion.

\textcite{bonisoli2007} used an experimental rig to analyse the nonlinear
behaviour of a magnetic and linear spring in parallel, with more theoretical
analysis published later in the year \cite{bonisoli2007b}. They showed a
configuration of linear and magnetic springs with the notable feature that the
resonance frequency exhibits little dependence on mass loading and nonlinear
effects can be seen. This effect was further demonstrated in parallel research
by the same authors.

The field of nonlinear dynamics is very large, and applications to vibration
suppression can result in surprising results. \textcite{oueini1999} considered
the response of a nonlinear (including cubic) plant with an additional cubic
nonlinear feedback law and established that vibration attenuation was possible
(like a tuned mass damped?) and that nonlinear phenomena such as chaos
existed. Presumably, such phenomena are undesirable for vibration isolation!

\textcite{zhao2007} examined the vibratory behaviour of a two degree of
freedom nonlinear system with a time-delayed positional feedback. This results
of this paper emphasises the importance of accounting for time delays in
feedback control systems.

\fixme{put this somewhere up there:} \textcite{starosvetsky2008}, with a good
review of the literature, talk about the `nonlinear energy sink' idea in which
a nonlinear system provides better and broader vibration absorbing.

\cite{zuo2004}

Anyway, it is an interested argument to make that digital control in a
vibration isolation situation is inappropriate. When absence of all
vibrations are required, the side-effects of digital control (\ie,
chaotic effects due to quantisation and time delays
\cite{csernak2007}) may prove deleterious for extreme applications. On
the other hand, any sufficiently expensive system should be able to
reduce these effects to be arbitrarily negligible. For the purposes of
this thesis, such small effects are of no concern.


Nonlinear control is better than linear control, as argued cogently by
\textcite{kokotovic1992}. \textcite{queiroz2007} used nonlinear
control to stabilise a magnetic bearing with pull--pull electromagnets
with parameter uncertainties, while also minimising power consumption
of the system.



\section{Energy harvesting}
\seclabel{energy-harvesting}

When electrical circuits are used to absorb resonant vibrations, the energy
absorbed can be redirected to produce a power output \cite{stephen2006}. Such
devices are gaining popularity for ambient vibration--powered applications
such as remote sensing \cite{arnold2007}. Another field of interest for
regenerative damping is in vehicle suspensions, in which useable power can be
extracted with the same mechanism as providing greater ride comfort
\cite{graves2000thesis}.

A perfect energy harvesting devices would manage to absorb all of the kinetic
energy of the disturbances to generate its power. While this is impossible in
practise, the vibration absorption properties of such a device warrant its
inclusion in this literature review.

\textcite{challa2008} recently investigated a
semi-active device for tunable energy harvesting that used
variable-displacement attractive and repulsive magnets to adjust the
resonance frequency of a piezoelectric cantilever. This is the same
mechanism that is examined in this thesis for `\qzs' suspensions
(crossref).

Energy generation can also be accomplished through electromagnetic damping
through induced eddy currents in a coil experiencing relative displacement to
a magnetic field, such as investigated by \textcite{graves2000}.
\Secref{damping} discusses the damping possibilities of electromagnetic
systems for the purposes of vibration isolation devices.

\textcite{stephen2006}
talks about energy harvesting with micro-actuators, and the analysis
is pretty thorough. Furthermore, it all makes sense, which is quite
nice. There is a moderate amount of single degree of freedom
mass-spring-damper maths for future reference, which is coupled with a
simple electrical circuit. Not surprisingly (it's easy to say this
when simply reporting their results) they want energy dissipated
primarily by the electric components, not the mechanical damping, since
in the former the energy is retrievable whereas in the latter it's simply
dissipated as heat.


\textcite{mann2008} performed a preliminary investigation into the use of a
nonlinear vibration mount for energy harvesting, using a magnetic suspension
of repulsive magnets to create a Duffing-like oscillator. Large damping
ensured that the nonlinear regimes were only realised at large excitation
amplitudes, but the idea is that highly nonlinear resonances have a much
broader resonance peak through the higher branch around the jump phenomenon.
\fixme{reference this elsewhere}

A similar idea is explored by \textcite{shahruz2008} for an energy scavenging
cantilever beam that uses an arrangement of attracting magnets to shape the
force characteristic of the response. The aim is to achieve a power spectrum
of the response to a random excitation that is greater than the predominantly
linear response that is obtained without the magnets present. No analysis of 
damping or energy harvesting potential was explored in this paper.

    
