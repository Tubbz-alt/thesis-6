%!TEX root = thesis.tex

\chapter{Vibration isolation}

\chapterprecis{
  A literature review on vibration isolation and other things.
  Right now this chapter is a mess of ideas, covering single degree of
  freedom isolators, skyhook damping, tuned vibration absorbers, 
  energy harvesting/scavenging devices, active control, \qzs\ devices, and
  more. It is intended to set the scene for many of the vibration-related aspects
  of the project.
}

The field of active vibration isolation is a broad topic to cover in review;
not everything will be able to be covered here, but it is important to have an
overview of what people are doing to place this work in context. The
literature review that follows is strongly biased towards papers that have
been recently published. Tracking their citations backwards will yield a
tangled web of prior art in the field of vibration control.

\section{Passive systems}

\textcite{wu2006} discusses the effects of including the inertial effects of a
non-mass-less spring into the vibration characteristics of a
mass-spring-damper system.

\textcite{virgin2008} proposes an interesting alternative to the
classical helical spring for vibration isolation support: a `pinched
loop' created by a slender beam with both ends clamped has a wide
range of tuning possibilities and offers isolation in two degrees of
freedom.

\subsection{Internal resonances}

Idealised models assume system properties that are independent: a
spring has no mass, a damper has no stiffness, \etc. In practise,
however, such simplifications can be far from actual, and internal
resonances of actuators and passive absorbers can decrease the
performance of their intended vibration isolation
mechanisms. 

\textcite{du2003thesis} examined the effect of additional dynamics
from tuned vibration absorbers and the consequence that had on the
transmissibility and noise radiation of the structure under
control. While these additional high frequency dynamics would not be a
problem for certain problems (and the effects were worse in
viscoelastic-type springs; \ie, rubber springs over mechanical springs),
their effect must be considered for high-sensitivity vibration control
and cases where it is important to minimise radiated sound. In his
thesis he proposes methods to work around the internal resonance
problem using both passive and active means.

\textcite{wu2006} analysed the vibration response of a spring-mass-damper in
which the spring element had some inertial effect on the structure due to its
mass. The results presented show only a small magnification of the vibration
amplitude transmitted to the primary mass, particularly for small spring
relative masses; however, for highly sensitive equipment this phenomenom
should not be neglected.

Such internal resonance effects discussed above can be completely avoided by
using noncontact magnetic springs instead. On the other hand, the nonlinear
force/displacement relationship of magnetic springs will have some nonlinear
behaviour that could affect the vibration properties of such springs in other
ways. An example of this behaviour is demonstrated in \secref{qzs-not-zerk}.


\section{Feedback control}

\subsection{Single degree of freedom}

\Figref{simple-isolation} shows the simplest form of vibration
isolation in which a mass $\massMass$ is mounted with stiffness
$\stiffnessRel$ with disturbance input $\dispBase$. Passive
isolation is achieved for zero input force, $\forceIn=0$. Feedback
control allows the properties of the system to be adjusted
according the desired vibration response.

\begin{figure}
  \asyfig{Systems/vibration-base}
  \caption{Single degree of freedom vibration isolation mount;
    mass $\massMass$ is being isolated from disturbance
    $\dispBase$. Input force $\forceIn$ can be generated using
    feedforward or feedback control.}
  \figlabel{simple-isolation}
\end{figure}

The dynamic response of the isolated mass is
\begin{dmath}[label=simple-isolation]
  \massMass\ddot\dispMass +
  \dampingRel\gp{\dot\dispMass-\dot\dispBase} +
  \stiffnessRel\gp{\dispMass-\dispBase} + \forceIn = 0 .
\end{dmath}
First assume that there is no input force; taking the Laplace
tranform and rearranging produced the transmissibility
$\transmissibility$ of the system in the frequency domain:
\begin{dmath}[compact,label=simple-isolation-freq]
  \transmissibility\fn\freq = \frac{\laplaceMass}{\laplaceBase} = 
  \frac{\ii\freq\dampingRel + \stiffnessRel}
  {-\massMass\freq^2 + \ii\freq\dampingRel + \stiffnessRel}.
\end{dmath}

For ideal linear state feedback control, the input force can be
represented as a linear combination of measured relative
displacement $\gp{\dispMass-\dispBase}$, relative velocity
$\gp{\dot\dispMass-\dot\dispBase}$, and acceleration
$\ddot\dispMass$ of the mass. Let's also assume for now that
integrating the acceleration signal yields an exact absolute
velocity $\dot\dispMass$ of the mass as well. The generalised
feedback force can then be represented by
\begin{dmath}
 \forceIn = 
   \gainDisp\gp{\dispMass-\dispBase} + 
   \gainVel \gp{\dot\dispMass-\dot\dispBase} +
   \gainAcc \ddot\dispMass + 
   \gainSky \dot\dispMass
\end{dmath},
where $\gainArbitrary$ are feedback gains chosen appropriately for a given application.
The term involving $\gainSky$ is known as skyhook damping. Substituting this force equation
into \eqref{simple-isolation} yields
\begin{dmath}
  \gp{\massMass+\gainAcc}\ddot\dispMass +
  \gp{\dampingRel+\gainVel}\gp{\dot\dispMass-\dot\dispBase} +
  \gp{\stiffnessRel+\gainDisp}\gp{\dispMass-\dispBase} +
  \gainSky\dot\dispMass 
  = 0
\end{dmath}.

It can be clearly seen that there is an exact equivalence between the feedback
gains of the different signals and a corresponding physical parameter of the
system. For example, adding gain on the displacement feedback is analogous to
increasing the stiffness of the system. The one standout here is the `skyhook'
damping term, whose influence can be better seen in the equation for
transmissibility of the closed loop system. For this example, let displacement
and acceleration gains equal zero, $\gainDisp=\gainAcc=0$, resulting in
\begin{dmath}[compact,label=skyhook]
  \transmissibility\fn\freq = \frac{\laplaceMass}{\laplaceBase} = 
  \frac{\ii\freq\gp{\dampingRel+\gainVel} + \stiffnessRel}
    {-\massMass\freq^2 + \ii\freq\gp{\dampingRel+\gainVel+\gainSky} + \stiffnessRel}.
\end{dmath}
The transmissibilities due to the influence of relative velocity gain and
absolute velocity gain are shown in \figref{vel-vs-sky} with parameters
$\massMass=\SI{1}{kg}$, $\dampingRel=\SI{1}{kg/s}$,
$\stiffnessRel=\SI{10}{N/m}$, and the respective gains $\gainVel$ and
$\gainSky$ ranging from $0$ to $5$ corresponding from light to dark plot
lines.

\begin{figure}
   \begin{wide}
     \psfragfig{\phdpath Simulations/Springs/fig/sdof-vel}\hfil
     \psfragfig{\phdpath Simulations/Springs/fig/sdof-sky}
   \end{wide}
   \caption{Relative and absolute (`skyhook') velocity feedback control on the 
   system shown in \figref{simple-isolation}. Feedback gain increases as the 
   plot line ranges from light to dark.}
   \figlabel{vel-vs-sky}
\end{figure}

In comparing the \RMS\ transmissibilities over the frequency range of interest
($\sqrt{\Int{\transmissibility\fn\freq}{\freq,\freq_1,\freq_2}}$) as a function of
increasing feedback gain for the two cases, it is clear in the ideal case that
skyhook damping is more efficacious at reducing the total vibration of a
system. This is shown in \figref{rms-transmissibility}, where the relative
feedback \RMS\ transmissiblity has a local minimum whereas the skyhook case
continuously decreases. (The maximum frequency in this case was chosen to be
much greater than the resonance frequency;
$[\freq_1,\freq_2]=[0,\SI{1000}{rad/s}]$.)

\begin{figure}
   \psfragfig{\phdpath Simulations/Springs/fig/rms-transmissibility}
   \caption{\RMS\ transmissibility versus feedback gain of relative and 
   absolute (`skyhook') velocity feedback control on the system shown in 
   \figref{simple-isolation}.}
   \figlabel{rms-transmissibility}
\end{figure}

Velocity feedback control by using integrated accelerometer measurements at
the location of actuation has been used for some time as an effective
vibration isolation mechanism, shown for example by \textcite{kim1999} at the
ISVR at the University of Southampton, an institution that has been very
active in this field.

The limits of control in real-world devices was investigated by
\textcite{ananthaganeshan2001}, where time delays and phase lags introduced by
digital filters and integrators can have quite significant effects over the
ideal case of pure displacement or velocity or acceleration feedback (as
posited in the previous section). Real integrators (as opposed to ideal
integrators) and high pass filters cause instabilities at low frequencies.
Acceleration feedback has much smaller stability limit than displacement and
velocity. And the effect of displacement feedback is strongly reduced even
with small time delays. Therefore velocity feedback control from integrated
accelerometer measurements should be consided the better choice. Also, the
presence of a low pass filter does not significantly effect the efficacy of
velocity feedback.

\textcite{zhu2006} look at using such feedback in a highly sensitive
micro-gravity environment and show that `PID' feedback is unsuitable due to
instability at high (\ie, useful) control gains from quantisation and
anti-aliasing side-effects. Rather, integral and double-integral feedback from
an accelerometer give better results for them.

\subsection{Sensing}

The sensors used for active vibration control systems can make a
significant impact on the performance that the control can
achieve. \textcite{brennan2007} discuss the impacts of using
displacement, velocity, and acceleration feedback for a single degree
of freedom system in the context of signal conditioning filters and
digital time delays in the loop. Integrating an accelerometer signal
may be a convenient method to approximate the velocity, but it limits
the control performance that can be achieved compared to measuring the
velocity directly with a laser sensor (say).

It is always desired to minimise the number of sensors due to cost and
implementation complexity, and various methods may be used to estimate
states of a system that are not directly measured. If displacement and
acceleration are measured, a reasonable estimate of the velocity may
be computed by differentiating and integrating, respectively, the
measured signals and combining the results.

This method has been practised by \textcite{bennett2007}, and it works
acceptably well because the low frequency components can be obtained
from the derivative of the displacement and the high frequency
components can be obtained by integration of the acceleration. The
only question is where to draw the line in the frequency domain
between the two signals.

I've lost my paper that does this much more cleverly. Actually I haven't looked.


\section{Skyhook damping}
\seclabel{skyhook}

An attractive method of reducing the effects of resonance peaks is to add
damping to the structure. If this is performed by an actuator attached to the
`sky' instead of the ground, the advantage of knocking out the resonance peak
is not counteracted by an increase in higher frequency vibration (less
roll-off), which is the case for damping control with ground-fixed actuators.
This scheme is known as `skyhook' damping.

In the context of isolating vibration of a train, \textcite{li1999} discuss
the disadvantages in tracking when applying skyhook damping for systems that
experience very low frequency (\ie, steady state) changes in the base
position, but for the purposes of vibration isolation this is not a problem if
the ground is assumed to experience disturbance with zero mean. (They also use
Kalman filters for estimating velocity, an idea I'm keen on.)

A similar result \fixme{check/qualify this} can be achieved with dynamic stiffness control,
which can be effected with `semi-active' variable stiffness feedback control
\cite{leavitt2007}.

In the original and crude method of implementing skyhook control, the
switching mode that is necessary to emulate the `skyhook' in the controlled
damping introduces higher order (odd-multiple) harmonics in the frequency
response, as shown by \textcite{ahmadian2001}, who later proposed two
`jerk-free' skyhook algorithms \cite{ahmadian2004}.

Further work in the area was established by \textcite{liu2002}, who
demonstrated a number of skyhook-like semi-active damping methods, including
smoothing functions to eliminate the problem of jerk
\cite{liu2005}.\footnote{These references by Liu contain essentially the same
material, with greater detail in the first. The 2005 paper is easier to read
than the 2002 Technical Memorandum simply due to the formatting of the
documents.}

\textcite{song2007} used the nonlinearities of semi-active skyhook damping as
justification for the design of an adaptive controller for vibration
isolation; they did not compare their results with recent work in the skyhook
area, however.

Skyhook damping can also be implemented with fully active control systems, an
idea explored by, \eg, \textcite{elliott2001,elliott2004,yan2006,kim2008a}
\fixme{describe these more}. These methods are conceptually simpler, as the
inertial velocity can easily be obtained from integrated acceleration
measurements. Instability can occur in cases when the applied control force
affects the dynamics of the base, but in practice this tends not to be a
problem for comparatively massive support structures.

\textcite{serrand2000} discuss active skyhook damping applied to a two degree
of freedom structure with emphasis on the effects, for their system, due to
possible base flexibility. One of their results shows that using integrated
accelerometer measurements as a velocity feedback term has a low frequency
phase shift that can induce stability for high enough control gains.
Therefore, very high precision accelerometers or laser velocity vibrometers
are required for very low frequency vibration control (below \SI{1}{Hz}).




\section{Inertial actuators}

Sometimes it is not possible to integrate the control mechanism into the
support of a structure, in which case external actuators need to be added to
the device to provide the control forces. These tend to be inertial
electromagnetic actuators, also known as `proof-mass' actuators, where the
mass of the moving element provides an external force via coupling to the
structure. This is shown schematically in \figref{vibration-absorb}

\begin{figure}
   \asyfig{Systems/vibration-inertial} 
   \caption{An inertial force $\forceAbsorb$ designed to reduce the vibration 
   response $\dispMass$ due to disturbance $\dispBase$. The inertial actuator 
   has dynamics of its own ($\stiffnessAbsorb$, $\dampingAbsorb$) that 
   influence the overall vibration of the structure.}
   \figlabel{vibration-absorb}
\end{figure}

In this case, velocity feedback on its own can have problems with stability
margins at low frequencies. \textcite{benassi2002a} (using inertial actuators)
showed in experiment that this can be improved by the addition of a phase lag
controller with force feedback in conjuection with velocity feedback. The
theory for such `combined-state' feedback cases was subsequently developed
\cite{benassi2002b}.

A wide combination of feedback combinations was analysed by
\textcite{diaz2005} focussing on various forms of velocity feedback for
single-, double-, and multi-degree of freedom vibration isolation systems. In
the two degree of freedom system, an inertial actuator is used to provide
control force; as well as additional stability contraints due to this
arrangement, the resonance at low frequencies of the actuator itself
compromised the control performance.

(A possible solution to this problem might be to select an actuator with a
much \emph{greater} resonance frequency than the plant, but this arrangement
has been shown to be ineffective at controlling the system
\cite[][Appendix~A]{benassi2002}.)

A comparison of some of these methods, including skyhook damping and
semi-active vibration absorber methods, was done by \textcite{huyanan2007},
contrasting the performance and implementation differences between them.

\textcite{paulitsch2003} uses an electromechanical actuator that serves as a
self-sensing device for vibration control. The idea of self-sensing for
magnetic levitation purposes (in both cases using a electromagnet) has been
shown by \textcite{bleuler1992,vischer1993}. These self-sensing devices
(perhaps obviously, since the back--electromotive force is such a noisy
signal) do not perform nearly as well as when using a dedicated sensor, but
the technique is particularly interesting for low-cost, low-precision devices.





\section{Tuned vibration absorber}

One method of reducing vibration on a supported mass is to attach a
supplementary mass that resonanates in concert with the disturbance; this has
the effect of adding an anti-resonance to the original system at the frequency
of interest. These are known under various names as `tuned mass dampers',
`tuned vibration neutralisers', or `dynamic vibration absorbers' (presumably
other variations exist).

The concept of tuned mass dampers is not entirely the focus of this research,
but falls into the category of literature that crops up in association with
it. In the passive application, a tuned vibration absorber consists of
attaching a supplementary mass with a spring to the structure for which
vibration is to be removed. Excellent results can be achieved in knocking out
a resonance peak by matching the natural frequencies of the structure and the
additional mass. This is shown in \figref{tuned-mass-vs-fig} where a response
is shown for a range of stiffness values. There is a compromise between broad-
and narrow-band vibration attentuation. \Figref{inertial-trans-delta}
illustrates the vibration attenuation at resonance for a system with a
vibration absorber. It can be seen that for narrowband reduction, low absorber
damping produces greater vibration attenuation. Conversely, if the \RMS\
transmissiblity of the entire frequency band is calculated, as shown in
\figref{rms-inertial}, it can be seen that lower absorber damping
\emph{decreases} the overall vibration reduction.

\begin{figure}
   \psfragfig{\phdpath Simulations/Springs/fig/tuned-mass-vs-freq}
   \caption{Vibration absorber with a range of absorber frequencies. 
   When the absorber matches the resonance of the structure the vibration 
   amplitude around that frequency is greatly reduced.}
   \figlabel{tuned-mass-vs-fig}
\end{figure}

\begin{figure}
\begin{wide}
  \begin{subfigure}
    \psfragfig{\phdpath Simulations/Springs/fig/inertial-trans-delta}
    \caption{Transmissibility reduction at resonance due to a vibration 
    absorber system, versus absorber stiffness, for a range of absorber 
    damping values. Greater reductions result from \emph{lower} damping.}
    \figlabel{inertial-trans-delta}
  \end{subfigure}
  \begin{subfigure}
    \psfragfig{\phdpath Simulations/Springs/fig/rms-inertial}
    \caption{\RMS\ transmissibility of the vibration absorber system versus 
    absorber stiffness for a range of absorber damping values. Greater 
    broadband reductions result from \emph{higher} damping.}
    \figlabel{rms-inertial}
  \end{subfigure}
\end{wide}
\caption{A comparison of the effects of changing the damping ratio of the 
absorber on single-frequency and broadband transmissibility of a tuned mass 
damper.}
\end{figure}

\textcite{brennan2006} discusses a wide variety of actuators that may
be used to construct such a device:
\begin{quote}
  There is not a single ``best'' way of making an [adaptive tuned
  vibration absorber]. It depends upon the required frequency range,
  the agility (speed of reaction) and cost.
\end{quote}

The efficacy of the vibration absorber is related to the damping between it
and the structure; better results are achieved with lower damping, as shown in
\figref{rms-inertial}. The damping of the absorber can be reduced with an
active control system as shown by \textcite{kidner1998}. This can be
understood with the realisation that energy is not dissipated by the vibration
`absorber'; rather, motional energy from the vibration structure is
\emph{transferred} to the supplementary mass, and this process is degraded by
the presence of damping. This theme of the desirability of low damping
re-occurs at a later point in this thesis (crossref).

\textcite{felix2008} also recently demonstrated the use of coupling between a
cantilever under direct disturbance and a electromechanical damper with
nonlinear resistive and capacitive elements.

A paper I haven't read yet uses magnetorheological elastomer as a variable
stiffness element \cite{holdhusen2007}.

\textcite{casciati2007} uses a semi-active tuned mass damper to
control vibrations of a suspended cable; some care is required for
modal structures as their higher frequency (often nonlinear) behaviour
can pose an influence even as the targeted (low frequency) mode is
damped as desired.

Tuned vibration absorbers can also be effected artificially with a control
system \cite{kim2008a}.

Another application: \cite{moradi2008}

An extremely interesting novel device is demonstrated in the literature that
is a mechanically self-tuning vibration absorber \cite{ivers2008}. While not
as efficacious as a `true' vibration absorber, the ability to adapt to the
excitation frequency is very useful.

Tuned vibration absorbers are occasionally used to mitigate seismic vibrations
in large buildings, but their mass dependence makes their application rather
tricky and often impractical. \textcite{matta2008} propose a nonlinear
rolling structure via which a vibration absorber can be mounted to good effect
despite uncertain masses of either or both of the building and absorber. Their
work focuses on the novel idea of using a roof-top garden as a tuned mass
damper \parencite{matta2008a}.


\subsection{Fully active tuned vibration absorbers}

Some researchers have looked at using electromagnetic actuators to
provide an active force with which to cancel system resonances. The
advantage for such a system is the same as the semi-active case: with
a suitable controller, changes in the plant can be taken into account
in the vibration neutraliser. However, using a fully active system for
this task is not very energy efficient.

One example is the work by \textcite{chen2005a}, who use a linear
`voice coil motor' (crossref) as the actuator for a controller based
around a simple notch filter to remove the system resonance. Suffice
it to say that basic velocity feedback would have done a better job,
as the notch filter \emph{amplifies} vibration response in the
side-lobes of the filter.

While his work classically uses linear (\PID) control, a recent paper
by \textcite{wu2007} approaches this idea using an linear actuator
with a controller to simulate the dynamics of such a tuned absorber,
with additional dynamics to tune the resonances in the case of plant
changes. This is a good example of a poor choice of control strategy
by using empirical algorithms for a specific application rather than
applying a more general technique. \fixme{Don't be too harsh.}


\section{Multi-degrees of freedom}

This thesis primarily considers single degree of freedom systems, as
they are a suitable approximation for systems appropriate for magnetic
springs. (Well...) The vibration behaviour of complex structures is
almost another topic entirely, where multi-modal behaviour and
broadband (or wideband) attenuation are the important metrics. 

Many of the concepts discussed can be applied broadly for more complex
systems; a short way along this spectrum is the work of \textcite{febbo2008}
who applied coupled dual--vibration absorbers to suppress modal vibration of a
beam. Further examples of the application of many mass absorbers can be found
in the literature \parencite{kim2008}.

\textcite{howard-thesis} designed a system for six-axis isolation in
systems such as a cylindrical shell coupled to a vibrating mass. In
such systems, it is important to ensure best results to measure and
minimise the vibratory energy in the system, rather than the
acceleration amplitude. His six axis isolator consists of three
orthogonal pairs of shakers to each apply force and/or moment in a
single plane.

\textcite{yoshioka2001} designed and built a prototype vibration
isolation platform with a load bearing capacity about
\SI{1000}{kg}. Their device used a combination of pneumatic,
voice-coil linear motors, and piezoelectric actuators to control the
first half dozen vibration modes of the platform. Feedback velocity of
the table and feedforward velocity and displacement of the floor were
used as control signals (with integrated accelerometer
measurements). A genetic algorithm was used to generate a set of
control gains for the system, and very good results were achieved.

\subsection{Modal systems}

The issue of selecting appropriate control gains for velocity feedback control
systems was examined by \textcite{engels2008} for the cases of centralised and
decentralised control devices. In centralised control, a global model of the
system is used when allocating the feedback signals. In decentralised control,
each sensor/actuator pair operates independently to minimize the energy at the
mounting point. Generally centralised control is more difficult in practise
but can give better results. For the simple two-degree of freedom vibrating
system examined by \textcite{engels2008}, the centralised control performed
better although the differences to the decentralised control were small. An
analogous result was shown by \textcite{hoque2006} for a three-axis vibration
platform supported by so-called `infinite stiffness' magnets (see
\secref{infstiff}).





\section{Higher-level control}

Vibration isolation can be considered exclusively as a control
problem. \textcite{guo2005}, for example, used system identification
and a variety of feedforward/feedback control techqniques to isolate a
multi-degree of freedom structure in a single direction. Adaptive
control in six degrees of freedom for broadband noise has also been
shown by \textcite{duindam2005}.

\textcite{balandin1998} review the field of optimal control as applied
to shock and vibration isolation problems.\footnote{Their comment
  ``The number of papers is so great that there is little incentive to
  discuss them here'' does not bode well for any attempts by me to
  even summarise their review.} 
  They differentiate shock and vibration isolation succintly:
\begin{quote}
  The operating quality of shock isolators is usually described in
  terms of certain characteristics of the transient motion of the body
  being isolated, whereas the quality of vibration isolators is
  determined by the characteristics of steady-state forced
  oscillations.
\end{quote}
That is, optimising for shock will result in minimising, say, the peak
displacement of the mass, whereas optimising for vibration will result
in a low natural frequency (characterised by a minimum achievable
\RMS\ displacement). Since an optimal controller is based around a
cost function that will be dependent both on the vibratory system
itself and the mode of disturbance (transient, broadband noise, \etc),
such control approaches are heavily case-specific and are best used
when a plant is pre-determined and a vibration problem needs to be
considered after the fact. \textcite{bolotnik2001} discusses these
methods in more detail.

I haven't finished reading the paper by \textcite{savaresi2007}, but
it's about a combination skyhook/acceleration-feedback controller that
is really nice. It is in the context of vehicle vibration control for
ride `smoothness'; the premise is that skyhook damping knocks out
resonance peaks (without any other side-effects; \cf\ relative
velocity feedback) while acceleration ($\equiv$ `mass') feedback
decreases the overall transmissibility at higher frequencies.

Earthquake protection systems deal with a form of vibration isolation,
but focus more on transient response than continuous disturbance. The
nature of the problem is important to analyse before control
strategies are applied. \textcite{gavin2007} discuss using negative
stiffness and skyhook damping in the context of isolating machinery
in a building. (These guys also wrote the paper ``Drift-free
integrators'', which I may find useful to look into.)

Skyhook damping is considered an ideal case (I need some references
— \ie, detail — for this) and is implemented with a switching
controller in practice. This is because the abstraction of skyhook
damping is to attach a damper to the `sky' itself; that is, the added
damping is proportional to the absolute velocity of the damped
mass. In practice, however, a velocity sensor can only measure a
\emph{relative} velocity between the mass and the position of the
fixed sensor. I find it interested than observer techniques haven't
been investigated for this purpose. Perhaps I've just missed that
particular research cul-de-sac.

Incredibly enough, this paper mentions the article
\cite{guardabassi2001} as a reference for the fact that frequency
response of nonlinear systems is somewhat undefined (but I can't see
how it is a particularly relevant citation!). Here's how they compute
their nonlinear `frequency response', for which they calculate a
\emph{variance gain}:
\begin{dmath}
\hat F_{\text{acc}}(j\omega_i) = \sqrt{\left.
  {\color{gray}\tfrac{1}{T}}\Integrate{\ddot z_i(t)^2}{t,0,T}
  \middle/
  {\color{gray}\tfrac{1}{T}}\Integrate{z_{r_i}(t)^2}{t,0,T}
  \right.} 
  \condition*{i=1,2,\dots,N}
\end{dmath},
for input signals
\begin{dmath}
z_{r_i}(t) = A \Sin{\omega_i t} 
\condition*{t\in[0,T]}
\end{dmath},
and measured output signals $\ddot z_i(t)$. Note that the $1/T$ terms (above,
greyed out) haven't been cancelled as the numerator and denominator are both
expressions for the \RMS\ value of each signal over the sampled time period.

This technique is useful for quantifying the behaviour of nonlinear
systems. For measuring the frequency response function of a
predominantly linear system that is contaminated by nonlinear signals
whose effects should be discarded, other techniques are available
\cite[for example]{schoukens2001}.

\textcite{kerber2007} modelled a plate-plate coupling through four
contact points in six degrees of freedom system and applied active
vibration isolation in the vertical direction using a range of control
methods. Of those trialed, \Hinf\ control had the best low frequency
response, and \PI\ control (with anti-windup) the best high
frequency response. Output feedback, state feedback with an observer,
and velocity feedback methods all performed worse to various extents.

\textcite{chen2007} used \Hinf\ control to suppress the resonance peak at
around \SI{3}{Hz} of a pneumatic vibration isolation mount. Very good
reductions in transmissibility were achieved; this technique highlights the
appeal of vibration control using an isolator with a damped resonance.

A modern look at output feedback methods is shown by \cite{mottershead2008},
where the dynamics of complex structures is altered via robust pole-placement
techniques that do not require models of the system.

And here's a method using delayed acceleration feedback by
\textcite{chatterjee2008}.



\subsection{Adaptive observation}

In a recent paper, \textcite{madkour2007} compare a slew of nonlinear
`artifical intelligence' adaptive algorithms to observe the parameters
of a vibrating pinned beam in order to apply feedback cancelation. The
algorithms compared are traditional RLS, a genetic algorithm with a
specific inheritance method, a neural network for estimating a smooth
function, and a combination fuzzy modelling/neural network procedure
for system identification. Control based on the genetic algorithm
performed the best.

Such techniques are appropriate when system identification is required
during operation; \ie, when the plant cannot be measured beforehand or
it slowly changes over time. However, such `artifical intelligence'
algorithms are probably sub-optimal for the observation of such a
dynamics problem when the system can be broadly modelled as a set of
differential equations in the time domain, for which nonlinear
techniques such as backstepping or sliding mode control are directly
suitable. Leave the heuristic approaches to systems that cannot be
modelled.

Note that this statement only applies to the use of such methods in
the control phase of vibration control; genetic algorithms, \eg,
provide a useful technique in the design phase when limited numbers of
sensors and actuators must be placed in `optimal' positions on a
complex structure. \fixme{cite us}




\section{Infinite stiffness}
\seclabel{infstiff}

Both \textcite{nijsse2001} and \textcite{mizuno2003a} (with related
publications) talk about using a combination of a positive and
negative stiffness springs to achieve affects like
\begin{dmath*} 
  k_T = \frac{ k_1 k_2 }{ k_1 + k_2 } = \infty 
  \condition{when $ k_2 = -k_1 $.}
\end{dmath*}

\textcite{xing2005} formalises this idea in their paper, or at least claim to.

A swath of papers have been published by Mizuno over the years 
\cite{mizuno2001,mizuno2003a,mizuno2003b,mizuno2007,mizuno2002}. 
The term `infinite stiffness' is perhaps unfortunate, as it's not true in
any useful sense of the term.

The work was extended by \textcite{hoque2006} for three--degree of freedom
vibration control.

\textcite{mizuno2003c} also present a modified version of the idea, whereby
rather than using magnetic springs in attraction, as in their other work, this
paper looks at using a voice coil actuator with designed negative stiffness. A
controller must be used to obtain this negative stiffness, but it is not clear
to me, at this stage, if their system will have the same sort of problem as
I've shown in my Active 2006 paper with the cancelling dynamics of a positive
stiffness mechanism.

The papers dealing with infinite stiffness are not doing what they
claim. That is, their static behaviour converges to something that
looks like infinite stiffness, but \emph{dynamically} they're the same
as any other structure.

This is because the negative spring in combination must be
stabilised somehow, and this stabilisation removes the nice property
of summing to zero with the other stiffness in the system.

You would think that my zero stiffness spring is in the same boat.
But writing out the equations seems to refute this assertion.
Recall,
\begin{dgroup}
\begin{dmath}
m\ddot x = K(x-y)^2 + u
\end{dmath},
\begin{dmath} 
u = \{K(x-y)^2\}_{\text{est}} + k_c x + c_c \dot x 
\end{dmath},
\begin{dmath}
m\ddot x = e[x,y] + k_c x + c_c \dot x 
\end{dmath}.
\end{dgroup}
So the final vibration of the support, $x$, is independent of $y$
(the base vibration) except for a small amount coming through the
error term, $e$.

On the other hand, the dynamics of the support are defined by the
controller gains $k_c$ and $c_c$, which seems to imply that the
tradeoff between base vibration isolation and direct force
disturbance in a regular system is not present here.


\section{Negative stiffness and \qzs}
\seclabel{vibrations-qzs}

Addition of negative stiffness elements in a design reduces the resonance
frequency, which improves vibration isolation. Early examples of such designs
using inclined springs were shown by \textcite{molyneux1957}. Such systems may
tuned to achieve a local region of zero stiffness, which is often termed
`\qzs'. \textcite{alabuzhev1989} looked at the nonlinear behaviour of such
systems, and also more recently by
\textcite{carrella2006,carrella2007,carrella2008} and improved by
\textcite{kovacic2008}. More recently, the dynamic response of these systems
has been analysed \parencite{carrella2009,carrella2008thesis} and shown to
have prominent nonlinearities that distort the frequency response but that do
not decrease the vibration isolation efficacy in general.

The use of buckling beams as a negative stiffness element to achieve \qzs\ has
also been implemented in practice~\cite{platus1999,tarnai2003,lee2007}.

A brief discussion of zero stiffness elements in hand-held vibrating
machinery yields perhaps little insight into our problem
\cite{sokolov2007}, but mentions that friction is the biggest problem
once zero stiffness has actually been achieved; this problem is
obviated when non-contact supports are used.

An interesting collection of literature for very low frequency vibration
isolation resulted from various gravity wave interferometers built around the
world. Some of these used the negative stiffness of the buckling beam, in
various forms, to reduce the resonance frequency of an isolator
\cite{cella2005}.

\QZS\ can also be achieved with magnetic systems. Magnetic configurations with
negative stiffness can be used to augment a positive stiffness support (which
can be simply a conventional spring) to lower the resonance frequency. For
example, \textcite{beccaria1997} used this technique (under the term `magnetic
antisprings') to improve the isolation for gravity wave detectors.
\textcite{carrella2007a,carrella2008} has also used a attractive magnets in
parallel with conventional springs to reduce the resonance frequency of the
system. Purely non-contact magnetic systems can also be used to similar effect
\cite{robertson2006,robertson2007}. Generally, systems that use the negative
stiffness between attracting magnets cannot be brought to a stable \qzs\
region due to their `softening spring' characteristic.

\textcite{hol2006} discuss a `gravity compensator' that uses the idea of an
axial bearing with \ang{90} rotated magnetisations to bear load in the
vertical direction. (Also see \textcite{yonnet1981} for related bearings that
I hadn't thought about in enough detail yet.) This creates something like a
negative quadratic force relationship, which has zero stiffness in the centred
vertical position.

Analysis of the nonlinear dynamics of such systems such as performed by
\textcite{lee2004b,kovacic2008} can be quite involved and is outside the scope
of this research.

\subsection{\QZS\ is not zero stiffness}
\seclabel{qzs-not-zerk}

It has been established that the goal of a `zero stiffness' device is to
reduce the resonance frequency of the system to as low a value as possible. In
the limiting case, if the system is stable and the nominal force of the spring
indeed matches the weight of the mass, then the gradient of the force at the
operating point will equal zero.

However, it is necessary to use a nonlinear spring to achieve this zero
stiffness condition. And the behaviour of a nonlinear oscillator varies
considerably from that of the classic linear spring. Most obviously, the shape
of the frequency response is not independent of the amplitude of the forcing
disturbance.

Consider the stable single degree of freedom system
\begin{dmath}
m \ddot x + b \dot x + \stiffnessDuffing (x+s)^3 = 0, 
\end{dmath}
where $s$ is an induced displacement disturbance. At the operating
position $x=0$, the nonlinear spring stiffness is
$3kx^2|_{x=0}=0$. For a disturbance $s$, the spring is perturbed and
generates a reaction force of $ks^3$ on the mass. The stiffness here
is $3ks^2$; \ie, dependent on the amplitude of disturbance. The
ramifications of this nonlinear force on the vibratory response of the
system are not exactly straightforward.

\textcite{tentor2001} analysed a spring generated by repulsion magnets
which behaved as a Duffing oscillator for large amplitude
vibrations. \fixme{(The Duffing oscillator is well-known in the literature of
nonlinear vibrations. Please reference it more.)} The difference for
his system was a significant linear component in the force equation:
\begin{dmath}
F_{\text{Duffing}} = \stiffnessLinear x + \stiffnessDuffing x^3.
\end{dmath}
The nonlinear dynamics only affected the response of the system when
the nonlinear term dominated over the linear term. For a zero
stiffness spring, $\stiffnessLinear=0$ and the nonlinear dynamics are
more significant.

Rather than perform a nonlinear analysis on the system above (which is
known in the literature as a Duffing oscillator — anything else I
need to add?), it is instructive to examine the power spectra generated
with a range of spring stiffnesses and Gaussian inputs.

\Figref{cubic-resonance-disturb,cubic-resonance-stiffness} show the
square root of the ratio of the power spectrum of $s$ and $x$, where
$s$ is a white noise signal of variance $S$. (The transfer function is
not examined because it removes nonlinear components of the original
signals.) The results are compared with the system
\begin{dmath}
m \ddot x + b \dot x + k_{\text{lin}}(x+s) = 0, 
\end{dmath}
which has a linear stiffness $k_{\text{lin}}=3kS^2$ equivalent to the
stiffness of the nonlinear spring at the variance displacement.

In both nonlinear systems simultions, significant nonlinearities in the
response can be seen. Maximum frequency and amplitude of the response increase
with both greater input disturbance amplitude and greater nonlinear spring
stiffness.

\begin{figure}
  \grf{Simulations/Zero_stiffness/eps/cubic-resonance-disturb}
  \caption{Cubic stiffness response with various amplitudes of
    disturbance in comparison to some approximately similar linear
    systems.}
  \figlabel{cubic-resonance-disturb}
\end{figure}

\begin{figure}
  \grf{Simulations/Zero_stiffness/eps/cubic-resonance-stiffness}
  \caption{Cubic stiffness response with varying values of the
    stiffness coefficient (with $S=1$ to compare with
    \figref{cubic-resonance-disturb}) in comparison to some
    approximately similar linear systems.}
  \figlabel{cubic-resonance-stiffness}
\end{figure}

When compared to the linear response of stiffness equivalent to the maximum
stiffness reached by the white noise input, the linear system response is
\emph{smaller} than the \qzs\ response. This restricts the applications of
using nonlinear springs to achieve low resonance frequencies: only when it
becomes infeasible to decrease the stiffness of a conventional linear system
any further should a nonlinear system be chosen instead.




\section{Nonlinear systems}

Interesting results have been shown using nonlinear springs to attach the
vibration absorber to the structure, such as \textcite{jo2008} who use
repulsive magnetic springs to produce a tuned absorber with a resonance at
double the frequency of the main resonance of the structure. \fixme{Why would
you do this?!}

A similar idea with a different approach is taken by
\textcite{liu2006a}, who used a permanent magnet and an electromagnet
to provide a variable stiffness in the absorber.

\textcite{zhang2008} use a nonlinear damper (specifically of the form viscous
plus cubic with velocity; \ie, $a_1\dot x\fn{t}+a_3\ddot x^3\fn{t}$) to excite
the structure at harmonic frequencies of the resonance. This results in less
energy at the frequency of vibration, although the resonance peak does remain.
While it does not seem likely that this method can compete with the reductions
seen with the approach of a tuned vibration absorber, this nonlinear damping
method does have the advantage that it does not require tuning for a
particular frequency and its effectiveness will not change with a time-varying
resonance frequency.

A very recent and comprehensive review of nonlinear vibration
isolation systems for a broad range of techniques was published by
\textcite{ibrahim2008}. It highlights the importance of nonlinear
analysis in this field: in some cases, better results can achieved
by using nonlinear spring forces to couple to and absorb vibration
energy (\fixme{eg}); in other cases, the behaviour of an isolator
cannot be adequately modelled by using linear systems theory.

Various combinations of magnetic springs create different classes of
approximate nonlinear oscillators, for which standard methods exist to analyse
their behaviour \fixme{can I cite a book here I haven't read?}. The
characteristic property of nonlinear springs is their exhibition of so-called
`jump phenomena' whereby the frequency response curve can take multiple values
at a single frequency, depending on the initial conditions.

\textcite{2006} analysed the behaviour of a nonlinear isolation mount in
detail, developing analytical models for the jump phenomena of a system with
cubic stiffness and quadratic damping. Critical values were illustrated to
avoid the ill effects of the nonlinearities; additional damping had the
general effect of decreasing the adverse nonlinear response.

For the purpose of vibration control, augmenting a linear spring
system with nonlinear magnetic springs alters the behaviour of the
natural frequency of the system to be weakly coupled to the mass of
the system. \textcite{dangola2006} analysed the nonlinear dynamics of
the nonlinear system approximated by (in non-dimensional form)
\begin{dmath}
  \eqlabel{dangelo}
  \D[2]{u}{\tau} + c_s\D{u}{\tau} 
  + \underbrace{(1+\alpha_1)u  + \alpha_2 u^2 
    + \alpha_3 u^3}_{\text{nonlinear spring}}
  = \delta\Cos{\rho\tau} .
\end{dmath}
For this system, a variation of both stiffness and mass by up to 50\%
yields an increase of stiffness of 6\%. For an equivalent linear
system, the natural frequency variation is ten times greater. This is
a very interesting result for loading elements for which the mass to
be supported is largely variable, in that the frequency response will
vary significantly less than for conventional linear springs. 

However, for weakly nonlinear magnetic springs, variation in the mass will
lead to changes in the resonance frequency. \textcite{todaka2001} created a
mechanical linkage to support two magnets in repulsion such that as their air
gap increased (due to less mass being supported), a horizontal offset between
them was created to lower the linearised operating stiffness. This allowed a
much smaller variation between the $k/m$ ratio than for flush magnets in
repulsion.

\textcite{bonisoli2007} used an experimental rig to analyse the nonlinear
behaviour of a magnetic and linear spring in parallel, with more theoretical
analysis published later in the year \cite{bonisoli2007b}. They showed a
configuration of linear and magnetic springs with the notable feature that the
resonance frequency exhibits little dependence on mass loading and nonlinear
effects can be seen. This effect was further demonstrated in parallel research
by the same authors.

The field of nonlinear dynamics is very large, and applications to vibration
suppression can result in surprising results. \textcite{oueini1999} considered
the response of a nonlinear (including cubic) plant with an additional cubic
nonlinear feedback law and established that vibration attenuation was possible
(like a tuned mass damped?) and that nonlinear phenomena such as chaos
existed. Presumably, such phenomena are undesirable for vibration isolation!

\textcite{zhao2007} examined the vibratory behaviour of a two degree of
freedom nonlinear system with a time-delayed positional feedback. This results
of this paper emphasises the importance of accounting for time delays in
feedback control systems.

A paper came through in the Transactions of the ASME by \textcite{chen2006}:
``Residual vibration suppression for Duffing nonlinear systems with
elecromagnetical actuation using nonlinear command shaping techniques''. It's
interesting but irrelevant to my research. They talk about this command
shaping technique, which in linear systems is something along the lines of
providing a tracking command in a series of steps, rather than a single one,
in order to control how the ringing is produced. Two steps half a period apart
will create ringing that deconstructively interferes, for example; a three
step process works even better, apparently. This paper looks at an actuator
with an electromagnet in parallel with a nonlinear spring and designs an
analogous nonlinear command shaping for it. Since I'm not looking at tracking
in this stage of the project (probably not ever) the paper is probably only of
academic interest.

\fixme{put this somewhere up there:} \textcite{starosvetsky2008}, with a good
review of the literature, talk about the `nonlinear energy sink' idea in which
a nonlinear system provides better and broader vibration absorbing.

\cite{zuo2004}

Anyway, it is an interested argument to make that digital control in a
vibration isolation situation is inappropriate. When absence of all
vibrations are required, the side-effects of digital control (\ie,
chaotic effects due to quantisation and time delays
\cite{csernak2007}) may prove deleterious for extreme applications. On
the other hand, any sufficiently expensive system should be able to
reduce these effects to be arbitrarily negligible. For the purposes of
this thesis, such small effects are of no concern.

(I'd be interested to see if FPGA or similar flexibly-configurable
analogue controllers  could be used instead.)

Nonlinear control is better than linear control, as argued cogently by
\textcite{kokotovic1992}. \textcite{queiroz2007} used nonlinear
control to stabilise a magnetic bearing with pull--pull electromagnets
with parameter uncertainties, while also minimising power consumption
of the system.


\textcite{agamennoni2004} were able to identify the nonlinear dynamics
of a `zero power suspension'.


\subsection{Analysing nonlinear systems in the frequency domain}

Who knows what the relationship is between a transfer function for a
linear system and the same maths applied to the output of a nonlinear
system? The previous section highlighted a problem here.

There are generalisations of the frequency response function for a
class of nonlinear systems \cite{lang2007}. I'm not sure how
much effort it would be to look into it for the systems I'm looking at.

Here's another approach: \cite{peng2008}

The problem with nonlinear analysis is that different techniques focus on
examining different behaviour in the response. For example,
\textcite{peng2008a} compare two methods, one of which captures `jump
phenomena', and the other which shows superharmonic behaviour. But there is
not one general method for classes of nonlinear system, nor perhaps can one
exist.


\section{Energy harvesting}
\seclabel{energy-harvesting}

When electrical circuits are used to absorb resonant vibrations, the energy
absorbed can be redirected to produce a power output \cite{stephen2006}. Such
devices are gaining popularity for ambient vibration--powered applications
such as remote sensing \cite{arnold2007}. Another field of interest for
regenerative damping is in vehicle suspensions, in which useable power can be
extracted with the same mechanism as providing greater ride comfort
\cite{graves2000thesis}.

A perfect energy harvesting devices would manage to absorb all of the kinetic
energy of the disturbances to generate its power. While this is impossible in
practise, the vibration absorption properties of such a device warrant its
inclusion in this literature review.

\textcite{challa2008} recently investigated a
semi-active device for tunable energy harvesting that used
variable-displacement attractive and repulsive magnets to adjust the
resonance frequency of a piezoelectric cantilever. This is the same
mechanism that is examined in this thesis for `\qzs' suspensions
(crossref).

Energy generation can also be accomplished through electromagnetic damping
through induced eddy currents in a coil experiencing relative displacement to
a magnetic field, such as investigated by \textcite{graves2000}.
\Secref{damping} discusses the damping possibilities of electromagnetic
systems for the purposes of vibration isolation devices.

\textcite{stephen2006}
talks about energy harvesting with micro-actuators, and the analysis
is pretty thorough. Furthermore, it all makes sense, which is quite
nice. There is a moderate amount of single degree of freedom
mass-spring-damper maths for future reference, which is coupled with a
simple electrical circuit. Not surprisingly (it's easy to say this
when simply reporting their results) they want energy dissipated
primarily by the electric components, not the mechanical damping, since
in the former the energy is retrievable whereas in the latter it's simply
dissipated as heat.


\textcite{mann2008} performed a preliminary investigation into the use of a
nonlinear vibration mount for energy harvesting, using a magnetic suspension
of repulsive magnets to create a Duffing-like oscillator. Large damping
ensured that the nonlinear regimes were only realised at large excitation
amplitudes, but the idea is that highly nonlinear resonances have a much
broader resonance peak through the higher branch around the jump phenomenon.
\fixme{reference this elsewhere}

A similar idea is explored by \textcite{shahruz2008} for an energy scavenging
cantilever beam that uses an arrangement of attracting magnets to shape the
force characteristic of the response. The aim is to achieve a power spectrum
of the response to a random excitation that is greater than the predominantly
linear response that is obtained without the magnets present. No analysis of 
damping or energy harvesting potential was explored in this paper.

    
