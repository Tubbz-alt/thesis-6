%!TEX root = thesis.tex
\chapter{Magnetic and electromagnetic forces}
\chaplabel{magnet-theory}

\epigraph{With an eye to the practical importance of levitation we
feel justified here in disregarding those aspects of it
associated with magic, spiritualism, and psychic
phenomena\dots}{\textcite{boerdijk1956b}}

\chapterprecis{
  This is the most analytical of the theory/literature sections, since it is
  here that I present and abstract a body of work that deals with `forces
  between magnets', also including some quick mentions of magnetic field
  theory and some associated concepts (such as eddy current damping).
}


\section{Forces}

In this chapter I discuss forces generated by electromagnetic
systems. Originally, I used \FEA\ techniques, but these proved time
consuming and crude. Integration is a better idea.

[\textcite{won2005}] Uses a \twoD/ \FEM\ to calculate the torque and
vibration of a motor in a mobile phone.

For fast magnetic field computation, the `fast multipole method' can
be used \cite{adedoyin2007}. But I haven't used it.

Magnetic fields can induce forces through a variety of
methods. Electromagnetic actuators often use Lorentz forces
\cite{hollis1993}.

\section{Magnetic fields}

It's instructive but misleading to think about magnetic fields
`squashing' as they come in proximity to each other. Just ask
\textcite{sodano2006} who received comments nitpicking comments about
their terminology \cite{marneffe2007}.

\section{Sources of the magnetic field}
\seclabel{bhm}

Magnetic fields are created by moving electrons. A long straight wire
will create cylindrical magnet fields, and a small loop will create a
magnetic dipole. Thus, an electron orbiting a proton is the smallest
magnetic element.  This is a hydrogen atom. In nature, however,
hydrogen exists as H$_2$, two protons orbited by two electrons---and
it happens that the two electrons orbit in opposite directions and the
magnetic fields of each cancel each other out.  Most material is like
this: basically, not magnetic.

The following is a very short introduction to the physics behind
magnets. A good reference book which inspired this section is
\textcite{campbell1994}.

The \define{magnetic dipole} is designated as the microscopic quantity
$\mdip = i\vect{A}$, for a current $i$ and a vector area $\vect{A}$
(direction normal to plane). For a collection of magnetic dipoles (as
in a permanent magnet), their net effect may be quantified with the
macroscopic \emph{magnetisation} of the material, $\magM$:
\begin{dmath}
  \magM =  \lim_{\Delta V \rightarrow 0} \frac{\sum \mdip}{\Delta V}.  \eqlabel{M}
\end{dmath}

The magnetisation of a permanent magnet creates the magnetic fields
that are of such great interest. \emph{Inside} the magnet (with no
other external fields present), the magnetic field, $\magB$, is given by
the simple relation
\begin{dmath}[label=BM]
  \magB = \permVac \magM
\end{dmath},
where $\permVac$ is the `permeability of the vacuum', an essentially
meaningless name given to the necessary constant of proportionality.

So, macroscopic magnetic fields have been derived from microscopic
currents. It is possible to take this full circle and now create
macroscopic current terms from the magnetic field. The equivalent
current density, $\vect{J}_m$ around a permanent magnet, created by
aligned microscopic orbiting electrons is given by:
\begin{dmath}
  \vect{J}_m = \curl\magM.
\end{dmath}
This is a good beginning for describing the effects of an \emph{external}
current density ($\vect{J}$) acting on the magnet. To separate the effects of
induced magnetisation and that caused spontaneously by magnetic material, a
new term is created: the magnetic field strength, $H$:
\begin{dmath}
  \vect{J} = \curl\magH.
\end{dmath}
Now the earlier \eqref{BM} can be adjusted to allow for both
internal and external forms of magnetisation (\ie, magnetic field
caused by permanent magnets or by current carrying conductors). This
is the fundamental equation relating the three important terms in
magnetics,
\begin{dmath}
  \magB = \permVac (\magM + \magH)  \eqlabel{BHM}
\end{dmath},
allowing the terms to be unambiguously defined. Henceforth, $\magB$ is
called the \define{magnetic flux density}, which a measure of the
total flux per area; and $\magH$ is the \define{magnetic field strength},
the effect of external current sources that creates the magnetic
field.
\note{The names of these terms are not always consistent in the
  literature. $\magM$ is also known as polarisation, and $\magB$ and $\magH$
  are both sometimes known as the magnetic field.}

Finally, the \define{permeability} $\perm$ of a material (generally
only used when the material is not a permanent magnet) is a varying
term describing the ratio between the magnitudes of the $\magB$ and $\magH$
fields. The \define{relative permeability} $\permrel$ is the ratio of
the permeability to $\permVac$.
\begin{dgroup}
  \begin{dmath}
    \perm = \frac{B}{H}
  \end{dmath},
  \begin{dmath}
    \permrel = \frac{\perm}{\permVac}
  \end{dmath}.
\end{dgroup}
\Eqref{BHM}, may now be used to describe the situation at all
points in space. Use \figref{BHM} to note that while inside the
magnet, the magnetic field is the vector sum of two components,
whereas outside the magnet, the magnetisation is zero and the magnetic
field is related to the magnetic field strength by a constant. This
results in $\magB$ being continuous everywhere, and both $\magM$~and $\magH$
being discontinuous.

\begin{figure}[htbp]
   \centering
   \grf{Figures/Theory/BHM}
   \caption{The magnetic field, $\magB$, both inside and outside a magnet.}
   \figlabel{BHM}
\end{figure}

This equivalence in air is essentially the reason that there is often
confusion between $\magB$ and $\magH$. It can be seen that within a magnet,
however, their relationship is more complex and important. The
performance of a magnet is shown by its \bhcurve/, which is shown for
an ideal magnet in \figref{BHcurve}. This curve demonstrates the
nonlinear and hysteresic effects of the magnetic flux density of a
magnetic material as external magnetic field is applied to it.

Two important features are shown in the \bhcurve/. First, the
\define{remanence} of the magnet, $B_r$. This value is equal to
$\permVac\Msat$ and occurs when there is no external
magnetic field. The other is the \define{coercive force}, $H_c$, which
is the amount of magnetic field strength required to reduce the flux
density of the magnet to zero.

\begin{figure}[htbp]
   \centering
   \grf{Figures/Theory/BHcurve}
   \caption{The characteristic $B$ \vs\ $H$ curves for an ideal rare-earth magnet.}
   \figlabel{BHcurve}
\end{figure}

When the magnet is applying energy, it is operating in the second
quadrant of its \bhcurve/, \ie, the section of the curve between the
point of remanence and the coercive force, known as the
\define{demagnetisation curve}. This is because the energy used for
the demagnetisation is being taken by the causes of the
demagnetisation. At some point in this second quadrant, $-\magB\times\magH$
will have a maximum. Since this is proportional to the energy
potential, this is known as the \define{maximum energy product}
$\BHmax$, and can be shown to be equal in an ideal magnet to:
\begin{dmath}
  -\BHmax = \frac{\Msat}{2}
\end{dmath}
The potential energy of the magnet is thus directly related to its
magnetisation.

And what do you know. Just a little while ago, while reading
\cite{furlani2001}, I wondered what was the constitutive relation
with rare earth magnets (the book wasn't entirely clear, unless I
missed that part). So here we are: due to \textcite{nagaraj1988}, for
high coercivity permanent magnets,
\begin{dmath*}
  \magB = \permVac\magH + \magM ,
\end{dmath*}
assuming that the air approximates a vacuum (magnetically) and the
relative permeability of the magnet is unity. The rest of the paper
goes on to derive, in familiar ways, integrals for the forces between
magnets of cuboïd and cylindrical shapes. (This latter will be a
useful reference later on.)


\section{Properties of magnetic flux}
\seclabel{flux}

The previous section introduced $\magB$, the magnetic flux density. An
analysis of how to derive the paths of magnetic flux is a little
beyond the scope of this document, but it is important to discuss the
flux lines themselves.

`Magnetic flux' derives its name from archaic models of magnetism,
whose proponents believed in the literal flow of a magnetic fluid
called the `luminiferous æther'. Nowadays, scientists tend toward more
modern interpretations using electromagnetic fields involving quantum
theory. Nonetheless, the name sticks. Magnetic flux, $\flux$, is
therefore defined as the amount of `fluid' passing through an area:
\begin{dmath}
  \flux = \magB \cdot A
\end{dmath}
This flux is almost analogous to electric current; the only difference
being that electric current is constrained by the conductor it is
flowing through, whereas while magnetic flux is known to \emph{prefer}
areas of greater permeability, it occasionally can deviate from these
simple paths

It is more instructive for a basic understanding of how magnets behave
to look at the ways their flux lines interact. The following `magnet
design axioms' are adapted from \textcite{moskowitz1995}, whose book
covers permanent magnet design for a wide range of uses.

\begin{enumerate}
\item Flux lines follow the path of least resistance. This means that they will
travel through the shortest path possible,
through the material with the
\emph{greatest} permeability---so they will travel more readily through
magnetic or ferrous material than air, and more readily through air
(although only slightly) than diamagnetic material.

\item Flux lines travelling in the same direction repel each other. This means
flux lines will never cross.

\item Flux lines enter ferrous material at right angles.

\item Permeability of ferrous material is `used up' by flowing flux; when the
material reaches saturation, flux lines travel as easy though air as through
the saturated material.

\item Flux lines travel from North to South poles in closed loops.

\item Magnets are made up of a very large number of unit poles.
\end{enumerate}

From these axioms, one can generate incorrect, yet applicable,
theories how and why magnets attract and repel each other. For
example, two magnets in repulsion have flux lines opposing each
other. It can be imagined that the reason forces occur between them is
due to a `squashing' of the flux lines which the magnets try to
oppose---but theories like this only help visualising magnetic
behaviour, \emph{not} for explaining the reasons behind it.

\section{Magnetic materials}

There are several materials from which permanent magnets can be
made. Short attention will be placed on the cheaper, legacy magnetic
materials such as the ferrite magnets and alnico magnets due to their
poor performance. Rare-earth neodymium magnets are now quite
commonplace and rather cheap, and have much more desirable properties
than these old fashioned magnets.

\Tabref{magnets} shows some approximate ranges comparing the
properties of the various magnet types available. Clearly, rare earth
magnets are capable of much greater energy output, and their high
coercivity precludes them from losing their magnetisation through
physical impact or proximity with other magnets---unlike the older
ferrite and alnico magnets. Their only disadvantage is their low
operating temperatures that perhaps will be inconvenient using them
for biased, or hybrid, electromagnets, in which a current carrying
coil is wrapped around a permanent magnet to increase the output of
magnetic field.

\begin{table}
  \centering
  \begin{tabular}{@{} l r@{\,--\,}l r@{\,--\,}l r@{\,--\,}l @{}}
    \toprule
    & \multicolumn{6}{c}{Magnet type}\\
    \cmidrule{2-6}
    Property            & \multicolumn{2}{c}{Ferrite}
                        & \multicolumn{2}{c}{Alnico}
                        & \multicolumn{2}{c}{Neodymium}  \\
    \midrule
    Max.\ temperature (°C)    & \num{400} & \num{500} & \num{800} & \num{900} &    \num{ 80} & \num{200}  \\
    Remanence (T)             & \num{0.2} & \num{0.4} & \num{0.5} & \num{1.3} &    \num{  1} & \num{1.3}  \\
    Coercivity (\si{kA/m})    & \num{100} & \num{200} & \num{50 } & \num{160} & ~~~\num{800} & \num{900}  \\
    Max.\ energy product
               (\si{kJ/m^3})  & \num{6}   & \num{33}  & \num{10}  & \num{80}  &    \num{200} & \num{300}  \\
    \bottomrule
  \end{tabular}
  \caption[Typical values for various permanent magnets.]
  {Typical values for various permanent magnets.
   Adapted from information from \url{http://www.magtech.com.hk/}.}
  \tablabel{magnets}
\end{table}


\section{Forces between magnets}


\textcite{bassani2006}

A general technique for finding the forces between two magnets is
simple to describe. The first magnet creates a magnetic field in the
region of the second magnet; the force is calculated due to the
interaction of the first magnet's field and the internal field of the
second magnet. Or vice versa; reciprocity holds here.

The magnetic field due to a magnet can be calculated by modelling the
magnet either as a body containing a current density or containing
some magnetic charge. Rare-earth magnets have the convenient property
of high coercivity that prevents demagnetisation; hence, assumptions
of a fixed and volume-constant magnetisation hold very well.

In the first step, the integration takes place over the region of the
first magnet:
\begin{dmath}
\magB_1\fn{\pos_2} =
  \magconst\Int{
    \gp{-\Div\magM_1}
    \frac{\pos_2-\pos_1}{\Abs{\pos_2-\pos_1}^3}}{v_1,V_1}
+ \magconst\oint\limits_{S_1}
    \gp{\magM_1\dotprod\normn}
    \frac{\pos_2-\pos_1}{\Abs{\pos_2-\pos_1}^3}
    \dee s_1
\end{dmath}.

In the second step, the integration of the function of the magnetic
field of the first magnet takes place over the region of the second
magnet:
\begin{dmath}
\force =
  \int\limits_{V_2}
  \gp{-\Div\magM_2}
  \magB_1\fn{\pos_2} \dee v_2
+ \oint\limits_{S_2}
  \gp{\magM_2\dotprod\normn}
  \magB_1\fn{\pos_2} \dee s_2
\end{dmath}.

Torque is similar.
\begin{dmath}
\torque =
  \int\limits_{V_2}
  \gp{-\Div\magM_2}
  \gp{  \lever\crossprod\magB_1\fn{\pos_2} } \dee v_2
+ \oint\limits_{S_2}
  \gp{  \magM_2\dotprod\normn       }
  \gp{  \lever\crossprod\magB_1\fn{\pos_2} } \dee s_2
\end{dmath}.

The volume integrals contain $\Div\magM$ terms, which will equal zero
in the regions of constant magnetisation within the regions of the
magnetic volumes. This is a reasonable assumption for modern rare
earth magnetic material. The equations above can therefore be
simplified.

\begin{dmath}
\magB_1\fn{\pos_2} =
  \magconst\oint\limits_{S_1}
    \gp{  \magM_1\dotprod\normn  }
    \frac{\pos_2-\pos_1}{\Abs{\pos_2-\pos_1}^3}
    \dee s_1
\end{dmath},
\begin{dmath}
\force =
  \oint\limits_{S_2}
  \gp{  \magM_2\dotprod\normn  }
  \magB_1\fn{\pos_2} \dee s_2
= \magconst
  \oint\limits_{S_2}
  \oint\limits_{S_1}
    \gp{  \magM_1\dotprod\normn_{s_1}  }
    \gp{  \magM_2\dotprod\normn_{s_2}  }
    \frac{\pos_2-\pos_1}{\Abs{\pos_2-\pos_1}^3}
    \dee s_1
  \dee s_2
\end{dmath},
\begin{dmath}
\torque =
  \oint\limits_{S_2}
    \gp{  \magM_2\dotprod\normn       }
    \gp{  \lever\crossprod\magB_1\fn{\pos_2} }
  \dee s_2
 =
  \magconst
  \oint\limits_{S_2}
    \gp{  \magM_2 \dotprod \normn_{s_2} }
    \gp{
      \lever\crossprod
      \gp{
        \oint\limits_{S_1}
          \gp{  \magM_1\dotprod\normn_{s_1}  }
          \frac{\pos_2-\pos_1}{\Abs{\pos_2-\pos_1}^3}
        \dee s_1
      }
    }
  \dee s_2
\end{dmath}.

\subsection{Analytical force expressions}

Expressions for the forces between magnets follow directly from
Maxwell's equations, and may be derived in a number of ways. For
cuboid-shaped magnets with magnetisations in the $z$-direction that
experience translation in three degrees of freedom but no rotation,
a relatively concise analytical solution for the forces in each
direction, $\vect F = \vect{F}{x,y,z}$, has been known for over
twenty years \cite{akoun1984}.

\subsubsection{(Anti-)parallel magnets}

A variety of analytical solutions have been developed to calculate the
force between cuboid-shaped magnets with parallel/anti-parallel
magnetisations \cite{akoun1984,nagaraj1988,bonisoli2006}. More complex
geometries can be realised through superposition of the solutions
\cite{bancel1999}. \citeauthor{akoun1984}'s expression is
\begin{dmath}[label=akoun]
\vect F = \frac{JJ'}{4\pi\permVac}
  \sum_{(i,j,k,l,p,q)\in\{0,1\}^6}
  \vect f\fn{u_{ij},v_{kl},w_{pq},r}
\end{dmath},
\begin{dgroup}
\begin{dmath}
f_x = \half\gp{v^2-w^2}\Log{r-u}+uv\Log{r-v}+vw\ArcTan{\frac{uv}{rw}}+\half ru
\end{dmath},
\begin{dmath}
f_y = \half\gp{u^2-w^2}\Log{r-v}+uv\Log{r-u}+uw\ArcTan{\frac{uv}{rw}}+\half rv
\end{dmath},
\begin{dmath}
f_z = -uw\Log{r-u}-vw\Log{r-v}+uv\ArcTan{\tfrac{uv}{rw}}-rw
\end{dmath},
\end{dgroup}
where
\begin{dgroup}
\begin{dmath}
u_{ij} = \alpha + \gp{-1}^j A - \gp{-1}^i a
\end{dmath},
\begin{dmath}
v_{kl} = \beta + \gp{-1}^l B - \gp{-1}^k b
\end{dmath},
\begin{dmath}
w_{pq} = \alpha + \gp{-1}^q C - \gp{-1}^p c
\end{dmath},
\begin{dmath}
r = \sqrt{u_{ij}^2+v_{kl}^2+w_{pq}^2}
\end{dmath},
\end{dgroup}
where $\cuboida{1,2,3}$ and $\cuboidb{1,2,3}$ are the half dimensions of
the fixed and floating magnets, respectively, and
$\cuboidoffset{1,2,3}$ is the distance between their centres;
$J$ and $J'$ are the magnetisations, and $\permVac$ is the permeability
of free space.

And here are the stiffnesses:
\begin{dgroup}
\begin{dmath}
k_x = -\frac{v u^2}{u^2+w^2}-r-v \Log{r-v}
\end{dmath},
\begin{dmath}
k_y = -\frac{u v^2}{v^2+w^2}-r-u \Log{r-u}
\end{dmath},
\begin{dmath}
k_z = \frac{v w^2}{u^2+w^2}
  + \frac{u w^2}{v^2+w^2}
  + 2r+u\Log{r-u}+v \Log{r-v}
\end{dmath}.
\end{dgroup}
These are substituted similarly to the force equations:
\begin{dmath}[label=akounk]
\vect K = \frac{JJ'}{4\pi\permVac} \sum_{(i,j,k,l,p,q)\in\{0,1\}^6} \vect k\fn{u_{ij},v_{kl},w_{pq},r} .
\end{dmath}

Note that the result $K_x+K_y+K_z=0$ follows from Earnshaw's theorem
\cite{earnshaw1842} following from the solution to Laplace's equation.

\section{Simplified force and stiffness expression for cube magnets}

The function $\nforce$ is the simplication of \eqref{akoun}
formula after the assumptions given in \secref{mag}, where $J_1$ and
$J_2$ are the magnetisations of the two magnets and
$\permVac=4\pi\times10^{-7}$ is the `permeability of the vacuum':
\begin{dmath}[label=nforce]
  \nforce = \frac{J_1J_2}{\pi\permVac} \baraccent{\nforce}
\end{dmath}
where
\begin{dmath}
  \baraccent{\nforce} = \gp{-2+\ndisp}\cdot\Abs{-2+\ndisp}-2 \ndisp \Abs{\ndisp}+\gp{2+\ndisp}\cdot
  \Abs{2+\ndisp}+4 \ndisp \sqrt{4+\ndisp^2}
  -2 \ndisp \sqrt{8+\ndisp^2}+\gp{4-2 \ndisp} \sqrt{8-4
    \ndisp+\ndisp^2}+\gp{-2+\ndisp} \sqrt{12-4 \ndisp+\ndisp^2}
  +\gp{-4-2 \ndisp} \sqrt{8+4 \ndisp+\ndisp^2}+\gp{2+\ndisp}
  \sqrt{12+4 \ndisp+\ndisp^2}
  +2 \left[ 4 \mathatan\left[\frac{4}{\ndisp \sqrt{8+\ndisp^2}}\right]+2
    \mathatan\left[\frac{4}{\gp{2-\ndisp}
        \sqrt{12-4 \ndisp+\ndisp^2}}\right]\right.
  -2 \mathatan\left[\frac{4}{\gp{2+\ndisp} \sqrt{12+4 \ndisp+\ndisp^2}}\right]+2 \ndisp
  \mathlog\left[-2+\sqrt{4+\ndisp^2}\right]
  -2 \ndisp \mathlog\left[2+\sqrt{4+\ndisp^2}\right]-2 \ndisp
  \mathlog\left[-2+\sqrt{8+\ndisp^2}\right]
  +2 \ndisp \mathlog\left[2+\sqrt{8+\ndisp^2}\right]+2
  \mathlog\left[-2+\sqrt{8-4
      \ndisp+\ndisp^2}\right]
  -\ndisp \mathlog\left[-2+\sqrt{8-4 \ndisp+\ndisp^2}\right]-2
  \mathlog\left[2+\sqrt{8-4 \ndisp+\ndisp^2}\right]
  +\ndisp \mathlog\left[2+\sqrt{8-4 \ndisp+\ndisp^2}\right]-2
  \mathlog\left[-2+\sqrt{12-4 \ndisp+\ndisp^2}\right]
  +\ndisp \mathlog\left[-2+\sqrt{12-4 \ndisp+\ndisp^2}\right]+2
  \mathlog\left[2+\sqrt{12-4 \ndisp+\ndisp^2}\right]
  -\ndisp \mathlog\left[2+\sqrt{12-4 \ndisp+\ndisp^2}\right]-2
  \mathlog\left[-2+\sqrt{8+4 \ndisp+\ndisp^2}\right]
  -\ndisp \mathlog\left[-2+\sqrt{8+4 \ndisp+\ndisp^2}\right]+2
  \mathlog\left[2+\sqrt{8+4 \ndisp+\ndisp^2}\right]
  +\ndisp \mathlog\left[2+\sqrt{8+4 \ndisp+\ndisp^2}\right]+2
  \mathlog\left[-2+\sqrt{12+4 \ndisp+\ndisp^2}\right]
  +\ndisp \mathlog\left[-2+\sqrt{12+4 \ndisp+\ndisp^2}\right]-2
  \mathlog\left[2+\sqrt{12+4 \ndisp+\ndisp^2}\right]
  \left.  -\ndisp \mathlog\left[2+\sqrt{12+4 \ndisp+\ndisp^2}\right]\right]
\end{dmath}

The normalised stiffness $\nstiffness$ is calculated by differentiating the force equations given by \textcite{akoun1984} before simplifying as with the force terms above:
\begin{dmath}[label=nstiffness]
  \nstiffness = -\frac{2J_1J_2}{\pi\permVac} \baraccent{\nstiffness}
\end{dmath}
where
\begin{dmath}
  \baraccent{\nstiffness} = \Abs{-2+\ndisp}-2 \Abs{\ndisp}+\Abs{2+\ndisp}+4
  \sqrt{4+\ndisp^2}-2\sqrt{8+\ndisp^2}
  -2 \sqrt{8-4 \ndisp+\ndisp^2}+\sqrt{12-4 \ndisp+\ndisp^2}-2 \sqrt{8+4\ndisp+\ndisp^2}
  +\sqrt{12+4 \ndisp+\ndisp^2}+2 \mathlog\left[-2+\sqrt{4+\ndisp^2}\right]
  -2\mathlog\left[2+\sqrt{4+\ndisp^2}\right]
  -2\mathlog\left[-2+\sqrt{8+\ndisp^2}\right]
  +2\mathlog\left[2+\sqrt{8+\ndisp^2}\right]
  -\mathlog\left[-2+\sqrt{8-4 \ndisp+\ndisp^2}\right]
  +\mathlog\left[2+\sqrt{8-4\ndisp+\ndisp^2}\right]
  +\mathlog\left[-2+\sqrt{12-4\ndisp+\ndisp^2}\right]
  -\mathlog\left[2+\sqrt{12-4\ndisp+\ndisp^2}\right]
  -\mathlog\left[-2+\sqrt{8+4 \ndisp+\ndisp^2}\right]
  +\mathlog\left[2+\sqrt{8+4\ndisp+\ndisp^2}\right]
  +\mathlog\left[-2+\sqrt{12+4\ndisp+\ndisp^2}\right]
  -\mathlog\left[2+\sqrt{12+4 \ndisp+\ndisp^2}\right]
\end{dmath}

\subsubsection{Orthogonal magnets}

Much more recently, \textcite{janssen2009-sensorletters} have shown analytical expressions to calculate the force between two magnets with orthogonal magnetisations.

\fixme{include}

\subsection{Other approaches}

\fixme{file}
\noindent Say $f$ integrates to $F$:
\begin{dmath*}
\iiint f(x,y,z)\dee{x}\dee{y}\dee{z} = F(x,y,z),
\end{dmath*}
Is it possible to express the definite integral as something along the lines of the following?
\begin{dmath*}
\iiint f(x,y,z) \dee{x}\dee{y}\dee{z} =
  \mathop{\sum\sum\sum}\limits_{i,j,k\in\{1,2\}} F(x_i,y_j,z_k)\cdot(-1)^{i+j+k}
\end{dmath*}
To my feeble brain, it seems likely.
If so, why haven't I seen anything like this before?

To account for magnets that do not have parallel/anti-parallel
magnetisation, \textcite{chen2002,chen2003} derived an expression in two
dimensions for the force induced between two L-shaped current
sheets, then applies this model with the length
of various legs of the `L' in different configurations set to zero. This
takes the assumption, like \textcite{yonnet1981}, of infinitely long
planar cross sections.
This was suitable for application to a magnetic bearing where
the curvature of the magnet has a small or negligible effect on the
force between the magnets. However, for true three dimensional cases
(such as the forces between two cube magnets) this technique would not
be suitable.

Rotation of a magnet around an axis is treated by Eliès, Charpentier and
Lemarquand in numerous papers
\cite{elies1998,charpentier1999,charpentier1999a,elies1999a}, giving equations
for the forces in the directions other than around which the magnet is
rotated. Since the equations they present are not self-consistent (some typing
mistakes have occurred in the publications of the papers), the correct form is
printed in \secref{french-equations} in an easier to read, abbreviated form.

\textcite{nagaraj1988} derived integrals using the current method
\fixme{crossref} to calculate the forces between cuboid magnets and
between cylindrical magnets, and compared the differences in
behaviour.

More complex behaviours, however, cannot be solved analytically as the
integrals become untractable; for anything approaching the forces
between two magnets with arbitrary rotations, \emph{numerical}
integration must be used \cite{elies1999a}. This technique is capable
of obtaining results in a much more straightforward manner than finite
element analysis, and must be used for the more complex situations.

\subsection{Abbreviated equations}
\seclabel{french-equations}

In 1999, a slew of papers were published by researchers at \emph{Laboratoire
d'Electrotechnique et de Magnétisme de Brest} investigating the forces between
non-contact magnetic rotational force couplers. These are of interest to this
work because they use an analytical expression for the forces between two
cuboidal magnets under arbitrary translation, with one inclined at any angle
around the \x-axis.

This article talks about the forces between two cuboid magnets at arbitrary
angles to each other, and shows the integral used to solve the problem.
Numerical techniques are used, however, so I'm only interested in their pretty
picture and the integral itself.

Incidentally, there's nothing special about this particular integral; it's the
same one used by Bancel in his Magnetic Nodes paper. You kind of need a
picture, and I don't like the notation, but:
\begin{dmath}
  \magB_1 =
    \iint_{S_{1+}}
      \frac{J}{4\pi}
      \frac{\vect{P}_{S_{1+}}\vect{P}}
           {\overline{P_{S_{1+}}P}^3}
    \dee{S_{1+}}
    -
    \iint_{S_{1-}}
      \frac{J}{4\pi}
      \frac{\vect{P}_{S_{1-}}\vect{P}}
           {\overline{P_{S_{1-}}P}^3}
    \dee{S_{1-}},
  \\
  \vect{F} =
    \iint_{S_{2+}}
      \frac{J}{\permVac}\magB_1
    \dee{S_{2+}}
    -
    \iint_{S_{2-}}
      \frac{J}{\permVac}\magB_1
    \dee{S_{2-}}.
\end{dmath}
This is different to the work of \textcite{akoun1984}, who derive an
expression for the interaction energy between the two magnets and then
differentiate to get the forces.

These expressions follow the work of \textcite{bancel1999}, who defined his
equations as a set of nested functions, unlike the equations of
\textcite{akoun1984} that were six-nested summations. Back to these new
papers, three were published \cite{elies1998,charpentier1999,charpentier1999a}
that all contain the expression I am interested in (why didn't they just cite
the original?), with a fourth \cite{elies1998} containing just the force
expression in a single direction. There's a little bit extra in the paper by
\cite{elies1999a} that looks at the forces of two magnets inclined around the
vertical axis, calculated with a numerical integration. The final paper
\cite{elies1999} involves couplings exclusively, making it less applicable to
this work. Its inclusion here is for completeness.

So, what is this equation that is so useful for me? Given two magnets located
in \threeD/ space, of sizes $\inlinevect{A,B,C}$ and $\inlinevect{a,b,c}$,
with the plane of the second rotated by $\theta$ around the \x-axis and their
origins offset by $\inlinevect{x_0,y_0,z_0}$, the forces in the \y- and
\z-directions ($F_y$ and $F_z$) between the two magnets can be calculated.

Forces in both the $y$ and $z$ direction use the $f_3$ function in
their calculations.
\begin{dgroup*}
\begin{dmath}
F_y = \frac{J^2}{4\pi\permVac}
  \sum_{i_{a,b,c,A,B,C}\in\{0,1\}}
  f_{y_2}\cdot\gp{-1}^{i_a+i_b+i_c+i_A+i_B+i_C}
\end{dmath},
\begin{dmath}
F_z\bigg|_{\theta\neq k\pi} =
       \frac{-J^2}{4\pi\permVac}\cdot\smash{\sum_{i_{a,b,c,A,B,C}\in\{0,1\}}}
        f_{z_2}\cdot\gp{-1}^{i_a+i_b+i_c+i_A+i_B+i_C}
\end{dmath},
\begin{dmath}
f_{y_2} = f_3\fn{u_0 , y_0 , z_0 , \theta , c i_c , C i_C}
\end{dmath},
\begin{dmath}
f_{z_2} =  f_3\fn{u_1,v_1,w_1,-\theta,0,0}/{\sin\theta}
         + f_3\fn{u_2,v_2,w_2,\theta,0,0}/{\tan\theta}
\end{dmath},
\begin{dmath}
u_0 = x_0 - ai_a + Ai_A
\end{dmath},
\begin{dmath}
u_1 = u_0-2x_0
\end{dmath},
\begin{dmath}
v_1 = -v_2\cos\theta - w_2\sin\theta
\end{dmath},
\begin{dmath}
w_1 = v_2\sin\theta - w_2\cos\theta
\end{dmath},
\begin{dmath}
v_2 = y_0-Ci_C\sin\theta
\end{dmath},
\begin{dmath}
w_2 = z_0-ci_c+Ci_C\cos\theta
\end{dmath}.
\end{dgroup*}
This is the auxiliary function used in the above. All dashed variables are
local to this function.
\begin{dgroup*}
\begin{dmath}
f_3\fn{u',v',w',\theta',c',C'} =
  u' f_5\gp{\Log{f_4-u'}-1}
  +\half\gp{f_6^2-{u'}^2}\Log{f_4+f_5}
  +\half u' \pi \Sign{f_5}\Abs{f_6}
  +u' f_6 \ArcTan{\frac{u' f_4 - u^2 -f_6^2}{f_5 f_6}}
  +\half f_4 f_5
\end{dmath},
\begin{dmath}
f_4 = \sqrt{{u'}^2+f_5^2+f_6^2}
\end{dmath},
\begin{dmath}
f_5 = \gp{v'-b i_b}\cos\theta'+\gp{w'-c'}\sin\theta'+B i_B
\end{dmath},
\begin{dmath}
f_6 = -\gp{v'-b i_b}\sin\theta'+\gp{w'-c'}\cos\theta'+C'
\end{dmath}.
\end{dgroup*}

The force in the $z$ direction is calculated separately if the second
magnet is not rotated around its axis ($f_{z_2}$ has a singularity
at $\theta=k\pi$).
\begin{dgroup}
\begin{dmath}
F_z\bigg|_{\theta=k\pi} =
  \cos\theta\cdot\frac{-J^2}{4\pi\permVac}
  \sum_{i_{a,b,c,A,B,C}\in\{0,1\}}f_{z_1}
  \cdot\gp{-1}^{i_a+i_b+i_c+i_A+i_B+i_C}
\end{dmath},
\begin{dmath}
f_{z_1} =
  \half uw\Log{\frac{r+u}{r-u}}+\half vw\Log{\frac{r+v}{r-v}}+
  uv\ArcTan{\frac{uv}{wr}-wr}
\end{dmath},
\begin{dmath}
u = x_0-ai_i+Ai_A
\end{dmath},
\begin{dmath}
v = y_0-bi_b+Bi_B+\half B\gp{\cos\theta-1}
\end{dmath},
\begin{dmath}
w = z_0-ci_c+Ci_C+\half C\gp{\cos\theta-1}
\end{dmath},
\begin{dmath}
r = \sqrt{u^2+v^2+w^2}
\end{dmath}.
\end{dgroup}

This equation is very similar, but not identical, to Akoun~\& Yonnet's
equation for the same thing. I suspect theirs is more simple because they
integrate over symmetric limits of integration (\eg, $\int^{+a}_{-a}$). Terms
that are odd functions integrate to zero in this case.

In order to allow this equation to be used for rotations greater than
\SI{\sipi}{radians}, additional terms have been added to accommodate for the
translational offset induced by rotating the magnet around its lower-left
edge. The preceding $\cos\theta$ provides the sign change of the
magnetisation, and the $\half\gp{\cos\theta-1}$ terms are used to return the
origin of the magnet after 180\textdegree\ rotation to the lower-left corner,
where it is expected.

\subsection{Quick results}

I can confirm that Charpentier's equations, and my simplifications thereof,
are consistent and give reasonable results. \Figref{charp-rotate} shows the
forces produced between two \SI{10}{mm} cube magnets as a function of rotation
angle $\theta$ of the second magnet, with a \SI{20}{mm} offset between their
centres. Two cases are shown: in the first , the magnets are displaced
vertically; in the second, the magnets are displaced horizontally.

\begin{figure}
  \begin{wide}
  \begin{subfigure}
    \psfragfig{\phdpath Simulations/Theory/latex/charp-rotation-forces}
    \caption{Vertical displacement.}
  \end{subfigure}
  \hfil
  \begin{subfigure}
    \psfragfig{\phdpath Simulations/Theory/latex/charp-horiz-rotation-forces}
    \caption{Horizontal displacement.}
  \end{subfigure}
  \hfil
  \null
  \end{wide}
  \caption{Vertical and horizontal forces on a rotating magnet
    as a function of rotation angle for fixed horizontal and vertical displacements.}
  \figlabel{charp-rotate}
\end{figure}

\Figref{charp-rotate} shows that the maximum force between two magnets is
exhibited when the direction of displacement is in the same direction as their
magnetisations. For displacements perpendicular to the direction of
magnetisation of the fixed magnet, forces of equal magnitude are obtained for
rotation of the second magnet $\theta=k\pi/2$ for $k\in\{0,1,2,\dots\}$.

\fixme{cross ref yonnet (?) old result for bearings having same force}

In the following simulations, the second
magnet is held at a constant height (one magnet height's separation distance)
and moved from left to right over a range of twice the magnet width
symmetrically above the first magnet (with respect to the magnets' centres of
gravity). Four such displacements are made with four different rotations for
the second magnet: $0$, $\half\pi$, $\pi$ and $-\half\pi$.
\Figref{charp-0to2pi-Fy} shows the forces in the horizontal \y-direction;
\Figref{charp-0to2pi-Fz} the forces in the vertical \z-direction. It can be
seen that the opposite rotations result in symmetric forces curves, as is
expected.

Now these expressions can be used to calculate analytically the forces between
multipole arrays with arbitrary numbers of magnet per wavelength.

\begin{figure}
  \centering
  \psfragfig{\phdpath Simulations/Theory/latex/charp-0to2pi-Fy}
  \caption{Horizontal forces on a rotating magnet situated directly above another.}
  \figlabel{charp-0to2pi-Fy}
\end{figure}

\begin{figure}
  \centering
  \psfragfig{\phdpath Simulations/Theory/latex/charp-0to2pi-Fz}
  \caption{Vertical forces on a rotating magnet situated directly above another.}
  \figlabel{charp-0to2pi-Fz}
\end{figure}



\section{Multipole}

Applications:
artificial hearts \parencite{finocchiaro2008,samiappan2008}
wave power generation \parencite{kimoulakis2008}
wind power generation \parencite{liu2008-ietm}

\textcite{backers1961} developed a theory for magnetic bearings
by assuming the effect of curvature of the ring magnets to be
negligible and looking at the forces between two large flat
plates. He showed that a periodically recurring magnetisation
yields stronger forces than a simple homogeneous one. He
implements this coarsely in an active magnetic bearing with
cross-section shown previously in \figref{backers-bearing}.

It was not until the wide-spread availability of the high energy
density rare-earth magnets in the late 1970s that magnetic
geometries more complicated than
\citeauthor{backers1961}' % no extra 's'
became feasible. \textcite{halbach1980} pioneered the research
on these `multi-pole' magnet arrangements, showing how the ideal
case of sinusoidal magnetisation may be approximated with a
structure made up from discrete magnets as shown in
\figref{halbach}. This figure shows magnets with \ang{45}
increments of magnetisation, but \ang{90} are much more
common. In this document, the term `Halbach array' will refer to
the latter form.

In his subsequent paper, \cite{halbach1981} shows some uses for
his multipole magnets; namely, for creating flux patterns for
electron beam undulators or wigglers.  While his purposes were
radically different from those for rotary bearings or this
research, he founded the principle that by arranging magnets
with rotating magnetisation directions (generally referred to as
`Halbach arrays') the magnetic field is concentrated on one side
of the array, and reduced on the other. \figref{halbach} shows
this effect with magnetic flux almost exclusively above the
array, so the magnetic field strength is much stronger above it
than below. Compare this figure to the flux lines of a single
magnet as shown in \figref{magflux}—notice how in the latter,
the flux is symmetrical both above and below the magnet.

As well as increasing the stength of the magnetic field in the desired area of use, this had the added advantage of `shielding' external equipment from the field where the magnetic forces are not required, although there are other possiblities to effect magnetic shielding \cite{becherini2009}.

\textcite{choi2008}

Forces between magnets are governed by their field strengths, so
it is a natural conclusion that two facing Halbach arrays will
have greater forces between them than single magnets of the same
size. \textcite{yonnet1991} examined this effect relating to the
stiffness of rotary bearings. They show that with bearings with
\ang{90} rotations between successive magnets (as opposed to
\citeauthor{backers1961}' % no extra 's'
bearing which used \ang{180}
increments), the stiffness can be approximately doubled while
keeping the magnet volume constant.

\begin{figure}
\grf{Figures/Magnets/magflux}
\caption{The magnetic flux lines of a single magnet.}
\figlabel{magflux}
\end{figure}

\textcite{bancel1998} look at the field produced by Halbach
arrays used for a linear position sensor. Their paper derives
equations for the magnetic flux density of an arbitrary magnet,
the geometry and magnetisation of which they optimise for their
purposes. They then use this equation to compute the effects of
a Halbach array (which they plot), but their analytical
solutions are not shown. This effort is based around the flux
density, so naturally no attempt is made to find the force
equations between two arrays.

There have been some very recent papers dealing with the
magnetic effects of Halbach arrays used in \maglev/
transportation \cite{hoburg2004}, predominantly to ensure
that the magnetic field is not strong enough to affect
passengers.

\begin{figure}
\grf{Figures/Multipole/halbach}
\caption[Magnetic flux lines of a Halbach array.]{%
  The magnetic flux lines of a Halbach array made from magnets with \ang{45}
  magnetisation increments (indicated by the arrowheads). The magnetic field
  can be seen to be much stronger above the array than below it.}
\figlabel{halbach}
\end{figure}

For some time it seemed that multipole or Halbach arrays were a
promising avenue of investigation for building a large-load magnetic
spring. Multipole arrays are structures composed of a multitude of
smaller magnets with (consistently) varying directions of
magnetisation. The advantages of these arrays over homogeneous magnets
are their single-sided flux patterns and their larger induced forces
(albeit over smaller distances).

\textcite{moser2006} showed an optimum configuration for a bearing
with alternating magnetisations in the axial direction (it is rather
difficult to radially magnetise a ring or disc magnet).

\textcite{chen2002} used stacked magnets of various sizes (with the
largest magnet rings midway between the inner and outer radii of the
bearing) in order to maximise the stiffness per cross sectional area of the
array.


\begin{description}
\item[\textcite{jang2005a,jang2005b}] These are two papers that are probably
of little interest to me, but they were flagged because they talk about
multipole or Halbach arrays. They talk about linear actuators, but it's tricky
for me to see what's exactly new in them.
\item[\textcite{filho1999,filho2003}] Yet more literature on planar actuators
that aren't \emph{directly} related to what I'm doing. Their analyses are
interesting, however, so I've kept them around in case they come in handy
later.
\end{description}


The next requirement of the design of the magnetic spring is high load
capacity. Since radial magnetic bearings work directly with trying to achieve
large forces from arrangements of magnets, it is possible to look to this
literature to find prior art. Recall from the literature (see
\secref{multipole}) that \citeauthor{yonnet1991} were able to increase the
stiffness of their radial bearing by applying Halbach arrays. This application
uses opposing Halbach arrays, but other configurations are possible.

An examination of the flux lines of various facing Halbach arrays, as shown in
\figref{halbach-flux-repl,halbach-flux-attr}, shows how flux interacts between
the two separate linear arrays. It can be seen that simply by varying the
direction of magnetisation of the first magnet in the second array, a full
sinusoid of forces can be achieved between the arrays—at the limits, total
attraction or repulsion in the vertical direction, as for
\figref{halbach-flux-repl,halbach-flux-attr} respectively.

\begin{figure}
  \begin{subfigure}
  \grf[scale=0.67]{Figures/Multipole/2-halb-repl}
  \caption{In repulsion.}
  \figlabel{halbach-flux-repl}
\end{subfigure}
\begin{subfigure}
  \grf[scale=0.67]{Figures/Multipole/2-halb-attr}
  \caption{In attraction.}
  \figlabel{halbach-flux-attr}
\end{subfigure}
\caption{Magnetic field lines of linear multipole arrays.}
\end{figure}

In this configuration, then, this restricts the relative lateral motion to
each other that the springs can experience. A slip by two magnet lengths, and
the spring will no longer be supporting load---it will be firmly attracted to
its former opposer! In practise, this is not a concern. The operation of the
spring is design such that free motion is opposed in the lateral directions,
and will be implemented with a physical stop (only at the extremes of
displacement so as to remain non-contact, of course) to prevent calamity of
this sort.

\section{Planar arrays}

The previous work covered deals with magnet arrays of homogeneously-magnetised
cross-section. \textcite{kim1997} cites a patent, in which he is involved,
that superimposes two orthogonal linear Halbach arrays to create a
\emph{planar} structure. \figref{pa-trumper} shows the structure, but in his
own planar levitator, \citeauthor{kim1997} actually uses a linear Halbach
array (later shown in \figref{pa-x2d}).

Planar arrays are used quite extensively in the field of planar permanent
magnet motors, and there have been some other array configurations developed.
Analysis of the magnetic flux produced by various planar arrays, including
Kim's, has been done by \textcite{cho2001}, culminating with the
design shown in \figref{pa-cho}. However, their uses have so far
been confined to applications that use the magnetic flux for providing motive
force; no analysis has been made on the \emph{forces} between arrays.

The author is unaware of any attempt to use Halbach arrays to increase the
load bearing efficiency of linear magnetic springs.

The `patchwork' array shown in \figref{pa-ns} is the most simple. See Hinds
cited by \textcite{kim1997}, or Chitayat cited by \textcite{cho2001} for its
use in other circumstances. The array used by \citeauthor{kim1997} is shown in
\figref{pa-x2d}, despite the fact that he demonstrated the Halbach
superposition. \fixme{crossref}

\begin{figure}
  \subfloat[Patchwork.]
    {\grf{Figures/Multipole/pa-ns}\figlabel{pa-ns}}
  \hfil
  \subfloat[Extended Halbach array.]
    {\grf{Figures/Multipole/pa-x2d}\figlabel{pa-x2d}}
  \caption{Two simple planar magnetic arrays.}
  \figlabel{pa-simple}
\end{figure}

\begin{figure}
   \grf{Figures/Multipole/pa-cho}
   \caption[Novel planar array shown in the literature.]{%
     The planar array by \textcite{cho2001}. Flux travels out of the page
     from the north- to south-faced magnets, and back through the \emph{array}
     in the triangular magnets.}
   \figlabel{pa-cho}
\end{figure}



\section{Damping}
\seclabel{damping}

\cite{tentor2001}

Henry Sodano completed his \PhD/ in 2005 \cite{sodano2005thesis} on
eddy current damping for flexible structure and has published several
papers based on that work.\footnote{I recommend the thesis for the
  additional context and literature review.} His work investigates the
use of non-contact magnetic (permanent \cite{sodano2005,sodano2006,sodano2008} or
electric \cite{sodano2007}) elements to passively \cite{sodano2005} or
actively \cite{sodano2006,sodano2007,sodano2008} add forces to a structure via
induced eddy currents. His work investigates the potential for use
with flexible structures primarily for use in space
applications.\footnote{Note that not all permanent magnets are created
  equal for suitability in space: the cheapest and most common class
  of rare earth magnetic alloy, neodymium-iron-boron, will become
  demagnetised in the influence of radiation due to localised heating
  effects. Samarium-cobalt magnets are less susceptible to this
  problem due to their higher Curie temperature and have been
  previously used in space applications \cite{chen2005}.}

The damping effects of magnetic forces has also been examined by
\textcite{bonisoli2006}. The use of electromagnetic damping can be
advantageous in applications where the absorbed energy is converted to
electrical energy for re-use and storage, increasing the overall
efficiency of the device; a good example is in the automotive industry
\cite{graves2000thesis}. \fixme{is mentioning regenerative damping
appropriate in this spot?}

Another recent investigation of an eddy current damper using an aluminium
plate is shown by \textcite{ebrahimi2008}. Damping of levitated permanent
magnets with a similar technique was shown by \textcite{elbuken2006}. Their
emphasis lay on the problem of micro-levitation, where small stiffnesses (and
damping) results in large amplitudes of disturbance. One of the noted
advantages in this case is the fact that the eddy current damping does not add
\emph{other} dynamics to the system it is applied to; equilibrium positions
and controller designs are unaffected with this technique.

Eddy currents may generate force on a conductor through two
mechanisms: change in magnetic field and/or change in velocity. Change
in velocity leads to a (possibly nonlinear) viscous damping force that
is dissipative: it can only decrease the enery of the system. However,
change in magnetic field generates forces that can be used to apply
work. This is the same mechanism used by \AC/ current levitating
devices. \fixme{crossref} It is unclear whether significant advantages
exist over using conductive material rather than ferrous material to
generate vibration suppression forces. For the conductive case, a
permanent magnet may be used to increase the field strength of the
electromagnet at the expense of added viscous damping (which may not
be undesirable). Also, constant and low frequency forces will not be
able to be generated without significant control design \dash\ this is
only possible with \AC/ current resonating with the \backemf/
of the induced eddy currents in the conductor. \fixme{check that}

The generation of eddy currents is an involved calculation, and has
not been investigated in detail for this thesis. The eddy current
density $\magJeddy$ induced in a conductive sheet moving through a
magnetic field at velocity $\velocity$ is given by
\begin{dmath}
\magJeddy = \conductivity\gp{\velocity \cross \magflux}.
\end{dmath}
where $\conductivity$ is the conductivity of the sheet. The force
$\forceEddy$ due to these eddy currents is
\begin{dmath}
\forceEddy = \Int{\magJeddy \cross \magflux}{V,V} .
\end{dmath}
Now,
\begin{dmath}
  \magJeddy \cross \magflux
  \propto
  \gp{\velocity \cross \magflux}\cross \magflux ,
\end{dmath}
which is anti-parallel to $\velocity$ \fixme{small figure}. Due to the
cross terms, the maximum force is obtained for magnetic fields
perpendicular to the motion of the conductor. Associatively, it is
only the component of magnetic field in the perpendicular direction
that influences the eddy force. This has implications on the
arrangement of eddy current dampers for vibrating structures.

Two configurations of eddy current dampers are shown in
\figref{eddy}. The two systems behave somewhat differently. A
comparison (\fixme{DOES IT?!?} of the flux through the conductive
sheets perpendicular to the motion of the mass shows that the vertical
configuration results in greater damping force. However, this system
will also exhibit a nonlinear damping force as the gap between the
magnet and conductor vary with displacement of the mass. In practice
this should not be a significant problem as the gap will usually be
somewhat larger than the motion of the mass.

\begin{figure}
  \begin{subfigure}
    \asyfig{Damping/eddy-h}
    \caption{Horizontal configuration; eddy
      currents induced via axial magnetic
      flux.\figlabel{eddy-h}}
  \end{subfigure}
  \begin{subfigure}
    \asyfig{Damping/eddy-v}
    \caption{Vertical configuration: eddy
       currents induced via radial magnetic
       flux.\figlabel{eddy-v}}
  \end{subfigure}
  \caption{Orthogonal configurations of eddy current dampers for a vibrating
    (non-magnetic) mass. The shaded section indicates conductive material.}
  \figlabel{eddy}
\end{figure}

Experimental results of a magnetically levitated mass
(\fixme{crossref}) later show that the damping in this case is very
low, making a passive non-contact eddy current damper of high
importance for vibration suppression in systems that will not be
receiving significant control effort.




