\documentclass[11pt,a4paper]{memoir}
\def\asydir{\jobname}
\usepackage{thesis-preamble}
\EndPreamble
\begin{document}

\chapter{Magnetic and electromagnetic forces}
\chaplabel{magnet-theory}

\epigraph{With an eye to the practical importance of levitation we
feel justified here in disregarding those aspects of it
associated with magic, spiritualism, and psychic
phenomena\dots}{\textcite{boerdijk1956b}}

\referpaper{The work presented in \secref{cyl-forces} is based on material that has been published as a journal paper~\cite{robertson2011-ietm}.}

\section{Magnetic fields}

The following is a very short introduction to the physics behind magnets, largely to introduce the notation used later in the thesis.

Magnetic fields are created by moving electrons.
A long straight wire will create a cylindrical-like magnet field, and a small loop will create a magnetic dipole.
Thus, an electron orbiting a proton is the smallest magnetic element.
This is a hydrogen atom.
In nature, however, hydrogen exists as H$_2$, two protons orbited by two electrons \dash and it happens that the two electrons orbit in opposite directions and the magnetic fields of each cancel each other out.
Most material is like this: basically, not magnetic.
However, there are a number of compounds that do retain an asymmetry in their electron composition, and this allows them to act magnetically under the right conditions.


\subsection{Magnetic parameters}
\seclabel{bhm}

The magnetic dipole is designated as the microscopic quantity
$\mdip = i\vect{A}$, for a current $i$ and a vector area $\vect{A}$
(direction normal to plane). For a collection of magnetic dipoles (as
in a permanent magnet), their net effect may be quantified with the
macroscopic magnetisation of the material, $\magM$:
\begin{dmath}
  \magM =  \lim_{\Delta V \rightarrow 0} \frac{\sum \mdip}{\Delta V}  \eqlabel{M}
\end{dmath}.
Inside the magnet itself (with no other external fields present), the magnetic field, $\magB$, is given by the relation \cite{campbell1994}
\begin{dmath}[label=BM]
  \magB = \permVac \magM
\end{dmath},
where the proportionality constant $\permVac$ is known as the permeability of the vacuum.

The equivalence of the magnetic field produced by a current-carrying coil and a permanent magnet is well established.
Therefore, it is possible to consider the magnetisation of a permanent magnet to be the result of an (abstract) surface current density $\vect{J}_m$ defined by
\begin{dmath}
  \vect{J}_m = \curl\magM
\end{dmath}.
This is a good starting point for describing the magnetic effects of an external current density $\vect{J}$ acting on a magnet.
To differentiate between such an induced magnetic field and that caused spontaneously by the magnetic material $\magM$, the induced magnetic field is denoted $\magH$, defined equivalently
\begin{dmath}
  \vect{J} = \curl\magH
\end{dmath}.
These three terms involving magnetic field can now be related to account for both internal and external forms of magnetisation (\ie, magnetic field caused by permanent magnets or by current carrying conductors) as \cite{campbell1994}
\begin{dmath}[label=BHM]
  \magB = \permVac (\magM + \magH)
\end{dmath}.
For clarify, $\magB$ is referred to as the magnetic flux density, $\magH$ is the magnetic field strength, and $\magM$ is the magnetisation.
\note{The names of these terms are not always consistent in the
  literature. $\magM$ is also known as polarisation, and $\magB$ and $\magH$
  are both sometimes known as the magnetic field in different contexts.}

\Eqref{BHM} may now be used to describe the situation at all points in space (\figref{BHM}).
Inside a permanent magnet, in the presence of an external magnetic field, the magnetic flux density is the vector sum of $\magH$ and $\magM$, whereas outside the magnet, the $\magM=0$ and the magnetic flux density is related to the magnetic field strength by a constant.
This results in $\magB$ being continuous everywhere, and both $\magM$~and $\magH$ being discontinuous.

\begin{figure}[htbp]
   \centering
   \includegraphics{PhD/Figures/Theory/BHM}
   \caption{The magnetic field, $\magB$, both inside and outside a magnet.}
   \figlabel{BHM}
\end{figure}


\subsection{Relationship between magnetic parameters}

The relationship established by \eqref{BHM} is true only for ideal permanent magnets whose own internal composition is not affected by an applied magnetic field, nor does it say much about other materials in the presence of a magnetic field.
Within such materials, there may be some relationship between the applied magnetic field strength and the resultant magnetic flux density, which is described by the permeability $\perm$ of a material where
  \begin{dmath}[label=perm]
    \perm = \frac{\magB}{\magH}
  \end{dmath}.
The permeability for materials with a non-negligible magnetic interaction will not necessarily be constant with applied field strength or temperature, and for anisotropic materials the permeability will be direction-dependent.
It is often convenient to express permeability as a relative measure
  \begin{math}
    \permrel = \frac{\perm}{\permVac}
  \end{math},
known as the relative permeability.

The relativel permeability of the vacuum is unity, and materials considered `non-magnetic' such as air, wood, water, and so on, have permeabilities very close to unity (within \num{1e-5}) as a result.
(The consequences of permeability less than unity, diamagnetism, has been discussed on page~\pageref{diamag}.)
Materials which are more strongly affected by magnetic fields have greater permeabilities; \eg, within the soft iron core of an electromagnet, current in the coil generates an applied magnetic field on the core, which induces its own internal magnetic field as a result which increases the overall strength of the electromagnet. The larger the permeability, the larger this `amplification' effect.
Despite being having strong remanence and coercivity values, rare earth magnets have a relative permeability of $\permrel=\num{1.05}$.

Refering to \eqref{BHM} in free space where $\magM=0$ the relationship between $\magB$ and $\magH$ is quite simple, which is essentially the reason that there is a historical terminological confusion between the two.
It can be seen that within a magnet, however, their relationship is more complex and important.
Consider a magnetic material which has not yet been magnetised.
As an external magnetic field is applied to it, the magnetic dipoles within the material begin to align along the direction of the applied field.
If the external magnetic field is sufficiently large, the material becomes completely magnetised (such that all of its dipoles are in alignment) and this magnetisation will be retained even once the applied magnetic field is removed.
If the magnetic field is increased further, the magnitude of magnetisation of the magnet cannot increase, and it is thus said to be `saturated'; this saturation magnetisation is denoted $\Msat$.
This process is conducted under high temperatures; once the magnetic material cools (with the applied magnetic field still in place) the magnetisation `sets' and the magnet is formed.
This magnetisation is retained unless a large demagnetisation field (which is simply a magnetic field applied in the opposite direction to its magnetisation) is applied to it or the magnet is heated sufficiently to allow its magnetic dipoles to lose their alignment due to thermal effects.

The performance of a magnet is linked to its behaviour under demagnetisation, as a strong magnet that demagnetises easily is of little use in practical application.
The behaviour is shown by the magnetic material's \bhcurve/, an example of which is shown for an ideal magnet in \figref{BHcurve}.
Two important features are shown in the \bhcurve/ indicated in \figref{BHcurve}.
First, the remanence of the magnet, $\remanence=\permvac\Msat$, is the value used to indicate the `strength' of the magnet alone corresponding to its saturation magnetisation.
The remanence refers to the amount of magnetic flux density that is measured in the absence of an applied external magnetic field, and it is a common term in practise since the internal magnetisation of a magnet $\Msat$ cannot be measured directly.

The second feature of interest on the \bhcurve/ is the coercivity, $\coerce$, which is the amount of magnetic field strength required to reduce the magnetic flux density within the magnet to zero.
The larger the coercivity, the greater the ability of a permanent magnet to resist demagnetisation due to the influence of external magnetic fields.
For the purposes of this thesis, rare earth magnets will be considered which have sufficiently large coercivities to avoid demagnetisation effects.

For theoretical magnetic force analysis (covered in \secref{magnet-forces} and later), the `strength' value to model the permanent magnets is defined in practise by the remanence of the magnet.
For cases such as finite element analysis, it can be necessary to instead define the permeability and coercivity of the magnets.
In which case, it can be convenient to calculate the coercivity with the relation
\begin{dmath}
\coerce = \frac{\remanence}{\perm\permvac}
\end{dmath}.

\begin{figure}[htbp]
   \centering
   \includegraphics{PhD/Figures/Theory/BHcurve}
   \caption{The characteristic $B$ \vs\ $H$ curves for an ideal rare-earth magnet.}
   \figlabel{BHcurve}
\end{figure}

As well as the remanence $\remanence$ and coercivity $\coerce$, two other parameters is often seen to describe the `strength' of a permanent magnet.
The first is known as the `maximum energy product' $\BHmax$, relating to the amount of potential energy that can be supplied by the magnetic field in the second quadrant of the \bhcurve/.
In an ideal magnet, $\BHmax$ is directly related to its saturation magnetisation via \cite{campbell1994}
\begin{dmath}[label=bhmax]
  -\BHmax = \permvac\gp{\frac{\Msat}{2}}^2
\end{dmath}.
The final term is known as `intrinsic coercivity' $\coerce{}_i$, which is the magnetic field strength required to completely magnetise a magnetic domain (\ie, a collection of dipoles) or to completely reverse the polarity of magnetisation for the same.
Since a permanent magnet is made up of a very large number of magnetic domains, the magnetic field strength used to initially magnetise a permanent magnet is said to be around $5\coerce{}_i$ as a rule of thumb.


\subsection{Properties of magnetic flux}
\seclabel{flux}

The previous section introduced $\magB$, the magnetic flux density.
`Magnetic flux' derives its name from archaic models of magnetism,
whose proponents believed in the literal flow of a magnetic fluid
called the `luminiferous æther'. Nowadays, scientists tend toward more
modern interpretations using electromagnetic fields involving quantum
theory. Nonetheless, the name sticks. Magnetic flux, $\flux$, is
therefore defined as the amount of `magnetic fluid' passing through an area:
\begin{dmath}
  \flux = \magB \cdot A
\end{dmath}
This flux is almost analogous to electric current; the only difference
being that electric current is constrained by the conductor it is
flowing through, whereas while magnetic flux is known to \emph{prefer}
areas of greater permeability, it occasionally can deviate from these simple paths.

\begin{figure}
\includegraphics{PhD/Figures/Magnets/magflux}
\caption{The magnetic flux lines of a single magnet.}
\figlabel{magflux}
\end{figure}

An analysis of how to derive the paths of magnetic flux is a beyond the scope of this document, but it is important to discuss the flux lines themselves.
Typical flux lines for a rectangular cross-section magnet are shown in \figref{magflux}.
It is more instructive for a basic understanding of how magnets behave
to look at the ways their flux lines interact. The following `magnet
design axioms' are adapted from \textcite{moskowitz1995}, whose book
covers permanent magnet design for a wide range of uses.
\begin{enumerate}
\item Flux lines follow the path of least resistance. This means that they will
travel through the shortest path possible,
through the material with the
\emph{greatest} permeability---so they will travel more readily through
magnetic or ferrous material than air, and more readily through air
(although only slightly) than diamagnetic material.
\item Flux lines travelling in the same direction repel each other. This means
flux lines will never cross.
\item Flux lines enter ferrous material at right angles in a low-permeability surrounding.
\item Permeability of ferrous material is `used up' by flowing flux; when the
material reaches saturation, flux lines travel as easy though air as through
the saturated material.
\item Flux lines travel from North to South poles in closed loops.
\item Magnets are made up of a very large number of unit poles.
\end{enumerate}
From these axioms, one can generate incorrect, yet applicable,
theories how and why magnets attract and repel each other. For
example, two magnets in repulsion have flux lines opposing each
other.
It can be imagined that the reason forces occur between them is
due to a `squashing' of the flux lines which the magnets try to
oppose---but theories like this only help visualising magnetic
behaviour, \emph{not} for explaining the reasons behind it.
\note{Just ask \textcite{sodano2006}, who received nitpicking comments about their terminology \cite{marneffe2007}.}

\section{Permanent magnets and magnetic materials}

There are several materials from which permanent magnets can be
made. Short attention will be placed on the cheaper, legacy magnetic
materials such as the ferrite magnets and alnico magnets due to their
poor performance. Rare-earth neodymium magnets are now readily 
available and rather cheap, and have much more desirable properties
than these old fashioned magnets.

\Tabref{magnets} shows some approximate ranges comparing the
properties of the various magnet types available. Clearly, rare earth
magnets are capable of much greater energy output, and their high
coercivity precludes them from losing their magnetisation through
physical impact or proximity with other magnets---unlike the older
ferrite and alnico magnets. Their only disadvantage is their low
operating temperatures that perhaps will be inconvenient using them
for biased, or hybrid, electromagnets, in which a current carrying
coil is wrapped around a permanent magnet to increase the output of
magnetic field.

\begin{table}
  \caption[Typical values for various permanent magnets.]
  {Typical values for various permanent magnets.
   Adapted from information from \url{http://www.magtech.com.hk/}.}
  \tablabel{magnets}
  \begin{tabular}{@{} l r@{\,--\,}l r@{\,--\,}l r@{\,--\,}l @{}}
    \toprule
    & \multicolumn{6}{c@{}}{Magnet type}\\
    \cmidrule(l){2-7}
    Property            & \multicolumn{2}{c@{}}{Ferrite}
                        & \multicolumn{2}{c@{}}{Alnico}
                        & \multicolumn{2}{c@{}}{Neodymium}  \\
    \midrule
    Max.\ temperature (°C)    & \num{400} & \num{500} & \num{800} & \num{900} &    \num{ 80} & \num{200}  \\
    Remanence (T)             & \num{0.2} & \num{0.4} & \num{0.5} & \num{1.3} &    \num{  1} & \num{1.3}  \\
    Coercivity (\si{kA/m})    & \num{100} & \num{200} & \num{50 } & \num{160} & ~~~\num{800} & \num{900}  \\
    Max.\ energy product
               (\si{kJ/m^3})  & \num{6}   & \num{33}  & \num{10}  & \num{80}  &    \num{200} & \num{300}  \\
    \bottomrule
  \end{tabular}
\end{table}


\section{General technieques for calculating forces between magnets}
\seclabel{magnet-forces}

A general technique for finding the forces between two magnets is
simple to describe. The first magnet creates a magnetic field in the
region of the second magnet; the force is calculated due to the
interaction of the first magnet's field and the internal field of the
second magnet. Or vice versa; reciprocity holds here.

There are two methods that will be outlined here for calculating the magnetic field of a permanent magnet, known as the charge and current models.
Respectively, these consist of modelling the magnets as having two surfaces of `magnetic charge', or modelling the magnet as being circumscribed of an equivalent surface current density.
In the expressions to follow, the magnetisation of each magnet has been assumed to be homogeneous and constant, which is usually a reasonable assumption for modern rare earth magnetic material; hence terms involving $\Div\magM$ and $\Curl\magM$ equate to zero.

In the first step, the integration takes place over the surface of the
first magnet $S_1$, which is written for the charge model as:
\begin{dmath}
\magB_1\fn{\pos_2} =
 \magconst\oint\limits_{S_1}
    \gp{\magM_1\dotprod\normn_{s_1}}
    \frac{\pos_2-\pos_1}{\Abs{\pos_2-\pos_1}^3}
    \dee s_1
\end{dmath},
and for the current model as:
\begin{dmath}
\magB_1\fn{\pos_2} =
 \magconst\oint\limits_{S_1}
    \gp{\magM_1\cross\normn_{s_1'}}\cross
    \frac{\pos_2-\pos_1}{\Abs{\pos_2-\pos_1}^3}
    \dee s_1'
\end{dmath},
where $\normn$ is the normal vector from the surface of integration.

In the second step, the integration of the function of the magnetic field of the first magnet takes place over the surface of the second magnet $S_2$, and the integral for the charge model is:
\begin{dmath}[label=charge-force]
\force = \oint\limits_{S_2}
  \gp{\magM_2\dotprod\normn_{s_2}} \magB_1\fn{\pos_2} \dee s_2
\end{dmath},
and for the current model as:
\begin{dmath}[label=current-force]
\force = \oint\limits_{S_2}
  \gp{\magM_2\cross\normn_{s_2'}} \cross \magB_1\fn{\pos_2} \dee s_2'
\end{dmath}.

\Eqref{charge-force,current-force} are general recipes for calculating the forces between permanent magnets, and a similar formulation will allow for the modelling of electromagnetic coils as well.
Complex geometries, however, cannot be solved analytically as the integrals become intractable; numerical integration must be used in this case \cite{elies1999a}.
Generally when this is required the integrals are simplified as much as possible before numerical integration is applied to the remaining terms; for this reason this technique is often called a `semi-numerical' approach, and is capable of obtaining results in a  more straightforward and efficient manner than finite element analysis.

FURLANI

\subsubsection{Analytical expressions for the magnetic field}

For the purposes of this work, the analytical calculation of the magnetic flux $\magB$ is largely overlooked in favour of analytical force calculations, which will be dealt with in the next section.
However, since an analytical formulation for $\magB$ is a requirement for then calculating the force, a short literature review will be covered here for different magnet geometries.

The magnetic field for cuboid shaped magnets has a concise solution and has been known for some time for magnetisation in a direction orthogonal to the magnet face \cite{akoun1984}.
Much more recently, expressions were presented for calculating the magnetic field from a cuboid magnet with magnetisation in an arbitrary direction \cite{ravaud2009-pier98}.

The magnetic field from `pyramidal frustrum' magnets \cite{compter2010-ietm} can calculated by decomposing the pyramidal frustums into triangular and quadrilateral current sheets of non-square orientation.
This method is a positive step towards generating a general theory for calculating the magnetic field from arbitrary magnet shapes with flat sides.

The geometry of cylindrical magnets and coils results in elliptic integrals in the solution to their field equations.
For a cylindrical magnet, the field solutions have been published for axial magnetisations \cite{ravaud2010-ietm} and radial ones \cite{furlani1995-ietm}.
A comparison between the current and charge models for calculating the magnetic field for radially-magnetised arc-shaped magnets was made by \textcite{ravaud2009-pier-compare}, who state:
\begin{quote}
The problem is thus to guess what model is the most appropriate for calculating the three components of the magnetic field produced by permanent magnets.
It does not seem to be more difficult to use the Amperian current model rather than the Coulombian [`charge'] model for calculating the magnetic field created by parallelepiped magnets. [\dots] For arc-shaped permanent magnets, it seems to be more difficult to guess what model is the most appropriate.
\end{quote}

Finally, it is interesting from a historical perspective that new publications on the analytic magnetic field equation for a `thick coil' (or toroidal conductor with rectangular cross section, to be more precise/verbose) appear to be being published at an accelerating rate \cite{danilov1971-nim,urankar1982-ietm,babic1988-ietm,azzerboni1993-ietm,labinac2006-ajp,pechenkov2006-rndt,ravaud2010-emwaves,zhang2012-ietm}.
The newer equations tend to be more general and/or robust.



\section{Forces between parallel cuboid magnets}
\seclabel{cuboid}

In this section, the literature for forces between cuboid magnets is introduced in detail (this theory is used extensively in this work).
The theory for combining the force equations between parallel and orthogonal cuboid magnets is formalised for the purpose of calculating the forces between arbitrarily magnetised cuboid magnets.

\subsection{(Anti-)parallel alignment}

\def\e#1{e_#1}

A variety of analytical solutions have been developed to calculate the
force between cuboid-shaped magnets with parallel/anti-parallel
magnetisations \cite{akoun1984,nagaraj1988,bonisoli2006}. More complex
geometries can be realised through superposition of the solutions
\cite{bancel1999}.

\begin{figure}
  \asyinclude{PhD/Figures/Magnets/akoun}
  \caption
  [Geometry for parallel cuboid magnets.]
  {Geometry for the expression by \textcite{akoun1984} to
  calculate the forces between two parallel cuboid magnets with
  magnetisations in the vertical direction, distance between their centres
  $(\alpha,\beta,\gamma)$, and magnet sizes as shown.}
  \figlabel{akoun}
\end{figure}

The notation for the models to calculate the forces between cuboid magnets is as follows, with the geometry of the system depicted graphically in \figref{akoun}.
The first magnet has dimensions $[2a, 2b, 2c]\T$ and the second magnet has dimensions $[2A, 2B, 2C]\T$.
The distance between their centres is given by $\vect d=[\alpha,\beta,\gamma]\T$.
The magnetisations of the magnets are assumed to be constant and aligned in the $z$-direction (`facing up').
Anti-parallel magnets (`negative magnetisation') corresponds to the secondary magnet `facing down' and results in forces of reversed sign.
 \note{This relationship is only true of high-coercivity magnets; for magnets such as ferrites whose own magnetic fields can demagnetise each other, an approximation can be made that the repulsive force between two magnets will be approximately 40\% of the attractive force between them \cite{moskowitz1995}.}

The forces between two $\vect F_{z,z}$ parallel or anti-parallel magnets with remanences $\magn1$ and $\magn2$ and geometry as defined above is compactly written as six nested summations of intermediate expressions in $x$, $y$, and $z$ directions:
\begin{equation}\eqlabel{akoun}
\vect F_{z,z} = \frac{\magn1\magn2}{4\pi\mu_0}
  \sum_{i,j,k,l,p,q\in\{0,1\}^6}
  \hspace{-5mm}% space hack
  \vect \phi_{z,z}\fn{\vect \delta_{i,j,k,l,p,q}}
  \cdot
  \gp{-1}^{i+j+k+l+p+q} ,
\end{equation}
where $\vect\phi_{z,z}(\vect\delta) = [\,\phi_x(\vect\delta),\,\phi_y(\vect\delta),\,\phi_z(\vect\delta)\,]\T$ will be given later.
The $(z,z)$ subscript refers to the directions of magnetisation of the magnets.
Force calculations for magnets oriented in the $x$ or $y$ directions can be found using a coordinate system transformation on \eqref{akoun}.

This form of \eqref{akoun} arises as it is derived from six nested direct integrals.
Rather than expanding the limits of each integral, the following summation notation is used instead; say $f$ integrates to $F$:
\begin{equation}
\Int {f\fn{x}}{x,-a,a} = F\fn{a}-F\fn{-a} =
\!\!\! % space hack
\sum_{i\in\{0,1\}} F\fn{a\gp{-1}^i}\cdot \gp{-1}^i  =
\!\!\! % space hack
\sum_{e_i\in\{1,-1\}}
\!\!\! % space hack
F\fn{a e_i}\cdot e_i .
\end{equation}
And therefore for multiple integrations:
\begin{dmath}
\Int{ f\fn{x,y,z} } {x,x_0,x_1} {y,y_0,y_1} {z,z_0,z_1}=
  \sum_{i,j,k\in\{0,1\}^3} F\fn{x_i,y_j,z_k}\cdot\gp{-1}^{i+j+k}
\end{dmath}
For $N$ nested integrals, it may be convenient to express this in a product form instead where $f$ is integrated over variables $x_i$ from $u_i\fn{1}$ to $u_i\fn{-1}$:
\begin{dmath}
\Int{ f\fn{x_1,x_2,\dots} }{x_1}{x_2} \cdots =
  \sum_{e_1,e_2,\dots\in\{1,-1\}^N}
    \gp{ F\fn{u_1\fn{e_1},u_2\fn{e_2},\dots}\prod_{n=1}^{N} e_n }
\end{dmath}

As the limits of the integral occur at the corners of the cuboid magnets, $\vect \phi_{z,z}$ is an intermediate function acting between each combination of corners between the first and second magnet.
Bancel \cite{bancel1999} used this fact to invent an abstraction for these expressions known as `magnetic nodes' calling what has been written above as $\vect\phi_{z,z}\fn{\vect \delta_{i,j,k,l,p,q}}\cdot\gp{-1}^{i+j+k+l+p+q}$ as the `force' between two magnetic nodes $(i,k,p)$ and $(j,l,q)$.
Summing the magnetic node forces between every combination of corners of the first and second magnet yields the total force between them.
This abstraction allows a reduction in the number of calculations required when magnetic nodes overlap; i.e., when calculating the forces between arrays of touching magnets.

The distance between two corners/nodes of two respective magnets, $\vect\delta_{i,j,k,l,p,q}=[u_{i,j}, v_{k,l}, w_{p,q}]\T$, is given by the distance between the magnet centres, $\vect d$, plus and minus the distance between the magnet centre and corner position for the second magnet, $\vect R$ and for the first magnet $\vect r$:
\begin{equation}
\vect\delta_{i,j,k,l,p,q}=\vect d+\vect R_{j,l,q} - \vect r_{i,k,p} ~,
\end{equation}
where
\begin{align}
\vect R_{j,l,q}&=\begin{bmatrix}A\gp{-1}^j\\B\gp{-1}^l\\C\gp{-1}^q\end{bmatrix},&
\vect r_{i,k,p}&=\begin{bmatrix}a\gp{-1}^i\\b\gp{-1}^k\\c\gp{-1}^p\end{bmatrix}.
\end{align}
Complete expressions for the corner distances are therefore:
\begin{dmath}[compact]
\vect \delta_{i,j,k,l,p,q}=\begin{bmatrix}u_{i,j}\\v_{k,l}\\w_{p,q}\end{bmatrix}=
\begin{bmatrix}
  \alpha-a\gp{-1}^i+A\gp{-1}^j\\
  \beta-b\gp{-1}^k+B\gp{-1}^l\\
  \gamma-c\gp{-1}^p+C\gp{-1}^q
\end{bmatrix}
\end{dmath}.
The $\vect\phi_{z,z}$ terms required for calculating the `force between nodes' can now be written, where $r=\sqrt{u^2+v^2+w^2}$, as:
\begin{dmath}[label=phi-zz]
\vect\phi_{z,z}\fn{\vect \delta} =
\begin{bmatrix}
\half\gp{v^2-w^2}\Log{r-u}+uv\Log{r-v}+vw\ArcTan{\tfrac{uv}{rw}}+\half ru \\
\half\gp{u^2-w^2}\Log{r-v}+uv\Log{r-u}+uw\ArcTan{\tfrac{uv}{rw}}+\half rv \\
-uw\Log{r-u}-vw\Log{r-v}+uv\ArcTan{\tfrac{uv}{rw}}-rw
\end{bmatrix}
\end{dmath}
Note that when evaluating these functions, two numerical singularities must be accounted for:
\begin{align}
\lim_{x\to 0} x \log x &= 0 , & \lim_{x\to 0} \arctan(x/x) &= 0.
\end{align}

The stiffness characteristics can be derived by differentiating \eqref{akoun} with respect to displacement in each respective direction, resulting in
\begin{dmath}[label=akounk]
\vect K_{z,z} = \frac{\magn1\magn2}{4\pi\permVac} \sum_{(i,j,k,l,p,q)\in\{0,1\}^6} \vect k_{z,z}\fn{u_{ij},v_{kl},w_{pq},r}
\cdot \gp{-1}^{i+j+k+l+p+q} ,
\end{dmath}
where
\begin{dmath}
\vect k_{z,z} =
\begin{bmatrix}
-\frac{v u^2}{u^2+w^2}-r-v \Log{r-v} \\
-\frac{u v^2}{v^2+w^2}-r-u \Log{r-u} \\
 \frac{v w^2}{u^2+w^2}
  + \frac{u w^2}{v^2+w^2}
  + 2r+u\Log{r-u}+v \Log{r-v}
\end{bmatrix}
\end{dmath}
Note that the sum of the stiffness components ${K_{z,z}}_x+{K_{z,z}}_y+{K_{z,z}}_z=0$ follows from Earnshaw's theorem
\cite{earnshaw1842} following from the solution to Laplace's equation.

Allag and Yonnet \cite{allag2009-ietm} have published a description of how torques between cuboid magnets may be analytically calculated in a similar manner.
It is instructive to analyse their approach to the theory used to derive these equations, as it will be shown below that it is incorrect.
A short time after this paper was published a separate research group published the \emph{correct} equation \cite{janssen2010-ietm} for calculating torques between (anti-)parallel magnets.

The assertion given by Allag and Yonnet is that since the `forces' between the nodes of the magnets can be calculated, these corner forces can be used to calculate the applied torque on the magnets. Their expression for the torque (on the second magnet) can be written as per the style above in the following form.
\begin{equation}\eqlabel{torque}
\vect T_{z,z}=\frac{\magn1\magn2}{4\pi\mu_0}
  \sum_{i,j,k,l,p,q\in\{0,1\}^6}
  \vect\psi_{i,j,k,l,p,q}
  \cdot
  \gp{-1}^{i+j+k+l+p+q} ,
\end{equation}
where
\begin{equation}
\vect\psi_{i,j,k,l,p,q} = \vect R_{j,l,q}\vect\times \vect \phi_{z,z}\fn{\vect \delta_{i,j,k,l,p,q}} .
\eqlabel{torque-inner}
\end{equation}
The interpretation of \eqref{torque} and \eqref{torque-inner} is that the inner terms of the summation represent the torque applied on the second magnet due to the influence of one corner of the first magnet and one corner of the second magnet. Allag and Yonnet write this slightly differently in terms of the total torque applied from the entire first magnet on each corner of the second:
\begin{equation}\eqlabel{torqu-1}
\vect T = \sum_{j,l,q\in\{0,1\}^3} \vect R_{j,l,q}\vect\times \vect f_{j,l,q} ,
\end{equation}
where $\vect f_{j,l,q}$ is the summed force on a corner of the second magnet due to every corner of the first magnet, given by
\begin{equation}\eqlabel{torqu-2}
\vect f_{j,l,q}=\tfrac{\magn1\magn2}{4\pi\mu_0}
  \!\!\!\sum_{i,k,p\in\{0,1\}^3}\!\!\!
  \vect\phi_{z,z}\fn{\vect d_{i,j,k,l,p,q}}
  \cdot
  \gp{-1}^{i+j+k+l+p+q}.
\end{equation}
In other words, written in this form the `corner torque' for each node of the second magnet is given by
\begin{equation}\eqlabel{corner-torques}
  \vect \tau_{j,l,q} = \vect R_{j,l,q} \vect\times \vect f_{j,l,q}.
\end{equation}
It is easily seen that \eqref{torqu-1} and \eqref{torqu-2} are equivalent to \eqref{torque} since the inner summation in the former can be migrated out from inside the cross product.

In the interests of clarify, it should be noted that what has been referred to until now as a corner force is in fact not actually a force; rather, it is simply a mathematical abstraction (a limit of an integral) that has no basis in the physics of the situation.


\subsection{Non-symmetric corner forces}

Writing the interaction between two corners as $(i,k,p)\to(j,l,q)$, consider two geometrically-symmetrical cases from the previous example with zero horizontal displacement:
\begin{itemize}
\item $(0,0,0)\to(1,0,0)$, and
\item $(1,0,0)\to(0,0,0)$.
\end{itemize}
These two interactions are depicted in \figref{mag-nodes}.

\begin{figure}[t]
\centering
\asyinclude{PhD/Figures/Magnets/mag-nodes}
\caption{Two geometrically-symmetrical node interactions.}
\figlabel{mag-nodes}
\end{figure}

\subsection{Corner forces vs.\ distance}

Consider a fixed magnet of size $[2a,2b,2c]\T$ reacting with another magnet of some fixed vertical distance away. Compare two cases for the second magnet:
\begin{itemize}
\item	of equal size with $[2A, 2B, 2C]\T=[2a, 2b, 2c]\T$, and
\item	of greater size with $[2A, 2B, 2C]\T=[6a, 6b, 2c]\T$.
\end{itemize}
These two cases are shown in \figref{nodes-bigsmall} highlighting the interaction between two equivalent corners.

\begin{figure}
\centering
\asyinclude{PhD/Figures/Magnets/nodes-bigsmall}
\caption{Two node interactions for a smaller magnet and a larger magnet.}
\figlabel{nodes-bigsmall}
\end{figure}


\subsection{Forces between orthogonal cuboid magnets}

Two groups of researchers simultaneously published, in the same journal, equivalent methods to calculate the force between orthogonal cuboid magnets \cite{janssen2009-sensorletters,allag2009-sensorletters}.
The expressions of \textcite{allag2009-sensorletters} are slightly simpler\footnote{Also note a typographical error in the equations of \textcite{janssen2009-sensorletters}.} and are reproduced here for completeness.
The signs of these equations has been reversed for consistency with \eqref{akoun} in which the equations calculate the force on the second magnet.

\begin{dmath}[label=orth-magforce]
\vect F_{z,y} = \frac{\magn1\magn2}{4\pi\mu_0} \sum_{i,j,k,l,p,q\in\{0,1\}^6} \vect f_{zy}\fn{\vect \delta}\cdot\gp{-1}^{i+j+k+l+p+q}
\end{dmath}
Again, the distance between the `corner nodes' of each magnet is given by
\begin{equation}
\vect\delta_{i,j,k,l,p,q}=\vect d+\vect R_{j,l,q} - \vect r_{i,k,p} ~,
\end{equation}
where
\begin{align}
\vect R_{j,l,q}&=\begin{bmatrix}A\gp{-1}^j\\B\gp{-1}^l\\C\gp{-1}^q\end{bmatrix},&
\vect r_{i,k,p}&=\begin{bmatrix}a\gp{-1}^i\\b\gp{-1}^k\\c\gp{-1}^p\end{bmatrix}.
\end{align}
with magnet dimensions defined as earlier.
The $\vect f_{zy}$ terms required for calculating the `force between nodes' for orthogonal magnets can now be written as:
\begin{dgroup}
\begin{dmath}
\phi_x\fn{\vect \delta} = vw\Log{r-u}-uv\Log{r+w}-uw\Log{r+v}+\half u^2\ArcTan{\frac{vw}{ru}}+\half v^2\ArcTan{\frac{uw}{rv}}+\half w^2\ArcTan{\frac{uv}{rw}}
\end{dmath},
\begin{dmath}
\phi_y\fn{\vect \delta} = -\half\gp{u^2-v^2}\Log{r+w}+uw\Log{r-u}+uv\arctan\fn{\tfrac{uw}{rv}}+\half rw
\end{dmath},
\begin{dmath}
\phi_z\fn{\vect \delta} = -\half\gp{u^2-w^2}\Log{r+v}+uv\Log{r-u}+uw\arctan\fn{\tfrac{uw}{rv}}+\half rv
\end{dmath}.
\end{dgroup}



\subsection{Simplified force and stiffness expression for cube magnets}
\seclabel{cube-forces}

The function $\magforce$ is the simplication of \eqref{akoun} formula for cube magnets, where $\mdim$ is the side length, $\ndisp=\gamma/\mdim$ is the normalised vertical displacement, and $\magn1$ and $\magn2$ are the remanence magnetisations of the two magnets:
\begin{dmath}[label=magforce]
\magforce = \mdim^2 \nforce
\end{dmath},
where
\begin{dmath}[label=nforce]
  \nforce = \frac{\magn1\magn2}{\pi\permVac} \baraccent{\nforce}
\end{dmath},
and
\begin{footnotesize}
\begin{dmath}
  \baraccent{\nforce} = \gp{-2+\ndisp}\cdot\Abs{-2+\ndisp}-2 \ndisp \Abs{\ndisp}+\gp{2+\ndisp}\cdot
  \Abs{2+\ndisp}+4 \ndisp \sqrt{4+\ndisp^2}
  -2 \ndisp \sqrt{8+\ndisp^2}+\gp{4-2 \ndisp} \sqrt{8-4
    \ndisp+\ndisp^2}+\gp{-2+\ndisp} \sqrt{12-4 \ndisp+\ndisp^2}
  +\gp{-4-2 \ndisp} \sqrt{8+4 \ndisp+\ndisp^2}+\gp{2+\ndisp}
  \sqrt{12+4 \ndisp+\ndisp^2}
  +2 \left[ 4 \mathatan\left[\frac{4}{\ndisp \sqrt{8+\ndisp^2}}\right]+2
    \mathatan\left[\frac{4}{\gp{2-\ndisp}
        \sqrt{12-4 \ndisp+\ndisp^2}}\right]\right.
  -2 \mathatan\left[\frac{4}{\gp{2+\ndisp} \sqrt{12+4 \ndisp+\ndisp^2}}\right]+2 \ndisp
  \mathlog\left[-2+\sqrt{4+\ndisp^2}\right]
  -2 \ndisp \mathlog\left[2+\sqrt{4+\ndisp^2}\right]-2 \ndisp
  \mathlog\left[-2+\sqrt{8+\ndisp^2}\right]
  +2 \ndisp \mathlog\left[2+\sqrt{8+\ndisp^2}\right]+2
  \mathlog\left[-2+\sqrt{8-4
      \ndisp+\ndisp^2}\right]
  -\ndisp \mathlog\left[-2+\sqrt{8-4 \ndisp+\ndisp^2}\right]-2
  \mathlog\left[2+\sqrt{8-4 \ndisp+\ndisp^2}\right]
  +\ndisp \mathlog\left[2+\sqrt{8-4 \ndisp+\ndisp^2}\right]-2
  \mathlog\left[-2+\sqrt{12-4 \ndisp+\ndisp^2}\right]
  +\ndisp \mathlog\left[-2+\sqrt{12-4 \ndisp+\ndisp^2}\right]+2
  \mathlog\left[2+\sqrt{12-4 \ndisp+\ndisp^2}\right]
  -\ndisp \mathlog\left[2+\sqrt{12-4 \ndisp+\ndisp^2}\right]-2
  \mathlog\left[-2+\sqrt{8+4 \ndisp+\ndisp^2}\right]
  -\ndisp \mathlog\left[-2+\sqrt{8+4 \ndisp+\ndisp^2}\right]+2
  \mathlog\left[2+\sqrt{8+4 \ndisp+\ndisp^2}\right]
  +\ndisp \mathlog\left[2+\sqrt{8+4 \ndisp+\ndisp^2}\right]+2
  \mathlog\left[-2+\sqrt{12+4 \ndisp+\ndisp^2}\right]
  +\ndisp \mathlog\left[-2+\sqrt{12+4 \ndisp+\ndisp^2}\right]-2
  \mathlog\left[2+\sqrt{12+4 \ndisp+\ndisp^2}\right]
  \left.  -\ndisp \mathlog\left[2+\sqrt{12+4 \ndisp+\ndisp^2}\right]\right]
\end{dmath}
\end{footnotesize}

The stiffness $\magstiffness$ is calculated by differentiating the force equations given by \textcite{akoun1984} before simplifying as with the force terms above:
\begin{dmath}[label=magstiffness]
\magstiffness = \mdim \nstiffness
\end{dmath},
where
\begin{dmath}[label=nstiffness]
  \nstiffness = -\frac{2\magn1\magn2}{\pi\permVac} \baraccent{\nstiffness}
\end{dmath},
and
\begin{footnotesize}
\begin{dmath}
  \baraccent{\nstiffness} = \Abs{-2+\ndisp}-2 \Abs{\ndisp}+\Abs{2+\ndisp}+4
  \sqrt{4+\ndisp^2}-2\sqrt{8+\ndisp^2}
  -2 \sqrt{8-4 \ndisp+\ndisp^2}+\sqrt{12-4 \ndisp+\ndisp^2}-2 \sqrt{8+4\ndisp+\ndisp^2}
  +\sqrt{12+4 \ndisp+\ndisp^2}+2 \mathlog\left[-2+\sqrt{4+\ndisp^2}\right]
  -2\mathlog\left[2+\sqrt{4+\ndisp^2}\right]
  -2\mathlog\left[-2+\sqrt{8+\ndisp^2}\right]
  +2\mathlog\left[2+\sqrt{8+\ndisp^2}\right]
  -\mathlog\left[-2+\sqrt{8-4 \ndisp+\ndisp^2}\right]
  +\mathlog\left[2+\sqrt{8-4\ndisp+\ndisp^2}\right]
  +\mathlog\left[-2+\sqrt{12-4\ndisp+\ndisp^2}\right]
  -\mathlog\left[2+\sqrt{12-4\ndisp+\ndisp^2}\right]
  -\mathlog\left[-2+\sqrt{8+4 \ndisp+\ndisp^2}\right]
  +\mathlog\left[2+\sqrt{8+4\ndisp+\ndisp^2}\right]
  +\mathlog\left[-2+\sqrt{12+4\ndisp+\ndisp^2}\right]
  -\mathlog\left[2+\sqrt{12+4 \ndisp+\ndisp^2}\right]
\end{dmath}
\end{footnotesize}

These simplified equations are printed here to emphasise the $\mdim^2$ relationship for the force shown in \eqref{magforce} and the $\mdim$ relationship for the stiffness in \eqref{magstiffness}.
This is interesting because it is not evident from Akoun~and Yonnet's original equations that such a simplification is possible.

\subsection{Cuboid magnets with arbitary magnetisations}
\seclabel{magforce-arbitary}

Most force expressions are derived from magnetic field equations that are assumed from magnets with magnetisation parallel to one of their sides.
Superposition can then be used to combine the expressions for orthonogal magnets to generate the force from a magnet with arbitrary magnetisation.
\textcite{ravaud2009-pier98} instead show the magnetic field equations for a cuboid magnet with arbitrary magnetisation; their work is still to be extended to calculate the forces between such magnets.
Since their equation for calculating the magnetic field is necessarily more complex, it is not clear whether an equation derived using analytical integration to calculate the force directly (if the integral is even tractable) will be more efficient than the superposition approach outlined below.

The geometry of the two-magnet system is shown in \figref{akoun}, in which the magnets have side lengths $\vect s = [2a, 2b, 2c]\T$ and $\vect S = [2A, 2B, 2C]\T$ respectively and the distance between their centres is given by $\vect d=[\alpha,\beta,\gamma]\T$. The calculations always assume that the first magnet is fixed and force is acting on the second magnet. The signs must be reversed to obtain the forces acting on the first magnet.

As shown earlier in \eqref{akoun},
\textcite{akoun1984} provide the force expressions for magnets with vertical magnetisations.
This force is denoted below as $\vect F_{z,z}\fn{\vect s, \vect S, \vect d, \magn1, \magn2}$ as a function of the magnet sizes, the distance between them, and their magnetisation magnitudes $\magn1$ and $\magn2$.
From \eqref{orth-magforce}, \textcite{allag2009-electromotion} provide the force expressions for the first magnet with vertical magnetisation and the second magnet with magnetisation in the horizontal $y$-direction.
This force is denoted below as $\vect F_{z,y}\fn{\vect s, \vect S, \vect d, \magn1, \magn2}$.

The force between a vertically-magnetised magnet and one with magnetisation in the horizontal $x$-direction can be calculated by applied a rotational transformation to $\vect F_{z,y}$ around the $z$-axis.
That is,
\begin{equation}\eqlabel{fzx}
\vect F_{z,x}\fn{\vect s, \vect S, \vect d, \magn1, \magn2} = \mathbf R_z\fn{-\tfrac{\pi}2}\vect F_{z,y}\fn{\vect s_{z,x}, \vect S_{z,x},\vect d_{z,x}, \magn1, \magn2} ,
\end{equation}
where
\begin{align}
\vect s_{z,x} &= \Abs{\mathbf R_z\fn{\tfrac\pi2}\vect s}, \\
\vect S_{z,x} &= \Abs{\mathbf R_z\fn{\tfrac\pi2}\vect S}, \\
\vect d_{z,x} &= \mathbf R_z\fn{\tfrac\pi2}\vect d,
\end{align}
for which $\Abs{\cdot}$ is the \emph{element-wise} absolute value function and $\mathbf R_z\fn{\theta}$ is the rotation matrix around the $z$-axis:
\begin{equation}
\mathbf R_z\fn\theta = \begin{bmatrix}
\cos\theta & -\sin\theta & 0 \\
\sin\theta & \hphantom{-{}}\cos\theta & 0 \\
0 & 0 & 1
\end{bmatrix}.
\end{equation}

Using the force expressions $\vect F_{z,x}$, $\vect F_{z,y}$, and $\vect F_{z,z}$ in superposition allow the force to be calculated between a vertically magnetised magnet and another magnet with arbitrary magnetisation direction. By applying coordinate system transformations to these expressions, arbitrary magnetisation directions can be achieved for the first magnet as well.

For horizontal $x$-direction magnetisation,
\begin{equation}\eqlabel{fxxyz}
\vect F_{x,\{x,y,z\}}\fn{\vect s, \vect S, \vect d, \magn1, \magn2} =
  \mathbf R_y\fn{\tfrac\pi2}
  \vect F_{z,\{x,y,z\}}\fn{\vect s_x, \vect S_x, \vect d_x, \magn1, \magn2}
\end{equation}
where
\begin{align}
\vect s_x &= \Abs{\mathbf R_y\fn{-\tfrac\pi2}\vect s}, \\
\vect S_x &= \Abs{\mathbf R_y\fn{-\tfrac\pi2}\vect S}, \\
\vect d_x &= \mathbf R_y\fn{-\tfrac\pi2}\vect d,
\end{align}
and $\mathbf R_y\fn{\theta}$ is the rotation matrix around the $y$-axis:
\begin{equation}
\mathbf R_y\fn\theta = \begin{bmatrix}
\cos\theta & 0 & -\sin\theta \\
0 & 1 & 0 \\
\sin\theta & 0 & \hphantom{-{}}\cos\theta \\
\end{bmatrix}.
\end{equation}
Similarly, for horizontal $y$-direction magnetisation,
\begin{equation}\eqlabel{fyxyz}
\vect F_{y,\{x,y,z\}}\fn{\vect s, \vect S, \vect d, \magn1, \magn2} = \\
  \mathbf R_x\fn{-\tfrac\pi2}
  \vect F_{z,\{x,y,z\}}\fn{\vect s_y, \vect S_y, \vect d_y, \magn1, \magn2}
\end{equation}
where
\begin{align}
\vect s_y &= \Abs{\mathbf R_x\fn{\tfrac\pi2}\vect s}, \\
\vect S_y &= \Abs{\mathbf R_x\fn{\tfrac\pi2}\vect S}, \\
\vect d_y &= \mathbf R_x\fn{\tfrac\pi2}\vect d,
\end{align}
and $\mathbf R_x\fn{\theta}$ is the rotation matrix around the $x$-axis:
\begin{equation}
\mathbf R_x\fn\theta = \begin{bmatrix}
1 & 0 & 0 \\
0 & \cos\theta & -\sin\theta \\
0 & \sin\theta & \hphantom{-{}}\cos\theta \\
\end{bmatrix}.
\end{equation}

Given the results of the afore-referenced papers by Yonnet et al. and \eqref{fzx,fxxyz,fyxyz}, the force between two magnets of arbitrary magnetisation can be written as
\begin{equation}\eqlabel{total-force}
\vect F\fn{\vect s, \vect S, \vect d,\expandafter\vect \magn1,\expandafter\vect \magn2}=\sum_{i,j\in\{x,y,z\}^2} \vect F_{i,j}\fn{\vect s, \vect S, \vect d, J_{1_i}, J_{2_j}}
\end{equation}
where
\begin{align}
\expandafter\vect \magn1 &= [J_{1_x},J_{1_y},J_{1_z}]\T, &
\expandafter\vect \magn2 &= [J_{2_x},J_{2_y},J_{2_z}]\T.
\end{align}
Although it is well known that the principle of superposition can be used in this way,
this is the first formalisation of this theory to decompose a diagonal magnetisation into its orthogonal components for calculating the forces between diagonally-polarised magnets.




\subsection{Forces between magnets with relative rotation}
\seclabel{french-equations}

In 1999, a slew of papers were published by researchers at \emph{Laboratoire
d'Electrotechnique et de Magnétisme de Brest} investigating the forces between
non-contact magnetic rotational force couplings.
These are of interest to this
work because they use an analytical expression for the forces between two
cuboidal magnets under arbitrary translation, with one inclined at any angle
around the \x-axis.
Three papers were published \cite{elies1998,charpentier1999-ietm-mar,charpentier1999-ietm-sep}
that all contain the force expression of interest, with a fourth \cite{elies1998} containing just the force expression in a single direction.

Their expressions are re-written here because each separate publication contains different typographical errors.
The equations here have been reconstructed by comparing the differences and similarities between the equations in the different papers, and re-written in a more compact form.
Given two magnets located in \threeD/ space, of sizes $\inlinevect{A,B,C}$ and $\inlinevect{a,b,c}$, with the plane of the second rotated by $\theta$ around the \x-axis and their origins offset by $\inlinevect{x_0,y_0,z_0}$, the forces in the \y- and \z-directions ($F_y$ and $F_z$) between the two magnets can be calculated.

Forces in both the $y$ and $z$ direction use the $f_3$ function in their calculations.
\begin{dmath}
F_y = \frac{\magn1\magn2}{4\pi\permVac}
  \sum_{i_{a,b,c,A,B,C}\in\{0,1\}}
  f_{y_2}\cdot\gp{-1}^{i_a+i_b+i_c+i_A+i_B+i_C}
\end{dmath},
\begin{dmath}
F_z\bigg|_{\theta\neq k\pi} =
       \frac{-\magn1\magn2}{4\pi\permVac}\cdot\smash{\sum_{i_{a,b,c,A,B,C}\in\{0,1\}}}
        f_{z_2}\cdot\gp{-1}^{i_a+i_b+i_c+i_A+i_B+i_C}
\end{dmath},
\begin{dmath}
f_{y_2} = f_3\fn{u_0 , y_0 , z_0 , \theta , c i_c , C i_C}
\end{dmath},
\begin{dmath}
f_{z_2} =  \frac{f_3\fn{u_1,v_1,w_1,-\theta,0,0}}{\sin\theta}
         + \frac{f_3\fn{u_2,v_2,w_2,\theta,0,0}}{\tan\theta}
\end{dmath},
\begin{dgroup}
\begin{dmath}
u_0 = x_0 - ai_a + Ai_A
\end{dmath},
\begin{dmath}
u_1 = u_0-2x_0
\end{dmath},
\begin{dmath}
v_1 = -v_2\cos\theta - w_2\sin\theta
\end{dmath},
\begin{dmath}
w_1 = v_2\sin\theta - w_2\cos\theta
\end{dmath},
\begin{dmath}
v_2 = y_0-Ci_C\sin\theta
\end{dmath},
\begin{dmath}
w_2 = z_0-ci_c+Ci_C\cos\theta
\end{dmath}.
\end{dgroup}
This is the auxiliary function used in the above. All dashed variables are
local to this function.
\begin{dgroup*}
\begin{dmath}
f_3\fn{u',v',w',\theta',c',C'} =
  u' f_5\gp{\Log{f_4-u'}-1}
  +\half\gp{f_6^2-{u'}^2}\Log{f_4+f_5}
  +\half u' \pi \Sign{f_5}\Abs{f_6}
  +u' f_6 \ArcTan{\frac{u' f_4 - u^2 -f_6^2}{f_5 f_6}}
  +\half f_4 f_5
\end{dmath},
\begin{dmath}
f_4 = \sqrt{{u'}^2+f_5^2+f_6^2}
\end{dmath},
\begin{dmath}
f_5 = \gp{v'-b i_b}\cos\theta'+\gp{w'-c'}\sin\theta'+B i_B
\end{dmath},
\begin{dmath}
f_6 = -\gp{v'-b i_b}\sin\theta'+\gp{w'-c'}\cos\theta'+C'
\end{dmath}.
\end{dgroup*}

The force in the $z$ direction is calculated separately if the second
magnet is not rotated around its axis ($f_{z_2}$ has a singularity
at $\theta=k\pi$).
\begin{dgroup}
\begin{dmath}
F_z\bigg|_{\theta=k\pi} =
  \cos\theta\cdot\frac{-\magn1\magn2}{4\pi\permVac}
  \sum_{i_{a,b,c,A,B,C}\in\{0,1\}}f_{z_1}
  \cdot\gp{-1}^{i_a+i_b+i_c+i_A+i_B+i_C}
\end{dmath},
\begin{dmath}
f_{z_1} =
  \half uw\Log{\frac{r+u}{r-u}}+\half vw\Log{\frac{r+v}{r-v}}+
  uv\ArcTan{\frac{uv}{wr}-wr}
\end{dmath},
\begin{dmath}
u = x_0-ai_i+Ai_A
\end{dmath},
\begin{dmath}
v = y_0-bi_b+Bi_B+\half B\gp{\cos\theta-1}
\end{dmath},
\begin{dmath}
w = z_0-ci_c+Ci_C+\half C\gp{\cos\theta-1}
\end{dmath},
\begin{dmath}
r = \sqrt{u^2+v^2+w^2}
\end{dmath}.
\end{dgroup}
This equation is very similar, but not identical, to Akoun~\& Yonnet's
equation for the same (\eqref{akoun}).

In order to allow this equation to be used for rotations greater than \SI{\sipi}{radians}, additional terms have been added to accommodate for the translational offset induced by rotating the magnet around its lower-left edge.
The preceding $\cos\theta$ provides the sign change of the magnetisation, and the $\half\gp{\cos\theta-1}$ terms are used to return the origin of the magnet after 180\textdegree\ rotation to the lower-left corner, where it is expected.

As an example of using these equations to calculate forces as a function of magnet rotation, \figref{charp-rotate} shows the forces produced between two \SI{10}{mm} cube magnets as a function of rotation angle $\theta$ of the second magnet, with a \SI{20}{mm} offset between their centres.
Two cases are shown: in the first, the magnets are displaced vertically; in the second, the magnets are displaced horizontally.

\begin{figure}
  \begin{wide}
  \begin{subfigure}
    \psfragfig{\phdpath Simulations/Theory/latex/charp-rotation-forces}
    \caption{Vertical displacement.}
  \end{subfigure}
  \hfil
  \begin{subfigure}
    \psfragfig{\phdpath Simulations/Theory/latex/charp-horiz-rotation-forces}
    \caption{Horizontal displacement.}
  \end{subfigure}
  \hfil
  \null
  \end{wide}
  \caption{Vertical and horizontal forces on a rotating magnet
    as a function of rotation angle for fixed horizontal and vertical displacements.}
  \figlabel{charp-rotate}
\end{figure}

\Figref{charp-rotate} shows that the maximum force between two magnets is exhibited when the direction of displacement is in the same direction as their magnetisations.
For displacements perpendicular to the direction of magnetisation of the fixed magnet, forces of equal magnitude are obtained for rotation of the second magnet $\theta=k\pi/2$ for $k\in\{0,1,2,\dots\}$.

In the following simulations, the second magnet is held at a constant height (one magnet height's separation distance) and moved from left to right over a range of twice the magnet width symmetrically above the first magnet (with respect to the magnets' centres of gravity).
Four such displacements are made with four different rotations for the second magnet: $0$, $\half\pi$, $\pi$ and $-\half\pi$.
\Figref{charp-0to2pi} shows the forces in the horizontal \y-direction and vertical \z-direction, \resp.
It can be seen that the opposite rotations result in symmetric forces curves, as is expected.

\begin{figure}
  \begin{wide}
  \hspace{-1cm}%
  \begin{subfigure}
    \psfragfig{\phdpath Simulations/Theory/latex/charp-0to2pi-Fy}
    \caption{Horizontal forces.}
    \figlabel{charp-0to2pi-Fy}
  \end{subfigure}\hfil
  \begin{subfigure}
    \psfragfig{\phdpath Simulations/Theory/latex/charp-0to2pi-Fz}
    \caption{Vertical forces.}
    \figlabel{charp-0to2pi-Fz}
  \end{subfigure}
  \end{wide}
  \caption{Vertical and horizontal forces on a rotating magnet situated directly above another.}
  \figlabel{charp-0to2pi}
\end{figure}


\section{Forces between cylindrical magnets}
\seclabel{cyl-forces}

Ravaud et al.~\cite{ravaud2010-ietm} recently published an expression for the forces between two cylindrical magnets or thin coils (which are equivalent electromagnetically). At part of the work of these thesis, a simplification of their equation has been presented \parencite{robertson2011-ietm}.
This simplification results in a faster execution time and more convenient calculation with numerical software.

In previous literature on modelling the forces between cylindrical magnets,
Nagaraj~\cite{nagaraj1988} investigated and compared the force between cuboid and cylindrical magnets with arbitrary displacements using numerical integration to calculate his results; Furlani~\cite{furlani1993-ietm,furlani1993-ietm-coupl} calculated the force between radially-aligned ring magnets using a numerical discretisation of the magnet volume using theory developed in more detail in his book~\cite{furlani2001-magnetbook}. Hull et al.~\cite{hull1999-japplphys} presented integral equations for calculating the radial and axial forces between a cylindrical magnet and a superconductor, which is equivalent to the force between two cylindrical magnets, and \textcite{bassani2006-trib-int} presented integral equations for calculating the radial and axial forces between ring magnets. Such integral equations require numerical methods to evaluate.
Most recently, \textcite{ravaud2010-ietm} derived a closed-form solution using elliptic integrals for the forces between radially-aligned cylindrical magnets; their result is the most straightforward method yet presented for calculating forces in this configuration.

\begin{figure}
\centering
\asyinclude{PhD/Figures/Coil/coil-mag-equiv.asy}
\caption
[The equivalence between a permanent magnet and a current-carrying coil.]
{The equivalence between a permanent magnet of magnetisation $J$ (left) in the positive vertical direction, and a current-carrying coil (right) with equivalent magnetisation $J_{\text{eq.}}=\mu_0 N I/h$ for current $I$ shown flowing anti-clockwise from the top through $N$ axial turns across height $h$.}
\figlabel{coil-mag-equiv}
\end{figure}

The equation for the force between cylindrical magnets can also be used to calculate the force between thin coils with many axial turns, as both magnet and coil can be modelled as a surface current density around a cylinder (see \figref{coil-mag-equiv}). In related work, Kim et al.~\cite{kim1996-ietm} presented a different integral equation for the radial force between (single-turn) circular coils with eccentric radial displacement, for which further application of their results is required to calculate the forces between coils with many turns, such as for the system examined here. An expression for the force between thin coaxial coils has also been published by Babic et al.~\cite{babic2008-ietm}; it too is more complex than the expression to be presented in the current work.

The equation of Ravaud et al.~\cite{ravaud2010-ietm} is suitable for use for magnets or coils of any size with no restriction on axial displacement. For magnets, they are assumed to have constant magnetisation that is strong enough such that adjacent magnets will not affect the magnetisation of each other. For coils, they are modelled as surface current densities only; therefore for accurate coil calculations the radial thickness of the coil must be much smaller than its radius and the coil must be wound tightly in the axial direction. The force between thick coils with many radial turns has been shown also by Ravaud~et al.~\cite{ravaud2010-pier}; this expression requires some numerical integration and is accordingly slower to calculate than the expression for thin coils discussed herein.

The system consists of two coaxial cylindrical magnets or current-carrying coils which have a relative axial displacement between them, as shown in \figref{cyl-schem}.
A three-dimensional schematic is shown later in \figref{coil-param} that also illustrates the equivalence between a magnet and a coil in this system.

\begin{figure}
\centering
\asyinclude{PhD/Figures/Systems/cyl.asy}
\lofcaption{Two-dimensional side view of the system composed of two coaxial cylindrical magnets with a generated force on the second magnet.}
{
  (While magnets are shown, either or both may be replaced by a thin coil as shown in \figref{coil-mag-equiv}.)
  Axial displacement between the magnets may be positive or negative, and their volumes may overlap in the case of a magnet located inside a coil.
  Arrows within the magnets indicate direction of magnetic polarisation.
}
\figlabel{cyl-schem}
\end{figure}

\def\m#1{m_{#1}}
The force equation is given by~\cite{robertson2011-ietm}
\begin{dmath}[label=simpl4]
F_z = \frac{\magn1 \magn2}{2\mu_0} \sum_{i=1}^2 \sum_{j=3}^4 \m1\m2\m3 f_z \gp{-1}^{i+j}
\end{dmath},
where the intermediate expression is defined in terms of complete elliptic integrals of the first, second, and third kind ($\EllipticK{m}$, $\EllipticE{m}$, and $\EllipticPi{n,m}$, respectively)
\begin{dmath}[label=simpl4i]
f_z=
  \EllipticK{\m4}
  - \frac{1}{\m2}\EllipticE{\m4}
  +
\gdef\finalterm{
  \gp[2]{\frac{\m1^2}{\m3^2}-1} \invtimes
    \EllipticPi{\frac{\m4}{1-\m2},\m4}
}\finalterm
\end{dmath},
with parameters
\begin{align}
\m1 &= z_i - z_j, \\
\m2 &= \frac{\gp{r_1-r_2}^2}{\m1^2}+1,\\
\m3^2 &= \gp{r_1+r_2}^2+\m1^2, \\
\m4 &= \frac{4 r_1 r_2}{\m3^2}, \qquad 0<\m4\le 1.
\end{align}
This equation is particularly efficient to calculate as the complete elliptic integrals of the first, second, and third kind can all be calculated simultaneously with a single iteration of the arithmetic-geometric mean approach \cite[\S19.8(i)]{DLMF2010}.


\subsection{Numerical evaluation of the axial force}
\seclabel{numer}

Numerical singularities occur when an expression is mathematically continuous and terms within the expression approach infinity; care must be taken when evaluating such expressions numerically.
There are two numerical singularities in \eqref{simpl4}.
The first occurs when the radii are equal such that $\m2=1$ and the following term disappears as $\EllipticPi{\pm\infty,m}=0$:
\begin{dmath}
\finalterm = 0 \condition*{\m2=1}
\end{dmath}.

The second numerical singularity occurs when the magnets/coils have coincident faces such that $\m1=0$ for some values of $i$ and $j$ in the double summation. In this case, the parameter $\m2$ contains the coefficient $1/\m1^2=1/0$. This singularity can be avoided entirely since coincident faces generate no component of force between them, and hence the entire intermediate expression within the summation $\m1\m2\m3 f_z'$ can be defined as zero when $\m1=0$.

\subsection{Implementation efficiency}

Evaluated in Mathematica (including branching to avoid singularities), \eqref{simpl4} took an average of 0.26\,ms on a notebook computer to calculate the force at a single location (10000 samples with random input variables). The original equation by \citeauthor{ravaud2010-ietm} in the same configuration evaluated in 2.2\,ms on average, which is over eight times slower than the new equation. For researchers performing design optimisations with variations over a large number of parameters, such an efficiency improvement is useful in minimising the total computation time of the optimisation process.



\section{Summary of the magnetic theory}

An overview of the literature has been presented for calculating the forces, torques, and stiffnesses between permanent magnet configurations of various geometries.
This theory is used extensively in the remainder of this work.
Some shortcomings in the literature have been highlighted but are not investigated further here.


\end{document}
