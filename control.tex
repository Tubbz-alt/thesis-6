%!TEX root = thesis.tex

\chapter{Control}
\chaplabel{control}

\chapterprecis{
  More unfinished work; the work here deals with an impossible control system
  to stabilise a `\qzs/' magnetic spring and use it for vibration isolation
  system. Two works are presented that correspond to papers presented at
  separate conferences; both deal with backstepping controllers for the
  aformentioned system, the second more complex and robust than the first.
  Neither have been shown to be effective in practise, however.
}

I have to fit this somewhere in here: ``die
Gleichgewichtzustandwiederherstellungsmoeglichkeit'' is German for
``the possibility of re-establishing a condition of equilibrium.''


\section{Stabilities of nonlinear springs}

To start with, look at Lyapunov stability. Our system is
\begin{dmath}[label=general-system]
\dot x = f(x,t),
\end{dmath}
with some conditions \cite[pg.\,22]{krstic1995}. Then any $V$ that
is positive definite and unbounded, and that satisfies
\begin{dmath}
\dot V = \pd{V(x)}{x}f(x,t) \leq -W(x) \leq 0
\end{dmath}
for the continuous function $W$ means that solutions of
\eqref{general-system} are globally uniformly bounded, and for an
equilibrium defined at $x=0$ is globally uniformly asymptotically
stable. By the way, $\lim_{t\approaches\infty}W(x(t))=0$ in this
case.

Lyapunov functions provide a powerful method for establishing the
stability of a system. The states of a damped vibratory system are
\begin{dgroup}[label=linear-vibra-sys]
\begin{dmath*}
  \dot x_1 = x_2
\end{dmath*},
\begin{dmath*}
  \dot x_2 = -\stiffnessLinear/\mass x_1 -\damping/\mass x_2
\end{dmath*},
\end{dgroup}
with displacement $x_1$, velocity $x_2$, stiffness $\stiffnessLinear$, damping $\damping$
and mass $\mass$. This system can be shown to be stable by defining
the Lyapunov function
\begin{dmath}
  \lyapunov = \half \lyaGainA x_1^2 + \half \lyaGainB x_2^2 
\end{dmath},
and showing that its time derivative is negative semi-definite:
\begin{dmath}
  \dot\lyapunov
   = \lyaGainA x_1 \dot x_1 + \lyaGainB x_2 \dot x_2 
   = -\lyaGainB(\damping/\mass) x_2^2 + 
      (\lyaGainA-\lyaGainB\stiffnessLinear/\mass) x_1 x_2 
\end{dmath}.
The first term here is always negative semi-definite, but the second term is a
little more ambiguous. One of the arts in Lyapunov functions is to recognise
that the gains in the Lyapunov function $\lyapunov$ may be arbitrarily chosen;
since the sign of $x_1 x_2$ can vary with time the second term cannot be
forced negative definite. Instead, by choosing
$\lyaGainA=\lyaGainB\stiffnessLinear/\mass$ this troublesome term disappears,
and the desired result of $\dot\lyapunov\le0$ is achieved; thus it is shown
that \eqref{linear-vibra-sys} is stable.

A linear spring has equivalent state equation $\dot x = -\stiffnessLinear x$.
To show stability, the Lyapunov function $L=\half x^2$ with derivative $\dot L
= x \dot x$ is used here and below:
\begin{dmath}
  \dot L = x \dot x = -\stiffnessLinear x^2 \le 0 \forall x .  
\end{dmath}

A quadratic spring (for example, as created in a zero stiffness
magnetic system) is only stable for $x<0$:
\begin{dmath}
  \dot x = \stiffnessQuad x^2 , \\
  \dot L = x \dot x = \stiffnessQuad x^3 \le 0 \text{~for~} x\le 0 .  
\end{dmath}

On the other hand, a cubic spring (a zero stiffness mechanical spring)
is stable:
\begin{dmath}
  \dot x = -\stiffnessDuffing x^3 , \\
  \dot L = x \dot x = -\stiffnessDuffing x^4 \le 0 \forall x .  
\end{dmath}

\fixme{draw accompanying graphs}

These are all intuitive results. Both the quadratic and cubic force
curves have a local zero stiffness at $x=0$. However, the quadratic
force curve has the advantage in practice (\fixme{crossref}) of
quasi-zero stiffness in all degrees of freedom. Could a virtual cubic
stiffness be applied with a controller to stabilise this system while
ensuring quasi-zero stiffness? The state equation is
\begin{dmath}
  \dot x = \stiffnessQuad x^2 -\stiffnessDuffing x^3 
\end{dmath}
with $\stiffnessQuad > 0$ and $\stiffnessDuffing$ freely able to be
chosen in the control law. The derivative of the Lyapunov function is
\begin{dmath}
  \dot L  = \stiffnessQuad x^3 -\stiffnessDuffing x^4 
          = x^2 \gp{\stiffnessQuad x -\stiffnessDuffing x^2} .
\end{dmath}.
Since $x^2\ge0$, for stability 
\begin{dmath}[compact]
  \stiffnessQuad x -\stiffnessDuffing x^2 \le 0
  \implies x\gp{\gamma-x} \le 0
\end{dmath}.
This is only true for $x\le0$ and $x\ge\gamma$, and this system is
therefore not stable for all $x$. However, there is a local maximum in
the region $0<x<\gamma$, and the addition of a further linear term to
the controller can have the effect of moving this global maxima below zero.

The state equation now contains the augmenting linear term
$\stiffnessLinear$ that can also be freely chosen in the control law:
\begin{dmath}
  \dot x = -\stiffnessLinear x + \stiffnessQuad x^2 - \stiffnessDuffing x^3. 
\end{dmath}
Similarly to before (\fixme{crossref??}), for stability
\begin{dmath}
  -\stiffnessDuffing x^2 + \stiffnessQuad x - \stiffnessLinear \le 0.
\end{dmath}
This is true for all $x$ for $\stiffnessQuad^2 \le
4\stiffnessLinear\stiffnessDuffing$ (no real roots of the quatratic
characteristic equation).

This analysis does not take into account factors such as knowledge of
the system parameters, so it is only the very beginning of a nonlinear
control law for a quadratic-force system. Furthermore, it shifts the
position of the local zero stiffness operating point. However, it does
demonstrate the feasibility of adding virtual stiffnesses to the
system in order to add stability without affecting the possibility of
quasi--zero stiffness.


\section{Backstepping}

This section is based on work presented at the Active 2006 conference,
where I first showed that nonlinear backstepping control of a zero
stiffness system, in theory, had some merit.

\subsection{Introduction}
In a large load vibration isolation system, significant static
forces are required to counter the effects of gravity. For a
conventional linear isolator, this compensating force is $F=-k x_0$,
for stiffness $k$ and static deflection $x_0$. Accordingly, for
large forces the stiffness and/or the static deflection must also be
large.

While low stiffness is advantageous for vibration isolation, there
is a practical lower limit on obtaining a stiffness to support a
large load. {The disadvantage of increasing the stiffness of the
system is due to the corresponding increase in the resonant
frequency $\natfreq=\sqrt{k/m}$, for mass $m$. This results in poor
passive vibration isolation for frequencies below
$\sqrt{2}\cdot\natfreq$.} Accordingly, a method of supporting large
loads while keeping the stiffness low is desirable. For an active
system, it will be shown that, in this case, the further advantage
is gained of reduced control effort in actuating the device.

In this paper, a permanent magnetic configuration is demonstrated
that can reduce the passive stiffness to zero at unstable
equilibrium while still providing a supporting force.  The magnetic
design is scalable, which provides for the capability of large load
bearing. Non-linear control laws are proposed to stabilise the
system and are demonstrated via simulation.  Finally, implementation
issues are discussed for a practical system.

\subsection{Permanent magnet springs}

The simplest form of magnetic suspension is the vertically
attractive pair in \figref{spring-vattr}, in which a fixed upper
magnet supports the lower in an unstable manner. Due to the inherent
instability resulting from negative stiffness and the nonlinear
forces involved, this system is often used for demonstrations of the
efficacy of active control techniques.

While all permanent magnet levitations are unstable by nature
\cite{earnshaw1842,tonks1940}, in a completely passive magnetic
spring instability in directions other than the supporting direction
can be controlled with rigid guides (for example, see the isolator
of \textcite{puppin2002}). Such a configuration,
shown in \figref{spring-vrepl}, consists of a vertical pair of
magnets in repulsion, with the lower magnet providing the supporting
forces.

Zero stiffness structures for vibration isolation from the ground
have been examined by \textcite{nijsse2001}, who introduced the
magnet arrangement shown in \figref{spring-vcomb}, a combination of
the two aforementioned magnetic springs. \textcite{xing2005}
have examined the general solution for feedback control systems
achieving zero and infinite stiffness. \iffalse The latter, for
rejection of direct disturbance on the supported stage, has been
researched by \textcite{mizuno2003b}, in which the series
combination of a conventional positive spring and a negative
magnetic spring results in theoretically infinite static stiffness
in total:
\begin{align*}
\frac{1}{k} & = \frac{1}{k_1} + \frac{1}{k_2} , &
\therefore k & = \frac{k_1k_2}{k_1+k_2} = \infty, \text{~if~} k_1+k_2=0 .
\end{align*}
\fi

This paper further develops the concept of zero stiffness through
the parallel combination of two magnetic springs, $k = k_1+k_2 = 0$,
via theoretical modelling and simulated nonlinear control.

\begin{figure}
 %\subfloat[Vertically attracting spring \figlabel{spring-vattr}]
 %  {\parbox[s][\heightof{\empuse{spring-vcomb}}]{0.3\linewidth}
 %    {\centering\empuse{spring-vattr}}}%
 %\subfloat[Vertically repelling spring \figlabel{spring-vrepl}]
 %  {\parbox[s][\heightof{\empuse{spring-vcomb}}]{0.3\linewidth}
 %    {\centering\vfill\empuse{spring-vrepl}}}%
 %\subfloat[s][Combination spring \figlabel{spring-vcomb}]
 %  {\parbox{0.3\linewidth}
 %    {\centering\empuse{spring-vcomb}}}%
  \caption{%
    Magnetic springs for bearing vertical loads. The shaded magnet
    in each is supported against gravity ($F_g$) by the respective
    magnetic forces ($F_{m_i}$). Arrows within the magnets indicate their
    directions of magnetisation.}
  \figlabel{vsprings}
\end{figure}

\subsection{Zero stiffness}

The term `zero stiffness' strictly denotes a decoupling between
displacement and force for two disparate objects in space;
\begin{dmath}[compact,label=zero-stiffness]
  \func\force\displacement=
  \lim_{\stiffness\approaches0}-\stiffness\displacement=0
\end{dmath}.
However, for any practical quasi-zero stiffness system (also see the
mechanically zero stiffness spring of \textcite{carrella2006}) the stiffness varies by position, so
perturbations from the equilibrium point eliminate the zero
stiffness property. On the other hand, small perturbations lead only
to small changes in the stiffness, which will therefore remain close
to zero.

Taking the inverse of \eqref{zero-stiffness} naïvely implies that
for a totally zero stiffness system, applied force will effect a
displacement without bound:
$\func\displacement\force=\lim_{\stiffness
  \approaches0}-\force/\stiffness=\infty$. In a more realistic case, the lower
the stiffness the more easily the system can be actuated.

\subsection{Permanent magnet design}

In this section, the stabilities through displacement in the load
bearing direction for the three springs shown in \figref{vsprings}
are analysed in order to show the desirable properties of the zero
stiffness magnetic spring.

\subsubsection{Mathematical models}

The solution shown previously in \eqref{akoun}, and others like it, provide a
convenient way to analyse the behaviour of any simple permanent magnet
configuration. Since the coercivity of rare earth magnetic material is
great enough to ensure that nearby magnets will not demagnetise each
other, the forces follow the principle of superposition. That is,
referring back to \figref{vsprings}, $F_{m_3}=F_{m_1}+F_{m_2}$.

\subsubsection{Stability analyses}

The stability of a magnetic configuration may be investigated by
examination of the stiffnesses in each direction, which may be
calculated via application of \eqref{akounk}. Stability exists for
negative force gradients (positive stiffness), where the reaction
forces act to oppose perturbatory displacements. In the following
analyses, cube magnets of side length 20\,mm are used with
magnetisations of~1\,T.

The stiffness in each translatory direction for the vertically
attracting spring (\figref{spring-vattr}) experiencing vertical
deflection is shown in \figref{K-V-attr}. It can be seen that this
spring is unstable (i.e., it has negative stiffness) in the vertical
direction, but stable (positive stiffness) in both horizontal
directions.

\begin{figure}[p]
  \begin{subfigure}
    \grf{Active2006/work/eps/active-vattr-kz}
    \caption{Vertical stiffness (unstable).\figlabel{K-V-attr-Z}}
  \end{subfigure}
  \hfill
  \begin{subfigure}
    \grf{Active2006/work/eps/active-vattr-ky}
    \caption{Horizontal stiffnesses (stable).\figlabel{K-V-attr-Y}}
  \end{subfigure}
  \caption{Stiffnesses of the vertically attracting spring (\figref{spring-vattr}).}
  \figlabel{K-V-attr}
\end{figure}

A similar analysis has been performed on the vertically repelling
spring (\figref{spring-vrepl}), for which the results are displayed
in \figref{K-V-repl}. These show opposite tendencies to the previous
spring: both horizontal directions are unstable, but the vertical,
load-bearing, direction is stable.

\begin{figure}[p]
  \begin{subfigure}
    \grf{Active2006/work/eps/active-vrepl-kz}
    \caption{Vertical stiffness (stable).\figlabel{K-V-repl-Z}}  
  \end{subfigure}
  \hfill
  \begin{subfigure}
    \grf{Active2006/work/eps/active-vrepl-ky}
    \caption{Horizontal stiffnesses (unstable).\figlabel{K-V-repl-Y}}
  \end{subfigure}
  \caption{Stiffnesses of the vertically repelling spring (\figref{spring-vrepl}).}
  \figlabel{K-V-repl}
\end{figure}

When these two springs are combined to create the spring shown in
\figref{spring-vcomb}, the stiffnesses in each bearing direction
have each a point of inflexion, as shown in \figref{K-V-comb}. This
inflexion, equidistant between the two magnets, is the point of zero
stiffness in both vertical and horizontal directions. The supporting
force at this point is a function of the gap. This relationship is
shown in \figref{active-zerok-fxrange}, which plots vertical force
vs.\ vertical displacement for a zero stiffness spring with varying
gaps.

The curves shown in \figref{active-zerok-fgap} demonstrate the
nominal load-bearing force of a zero stiffness spring with
increasing cube magnet side-lengths. This figure indicates that a
large design space is possible with appropriately chosen parameters.

\begin{figure}[p]
  \begin{subfigure}
    \grf{Active2006/work/eps/active-vzerok-kz}
    \caption{Stiffness in the vertical direction 
     (stable for negative displacements).\figlabel{K-V-zerok-Z}}
   \end{subfigure}
  \hfill
  \begin{subfigure}
    \grf{Active2006/work/eps/active-vzerok-ky}
    \caption{Stiffnesses in the horizontal directions 
     (stable for positive displacements).\figlabel{K-V-zerok-Y}}
  \end{subfigure}
  \caption{Stiffnesses of the zero stiffness spring (\figref{spring-vcomb}).}
  \figlabel{K-V-comb}
\end{figure}

\begin{figure}
  \begin{subfigure}
    \grf{Active2006/work/eps/active-zerok-fxrange}
    \caption{Force/displacement curves for the magnetic arrangement shown
        in \figref{spring-vcomb}. 20\,mm cube magnets
        are used with various gaps.
        \figlabel{active-zerok-fxrange}}  
  \end{subfigure}
  \par
  \begin{subfigure}
    \grf{Active2006/work/eps/active-zerok-fgap}
    \caption{Supporting force/gap curves for various magnet sizes
     at zero displacement. In each case, the gap range is 1--2 magnet dimensions.
     Increasing force results from increasing magnet size.
     \figlabel{active-zerok-fgap}}   
  \end{subfigure}
     \caption{Dependence of the supporting force on geometry of the
       zero stiffness spring.}
\end{figure}

\subsubsection{Development of more advanced magnetic geometries}

\fixme{refer back to the lit review here}

The results in \figref{K-V-repl} show that the vertically repelling
spring is unstable in both horizontal directions. Since
$k_x+k_y+k_z=0$, there are two options for the design of the
behaviour of the horizontal dynamics. Stabilisation in one of these
directions is possible through the addition of horizontally located
magnets \cite{robertson2005}. In this case, at rest, $k_z=0$ and
$k_y=-k_x>0$; \emph{passive} stabilisation in one horizontal degree
of freedom is achieved.

Alternatively, with the proposed combination (zero stiffness) spring
(\figref{spring-vcomb}), the horizontal stiffnesses in
\figref{K-V-zerok-Y} are also zero at the midpoint: $k_x=k_y=k_z=0$.
With stabilising controllers (equivalent to those shown later in
this paper) in every degree of freedom, an object may be supported
with zero stiffness in each translatory direction.

For increasing the load bearing ability of a magnetic spring, simple
scaling of the magnetic volume is inefficient \cite{yonnet1991}.
Greater forces over shorter distances may be achieved with multipole
(`Halbach') arrays, which are an approximation of a sinusoidally
polarised magnet \cite{robertson2005}. The tenet of zero stiffness
for such structures remains the same as for homogeneous magnets,
however, and these structures will not be analysed further here.

\subsection{Control strategies}

The basic zero stiffness configuration that is under examination is
marginally stable at its operating point. The stable effects of the
lower magnet and the unstable effects of the upper magnet combine to
produce a unique force/displacement characteristic (zero slope) as
seen in \figref{active-zerok-fxrange} on the previous page.

To a good approximation, over the range of several millimetres this
relationship may be modelled as quadratic: $F_m = K x^2 + F_0$, for
displacement $x$ from the rest position at the point of zero
stiffness, balanced midway between the two fixed magnets, with a
supporting force $F_0$. When summed with the load due to gravity,
this simplifies to $F = F_m - mg = K x^2$.

For this stage of the analysis, the damping of the system is
neglected, as its effect will be small. The approximate dynamic
equation of motion of the zero stiffness spring is therefore (for
mass $m$)
\begin{dmath}[label=dynamics]
m\ddot x - K x^2 = 0
\end{dmath}.
A controller for this unstable system may now be designed for the
purposes of stable operation at the zero stiffness position. The
design for a linear controller might proceed from here by (Jacobian)
linearisation around the operating point; linear stiffness $k =
\partial F/\partial x |_{x=0}$. This linearised system in state space form is
\begin{dmath}[label=linearised-dynamics]
\inlinematr{ \ddot x ; 
              \dot x } = \inlinematr{ 0 , 0 ; 
                                      1 , 0 } 
                         \inlinematr{ \dot x ; 
                                      x      }
                       + \inlinematr{ 1/m ; 
                                      0   } u 
\end{dmath}.
A standard linear controller designed around this model fails to
stabilise the actual system robustly, due to the large gains that
would be required to overcome the strong nonlinear destabilising
term. This can be shown using Lyapunov stability criterion
\cite{khalil1992}. Consequently, a simple backstepping
controller~\cite{krstic1995} will be used that can accommodate such
nonlinear problems.

In order to compare the results from the controlled zero stiffness
spring to a more conventional system, the same analysis will also be
performed on the vertically repelling spring, which has behaviour
similar to a standard linear spring. For the purposes of vibration
isolation, it is desired to examine the vibration response due to
indirect excitation through the (previously assumed fixed) base,
representing undesirable ground vibration. Schematics for the two
systems are shown in \figref{base-excitation}, which defines $U_1$
and $U_2$ as inputs to the vertically repelling and zero stiffness
springs, respectively.

\begin{figure}
  \fixme{Draw asy schematics for magnetic springs}
 %\null\hfill
 %\subfloat
 %  [Vertically repelling spring.
 %   \figlabel{vrepl-base-excitation}]
 %  {\quad\empuse{vrepl-base-excitation}\quad}
 %\hfill
 %\subfloat
 %  [Zero stiffness spring.
 %   \figlabel{zspring-base-excitation}]
 %  {\empuse{zspring-base-excitation}}
 %\hfill
 %\null
  \caption{Schematics of the simulated springs, for displacement $x$,
     disturbed by base excitation $y$, and controlled by input force $U$.}
  \figlabel{base-excitation}
\end{figure}

The dynamic equation for these systems, assuming some linear viscous
damping, $d$, and a nonlinear stiffness force
$\func{F_k}{{\cdot}}$, is
\begin{dmath}
  m\ddot\x + d\group{\dot\x-\dot\vv} - \func{F_k}{\x-\vv} = U .
  \eqlabel{motion}
\end{dmath}
The actual forces $\func{F_k}{x-y}$ for each system may be
calculated via application of \eqref{akoun}. From such numerical
results, approximations can be derived as follows for use in the
control laws:
\begin{dmath}
  \func{F_{k_1}}{\bar x} \approx \q3\bar x^3+\q2\bar x^2+\q1\bar x , \text{(vertically repelling)} \eqlabel{cubic}
\end{dmath}
\begin{dmath}
  \func{F_{k_2}}{\bar x} \approx \Q\bar x^2 , \text{(zero stiffness)} \eqlabel{quadratic}
\end{dmath}
in which cancelation of the static force with the gravity load has
been taken into account.

\subsubsection{Controller derivation for the zero stiffness spring}

The approximate nonlinear system dynamics for the zero stiffness
spring may be written in the following form:
\begin{dgroup}
\begin{dmath}[label=xdot]
\dot \x = \y  
\end{dmath},
\begin{dmath}
\dot \y = k\xv^2 + u
\end{dmath},
\end{dgroup}
where $\x$ is the displacement state, $\y$ is the velocity state, $k
= \Q/m$ from \eqref{dynamics,quadratic}, and $u = U_2/m$ is the
normalised input force.

An initial control Lyapunov function is chosen (typical for
mechanical systems) around which to design a stability controller
for the system:
\begin{dgroup}
\begin{dmath*}
V_1 \eqdef \half \x^2
\end{dmath*},
\begin{dmath*}
\dot V_1 = \x\dot\x = \x\y
\end{dmath*}.
\end{dgroup}
The velocity term can now be considered (mathematically) as a
virtual input for this system, $\y = \yy + \z $, where $\yy$ is the
desired value of $\y$ and $\z$ is the associated error:
\begin{dmath}
\dot V_1 = \x\yy + \x\z.\eqlabel{vdot1}
\end{dmath}
The term $\yy$ is chosen such that the first term in \eqref{vdot1}
is negative definite:
\begin{align}
\yy & \eqdef -\cc\x , \quad\cc\in\mathbb{R}^+ & \dot\yy & = -\cc\dot\x
= -\cc\y , \\
\therefore \dot V_1 & = -\cc\x^2 + \x\z.
\end{align}
The virtual state error term, $\z=\y-\yy$, is now
\begin{dgroup}
  \begin{dmath*}
    \z = \y+\cc\x 
  \end{dmath*} ,
  \begin{dmath*}
    \dot\z = k\xv^2+u+\cc\y 
  \end{dmath*} .
\end{dgroup}
Backstepping one integrator to incorporate $\dot\z$, and hence the
input $u$, a second control Lyapunov function is defined:
\begin{dgroup}
  \begin{dmath}
    V_2  \eqdef V_1 + \half\z^2 
  \end{dmath},
  \begin{dmath}
    \dot V_2 = \dot V_1 + \z\dot\z 
             = \group{\x\z + \x\yy} + \z\group{ u+k\xv^2+\cc\y } 
             = -\cc\x^2 + \z\group{ u+\x+k\xv^2+\cc\y } 
  \end{dmath}.
\end{dgroup}
The simplest route to stability is taken when the nonlinearities
are simply cancelled by the input control force, $u$:
\begin{align}
u & \eqdef -\d\z - \x - k\xv^2 - \cc\y , \quad \d\in\mathbb{R}^+  \notag\\
  & = -\x\group{\cc\d+1} - \y\group{\cc+\d} - k\xv^2 ,
  \eqlabel{active-linearised-feedback}\\
\therefore \dot V_2 & = -\cc\x^2 -\d\z^2 < 0 \mathbin{\forall} \x,\z
\neq 0 \eqlabel{prove-stability}
\end{align}
For an ideal system, proof of global asymptotic stability of the
controlled system follows from \eqref{prove-stability}
\cite{khalil1992}.

This controller design is equivalent to feedback linearisation
(note, \emph{not} linearisation of the dynamics around the operating
point as in \eqref{linearised-dynamics}) since the choice for $u$
simply cancels the nonlinearity and applies controller gains on the
states $\x$ and~$\y$. Substituting
\eqref{active-linearised-feedback} into \eqref{xdot} results in the
closed loop system, for some controller gains $\cc$ and $\d$,
\begin{dmath}
\ddot \x = -\x\group{\cc\d+1} - \dot\x\group{\cc+\d} .
\eqlabel{cl-dynamics}
\end{dmath}
Note that these closed loop dynamics are independent of the
disturbance $\vv$. For an ideal system, attenuation is therefore
infinite. However, errors will occur as the true dynamics deviate
from the approximation used to derive the controller in
\eqref{active-linearised-feedback}. While these errors can lead to
instability, increasing the controller gains $\cc$ and $\d$ is
sufficient to create a local region of stability around the nominal
position that is large enough to contain the effects of the
disturbance. It is possible, however, to design a more involved
backstepping controller to compensate robustly for this problem.

An equivalent controller to \eqref{active-linearised-feedback} can
be designed for the vertically repelling spring, resulting in
identical closed loop dynamics (\eqref{cl-dynamics}). Referring to
\figref{base-excitation}, the two controllers are:
\begin{dgroup}
\begin{dmath}[label=controller1]
  U_2 = -mx\group{\cc\d+1} - m\dot x\group{\cc+\d}-\Q\xv^2
\end{dmath},
\begin{dmath}[label=controller2]
  U_1 = \underbrace{ 
          -mx\group{\cc\d+1} - m\dot x\group{\cc+\d}
        }_{\text{Controller dynamics}}
        - 
        \underbrace{
          \q3\xv^3-\q2\xv^2 - \q1\xv
        }_{\text{Cancelation of open loop dynamics}} 
\end{dmath}.
\end{dgroup}

\subsection{added bit?}

Recall,
\begin{dgroup}
\begin{dmath}
m\ddot x = K(x-y)^2 + u
\end{dmath},
\begin{dmath} 
u = \{K(x-y)^2\}_{\text{est}} + k_c x + c_c \dot x 
\end{dmath},
\begin{dmath}
m\ddot x = e[x,y] + k_c x + c_c \dot x 
\end{dmath}.
\end{dgroup}
So the final vibration of the support, $x$, is independent of $y$
(the base vibration) except for a small amount coming through the
error term, $e$.

On the other hand, the dynamics of the support are defined by the
controller gains $k_c$ and $c_c$, which seems to imply that the
tradeoff between base vibration isolation and direct force
disturbance in a regular system is not present here.

\subsection{Simulation results}

Results are obtained by investigating the dynamics in numerical
simulation of the more conventional vertically repelling spring and
the zero stiffness spring, both under equivalent forms of control,
as derived in the previous section.

For the simulations to follow, the `true' magnet dynamics are
modelled using forces obtained from \eqref{akoun}. The magnets are
\SI{20}{mm} cubes, and the zero stiffness spring is chosen to have a
magnet gap (see \figref{spring-vcomb}) of \SI{20}{mm}. Without loss of
generality, the mass supported by the springs is chosen to be equal
to the minima of the force vs.\ displacement curve of the zero
stiffness spring, producing the zero stiffness condition. That is,
mass $m=\SI{1.836}{kg}$. For a different mass, a different gap
would be chosen.

The disturbance input is assumed to be Gaussian-distributed random
displacement with a standard deviation of \SI{1}{mm}; this is
numerically differentiated to obtain the input velocities for a
linear damping of 5\%, shown by $d$ in \eqref{motion}. This damping
term is included to approximate within an order of magnitude the
real-world effects of air resistance and eddy currents.

Several simulations are presented with different controller gains.
Initially, they are chosen to be large enough to overcome
instability through the controller approximation errors given the
specified input disturbance. Successively greater values were then
selected to demonstrate the limits of the control performance. In
total, simulations were performed for three sets of controller gains
on each of the two springs with the controllers respectively shown
in \eqref{controller1,controller2}.

A summary of the results are shown in \tabref{rms}, tabulating
\textsc{rms} average values of displacement and force experienced by
the support and the control force exerted to effect the control. The
zero stiffness spring has both a greater vibration attenuation and
an order of magnitude smaller control effort compared to the
vertically repelling spring.

\begin{table}
  \caption{\RMS\ values of the displacements, total forces, and control
  forces on the two springs for increasing control gains $\cc$, $\d$.}
  \begin{tabular}{@{} lccccc @{}}
    \toprule
    \multicolumn{3}{c}{System} & \multicolumn{3}{c}{\textsc{rms} values} \\
    \cmidrule(r){1-3}  \cmidrule(l){4-6}
    Spring & $\cc$ & $\d$ & Disp. (mm) & Force (N) & Control (N) \\
    \midrule
    Vertically repelling & 20  & 10  & 0.376  & 2.04  & 2.14  \\
                         & 50  & 25  & 0.129  & 1.48  & 6.25  \\
                         & 100 & 50  & 0.042  & 1.32  & 8.20  \\
    \midrule
    Zero stiffness       & 20  & 10  & 0.146 & 0.197 & 0.142 \\
                         & 50  & 25  & 0.100 & 0.263 & 0.199 \\
                         & 100 & 50  & 0.006 & 0.201 & 0.167 \\
    \bottomrule
  \end{tabular}
  \tablabel{rms}
\end{table}

Figures \ref{fig:active-disp} and~\ref{fig:active-force}
show time traces of the displacements and forces experienced by the
controlled springs for the lowest value of the controller gains.

\figref{active-Txd} shows the vibration transmissibilities of the
springs, calculated as the ratio between the output displacement and
input disturbance power spectra. Power spectra ratios were used
rather than frequency response functions (the ratio of the
cross-spectra to output auto-spectra) due to the nonlinear
response.

\begin{figure}
\begin{wide}
  \begin{subfigure}
    \grf{Active2006/work/eps/active-sim5-disp}    
    \caption{Vertically repelling spring.\figlabel{active-sim5-disp}}
  \end{subfigure}
  \hfill
  \begin{subfigure}
    \grf{Active2006/work/eps/active-sim2-disp}
    \caption{Zero stiffness spring.\figlabel{active-sim2-disp}}  
     \end{subfigure}
  \end{wide}
  \caption{Displacement traces of the floating magnet and the base
      disturbance ($x$ and $y$ in \figref{base-excitation},
      respectively) for the two active springs, with the lowest
      controller gains.}
  \figlabel{active-disp}
\end{figure}

\begin{figure}
\begin{wide}
  \begin{subfigure}
    \grf{Active2006/work/eps/active-sim5-force}
    \caption{Vertically repelling spring (control 
     force $U_1$ in \eqref{controller2}).\figlabel{active-sim5-force}}  
  \end{subfigure}
  \hfill
  \begin{subfigure}
    \grf{Active2006/work/eps/active-sim2-force}
    \caption{Zero stiffness spring (control force 
     $U_2$ in \eqref{controller1}).\figlabel{active-sim2-force}}    
  \end{subfigure}
  \end{wide}
  \caption{Force traces for the two active springs with the lowest
    controller gains; both the total force on the spring and the
    control actuator force are shown.  Note that the force axes in (a)
    and (b) are not scaled equally due to the large discrepancy
    between them.}
  \figlabel{active-force}
\end{figure}

Although the \emph{ideal} closed loop dynamics should be equal (if
the controller cancellations were perfect) and the displacements of
the suspended magnet should tend towards zero, the mismatch between
the estimated dynamics, \eqref{cubic,quadratic}, and actual
dynamics, from \eqref{akoun}, shows discrepancy between the two
systems.

The lower control effort result can be considered a result of the
fact that the zero stiffness spring simply needs to be stabilised to
achieve vibration isolation, whereas the vertically repelling spring
has dynamics that need to be actively cancelled to achieve the same
result. The is very apparent for the high gain cases, where the
control effect changes little for the zero stiffness spring from the
lower gain case, but the isolation performance increases
significantly. In comparison, the control effort of the vertically
repelling spring increases fourfold.

\begin{figure}
  \begin{wide}
    \begin{subfigure}
      \grf{Active2006/work/eps/active-base-powerspec-x7}
      \caption{Vertically repelling spring.}
    \end{subfigure}
    \hfill
    \begin{subfigure}
      \grf{Active2006/work/eps/active-base-powerspec-x4}
      \caption{Zero stiffness spring.}
    \end{subfigure}
  \end{wide}
  \caption{Transmissibilities of the actively controlled springs for various gains.}
  \figlabel{active-Txd}
\end{figure}

\subsubsection{Rejection of direct force disturbances}

The second notable feature of the controller used is the behaviour
of the closed loop dynamics, shown in \eqref{cl-dynamics}. It has
already been shown that greater control effort will correspond to
greater vibration isolation from the ground. In addition, the
natural frequency of the closed loop system, $\natfreq =
\sqrt{\cc\d+1}$, will also increase. This will have the effect of
increasing the attenuation of direct disturbance forces applied to
the floating spring (for example, due to vibrations of machinery
being supported or external excitation from turbulent airflow).

\subsection{Inherent assumptions}

Many assumptions have been made to simplify the analyses in this
paper. These assumptions will be lifted in the practical development
of the vibration isolator. The zero stiffness property only holds
when the magnet distances are tuned to support the mass of the
isolator in the region of local force minima. Variations of the load
would require a variable magnet separation distance, which could be
effected with an actively controlled screw drive for (slow) online
tuning.

The horizontal dynamics may be analysed and stabilised through
similar analysis to that shown in this paper for the vertical
direction. Alternatively, the spring may be constrained in the
vertical direction.

It has been assumed here that the states of the system are fully
observable. In practice, only the relative displacement
$\group{x-y}$, relative velocity $\group{\dot x-\dot y}$, and the
absolute accelerations $\ddot x$ and $\ddot y$ are directly
measureable, but these parameters may be combined in an observer to
obtain the required states used for feedback in this paper.

The effects of the actuator dynamics have not been modelled, nor
have problems relating to achieving moment-free forces been
addressed. With careful placement of zero stiffness voice-coil
actuators, these problems should not be significant in practice.

\subsection{Summary}

The unique force characteristic of a magnetic configuration
combining vertically attracting and vertically repelling springs
allows non-contact load bearing with zero stiffness properties. This
is different from a classical spring that has a lower bound on its
stiffness and hence a lower limit on its vibration isolation
capabilities.

The nonlinear force-displacement relationship of this combination
spring required a nonlinear controller, which was developed using a
backstepping technique. Simulation results based on this controller
displayed the zero stiffness tendency anticipated; smaller than
unity transmissibility was achieved over the entire frequency
spectrum.





\section{More advanced controller synthesis}

The performance of a vibration isolation table in attentuating
disturbances from the ground is governed in the first order by the
stiffness of the primary supports. Higher stiffnesses are more
convenient to support large loads, but lower stiffnesses give better
passive vibration suppression into lower frequency ranges. In high
precision contexts, a resonance frequency of less than \SI{1}{Hz} is
often required, for which very soft pneumatic springs are typically
used~\cite{kawashima2007}.

However, as the stiffness of a spring decreases its bulk increases,
and there are practical limits on how large the springs can
grow. The use of more complex support structures is a promising method of
reducing the effects of low frequency disturbance noise when simply
reducing the stiffness of the basic spring supports becomes
infeasible.

One approach is to use a mechanical linkage~\cite{winterflood2001}. In
the limiting sense, the ideal support would have zero stiffness, in
which displacement of the ground produces zero disturbance force on
the load. Zero stiffness can be achieved with mechanical supports, but
only for local regions of displacement. Since the zero stiffness
property is not global, these devices are more accurately said to have
`quasi-zero stiffness' (although the `quasi' term is often omitted).

All zero stiffness systems are a combination of positive and negative
stiffness nonlinear springs. Others have shown clever arrangements of
mechanical springs that achieve this
\cite{alabuzhev1989,carrella2007,lee2007}, while permanent magnets
have also been used \cite{robertson2006,nijsse2001} and will be used
here.

Generally, the zero stiffness property only occurs at a point of
marginal stability, and some form of active control is required to
keep the system stable at such an operating position.  The focus of
this paper is to investigate nonlinear control techniques for this
purpose.

\subsection{System modelling}

For the preliminary investigation, a single degree of freedom system
is created with two pairs of magnets to generate the superposition of
positive and negative spring stiffnesses. A schematic of the model is
shown in \figref{schematic}, in which vertical displacement, $x$, of
the beam-tip constitutes vibration that is to be suppressed. In order
to restrict the motion to a single degree of freedom, the magnet pairs
are attached to a beam that is pinned at one end. The use of a large
lever arm results in motion that is almost completely constrained to
the vertical; horizontal and rotational displacements are
negligible. A push-pull electromagnet pair acting on a central magnet
is used to apply control force to the system via coil current $I$.

\begin{figure}
  \centering
  \grf{Figures/Rig/schematic}
  \caption{Schematic of the nonlinear magnetic system. Hollow blocks
    denote magnets with polarisation indicated by arrows. The centre
    magnet is encompassed within an electromagnet coil pair (blocks
    with crosses), wound in opposite directions, to apply control
    forces to the magnetic spring to stabilise the system. The frame
    around the system is rigid and attached to the ground.}
  \figlabel{schematic}
\end{figure}

\subsubsection{Zero stiffness spring}

The most simple form of the zero stiffness magnetic spring is the
combination of a magnet pair in repulsion (lower pair in
\figref{schematic}) and a magnet pair in attraction (upper pair in
\figref{schematic}). The force versus displacement curve for this
system can be described approximately by the quadratic~\cite{robertson2006}
\begin{equation}
  F_m = K_m(x-x_0)^2+{F_m}_0 ,
\end{equation}
where $K_m$ is the spring constant, ${F_m}_0$ the load bearing
capacity of the spring at zero stiffness, and $x_0$ the zero
stiffness position due to
\begin{equation}
  \left.\frac{\partial F_m}{\partial x}\right|_{x=x_0}=0 .
\end{equation}

The spring constant $K_m$ and the load bearing capacity ${F_m}_0$ are
functions of the strength of the magnets and the gaps between them. For
application purposes, the spring must be capable of bearing a variety
of loads at zero stiffness ($F_g=-Mg$ for a range of $M$), which
requires that the fixed magnet positions be adjustable. The mass being
supported is assumed here to be constant.

Air resistance on the moving body and eddy currents induced by the
permanent magnets will induce damping effects that dissipate energy
from the vibrating system. The overall damping force, $F_d$, can be
assumed to be viscous (and relatively small):
\begin{equation}
  F_d = -C_d\dot x ,
\end{equation}
where $C_d$ is the damping coefficient.

\subsubsection{Actuator dynamics}

The actuator is modelled as a simple dual-coil winding
surrounding a permanent magnet. To be specific, the force this coil
will produce, $F_c$, will be a function of both position and current, but a
reasonable approximation for small displacements is
\begin{equation}
  F_c = K_cI.
\end{equation}
\fixme{crossref to COIL}

When driving the coil with a voltage amplifier, there will be
electrical dynamics as well. In terms of coil impedance $L$, coil
resistance $R$, voltage gain $G$, and input voltage $u$, the coil
dynamics are given by
\begin{equation}
  L\dot I + IR + K_e\dot x = Gu,
\end{equation}
which also incorporates the back-emf term $K_e\dot x$ from the moving magnet
inside the coil.

\subsubsection{Dynamic model}

From the preceeding subsections, the complete model of the system is given by
\begin{align}
\begin{split}
M\ddot x & = F_g+F_m+F_c+F_d,\\
         & = -Mg + K_m(x-x_0)^2 + {F_m}_0 + K_cI - C_d\dot x,\\
L\dot I  & = Gu - IR - K_c\dot x.
\end{split}\eqlabel{sys}
\end{align}
This system must be expressed in the following form in order to implement a
standard backstepping controller~\cite{krstic1995}:
\begin{align}
\begin{split}
\eqlabel{tuning}
  \dot x_1 &= x_2 + \p1\T\theta,\\
  \dot x_2 &= b_2x_3 + \p2\T\theta,\\
  \dot x_3 &= b_3u + \p3\T\theta,
\end{split}
\end{align}
which is achieved by reformulating \Eqref{sys} with $[x_1, x_2, x_3] =
[x, \dot x, I]$ as
\begin{align}
\begin{split}
  \dot x_1 &= x_2 ,\\
  \dot x_2 &= \t1 + \t2 x_1 + \t3x_1^2 + \t4x_2 + b_2x_3 ,\\
  \dot x_3 &= \t5x_2 + \t6x_3 + b_3u ,
\end{split}
\end{align}
where
\begingroup
\def\frac#1#2{#1/#2}
\begin{equation}
\begin{aligned}
\theta_1&=\frac{K_m x_0^2}{M}-g+\frac{F_{m_0}}{M},&
\theta_3&=\frac{K_m}{M},&
\theta_5&=-\frac{K_c}{L},&
b_2&=\frac{K_c}{M},\\
\theta_2&=-\frac{2 x_0 K_m}{M},&
\theta_4&=-\frac{C_d}{M},&
\theta_6&=-\frac{R}{L},&
b_3&=\frac{G}{L},
\end{aligned}
\end{equation}
\endgroup
and for constant parameters $\theta$ and nonlinear functions $\varphi_i$
defined as
\begin{align}
\theta & = [\t1, \t2, \t3, \t4, \t5, \t6]\T , \\
\begin{split}
\p1\T & = [0,0,0,0,0,0] , \\
\p2\T & = [1,x_1,x_1^2,x_2,0,0] , \\
\p3\T & = [0,0,0,0,x_2,x_3] .
\end{split}
\end{align}

\subsection{Nonlinear control}

The authors have previously demonstrated a simple backstepping
controller that is able to stabilise a system similar to that derived
above, albeit without coil dynamics, for the purpose of vibration
isolation \cite{robertson2006}. That controller showed in simulation
the advantages of using backstepping control with a zero stiffness
system, but it required an explicit and exact model of the system to
calculate the control law. In turn, a very precise system
identification would be required for control to be possible; a
more robust technique is therefore necessary for a practical implementation.

An approach without these limitations is the backstepping method using
tuning functions \cite[\S4.5.1]{krstic1995}, which is a general method
to generate tracking control systems for systems (for any number of
states) of the form shown in \Eqref{tuning}. The controllers thus
created are adaptive and do not require knowledge of the coefficients
$b_2$, $b_3$ or $\theta$. The Lyapunov-based design ensures that the
system states converge to their desired values and the unknown
parameters converge to a bounded set.  This latter property means that
using this technique is not appropriate for system identification,
since some parameter uncertainty is probable even after convergence.

When applying the tuning functions backstepping technique to
\Eqref{tuning}, the system of `error variables' $z_1$, $z_2$, $z_3$ is
defined with respect to the setpoint $y$ and `stabilising functions'
$\alpha_1$ and $\alpha_2$ (with $\alpha_3$ to appear):
\begin{align}
z_1&=x_1-y ,&
z_2&=x_2-\alpha _1 ,&
z_3&=x_3-\alpha _2 .
\end{align}
The voltage controller can now be shown to be defined as the following:
\begingroup
\def\{{[}
\def\}{]}
\def\left#1{#1}
\def\right#1{#1}
\def\overset#1#2{\dddot #2}
\def\delta\left(#1,#2\right){%
  \partial_{#2}\lbrace#1\rbrace}
\begin{equation}
u=\varrho _3 \alpha _3 ,\eqlabel{control}
\end{equation}
with stabilising functions\footnote{The notation
  $\partial_x\{y\}\equiv\frac{\partial y}{\partial x}$ is used for
  clarity.}
\begin{multline}
\alpha _3=-\kappa _3 z_3 x_2^2-\vartheta _5 x_2+\delta \left(\alpha _2,x_1\right) x_2
-\left(x_1^4+x_1^2+x_2^2+x_3^2+1\right) \kappa _3 z_3 \delta \left(\alpha _2,x_2\right){}^2\nl+\delta \left(\alpha _3,\vartheta \right)\dot{\vartheta }-x_3 \vartheta _6-\beta _2 z_2-\kappa _3 z_2^2 z_3-c_3 z_3-x_3^2 \kappa _3 z_3\nl
+\left(\vartheta _1+x_1 \left(\vartheta _2+x_1 \vartheta _3\right)+x_2 \vartheta _4+x_3 \left(\beta _2+2 \kappa _3 z_2 z_3\right)\right) \delta \left(\alpha _2,x_2\right)+\dot{\varrho }_2 \delta \left(\alpha _2,\varrho _2\right),
\end{multline}
\vspace{-2\baselineskip}
\begin{multline}
\alpha _2=\varrho _2 \left(-\vartheta _3 x_1^2-\vartheta _2 
x_1-x_1+y-\vartheta _1-x_2 \vartheta _4-c_2z_2\nl
-\left(x_1^4+x_1^2+x_2^2+1\right) \kappa _2 z_2+x_2 \delta 
\left(\alpha _1,x_1\right)\right),
\end{multline}
\vspace{-2\baselineskip}
\begin{equation}
\alpha _1=-c_1 z_1 . \eqlabel{control-end}
\end{equation}
In the controller above, $c_1$, $c_2$, and $c_3$ are the controller
gains, and $\kappa_2$, $\kappa_3$ are the nonlinear damping
gains. These gains can be adjusted to obtain a desirable controller
response.

Parameter update laws for the estimates are shown in
\Eqrangeref{param}{param-end}, where $\vartheta$ are the estimates of
parameters $\theta$, $\beta_2$ and $\beta_3$ are the estimates of $b_2$
and $b_3$, and $\varrho_2$ and $\varrho_3$ are the estimates of
$p_2=1/b_2$ and $p_3=1/b_3$. $\varrho$ terms are used to avoid
$1/\beta$ terms in the controller/update laws, which become
problematic if $\beta\to0$. $\Gamma$ is the parameter update gain
matrix for $\vartheta$, and $\gamma_2$, $\gamma_3$ are update gains for $\varrho_2$ and $\varrho_3$.
\begin{multline}\eqlabel{param}
\dot{\vartheta }=\Gamma\left\{z_2-z_3 \delta \left(\alpha _2,x_2\right),x_1 \left(z_2-z_3 \delta \left(\alpha _2,x_2\right)\right),\nl
x_1^2 \left(z_2-z_3 \delta \left(\alpha _2,x_2\right)\right),x_2 \left(z_2-z_3 \delta \left(\alpha _2,x_2\right)\right),x_2 z_3,x_3 z_3\right\}\T ,
\end{multline}
and
\begingroup
\def\frac#1#2{#1/#2}
\begin{align}
  \dot{\beta _2} &=-\gamma _2 z_3 \left(-x_2+\alpha _1+x_3 \delta \left(\alpha _2,x_2\right)\right) ,
  \\
  \dot{\varrho _2}&=-\frac{\gamma _2 z_2 \alpha _2}{\varrho _2} ,
  \\
  \dot{\varrho _3}&=-\gamma _3 z_3 \alpha _3 , \eqlabel{param-end}
\end{align}
\endgroup
\endgroup
A characteristic of the tuning functions design is the tight coupling
between the controller and the parameter update laws. Note that
$\alpha_3$ contains terms involving both $\dot {\vartheta}$ and
$\dot {\varrho} _2$. No derivatives of the setpoint $y$ appear, as it
is assumed to be constant, but this restriction is not necessary in
the general form of the controller.

\subsection{Results}

To demonstrate the performance of the controller described above, a
selection of simulation results are shown. When the system is
initialised, the position of the fixed magnets needs to be set in
order for the magnet force curve to provide a nominal supporting force
to place the spring as close to the zero stiffness location as
possible. For this to occur, $F_{m_0}=Mg$, where $F_{m_0}$ is a
function of the gaps between the fixed and floating magnets.

However, this calibration cannot be performed in open loop, because
the desired location of the spring is in a position of marginal
stability; therefore, the controller must be active while the fixed
magnet positions are adjusted. 

The first simulation results demonstrate the stability and convergence
properties of the controller. The system parameters used in this
simulation are shown in \tabref{params}, with initial parameter
estimates 10\% below their actual value.  Adaptation gains $\gamma_2$,
$\gamma_3$, and $\Gamma$ were chosen as the inverse of the orders of
magnitude of the parameters they affect.

\Figref{converge-states} shows the states converging with setpoint
following ($x=x_0$). During parameter convergence, the system
approaches its desired position as the parameters converge to their
steady state values. \Figref{converge-params} demonstrates that the
parameters do not converge to their exact values, but to some nearby
constant values instead.

\begin{table}
\centering
\caption{System parameters for simulation results.}
\tablabel{params}
\vspace{-\baselineskip}% why??
\null\hfill
\begin{tabular}{@{}|cc|@{}}
\hline
\multicolumn{2}{|c|}{Mechanical} \\
\hline
Parameter & Value \\
\hline
$M$ & \SI{0.01}{kg} \\
$K_m$ & \SI{100}{N/m^2} \\
$F_{m_0}$ & $0.8 Mg$ \\
$C_d$ & \SI{0.01}{kg/s} \\
$x_0$ & \SI{0.02}{m} \\
\hline
\end{tabular}\hfill
\begin{tabular}{@{}|cc|@{}}
\hline
\multicolumn{2}{|c|}{Electrical} \\
\hline
Parameter & Value \\
\hline
$K_c$ & \SI{1}{N/V} \\
$K_e$ & \SI{0.1}{V/(m/s)} \\
$R$ & \SI{8}{Ohm} \\
$L$ & \SI{0.001}{H} \\
$G$ & 1 \\
\hline
\end{tabular}\hfill
\begin{tabular}{@{}|cc|@{}}
\hline
\multicolumn{2}{|c|}{Controller} \\
\hline
Parameter & Value \\
\hline
$c_1$, $c_2$, $c_3$ & $10$ \\
$\kappa_2$, $\kappa_3$ & $1$ \\
$\gamma_2$ & $0.01$ \\
$\gamma_3$ & $0.001$ \\
$\Gamma$ & See below \\
\hline
\end{tabular}
\hfill\null
\bigskip

\fbox{%
\begin{minipage}{0.65\textwidth}
\centering
$\Gamma = \operatorname{diag}(1,0.001,0.0001,1,0.01,0.0001)$
\end{minipage}}
\end{table}

\begin{figure}
  \centering
  \begin{subfigure}
    \grf{ICSV14/work/eps/icsv-ideal}
    \caption{States of the system\figlabel{converge-states}}
  \end{subfigure}
  \hfill
  \begin{subfigure}
    \grf{ICSV14/work/eps/icsv-ideal-params}
    \caption{Parameter estimates, normalised with their exact values 
    (that is, $\vartheta_i/\theta_i$, $\beta_2/b_2$, $\rho_i/p_i$ are plotted).\figlabel{converge-params}} 
  \end{subfigure}
  \caption{Example control with system parameters as shown in
    \tabref{params}.}
  \figlabel{control1}
\end{figure}

Once the parameters have converged and the magnet positions have been
adjusted to achieve the desired nominal supporting force, the system
is simulated again with perturbation on $M\ddot x$ introduced to
demonstrate the behaviour under direct disturbance situations. For
this simulation, the nominal force $F_{m_0}=0.99Mg$; the 1\%
difference imposes a small initial displacement offset. The
perturbation force on the isolated mass is Gaussian-distributed white
noise with a variance of \SI{0.001}{N}.

The simulation results are shown in \figref{control2}, for two cases:
with and without parameter adaptation. \Figref{adapt-noise} can be
seen to have a greater succeptibility to the noise disturbance
compared to \Figref{ssnoise}. These results show that once the system has
converged, it is desirable to disable the adaptation
($\gamma_2,\gamma_3=0$, $\Gamma=\bm{0}$) to avoid perturbation in the
system affecting the control performance.

\begin{figure}
  \centering
  \begin{subfigure}
    \grf{ICSV14/work/eps/icsv-adapt-noise-t}
    \caption{With parameter adaptation.\figlabel{adapt-noise}}
  \end{subfigure}
  \hfill
  \begin{subfigure}
    \grf{ICSV14/work/eps/icsv-ssnoise-t}
    \caption{Without parameter adaptation.\figlabel{ssnoise}}
  \end{subfigure}
  \caption{Control at steady state position with direct disturbance,
    with and without parameter adaptation. It is clear that noise
    upsets the performance of the parameter adaptation after convergence.
    See \figref{converge-states} for legend.}
  \figlabel{control2}
\end{figure}

\subsection{Summary and future work}

This article has investigated the feasibility nonlinear
control for a magnetic spring system. We have shown that standard
backstepping methods can be used to stabilise the system in the
presence of parameter uncertainties. However, we have not yet extended
this work to look at vibration isolation, which is the \emph{raison
  d'etre} for looking at this quasi-zero stiffness system.

To investigate vibration isolation, additional states need to be added
to the system to represent the vibration of the ground. The tuning
functions backstepping method accommodates this additional complexity
without trouble.

The current design uses full state feedback, which is not infeasible
in practice with the use of laser displacement and velocity
sensors. Such sensors are expensive, however, and a useful
modification to the controller will be to add state observers (either
velocity or displacement) in order to reduce the number of sensors
required.



\section{Unfinished}

\section{Backstepping}

\subsection{Trajectory initialisation}

Trajectory initialisation is used to improve the estimate
convergence. With the controller as designed above, it can be shown
that
\begin{dmath}
  (\theta-\vartheta(t))\T\Gamma^{-1}(\theta-\vartheta(t))\leq 
  z(0)\T z(0)+(\theta-\vartheta(0))\T\Gamma^{-1}(\theta-\vartheta(0)).
\end{dmath}
This shows that while the estimates may never converge to their exact
values, there are two things we can do to reduce their
uncertainty. Firstly, we want to have good estimates of the system
parameters from the beginning in order to minimise
$(\theta-\vartheta(0))\T\Gamma^{-1}(\theta-\vartheta(0)$. This can be
achieved through analytical calculation and \textit{a priori}
measurement; another viable option is to iterate the controller while
updating the estimates' initial conditions to their converged values
from the previous run.

Secondly, we want to minimise $z(0)\T z(0)$, recognising that $z(t)$
are error states incorporating the difference between the actual and
desired states of the system \cite{krstic1995}. This can be achieved
by choosing a trajectory that contains an term to bring the
tracking function to the actual states' measurement at $t=0$.

That is, rather than choosing a following function $y_f(t)$ that may
have $y_f(0)\neq x(0)$, we desire
\begin{dmath}
  y_r(t)=y_f(t)+\delta(t),\\
  \delta(0)=y_r(0)-y_f(0),\\
  \delta(t)\to0\text{ as }t\to\infty.
\end{dmath}

The tracking controller also requires the derivatives of the
trajectory to follow, which can be calculated from the plant estimate:
\begin{dgroup}
\begin{dmath}
  y_r(0) = x_1(0)  
\end{dmath}.,
\begin{dmath}
  \dot y_r(0) = x_2(0)  
\end{dmath},
\begin{dmath}
  \ddot y_r(0) = x_3(0)+\varphi_2\cdot\vartheta(0)  
\end{dmath}.
\end{dgroup}

Now, the function $\delta(t)$ can be any sufficiently differentiable
(up to the number of states of the nonlinear system) function that
converges to zero. A simple stable linear system satisfies these
requirements is
\begin{dmath}
  \begin{bmatrix}
    \dot{\delta}(t)\\\ddot{\delta}(t)\\\dddot{\delta}(t)
  \end{bmatrix}
  =
  \begin{bmatrix}
    0 & 1 & 0 \\ 0 & 0 & 1 \\ -m_0 & -m_1 & -m_2
  \end{bmatrix}
  \begin{bmatrix}
    \delta(t)\\\dot{\delta}(t)\\\ddot{\delta}(t)
  \end{bmatrix}
\end{dmath}
with $m_i>0$ and initial conditions $\dot{\delta}(0)=\dot y_r(0)$ and
$\ddot{\delta}(0)=\ddot y_r(0)$.





