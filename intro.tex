%!TEX root = thesis.tex

\chapter{Introduction}

\epigraph{In my experience I found that the most effective way to express something in order to make others understand is to use the simplest language. Also I learned from teaching that the more rigid the language the less effective it is.}{\textcite{mahathera1990}}



\section{Introduction to the central themes of the thesis}

Before launching into detail into the various topics under investigation, this section briefly touches on the issues addressed over the entire thesis: vibrations and their isolation and suppression in \secref{vibrations-summary-intro}; permanent magnets and their role in the design of supporting structures and other devices in \secref{magnets-summary-intro}; and `\qzs/' systems, which unify in this thesis the fields of vibrations and magnets, in \secref{qzs-summary-intro}.
These three broad areas are then investigated in more detail in \secrangeref{vibrations-explore}{qzs-explore}.


\subsection{The problem of vibrations}
\seclabel{vibrations-summary-intro}

The disturbing effects of vibrations are a well-known and continuing problem.
The transmission of vibrations from a source can only ever be reduced, not eliminated — that is, not without removing the source entirely from the local region affected.
And in many cases there is no single source; the very ground itself may be a medium through which undesirable vibrations are transmitted.
Earthquakes are an extreme example of this, but on a smaller scale there are continuous time-varying displacements of the `fixed' ground beneath us.
   \note{Not to be too earth-centric, but many of the ideas here have a different relevance in off-planet circumstances.}
The earth should in fact be considered as a distributed vibrating structure, of very great mass in total, that has a range of displacement profiles dependent on the local surrounding impedance conditions.

Whether the source of the disturbance is near or far, or how it propagates through the ground to arrive at the region of interest, these disturbances can cause a variety in problems on equipment that is required to, ideally, remain absolutely still.
A good example is during the electro-lithography performed to construct our microchips, in which nanometre-sized disturbances can affect the overall yield of the silicon wafer produced.
Mitigating the effects of a disturbance through the base on which a structure is supported is known as `vibration isolation' and is the over-arching problem in which the work of this thesis should be put in context.

A contrasting vibration problem occurs when some manner of machine causes its own vibration; a well-known example is a washing machine that exerts an oscillating disturbance force on itself through a mass imbalance.
This type of vibration problem requires a rather different set of design solutions and often its solution acts in opposition to the vibration isolation problem discussed above.
Reducing the effects of self-induced vibration disturbance will be termed `vibration suppression'
  \note{The descriptor `Vibration isolation' is sometimes used to refer to both problems, but it would be confusing here to avoid the clarification.}
for the purposes of this thesis and will be revisited on occassion herein.

There are a variety of `classic' solutions for both vibration isolation and vibration suppression.
A particularly simple solution for \emph{both} problems is to mount the equipment on a many-ton slab of concrete.
This is not always practical.
Another common approach for vibration isolation is to support the equipment with pneumatic springs.
When in operation, these springs provide a low supporting stiffness and low static deflection; they can typically be used to support hundreds of kilograms with resonance frequency of less than five hertz.

Other support methods besides pneumatics are able to achieve low stiffness; this is an area that will be further investigated in the literature review.
But one method in particular is interesting in this context: permanent magnets can provide low stiffness support without energy expenditure.
Their nonlinear forces in both attraction and repulsion allows the possibility of interesting supporting designs, and their non-contact nature allows their use in vacuum and `clean-room' environments.


\subsection{Permanent magnets used for mechanical design}
\seclabel{magnets-summary-intro}

The last twenty years has seen the maturation of the rare earth permanent magnet industry.
These magnets are now widely available in large sizes and strong magnetisation at relatively low prices.
They are now used in a variety of mechanical design, including bearings, couplings, \maglev/ trains, and so on, which take advantage of non-contact attractive and/or repulsive forces.
Magnets can be used to make other magnets move or to keep them in place — there is little limit to the ingenuity of their application.
However, in part to this complex behaviour between them, there are few design guidelines that can be used to aid their use [quote Moskowitz].

We can speak broadly about their integrated use in force design: magnets can be used in conjunction with current-carrying coils to effect time-varying forces (as in shakers and speakers); soft iron can be used to guide the magnetic fields into desired regions or away from unwanted areas (\eg, latches and motors); or magnets can be used alone for unique force--displacement characteristics or simply for applying non-contact forces (\eg, bearings).

These ideas in mechanics and dynamics have application back to the field of vibrations.
A `synergy' between the two fields is seen in areas such as energy harvesting from ambient vibration, the study of vibration in high-speed magnetic bearings, and one of the main themes of this thesis — nonlinear and/or noncontact forces for support equipment for vibration isolation.

Support mass with a noncontact force can also be called `levitation', an area which deserves its own mention.
\textcite{earnshaw1842} proved that levitation with the force of permanent magnets alone was impossible, although this did not become common knowledge \note{If it can even be said to be `commonly known' today.
Anecdotal evidence suggests otherwise.} until much later — given by the range of patents issued that assume the opposite.
Exceptions to `Earshaw's Theorem' include the use of diamagnetic materials and actively-controlled system, amongst some others.
It is the possibility of overcoming instability with active means that is of interest in this thesis.


\subsection{\QZS/ systems}
\seclabel{qzs-summary-intro}

The transition between stable and unstable forces becomes interesting in the context of vibration isolation.
Between positive and negative stiffness in a force \vs\ displacement characteristic, there is a inflexion point of zero stiffness.
This point is termed a `\qzs/' position to emphasise that the dynamic behaviour of the system in this condition can be rather complex and usually unstable.
`True' zero stiffness would imply \emph{no} connection between between the mass and the base, as if they were floating in free space — the motion of one would have no effect on the motion of the other.

As systems approach \qzs/, their vibration isolation inproves as the resonant frequency decreases.
Operation at the \qzs/ position is not possible as the system is, at best, only marginally stable, and the system must be tuned (based on the applied loading) to achieve best results.

Certain magnetic systems are not the only ones to exhibit \qzs/.
The phenomenon was first proposed using inclined springs to achieve a `buckling' effect.
Magnets are more convenient in many ways than inclined or buckling springs in that the negative stiffness can be applied directly without having to exploit the byproduct of a mechanical spring or linkage arrangement, which can be more bulky.

Active control systems can be used with \qzs/ systems to improve their  performance in one of three ways:
\begin{enumerate}
  \item Standard active vibration control with velocity feedback;
  \item Remove the instability at the \qzs/ location with a nonlinear controller;
  \item Online tuning of the system for load-independent operation.
\end{enumerate}
The first two of these strategies are investigated in this thesis.


\subsection{Project context}

The original goal of this project was to design and build a vibration isolation table using non-contact magnetic springs.
This goal can be split into two: the design of a non-contact magnetic spring (suitable for a vibration isolation table); and the design of the vibration isolation table itself.

Vibration isolation tables are generally designed to attenuate natural disturbances from the ground to the tabletop.
Current commercial models use pneumatic springs to perform this task, and this project arose out of curiosity: could magnetic springs be used instead?

Using magnets for load bearing brings its own set of challenges.
For completely non-contact support, active control must be used to stabilise \emph{at least} one degree of freedom.
For the design to be worth investigating, some advantage to using magnets should also be demonstrated.
  \note{Although I took much pleasure in explaining over the years that my \PhD/ project was to `build a table that floats on magnets'.}

However, the field of active control has been well-established and the feat of building a stabilising controller for a system with relatively simple dynamics is not worthy of the research for a \PhD/.
The work presented in this thesis is the investigations around the idea of building a `table that floats on magnets' while pulling out enough interesting nuggets to prove worthy of the title of `research'.



\section{Vibrations}
\seclabel{vibrations-explore}

The field of active vibration isolation is a broad topic to cover in review; not everything will be able to be covered here, but it is important to have an overview of what people are doing to place this work in context.
The literature review that follows is strongly biased towards papers that have been recently published.
Tracking their citations backwards will yield a tangled web of prior art in the field of vibration control.

\subsection{Forms of vibration control}

Many descriptions are given to various systems and types of vibration control.
As mentioned in the introduction to this chapter, `vibration isolation' is the main objective of this literature review.
It is instructive to illustrate some of the alternatives and define specifically the terminology used in this thesis.

The most basic vibratory or oscillating system is shown in \figref{simple-suppression}, in which a mass is excited by an external source and behaves with resultant dynamics determined by the stiffness and damping of the connection.
The ground to which it is fixed is assumed to have infinite mass and to have zero displacement.
In this system, motion of the mass can be suppressed by increasing the stiffness of the support, since as the mass becomes more greatly coupled with the ground, the input force has a diminishing influence.
This behaviour is referred to as `vibration suppression' in this thesis.
A practical example of this sort of vibration problem is a washing machine that causes itself to vibrate through a rotating imbalance; the imbalance causes oscillations in the drum which, when great enough, force the drum off its axis of rotation.
\note{A common problem in our household. Just have to learn to arrange the washing carefully inside the drum.
  \note{Our washing machine has since broken. Imbalance matters!}}

\begin{figure}
  \asyfig{Systems/vibration-sdof}
  \caption{
    Single degree of freedom vibration system; mass $\massMass$ is being excited by disturbance $\forceDisturb$.
    Input force $\forceIn$ can be generated using feedforward or feedback control to minimise the displacement $\dispMass$ of the mass.
  }
  \figlabel{simple-suppression}
\end{figure}

If the attachment of the mass in \figref{simple-suppression} is assumed \emph{not} to be infinitely massive and stiff, the problem becomes not only to suppress the motion of the mass but also to prevent force transmission from the input disturbance into the base itself.
A practical example is a piece of vibrating industrial equipment that radiates vibrations through the ground, causing noise and generating ground-bourne disturbances for other machinery.
This problem is not so easily solved; there is a trade-off in the self-induced displacements of the machinery and the force transmitted to the ground.
By lowering the stiffness of the support, the transmitted force is reduced but the self-induced displacements are increased.
Due to reciprocity, decreasing the transmitted force from the mass to the ground is equivalent to decreasing any disturbances transmitted from the ground to the mass.
Isolating the mass from ground vibration is known in this thesis as `vibration isolation'.
For the purposes of this discussion, there is assumed to be no self-induced vibration in the system.
A schematic of this type of vibration isolation system is shown in \figref{simple-isolation}, for which a practical example is protecting sensitive equipment from ground-based disturbances.

\begin{figure}
  \asyfig{Systems/vibration-base}
  \caption{
    Schematic of vibration isolation; mass $\massMass$ is being isolated from disturbance $\dispBase$.
    Input force $\forceIn$ can be generated using feedforward or feedback control to minimise the displacement $\dispMass$ of the mass.
  }
  \figlabel{simple-isolation}
\end{figure}

The examples of vibration suppression and vibration isolation have been depicted with active input forces that can be used to tune or adjust  oscillations of the systems.
In these examples, it is assumed that the input forces have negligible effect on the dynamics of the ground or base.
Further complications arise when this assumption no longer holds, such as when the mass is mounted on a flexible structure with mass that is not much greater than the mass being supported.
For the sake of this work, this more complex case is not examined in detail.

\subsection{Fundamentals of active vibration isolation}
\seclabel{fundavibes}

Introduced in the previous section, \figref{simple-isolation} shows the simplest form of vibration isolation in which a mass $\massMass$ is mounted with stiffness $\stiffnessRel$ with disturbance input $\dispBase$.
Passive isolation is achieved for zero input force, $\forceIn=0$.
Feedback control allows the properties of the system to be adjusted according the desired vibration response.

The dynamic response of the isolated mass is
\begin{dmath}[label=simple-isolation]
  \massMass\ddot\dispMass +
  \dampingRel\gp{\velMass-\velBase} +
  \stiffnessRel\gp{\dispMass-\dispBase} = \forceIn
\end{dmath}.
First assume that there is no input force; taking the Laplace tranform and rearranging produced the transmissibility $\transmissibility$ of the system in the frequency domain:
\begin{dmath}[compact,label=simple-isolation-freq]
  \transmissibility = \frac{\laplaceMass}{\laplaceBase} = 
  \frac{\ii\freq\dampingRel + \stiffnessRel}
  {-\massMass\freq^2 + \ii\freq\dampingRel + \stiffnessRel}.
\end{dmath}

For ideal linear state feedback control, the input force can be represented as a linear combination of measured relative displacement $\gp{\dispMass-\dispBase}$, relative velocity $\gp{\velMass-\velBase}$, and acceleration $\ddot\dispMass$ of the mass.
A force sensor can also be used for feedback purposes, but this yields results for lumped-parameter systems equivalent to acceleration feedback; for vibration isolation of flexible systems, there is evidence to suggest force feedback giving greater stability than acceleration feedback \cite{preumont2002-jsv}, especially for lightly damped structures.
Absolute velocity $\velMass$ and absolute displacement $\dispMass$ of the mass can be estimated by integrating the acceleration, and for completeness we also consider the relative acceleration between the mass and the base ($\accMass-\accBase$) as a possible signal for feedback control.
The generalised feedback force can then be represented by
\begin{dmath}
 \forceIn = 
   \gainDisp\gp{\dispMass-\dispBase} +
   \gainVel \gp{\velMass -\velBase}  +
   \gainAcc \gp{\accMass -\accBase}  +
   \gainSkymass   \dispMass +
   \gainSkyhook   \velMass  +
   \gainSkyspring \dispMass
\end{dmath},
where $\gainArbitrary$ are feedback gains chosen appropriately for a given application.
The term involving $\gainSkyhook$ is known as skyhook damping.
Substituting this force equation into \eqref{simple-isolation} yields
\begin{dmath}[label=isolation-feedback]
  \gp{\massMass+\gainSkyspring}\ddot\dispMass +
  \gp{\dampingRel+\gainVel}\gp{\velMass-\velBase} +
  \gp{\stiffnessRel+\gainDisp}\gp{\dispMass-\dispBase} +
  \gainAcc\accMass +
  \gainSkyhook\velMass +
  \gainSkyspring\dispMass
  = 0
\end{dmath}.
The transfer function between base and mass displacement for this generalised feedback case is
\begin{dmath}[label=tf-genfeedback]
  \frac\laplaceMass\laplaceBase = 
\frac{\gainDisp+\stiffness+s \gp{\damping+\gainVel+\gainAcc s}}{\gp{\gainAcc+\gainSkymass+\mass} s^2+\gp{\damping+\gainSkyhook+\gainVel} s+\gainDisp+\gainSkyspring+\stiffness}
\end{dmath}.
From \eqref{isolation-feedback} it can be seen that there is an exact equivalence between the feedback gains of three of the different signals and a corresponding physical parameter of the system.
Relative displacement and velocity feedback correspond to a variation in the stiffness and damping, respectively, and absolute acceleration feedback to the mass.
The system \eqref{tf-genfeedback} is stable for characteristic solutions of the denominator (\ie, the poles of the system) have negative real components. These can be analysed easily with the quadratic equation. Note that the feedback gains for the absolute versus relative signals always appear together in the denominator of the transfer function (\eg, $\gainAcc+\gainSkymass$), and hence affect the stability of the system identically.

To illustrate the effect of various control gains, from transfer function \eqref{tf-genfeedback} the transmissibility $\Abs{\transmissibility}$ drawn with parameters $\mass=\SI{1}{kg}$, $\damping=\SI{0.5}{kg/s}$ and $\stiffness=\SI{10}{N/m}$, and each of the six control gains $\gainArbitrary$ varied independently.

\subsubsection{Displacement feedback}
\Figref{disp-vs-sky} shows the effect of varying the control gains for relative displacement and absolute displacement ($\gainDisp$ and $\gainSkyspring$).
In both cases, the resonance frequency is increased with increased feedback gain.
This is usually detrimental to vibration isolation performance.
Relative displacement feedback corresponds to an increased stiffness and a higher resonance frequency, whereas absolute displacement feedback increases the vibration isolation at low frequencies; this scheme is notable for its less than unity response even as the frequency of excitation tends to zero, while the high frequency attentuation is unaffected.
As discussed in \secref{vibes-feedback}, this form of feedback is highly susceptable to low frequency instabilities and cannot reliably be implemented in practice.

The system with displacement feedback is stable according the inequality
\begin{dmath}
  \damping>\Real{\sqrt{\damping^2-4 \gp{\gainDisp+\gainSkyspring+\stiffness} \mass}}
\end{dmath},
which holds for all $\gainDisp+\gainSkyspring>-\stiffness$. Therefore, negative feedback gain may be used to lower the resonance frequency of the structure by decreasing the effective stiffness of the system.

\begin{figure}
   \begin{wide}
     \psfragfig{\phdpath Simulations/Springs/fig/sdof-relspring}\hfil
     \psfragfig{\phdpath Simulations/Springs/fig/sdof-skyspring}
   \end{wide}
   \caption{Relative and absolute displacement feedback control.}
   \figlabel{disp-vs-sky}
\end{figure}

This idea presupposes that the absolute displacement is a state that can be measured.
Disregarding slow and inaccurate sensors that can do this directly (such as using the Global Positioning System), the absolute displacement of an object can only be estimated based on other measurements of the system.
The simplest form of this is by double-integrating an accelerometer signal.
\textcite{zhu2006} used double-integral control in combination with other control techniques for vibration isolation in micro-gravity.

\subsubsection{Acceration feedback}
\Figref{disp-vs-sky} shows the effect of varying the control gains for relative acceleration and absolute acceleration ($\gainAcc$ and $\gainSkymass$).
Absolute acceleration feedback corresponds to an increased system mass, corresponding to a decreased resonance frequency; high frequency vibration isolation is improved.
Relative acceleration feedback is only included for completeness; it has the effect of reducing the resonance peak but effectively eliminating any vibration isolation characteristics at higher frequencies.
Other vibration schemes for reducing a tonal or narrowband disturbance are discussed briefly in \secref{vibneut}.

The system with acceleration feedback is stable according to the inequality
\begin{dmath}
  \Real{\frac{\damping \pm \sqrt{\damping^2-4 \stiffness \gp{\gainAcc+\gainSkymass+\mass}}}{\gainAcc+\gainSkymass+\mass}}>0
\end{dmath},
which is true for all $\gainAcc+\gainSkymass>-\mass$.
It is possible, albeit generally undesirable, to increase the resonance frequency of the system with negative acceleration gain, which corresponds to an effective decrease of the mass of the system.

\begin{figure}
   \begin{wide}
     \psfragfig{\phdpath Simulations/Springs/fig/sdof-accrel}\hfil
     \psfragfig{\phdpath Simulations/Springs/fig/sdof-skymass}
   \end{wide}
   \caption{Relative and absolute acceleration feedback control.}
   \figlabel{acc-vs-sky}
\end{figure}


\subsubsection{Velocity feedback}
\seclabel{skyhook-intro}

Acceleration and displacement feedback both primarily affect the resonance frequency of the system; velocity feedback is different.
The transmissibilities due to the influence of relative velocity and absolute velocity feedback ($\gainVel$ and $\gainSkyhook$) are shown in \figref{vel-vs-sky}.
    \note{In the literature, absolute velocity feedback is often referred to as `skyhook damping'. This terminology is largely refrained from use in this thesis.}
In both cases the resonance peak is lowered; the relative velocity feedback corresponds to a reduction in attenuation at higher frequencies, while for absolute velocity feedback the high frequency response is unaffected.

\begin{figure}
   \begin{wide}
     \psfragfig{\phdpath Simulations/Springs/fig/sdof-vel}\hfil
     \psfragfig{\phdpath Simulations/Springs/fig/sdof-sky}
   \end{wide}
   \caption{Relative and absolute velocity feedback control on the 
   system shown in \figref{simple-isolation}.}
   \figlabel{vel-vs-sky}
\end{figure}

Velocity feedback is stable for
\begin{dmath}
  \damping+\gainSkyhook+\gainVel \pm 
    \Real{\sqrt{\gp{\damping+\gainSkyhook+\gainVel}^2-4 \stiffness \mass}}>0
\end{dmath},
which is true for $\gainSkyhook+\gainVel>-\damping$.
While increased damping is usually desired, it is possible to reduce the effective damping in the system with negative velocity feedback gain, with the effect of increasing the amplitude of the resonance peak.
Active damping reduction has been performed to aid the efficiency of vibration neutralisers \cite{kidner1998}.

The absolute and relative velocity feedback results may be compared by calculating the \RMS\ transmissibilities over a frequency range of interest ($\sqrt{\Int{\transmissibility}{\freq,\freq_1,\freq_2}}$) as a function of increasing feedback gain for the two cases. 
This is shown in \figref{rms-transmissibility}, where the relative feedback \RMS\ transmissiblity has a local minimum whereas the absolute feedback case continuously decreases.
It is clear in the ideal case that skyhook damping is more efficacious at reducing the total vibration of a system.
(The maximum frequency in this case was chosen to be much greater than the resonance frequency; $[\freq_1,\freq_2]=[0,\SI{1000}{rad/s}]$.)

\begin{figure}
   \psfragfig{\phdpath Simulations/Springs/fig/rms-transmissibility}
   \caption{\RMS\ transmissibility versus feedback gain of relative and 
   absolute velocity feedback control on the system shown in 
   \figref{simple-isolation}.}
   \figlabel{rms-transmissibility}
\end{figure}

Active skyhook damping has been widely implemented in the vibrations literature \cite[for example]{elliott2001,elliott2004,yan2006,kim2008-iecst}.
Instability can occur in cases when the applied control force affects the dynamics of the base, but in practice this is not a problem when considering comparatively massive support structures.


\subsubsection{The analogy with passive control}

As mentioned above, three of the active approaches mentioned above correspond directly to variations for one of the physical parameters of the system: mass, damping, and stiffness.
Active vibration control is usually only attempted when such passive control is inadequate or impractical.
\Eg, adding mass reduces the resonance frequency, which improves vibration isolation; however, it is not always practical to do so.

Adding additional damping through viscous elements, equivalent to relative velocity feedback control, will reduce the amplitude of the resonance peak, but vibration isolation at higher frequencies will be degraded.
Energy generation can be accomplished through electromagnetic damping through induced eddy currents in a coil experiencing relative displacement to a magnetic field, such as investigated by \textcite{graves2000}.
An alternative has recently been explored of using nonlinear viscous damping instead, which provides similar benefits as active absolute velocity feedback  \cite{lang2009} with the advantages generally associated with passive control: robust, power-free, inexpensive.

Reducing the stiffness between the mass and its supporting base from has the same effect as increasing the mass; the resonance frequency is reduced and the bandwidth of vibration isolation is increased.
Passive stiffness reduction can be achieved by adding a negative stiffness element in parallel with the system \cite{lee2007-jsv,xing2005}.
This idea becomes a major theme of the thesis, explored later in this introduction in \secref{qzs-explore}.
Considering variable stiffness elements that can act in a `semi-active' regime, many methods exist to dynamically adjust the stiffness depending on the support being used \cite[for example]{kidner2002,liu2006-jsv,liu2008-jsv}.
Semi-active stiffness modification is used more commonly in \vibneut/ applications, described later in \secref{vibneut}, since for vibration isolation applications, in general, the lower the stiffness the better; there is no need to adjust the stiffness on-line if it is already as low as possible.
\note{Although an exception to this will be discussed later, in which the stiffness of a nonlinear system is adjusted upward to compensate for an increase in mass that would otherwise lead to instability.fixme{crossref}}

\subsection{Active vibration control in practice}

\subsubsection{Limitations of the above techniques}
\seclabel{vibes-feedback}

The limits of control in real-world devices was investigated by \textcite{ananthaganeshan2001} in a vibration suppression system, where time delays and phase lags introduced by digital filters and integrators can have quite significant effects over the ideal case of pure displacement or velocity or acceleration feedback (as posited in the previous section). Pithily stated by \textcite{williams2009}:
\begin{quote}
In theory integration of accelerometer signals is easily done; however, in practice, inertially referenced velocity proves to be as elusive as it is useful.
\end{quote}
Real integrators (as opposed to ideal integrators) and high pass filters cause instabilities at low frequencies.
Acceleration feedback has a much smaller stability limit than displacement and velocity feedbacks, and the effect of displacement feedback is strongly reduced even with small time delays.
Therefore velocity feedback control from integrated accelerometer measurements should be considered the better choice.
Also, the presence of a low pass filter does not significantly effect the efficacy of velocity feedback.

A similar problem is addressed by \textcite{zhu2006} in the design of a micro-gravity vibration isolation system. They look at using such feedback in a highly sensitive micro-gravity environment and show that `PD' feedback is unsuitable due to instability at high (\ie, useful) control gains from quantisation and anti-aliasing side-effects.
Rather, integral and double-integral feedback from an accelerometer give better results for them.

\textcite{zhao2007} examined the vibratory behaviour of a two degree of freedom nonlinear system with a time-delayed positional feedback.
This results of this paper emphasises the importance of accounting for time delays in feedback control systems.

\textcite{serrand2000} discuss active skyhook damping applied to a two degree of freedom structure with emphasis on the effects, for their system, due to possible base flexibility.
One of their results shows that using integrated accelerometer measurements as a velocity feedback term has a low frequency phase shift that can induce instability for high enough control gains.

Velocity feedback control by using integrated accelerometer measurements at the location of actuation has been used for some time as an effective vibration isolation mechanism, shown for example by \textcite{kim1999}.
They also look at the effect of non-infinite base mass, with results that I should have known about, I guess.


\subsubsection{Further discussion of velocity feedback (\ie, more skyhooks)}
\seclabel{skyhook}

The issue of selecting appropriate control gains for velocity feedback control systems was examined by \textcite{engels2008} for the cases of centralised and decentralised control devices.
In centralised control, a global model of the system is used when allocating the feedback signals.
In decentralised control, each sensor/actuator pair operates independently to minimize the energy at the mounting point.
Generally centralised control is more difficult in practise but can give better results.
For the simple two-degree of freedom vibrating system examined by \textcite{engels2008}, the centralised control performed better although the differences to the decentralised control were small.
An analogous result was shown by \textcite{hoque2006} for a three-axis vibration platform supported by a so-called `infinite stiffness' magnet/spring system (also see \secref{infstiff}).

An attractive method of reducing the detrimental effect of a resonance peak in vibration isolation is to add damping to the structure. As previously discussed, passive damping (or relative velocity feedback control) reduces the effectiveness of the vibration isolation at higher frequencies.
Semi-active methods have also been developed to approximate skyhook damping when fully active control is undesirable \cite{liu2002,liu2005,ahmadian2004} or its limitations too severe.
Semi-active control has the advantages of robustness and low power requirements, especially when large forces would be required for active control.

In the context of isolating vibration of a train, \textcite{li1999} discuss the disadvantages in tracking when applying skyhook damping for systems that experience very low frequency (\ie, steady state) changes in the base position, but for the purposes of vibration isolation this is not a problem if the ground is assumed to experience disturbance with zero mean.
(They also use Kalman filters for estimating velocity but such a signal is suitable only for tracking, not for control, due to the time delay due to the processing and calculation of the resultant signal.)

A similar result \fixme{check/qualify this} can be achieved with dynamic stiffness control, which can be effected with `semi-active' variable stiffness feedback control \cite{leavitt2007}.

In the original and crude method of implementing skyhook control, the switching mode that is necessary to emulate the `skyhook' in the controlled damping introduces higher order (odd-multiple) harmonics in the frequency response, as shown by \textcite{ahmadian2001}, who later proposed two `jerk-free' skyhook algorithms \cite{ahmadian2004}.

Further work in the area was established by \textcite{liu2002}, who demonstrated a number of skyhook-like semi-active damping methods, including smoothing functions to eliminate the problem of jerk
\cite{liu2005}.
  \note{These references by Liu contain essentially the same material, with greater detail in the first. The 2005 paper is easier to read than the 2002 Technical Memorandum simply due to the formatting of the documents.}

\textcite{song2007} used the nonlinearities of semi-active skyhook damping as justification for the design of an adaptive controller for vibration isolation; they did not compare their results with recent work in the skyhook area, however.


\subsubsection{Alternative control approaches}

This section could grow almost indefinitely. Only works that explicitly target vibration isolation systems are discussed, although there have been many papers published recently that the reader may find interesting albeit off-topic \cite{chatterjee2008,mottershead2008}.

In approximate order of complexity and effectiveness, a variety of vibration isolation control approaches are briefly discussed below.

\paragraph{Combination approaches of linear feedback}

The active feedback methods introduced in \secref{fundavibes} are orthogonal in the sense that more than one scheme may be applied in parallel.
For example, acceleration feedback could be used to reduce the resonance frequency, and absolute velocity feedback used to reduce the height of the resonance peak.
This approach was used by \textcite{savaresi2007} to implement a combination skyhook/acceleration-feedback controller in the context of vehicle vibration control for ride `smoothness'.
\textcite{gavin2007} discuss using negative stiffness and skyhook damping in the context of isolating machinery in a building.


\paragraph{Adaptive and state control}

\textcite{guo2005}, for example, used system identification and a variety of feedforward/feedback control techniques to isolate a multi-degree of freedom structure in a single direction.
Adaptive vibration control in six degrees of freedom for broadband noise has also been shown by \textcite{duindam2005}.


\paragraph{Optimal control}

\textcite{balandin1998} review the field of optimal control as applied to shock and vibration isolation problems.
\note{Their comment that the \enquote{number of papers is so great that there is little incentive to discuss them here} does not bode well for any attempts by me to even summarise their review.} 
They differentiate shock and vibration isolation succintly:
\begin{quote}
The operating quality of shock isolators is usually described in terms of certain characteristics of the transient motion of the body being isolated, whereas the quality of vibration isolators is determined by the characteristics of steady-state forced oscillations.
\end{quote}
That is, optimising for shock will result in minimising, say, the peak displacement of the mass, whereas optimising for vibration will result in a low natural frequency (characterised by a minimum achievable \RMS\ displacement).
Since an optimal controller is based around a cost function that will be dependent both on the vibratory system itself and the mode of disturbance (transient, broadband noise, \etc), such control approaches are heavily case-specific and are best used when a plant is pre-determined and a vibration problem needs to be considered after the fact.
\textcite{bolotnik2001} discusses these methods in more detail.

\paragraph{\Hinf/ control}

\textcite{kerber2007} modelled a plate-plate coupling through four contact points in six degrees of freedom system and applied active vibration isolation in the vertical direction using a range of control methods.
Of those trialed, \Hinf/ control had the best low frequency response, and \PI\ control (with anti-windup) the best high frequency response.
Output feedback, state feedback with an observer, and velocity feedback methods all performed worse to various extents.

\paragraph{Intelligent control}

\textcite{madkour2007} compared a slew of nonlinear `artifical intelligence' adaptive algorithms to observe the parameters of a vibrating pinned beam in order to apply feedback cancelation.
Such techniques are appropriate when system identification is required during operation; \ie, when the plant cannot be measured beforehand or it slowly changes over time.
However, such `artifical intelligence' algorithms are probably sub-optimal for the observation of such a dynamics problem when the system can be broadly modelled as a set of differential equations in the time domain, for which in even complex systems nonlinear techniques such as backstepping or sliding mode control are often directly suitable.
Leave the heuristic approaches to systems that cannot be modelled (or even simulated for a broad range of operating behaviours) through standard means.
\note{Note that these comments only apply to the use of such methods in the control phase of vibration control; genetic algorithms, \eg, provide a useful technique in the design phase when limited numbers of sensors and actuators must be placed in `optimal' positions on a complex structure \cite{simpson2003,howard2005}.}

\paragraph{Backstepping control}

\fixme{Cite ji pan?}

\paragraph{Sliding mode control}

While skyhook damping has been introduced above in \secref{skyhook-intro} as force feedback proportional to the absolute velocity of the base, others have considered the case where the influence of the ground velocity is actively reduced. In both cases, the ratio of the influence of mass velocity to base velocity is being increased. In the ideal sense, the dynamic equation of motion
\begin{dmath}
  \massMass\accMass + \dampingRel\gp{\velMass-\velBase} + 
    \stiffness\gp{\dispMass-\dispBase} = \forceIn
\end{dmath}
can be transformed into
\begin{dmath}
  \massMass\accMass + \dampingRel\velMass + 
    \stiffness(\dispMass-\dispBase) = 0  
\end{dmath}.
The adaptive sliding mode approach used by \textcite{zuo2004} was targetted towards vehicle suspension, in which ground vibration cannot be measured.
Sliding mode techniques has been recently shown to be more effective than classical techniques for an example of vehicle suspension control \cite{dong2009}.


\subsubsection{Examples of actual vibration isolation platforms}

The literature review until now has focussed mainly on specific systems and techniques for achieving vibration isolation and suppression.
This small section looks at the end result of much of this research: actual applications into constructing vibration isolation platforms.

\textcite{chen2007} used \Hinf/ control to suppress the resonance peak at around \SI{3}{Hz} of a pneumatic vibration isolation mount.
Very good reductions in transmissibility were achieved; this technique highlights the appeal of vibration control using an isolator with a damped resonance.

In high precision contexts, a resonance frequency of less than \SI{1}{Hz} is often required, for which very soft pneumatic springs are typically used \cite{kawashima2007}.

\textcite{yoshioka2001} designed and built a prototype vibration isolation platform with a load bearing capacity about \SI{1000}{kg}.
Their device used a combination of pneumatic, voice-coil linear motors, and piezoelectric actuators to control the first half dozen vibration modes of the platform.
Feedback velocity of the table and feedforward velocity and displacement of the floor were used as control signals (with integrated accelerometer measurements), and a genetic algorithm was used to generate a set of control gains for the system.


\subsection{Vibrations literature introduced and then ignored}

Until this point in the introduction, the only focus on the vibration control literature has rested upon the area directly related to vibration isolation through modification (whether active, semi-active, or passive) of the supporting structure for the mass.
The landscape for vibration control is much broader, however.
Here, two minor digressions are made to place the rest of the literature in context with alternate approaches of vibration reduction, and some discussion made on why the techniques herein were deemed unsuitable for the work of this thesis.


\subsubsection{Inertial actuators}

Sometimes it is not possible to integrate the control mechanism into the support of a structure, in which case external actuators need to be added to the device to provide the control forces.
These tend to be inertial electromagnetic actuators, also known as `proof-mass' actuators, where the mass of the moving element provides an external force via coupling to the structure.
This is shown schematically in \figref{vibration-absorb}.

\begin{figure}
   \asyfig{Systems/vibration-inertial} 
   \caption{An inertial force $\forceAbsorb$ designed to reduce the vibration 
   response $\dispMass$ due to disturbance $\dispBase$.
   The inertial actuator 
   has dynamics of its own ($\stiffnessAbsorb$, $\dampingAbsorb$) that 
   influence the overall vibration of the structure.}
   \figlabel{vibration-absorb}
\end{figure}

In this case, velocity feedback on its own can have problems with stability margins at low frequencies.
\textcite{benassi2002-part2} showed in experiment that this can be improved by the addition of a phase lag controller with force feedback in conjunction with velocity feedback.
The theory for such `combined-state' feedback cases was developed at the same time \cite{benassi2002-double}.

A wide combination of feedback combinations was analysed by \textcite{diaz2005} focussing on various forms of velocity feedback for \mbox{single-,} \mbox{double-,} and multi-degree of freedom vibration isolation systems.
In the two degree of freedom system, an inertial actuator is used to provide control force; as well as additional stability contraints due to this arrangement, the resonance at low frequencies of the actuator itself compromised the control performance.
(A possible solution to this problem might be to select an actuator with a much \emph{greater} resonance frequency than the plant, but this arrangement has been shown to be ineffective at controlling the system \cite[][Appendix~A]{benassi2002-part1}.)

A comparison of some of these methods, including skyhook damping and semi-active vibration absorber methods, was done by \textcite{huyanan2007}, contrasting the performance and implementation differences between them.

\textcite{paulitsch2003} uses an electromechanical actuator that serves as a self-sensing device for vibration control.
The idea of self-sensing for magnetic levitation purposes (in both cases using a electromagnet) has been shown by \textcite{bleuler1992,vischer1993}.
These self-sensing devices (perhaps obviously, since the \backemf/ is such a noisy signal) do not perform nearly as well as when using a dedicated sensor, but the technique is particularly interesting for low-cost, low-precision devices.

\fixme{tie into thesis}


\subsubsection{Narrow-band vibration isolation}
\seclabel{vibneut}

One method of reducing vibration on a supported mass is to attach a supplementary mass that resonanates in concert with the disturbance; this has the effect of adding an anti-resonance to the original system at the frequency of interest.
These are known under various names as `tuned mass dampers', `vibration neutralisers', `dynamic vibration absorbers', and so on.
  \note{No effort has been made to compile an exhaustive list.} 
The descriptions involving such terms as `damper' and `absorber' are not strictly accurate on the grounds that these devices do not act as energy dissipators; rather, they direct energy into a subsystem for which continuous disturbance is not undesirable.
In this thesis, the term `\vibneut/' is used, following \textcite{kidner1998} and others.

The concept of \vibneut/s is not entirely the focus of this research, but falls into the category of literature that crops up in association with it.
In the passive application, a \vibneut/ consists of attaching a supplementary mass to the structure for which vibration is to be removed.
The stiffness of the attachment is chosen by matching the natural frequencies of the structure with that of the additional mass.
This is shown in \figref{tuned-mass-vs-fig} where a response is shown for a range of stiffness values.

\begin{figure}
   \psfragfig{\phdpath Simulations/Springs/fig/tuned-mass-vs-freq}
   \caption{
     Vibration absorber with a range of absorber stiffnesses (labelled).
     When the absorber matches the resonance of the structure the vibration amplitude around that frequency is greatly reduced.}
   \figlabel{tuned-mass-vs-fig}
\end{figure}

There is a compromise between broad- and narrow-band vibration attentuation.
\Figref{inertial-trans-delta} illustrates the vibration attenuation at resonance for a system with a vibration absorber.
It can be seen that for narrowband reduction, low absorber damping produces greater vibration attenuation.
Conversely, if the \RMS\ transmissiblity of the entire frequency band is calculated, as shown in \figref{rms-inertial}, it can be seen that lower absorber damping \emph{decreases} the overall vibration reduction.

\begin{figure}
\begin{wide}
  \begin{subfigure}
    \psfragfig{\phdpath Simulations/Springs/fig/inertial-trans-delta}
    \caption{
      Transmissibility reduction at resonance due to a vibration absorber system, versus absorber stiffness, for a range of absorber damping values.
      Greater reductions result from \emph{lower} damping.}
    \figlabel{inertial-trans-delta}
  \end{subfigure}
  \begin{subfigure}
    \psfragfig{\phdpath Simulations/Springs/fig/rms-inertial}
    \caption{
      \RMS\ transmissibility of the vibration absorber system versus absorber stiffness for a range of absorber damping values.
    Greater broadband reductions result from \emph{higher} damping.}
    \figlabel{rms-inertial}
  \end{subfigure}
\end{wide}
\caption{
  A comparison of the effects of changing the damping ratio of the absorber on single-frequency and broadband transmissibility of a \vibneut/.
}
\end{figure}

The efficacy of a \vibneut/ is related to the damping between it and the structure; better results are achieved with lower damping, as shown in \figref{rms-inertial}.
The damping of the absorber can be reduced with an active control system as shown by \textcite{kidner1998}.
This can be understood with the realisation that energy is not dissipated by the vibration `absorber'; rather, motional energy from the vibrating structure is being \emph{transferred} to the supplementary mass, and this process is degraded by the presence of damping.

\Vibneut/ are tuned for a specific resonance frequency, which means that the frequency of the input disturbance must be known and unvarying for a passive device to achieve useful results.
Especially for low-damping neutralisers, their efficacy decreases rapidly as they become de-tuned.
To avoid the problem of slow variations in the input disturbance, semi-active methods can be used to observe the frequency of the disturbance and adjust the resonance frequency of the neutraliser to remain tuned.
Such neutralisers typically use a variable stiffness element, which can take many forms \cite{ting-kong1999,kidner2002,holdhusen2007}.
\textcite{brennan2006} discusses a wide variety of actuators that may be used to construct a \vibneut/:
\begin{quote}
There is not a single ``best'' way of making an [adaptive tuned vibration absorber].
It depends upon the required frequency range, the agility (speed of reaction) and cost.
\end{quote}

An interesting device is shown by \textcite{ivers2008} that can be described as a mechanically self-tuning \vibneut/.
While not as efficacious as an adaptive \vibneut/ that uses an external power source, the ability to adapt to the excitation frequency using only the energy of the disturbance itself is commendable if somewhat esoteric.

\Vibneut/ have been used to mitigate seismic vibrations in large buildings, but their mass dependence makes their application rather tricky and often impractical. \fixme{cite}
\textcite{matta2008} propose a nonlinear rolling structure via which a \vibneut/ can be mounted to good effect despite uncertain masses of either or both of the building and absorber.
Their work focuses on the interesting idea of using a roof-top garden as a \vibneut/ for a building \cite{matta2008a}.

Some researchers have looked at using electromagnetic actuators to provide a fully active force with which to cancel system resonances \cite{chen2005a,wu2007,kim2008-iecst}.
The advantage for such a system is the same as for semi-active controllers in general: with a suitable algorithm, changes in the plant can be taken into account in the vibration neutraliser.
However, using a fully active system for this task is not very energy efficient, since all `damping' is achieved artificially with the expenditure of actuator energy.
Nonetheless, good results can be obtained via this method and the potential flexibility of the control system could be a good reason to design such a system.
(Perhaps as a test platform for comparing various control techniques via emulation.)

\Vibneut/ can also be used for modal systems, in which case each neutraliser is designed at the specific resonance frequency of each mode.
In a recent example, \textcite{casciati2007} used a semi-active neutraliser to control vibration of a suspended cable; some care was required for their structure as the higher frequency (often nonlinear) behaviour posed an influence even as the targeted (low frequency) mode was damped as desired.
Optimization techniques can be used, if the mode shapes are known, to place multiple \vibneut/s in a modal system \cite{petit2009-jva}.

When electrical circuits are used to absorb resonant vibrations, the energy absorbed can be redirected to produce a power output \cite{stephen2006}.
Such devices are gaining popularity for ambient vibration--powered applications such as remote sensing \cite{arnold2007}, with practical implementations beginning to appear \cite{ferrari2009-sms}.
Electromagnetic systems tend to be more suitable for larger scale energy harvesting devices, whereas piezoelectric and electrostatic devices are more suitable at the micro-vibration scale \cite{beeby2009}.
Another field of interest for regenerative damping is in vehicle suspensions, in which useable power can be extracted with the same mechanism as providing greater ride comfort \cite{graves2000thesis}.
Recent work has self-powered \magnetorh/ dampers as \vibneut/ \cite{choi2009-jva}.

\textcite{stephen2006} performed a thorough analysis on energy harvesting with micro-actuators.
He considered a single degree of freedom mass-spring-damper coupled with a simple electrical circuit.
For best performance, energy should be dissipated as much as possible by the electric components, not the mechanical damping, since in the electrical network the energy is retrievable whereas as viscous damping the energy is dissipated as heat.
This idea has similarities with the concept discussed previously in this section that the effectiveness of the \vibneut/ is reduced with the presence of increased mechanical damping.

Semi-active methods have also been explored to tune energy harvesting devices to the frequency of disturbance.
\textcite{challa2008} investigated a semi-active device for tunable energy harvesting that used variable-displacement attractive and repulsive magnets to adjust the resonance frequency of a piezoelectric cantilever.
This is the same mechanism that is examined in this thesis for `\qzs/' suspensions (see \secref{qzs-explore} and \secref{qzs}).

\fixme{tie into thesis}


\subsection{Summary of the vibrations literature}

The cross-section of introductory concepts and literature shown in this section have been chosen to illustrate the broad approaches for vibration control of simple systems.



\section{Magnetics}
\seclabel{magnets-explore}


\subsection{Applications of magnetic forces}

\textcite{coey2002} discusses a broad range of magnetic applications,
summarised in \tabref{magnet-applications}.

\begin{table}
\begin{wide}
\begin{tabular}{@{}lll@{}}
\toprule
Field & Magnetic effect & Examples \\
\midrule
Uniform & Zeeman splitting & Magnetic resonance imaging \\
& Torque & Alignment of magnetic powder \\
& Hall effect, magnetoresistance & Sensors, read-heads \\
& Force on conductor & Dynamic Motors, actuators, loudspeakers \\
& Induced emf & Generators, microphones \\
Nonuniform & Force on charged particles & Beam control, 
radiation sources %(microwave, ultra-violet; X-ray) 
\\
& Force on magnet & Bearings, couplings, Maglev \\
& Force on paramagnet & Mineral separation \\
Time varying & Varying field & Dynamic Magnetometers \\
& Force on iron & Dynamic Switchable clamps, holding magnets \\
& Eddy currents & Metal separation, brakes \\
\bottomrule
\end{tabular}
\end{wide}
\caption{Applications of permanent magnet materials, 
adapted from \textcite{coey2002}.}
\tablabel{magnet-applications}
\end{table}

That magnets can apply forces to one another over a distance is quite a novel concept in a mechanical world accustomed to friction.
It has been a short while, relatively speaking, that it has been possible to even produce magnets with enough coercive force to apply useful mechanical forces.
Non-contact magnetics in mechanical systems is advantageous due to high precision and wear-free operation due to lack of friction.
This section looks broadly at some of the main applications of the field.

Loud-speakers are still being researched, with improved analytical methods to determine forces for optimisation purposes \cite{merit2009}.


\subsection{Maglev transportation}

The largest body of research into magnetic levitation is on so-called `\maglev/'
transportation, reviewed in 2006 by \textcite{lee2006}.
Its well-known goal is to use a levitated train or car to provide extremely fast and efficient transportation.
This field, which is rather diverse in terms of the techniques under investigation, is finally now achieving commercial application in the real world after some 20 or 30 years
of research.
While it has some concepts in common with this research (large loads, magnets), the techniques used tend to be rather distanced from those that will be applied for this project because they focus on transportation rather than \emph{elimination} of movement.

Earnshaw's theorem for stability is not applicable for \maglev/ systems, since the motion of the vehicle adds a time-varying element to the magnetic system.
Many approaches to the design of \maglev/ systems have been taken, including passively stable designs \cite{musolino2009}.


\subsection{Magnetic positioners}

In more recent years, another application for magnetic levitation has been investigated, which is the precision control of a levitated platform.
Commonly cited for use in the semiconductor industry for photolithography, these levitators were first researched around 20 years ago, and the field is still an active area of research \cite{fulford2008,fulford2009}.

Recent one degree of freedom linear magnetic bearing: \cite{ro2009-preeng}.

The first designs allowed travel in a single direction, \eg, \textcite{trumper1992}, while more recent developments allow more directions of control.
Such devices are capable of supporting small loads, and applying horizontal translation forces to effect displacements of up to around \SI{200}{mm} with nanometre precision.
Two planar devices are invented in the independent theses of \textcite{kim1997} and \textcite{molenaar2000}.
Most recently, a six degree of freedom non-contact actuator was demonstrated by \textcite{verma2004}.
See also the device built by \textcite{kim2007} to support around \SI{2}{kg} over a travel of $\SI{5}{mm}\times\SI{5}{mm}$ with nanometre precision.
The high performance and mechanical simplicity of their design is note-worthy.

Here're some more to look at: \textcite{boeij2008,zhang2008a}

The reason these devices are unsuitable for this research is due to their travelling capability.
Rather than using the primarly magnetic flux for load support, these designs use it in order to provide positioning control in the horizontal directions.

Here's something new: \textcite{shameli2008}.

A recent six degree of freedom actuator has been demonstrated by \textcite{jansen2008}, which supports loads in the order of \SI{10}{kg} with a relatively large air gap (\SI{1}{mm}--\SI{2}{mm}) and large stroke ($\SI{230}{mm}\times\SI{230}{mm}$).

[\textcite{dasilveira2005}] Analytical expression for the normal force between two magnets on a back-iron plate and a perpendicular coil.
The system is for a planar actuator, and the motivation is to be able to determine the amount of out-of-plane force generated by a particular design.
Could very well be useful for some of my ideas.


\subsection{Interesting devices}


I like magnetic levitation of objects in a wind tunnel \cite{higuchi2008}.

Here would be a good place to cite the brain, nerve, stuff, etc
\parencite{sekino2005,lu2008,demachi2008}

Measurements of memory encoding in the brain \cite{gjini2005}.

Measurements of the health of the heart \cite{lim2009}.

Stimulation of the human nervous system \cite{darabant2009}.

Wireless motion capture device \cite{hashi2005}.

Six degree of freedom remote localisation within the human body \cite{yang2009-ietm}.

Magnetic fields can be used for a vast array of scientific uses.
For example, noncontact sensing of material properties that involve variable conductivity, including fatigue cracks, defects in printed circuit boards, and even plastic landmine detection \cite{mukhopadhyay2005}.

Here's a fun application of magnetics where a computer input device is built with magnetic sensors placed on the wrist in order to sense single finger-tip motion from the opposite hand \parencite{han2008}.

I don't know if I care, but \textcite{vanwest2007} created a haptic interface for manuipulating small objects with magnetic levitation.

\textcite{park2008} demonstrates a MIMO controller for a flywheel energy storage mechanism using magnetic bearings while applying active vibration isolation.

\textcite{tomie2005} This paper's a bit of fun; uses a magnet attached to a cantilever, excited by an external field in a water tank, to propel a robotic fish.
Only left-right oscillations were produced, so the fish was constrained to move in a plane.


\subsection{Magnetic levitation is impossible!}
\seclabel{earnshaw}

The act of passively levitating a magnet by another is well known as impossible, although popular unlearned opinion is not aware of the fact. 
\textcite{earnshaw1842} proved that objects in the influence of fields that apply forces with an inverse-square relation to displacement cannot form configurations of stable levitation.
Approximately one hundred years later, \textcite{tonks1940} wrote a paper reminding his contemporaries of the work of \citeauthor{earnshaw1842} by applying the proof specifically to the field of magnetics:
\begin{quote} 
\dots no flexible assemblage of magnetic poles, in which readjustments in
position of the poles in the group can occur, can be stable in either a fixed
field or in the field from another such assemblage\dots
\end{quote}
Another interesting Earnshaw retrospective is given by \textcite{bassani2006-meccanica}, and an alternative formulation given by \textcite{reusch1994}.
A mathematical demonstration of Earnshaw's theorem is conceptually quite simple.
We start with the equation for the magnetic field; when there are no external current terms, it can be shown to be expressed as Laplace's equation:
\begin{dmath}[compact]
\grad\magB = 0 \implies \divgrad\magB = 0
\end{dmath}.
The potential energy of a magnet is proportional to the magnetic field it is subjected to, $U = -\magM\bdot\magB$, so when the magnetisation is time-invariant  (as in the case of a permanent magnet), we have:
\begin{dmath}[compact,label=earnshaw]
\divgrad U = \divgradxyz{U} = 0 
\end{dmath}.
The double differentiations of the energy are the stiffnesses in each direction.
But for stable equilibrium, these three terms must be greater than zero.
This cannot satisfy \eqref{earnshaw} and thus levitation cannot occur.

\begin{figure}
  \grf[width=0.4\textwidth]{Figures/Theory/saddle}
  \caption{A ball in unstable equilibrium on a saddle-shaped curve.}
  \figlabel{saddle}
\end{figure}

This is easy to visualise by analogy.
\figref{saddle} shows a ball balancing on a saddle-shaped curve, which we can take to be potential energy of a \twoD/ system.
Clearly it is stable in one direction, unstable in the other.
Perturbations `left' or `right' will result in reaction forces keeping it centred and stable, whereas small displacements `into' or `out from' the page will result in increased perturbation to instability.
So it is with any permanent magnet arrangement.


\subsection{Exceptions to Earnshaw}

Earnshaw's proof does not rule out all forms of `levitation' unconditionally, however.
\textcite{boerdijk1956a} reviewed the known methods for levitation, covering levitation by gravitation forces, pressure reaction forces, radiation field forces, and finally in detail, various magnetic and electromagnetic forces.
\textcite{bassani2006} revisited Earnshaw's work, in particular to highlight interesting exceptions to the theory against passive levitation.

Because Earnshaw's theorem looks only at the case for static equilibrium, cases when the magnetic field is dynamic are not covered.
This can occur broadly under three circumstances: when the magnetic field is generated with \AC/ currents; and when an unstable permanent magnet arrangement is stabilised with an active control system; and when the system is composed of elements with some dynamics associated with them 
\note{\Ie, when things are moving}.

Levitation using a magnetic field produced by \AC/ currents was covered in detail by \textcite{laithwaite1965}.
The basic mechanism of this type of levitation is that \AC/ currents create dynamic magnetic fields that induce eddy currents in a levitating object, and it is the interaction of the magnetic fields of these induced currents that causes the levitation.
The technique uses a large amount of power, and is not especially suitable for the purposes of this research for this reason.

For an actively stabilised levitation system, the levitation forces are created by permanent magnets (which have time-constant magnetic fields) and the necessary stabilisation applied with variable current electromagnets with a feedback control system.
For this reason the magnetic fields are known as quasi-static.
This system was first implemented by Holmes who levitated a magnetic needle, as cited by \textcite{boerdijk1956a}.
Some more practical examples of these types of system are covered in the next section.

This summary is fairly brief; \textcite{bleuler1992} wrote a more detailed overview.
His paper introduced the `self-sensing active magnetic bearing' \cite{vischer1993} which uses \backemf/ from the controlling electromagnet to sense the position of the floating element.
This eliminates the need for a more classical position sensor,
 \note{See \secref[vref]{xpmt-sensors} for a more detailed discussion of position sensors for non-contact purposes.}
but the control system is necessarily more complex and the behaviour not as precise.

Finally, levitation can be achieved in dynamic systems. \textcite{bassani2007} levitated a ring magnet above another by using continuous base excitation to find a small zone of stability in the nonlinear dynamics of the system. More well-known, the Levitron toy demonstrates stability of a magnetic spinning top above a ring magnet \cite{berry1997,berry1996,simon1997}.


\subsection{Diamagnetic forces}

Levitations involving diamagnetic material are also exempt from Earnshaw's theorem.
This was the motivation for the papers of \textcite{boerdijk1956b,boerdijk1956a} in which he cites Braunbek, who derived that magnetic material is governed by Earnshaw's theorem only because it has a relative magnetic permeability ($\permMag$) greater than one — \ie, a permeability greater than that of the surrounding medium.
A separate analysis of magnetic levitation systems provides a more specific measure for testing the stability of magnetic systems with various boundary conditions \cite{reusch1994}.
Material with $\permMag<1$ is \emph{not} covered by the theorem since the magnetic flux from the diamagnetic material becomes dependent on the displacement of the permanent magnet; this violates the condition of Earnshaw's theorem that fixed magnetic fields be used, and so static levitation involving magnets and such diamagnetic material is very possible.
To demonstrate this, \citeauthor{boerdijk1956b} levitated a small cylindrical magnet of dimensions \diameter$\,\SI{1}{mm} \times \SI{0.3}{mm}$.

More contemporary studies on diamagnetic levitation look at the levitation of larger objects including strawberries and frogs \cite{berry1997,geim1998,geim1999,simon2000,simon2001}, using high-power electromagnets (on the order of \SI{10}{T}).

Unfortunately, these techniques are not suitable for large load bearing.
Even the most diamagnetic substance known, pure bismuth, has $\permMag \approx 0.9998$ — hardly different than that of air.
The forces exchanged via magnetic flux between magnetic and diamagnetic materials, therefore, are incredibly small and not suited at all to the purposes of this research.

Superconducting material, on the other hand, behaves ideally diamagnetic with $\permMag=0$, so the forces produced between a superconductor and a magnet are equal to the forces between two (equal) permanent magnets themselves.
This allows many exciting possibilities for stable levitation.
However, even the so-called `high temperature' superconducting materials must be cooled to very low temperatures in order to remain superconductive.
Such a requirement renders this method financially and functionally impractical for this research.
A review of work in the area of superconducting levitation has been published by \textcite{ma2003}.


\subsection{Basic magnetic suspension}

The most simple variety of magnetic levitation or suspension is the counter-acting of gravity with an active electromagnetic force.
An advanced approach for this application used backstepping with a nonlinear model to provide robust control without system identification \cite{mahmoud2003}.
\textcite{gentili2003}.
\textcite{agamennoni2004} were able to identify the nonlinear dynamics of a coil-iron suspension.

\textcite{chang2001} applied nonlinear control to the problem of magnetic levitation, using coupled hybrid magnets (\ie, electromagnets biased with permanent magnet cores) that create a magnetic circuit with the levitated table of \SI{20}{kg}.
The paper looks at nonlinear analysis and neither passive nor active vibration isolation results are shown.

Even now, basic \PID\ control is still used for control purposes.
\textcite{li2007} report their success in suspending a magnetic table using a coupled electromechanical model of the system.

\textcite{banerjee2008} used a simple cascaded \PI\ and lead controllers to stabilise an electromagnetic suspension.
An optimisation technique was used to obtain the control gains necessary to achieve adequate performance over a range of displacement gaps; such control is usually only suitable for fixed-gap systems.

\textcite{gosiewski2008} applied \Hinf/ control to a magnetic suspension.
Eh.

The negative stiffness of electromagnetic actuators has been used with a low-stiffness membrane \cite{sato2001}.


\subsection{Magnetic vibration isolation}

\textcite{nagaya1993} constructed a non-contact vibration isolation table; they report a high-stiffness spring with transmissibility that \enquote{can be controlled to be nearly zero}.
Their table used small magnets in a simple design, which could not support large loads.
The authors showed later a better control system for their `perfect noncontact active vibration isolation table' \cite{nagaya1995a}.

\textcite{watanabe1996} wrote a paper detailing a functional vibration isolator using electromagnetic springs, which could support weights of up to
\SI{200}{kg}.
The control system used was quite advanced, utilising a combination of two independent control systems for stable levitation and robust vibration isolation.
The magnetic actuator design is not described, however.
It is believed that high-powered electromagnets were required for this design; with the current availability of cheap rare-earth magnets, a more efficient design is now possible.

\textcite{puppin2002} published a paper that most plainly demonstrates that magnetic springs can be used for vibration isolation, in which the authors make no attempt to achieve contactless suspension—the magnets are horizontally constrained in guides.
Furthermore, the springs are only used as passive isolators for vibrations in the vertical direction; no active control is used.
Nonetheless, their passive spring was still capable of significant vertical vibration attenuation.


\subsection{Forces between magnets}

See \chapref{magnet-theory}.



\section{Nonlinear systems, especially those related to magnetic forces}

A very recent and comprehensive review of nonlinear vibration isolation systems for a broad range of techniques was published by \textcite{ibrahim2008}.
It highlights the importance of nonlinear analysis in this field: in some cases, better results can achieved by using nonlinear spring forces to couple to and absorb vibration energy (\fixme{eg}); in other cases, the behaviour of an isolator cannot be adequately modelled by using linear systems theory.

Interesting results have been shown using nonlinear springs to attach the vibration absorber to the structure, such as \textcite{jo2008} who use repulsive magnetic springs to produce a tuned absorber with a resonance at double the frequency of the main resonance of the structure.
\fixme{Why would you do this?!}

\textcite{mann2008} performed a preliminary investigation into the use of a nonlinear vibration mount for energy harvesting, using a magnetic suspension of repulsive magnets to create a Duffing-like oscillator.
Large damping ensured that the nonlinear regimes were only realised at large excitation amplitudes, but the idea is that highly nonlinear resonances have a much broader resonance peak through the higher branch around the jump phenomenon. 
\fixme{reference this elsewhere}

A similar idea is explored by \textcite{shahruz2008} for an energy scavenging cantilever beam that uses an arrangement of attracting magnets to shape the force characteristic of the response.
The aim is to achieve a power spectrum of the response to a random excitation that is greater than the predominantly linear response that is obtained without the magnets present.
No analysis of damping or energy harvesting potential was explored in this paper.

\textcite{zhang2008} use a nonlinear damper (specifically of the form viscous plus cubic with velocity; \ie, $a_1\dot x\fn{t}+a_3\ddot x^3\fn{t}$) to excite the structure at harmonic frequencies of the resonance.
This results in less energy at the frequency of vibration, although the resonance peak does remain.
While it does not seem likely that this method can compete with the reductions seen with the approach of a \vibneut/, this nonlinear damping method does have the advantage that it does not require tuning for a particular frequency and its effectiveness will not change with a time-varying resonance frequency.

\textcite{starosvetsky2008}, with a good review of the literature, talk about the `nonlinear energy sink' idea in which a nonlinear system provides better and broader vibration absorbing.

Various combinations of magnetic springs create different classes of approximate nonlinear oscillators, for which standard methods exist to analyse their behaviour \fixme{can I cite a book here I haven't read?}.
The characteristic property of nonlinear springs is their exhibition of so-called `jump phenomena' whereby the frequency response curve can take multiple values at a single frequency, depending on the initial conditions.

\textcite{jazar2006} analysed the behaviour of a nonlinear vibration isolation mount in detail, developing analytical models for the jump phenomena of a system with cubic stiffness and quadratic damping.
Critical values were illustrated to avoid the ill effects of the nonlinearities; additional damping had the general effect of decreasing the adverse nonlinear response.

For the purpose of vibration control, augmenting a linear spring system with nonlinear magnetic springs alters the behaviour of the natural frequency of the system to be weakly coupled to the mass of the system.
\textcite{dangola2006} analysed the nonlinear dynamics of a nonlinear system in which a variation of both stiffness and mass by up to 50\% yields an increase of stiffness of 6\%.
For an equivalent linear system, the natural frequency variation is ten times greater.
This is a very interesting result for loading elements for which the mass to be supported is largely variable, in that the frequency response will vary significantly less than for conventional linear springs.

However, for weakly nonlinear magnetic springs, variation in the mass will still lead to changes in the resonance frequency.
\textcite{todaka2001} created a mechanical linkage to support two magnets in repulsion such that as their air gap increased (due to less mass being supported), a horizontal offset between them was created to lower the linearised operating stiffness.
This allowed a much smaller variation between the $k/m$ ratio than for flush magnets in repulsion.

\textcite{bonisoli2007-mssp} used an experimental rig to analyse the nonlinear behaviour of a magnetic and linear spring in parallel, with more theoretical analysis published later in the year \cite{bonisoli2007-mrc}.
They showed a configuration of linear and magnetic springs with the notable feature that the resonance frequency exhibits little dependence on mass loading and nonlinear effects can be seen.
This effect was further demonstrated in parallel research by the same authors.

The field of nonlinear dynamics is very large, and applications to vibration suppression can result in surprising results.
\textcite{oueini1999} considered the response of a nonlinear (including cubic) plant with an additional cubic nonlinear feedback law and established that vibration attenuation was possible (like a \vibneut/?) and that nonlinear phenomena such as chaos existed.
Presumably, such phenomena are undesirable for vibration isolation!

Anyway, it is an interested argument to make that digital control in a vibration isolation situation is inappropriate.
When absence of all vibrations are required, the side-effects of digital control (\ie, chaotic effects due to quantisation and time delays \cite{csernak2007}) may prove deleterious for extreme applications.
On the other hand, any sufficiently expensive system should be able to reduce these effects to be arbitrarily negligible.
For the purposes of this thesis, such small effects are of no concern.

Nonlinear control is better than linear control, as argued cogently by \textcite{kokotovic1992}.
\textcite{queiroz2007} used nonlinear control to stabilise a magnetic bearing with pull--pull electromagnets with parameter uncertainties, while also minimising power consumption of the system.


\subsection{Exploring \qzs/ systems}
\seclabel{qzs-explore}

In a conventional mass--spring system, the static deflection increases as the stiffness of the support is reduced, and a lower limit on the stiffness is imposed by constraints on the allowable displacement.
Novel approaches are required to reduce the resonance frequency below that possible with a linear spring.
One approach is to use a mechanical linkage \cite{winterflood2001}.

\textcite{virgin2008} proposes an interesting alternative to the classical helical spring for vibration isolation support: a `pinched loop' created by a slender beam with both ends clamped has a wide range of tuning possibilities and offers isolation in two degrees of freedom.

Addition of negative stiffness elements in a design reduces the resonance frequency, which improves vibration isolation.
Early examples of such designs using inclined springs were shown by \textcite{molyneux1957}.
These have an approximately cubic force \vs\ displacement characteristic, which may be tuned to achieve a local region of zero stiffness, which is often termed `\qzs/'.
\textcite{alabuzhev1989} looked at the nonlinear behaviour of such
systems, and also more recently by
\textcite{carrella2006,carrella2007-jsv,carrella2008-jsv,carrella2009} and improved by
\textcite{kovacic2008}.
More recently, the dynamic response of these systems has been analysed \parencite{carrella2009,carrella2008-thesis} and shown to have prominent nonlinearities that distort the frequency response but that do not decrease the vibration isolation efficacy in general.

A variety of mechanical linkages and arrangements can be designed for \qzs/ \cite{tarnai2003}.
Mechanical \qzs/ elements, generally using flexible beam supports in a buckling regime, have been used in application for vibration isolation platforms \cite{platus1999}, mounts for seismic noise attenuation \cite{cella2005}, vibration attenuation from hand-held machinery \cite{sokolov2007}, reduction of aircraft cabin noise \cite{baklanov2007-jsv}, and vehicle driver suspension \cite{lee2007-jsv}.
Friction is a limiting factor in these devices \cite{sokolov2007}; this particular problem is obviated when non-contact supports are used.
Further detail into the field  of nonlinear passive vibration isolators is given in the recent review by \textcite{ibrahim2008}.

\QZS/ can also be achieved with magnetic systems.
Magnetic configurations with negative stiffness can be used to augment a positive stiffness support (which can be simply a conventional spring) to lower the resonance frequency.
For example, \textcite{beccaria1997} used this technique (under the term `magnetic antisprings') to improve the isolation for gravity wave detectors.
\textcite{carrella2007-euromech,carrella2008-thesis} has also used a attractive magnets in parallel with conventional springs to reduce the resonance frequency of the system.
Purely non-contact magnetic systems can also be used to similar effect \cite{robertson2006,robertson2007}.
Generally, systems that use the negative stiffness between attracting magnets cannot be brought to a stable \qzs/ region due to their `softening spring' characteristic.

\textcite{hol2006} discuss a `gravity compensator' that uses the idea of an axial bearing with \ang{90} rotated magnetisations to bear load in the vertical direction.
(Also see \textcite{yonnet1981} for related bearings that I hadn't thought about in enough detail yet.)
This creates something like a negative quadratic force relationship, which has zero stiffness in the centred vertical position.

Analysis of the nonlinear dynamics of such systems such as performed by \textcite{lee2004-jsv,kovacic2008,kovacic2009} can be quite involved and is outside the scope of this research.


\subsection{\QZS/ is not zero stiffness}
\seclabel{qzs-not-zerk}

It has been established that the goal of a `zero stiffness' device is to reduce the resonance frequency of the system to as low a value as possible.
In the limiting case, if the system is stable and the nominal force of the spring indeed matches the weight of the mass, then the gradient of the force at the operating point will equal zero.

However, it is necessary to use a nonlinear spring to achieve this zero stiffness condition.
And the behaviour of a nonlinear oscillator varies considerably from that of the classic linear spring.
Most obviously, the shape of the frequency response is not independent of the amplitude of the forcing disturbance.

Consider the stable single degree of freedom system
\begin{dmath}
m \ddot x + b \dot x + \stiffnessDuffing (x+s)^3 = 0, 
\end{dmath}
where $s$ is an induced displacement disturbance.
At the operating position $x=0$, the nonlinear spring stiffness is $3kx^2|_{x=0}=0$.
For a disturbance $s$, the spring is perturbed and generates a reaction force of $ks^3$ on the mass.
The stiffness here is $3ks^2$; \ie, dependent on the amplitude of disturbance. 
The ramifications of this nonlinear force on the vibratory response of the system are not exactly straightforward.

\textcite{tentor2001} analysed a spring generated by repulsion magnets which behaved as a Duffing oscillator for large amplitude vibrations.
\fixme{(The Duffing oscillator is well-known in the literature of nonlinear vibrations. Please reference it more.)}
The difference for his system was a significant linear component in the force equation:
\begin{dmath}
F_{\text{Duffing}} = \stiffnessLinear x + \stiffnessDuffing x^3.
\end{dmath}
The nonlinear dynamics only affected the response of the system when the nonlinear term dominated over the linear term.
For a zero stiffness spring, $\stiffnessLinear=0$ and the nonlinear dynamics are more significant.

Rather than perform a nonlinear analysis on the system above (which is known in the literature as a Duffing oscillator — anything else I need to add?), it is instructive to examine the power spectra generated with a range of spring stiffnesses and Gaussian inputs.

\Figref{cubic-resonance-disturb,cubic-resonance-stiffness} show the square root of the ratio of the power spectrum of $s$ and $x$, where $s$ is a white noise signal of variance $S$.
(The transfer function is not examined because it removes nonlinear components of the original signals.)
The results are compared with the system
\begin{dmath}
m \ddot x + b \dot x + k_{\text{lin}}(x+s) = 0, 
\end{dmath}
which has a linear stiffness $k_{\text{lin}}=3kS^2$ equivalent to the stiffness of the nonlinear spring at the variance displacement.

In both nonlinear systems simultions, significant nonlinearities in the response can be seen.
Maximum frequency and amplitude of the response increase with both greater input disturbance amplitude and greater nonlinear spring stiffness.

\begin{figure}
  \grf{Simulations/Zero_stiffness/eps/cubic-resonance-disturb}
  \caption{Cubic stiffness response with various amplitudes of
    disturbance in comparison to some approximately similar linear
    systems.}
  \figlabel{cubic-resonance-disturb}
\end{figure}

\begin{figure}
  \grf{Simulations/Zero_stiffness/eps/cubic-resonance-stiffness}
  \caption{Cubic stiffness response with varying values of the
    stiffness coefficient (with $S=1$ to compare with
    \figref{cubic-resonance-disturb}) in comparison to some
    approximately similar linear systems.}
  \figlabel{cubic-resonance-stiffness}
\end{figure}

When compared to the linear response of stiffness equivalent to the maximum stiffness reached by the white noise input, the linear system response is \emph{smaller} than the \qzs/ response.
This restricts the applications for vibration isolation of using nonlinear springs to achieve low resonance frequencies: only when it becomes infeasible to decrease the stiffness of a conventional linear system any further should a nonlinear system be chosen instead.
Conversely, the broader peak of a hardening spring has been shown to be useful for energy harvesting purposes, especially for wandering narrowband excitation \cite{ramlan2009-nd}.


\subsection{(Quasi--)infinite stiffness systems}
\seclabel{infstiff}

Both \textcite{nijsse2001} and \textcite{mizuno2003a} (with related
publications) talk about using a combination of a positive and
negative stiffness springs to achieve affects like
\begin{dmath*}[compact]
  k_T = \frac{ k_1 k_2 }{ k_1 + k_2 } = \infty 
  \condition{when $ k_2 = -k_1 $.}
\end{dmath*}

\textcite{xing2005} formalises this idea in their paper.
A swath of papers have been published by Mizuno over the years
\cite{mizuno2001,mizuno2002,mizuno2003a,mizuno2003b,mizuno2007}.
The work was extended by \textcite{hoque2006} for three degree of freedom vibration control using the same ideas.
The term `infinite stiffness' is perhaps unfortunate, as it's not true in any useful sense of the term.
That is, their static behaviour converges to something that looks like infinite stiffness, but \emph{dynamically} they're the same as any other structure.
This is because the negative spring in combination must be stabilised somehow, and this stabilisation removes the nice property of summing to zero with the other stiffness in the system.

\textcite{mizuno2003c} present a modified version of the idea, whereby rather than using magnetic springs in attraction, as in their other work, this paper looks at using a voice coil actuator with designed negative stiffness.
A controller must be used to obtain this negative stiffness.

