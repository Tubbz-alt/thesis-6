\documentclass[11pt,a4paper]{memoir}
\usepackage{thesis-preamble}
\EndPreamble
\begin{document}
\chapter{Introduction}

%%%%%%%%%%%%%%%%%%%%%%%%%%%%%%%%%%%%%%%%%%%%%%%%%%%%%%%
\section{The central themes of this thesis}
%%%%%%%%%%%%%%%%%%%%%%%%%%%%%%%%%%%%%%%%%%%%%%%%%%%%%%%

Before launching into detail into the various topics under investigation, this section briefly touches on the issues addressed over the entire thesis: vibrations and their isolation and suppression in \secref{vibrations-summary-intro}; permanent magnets and their role in the design of supporting structures and other devices in \secref{magnets-summary-intro}; and `\qzs/' systems, which unify in this thesis the fields of vibrations and magnets, in \secref{qzs-summary-intro}.
The remainder of this chapter is then devoted to investigating these three broad areas in more detail in \secrangeref{vibrations-explore}{magnetsisolation-apps}, concluding with the structure and contributions of this thesis (\secref*{thesis-structure}).


\subsection{The problem of vibrations}
\seclabel{vibrations-summary-intro}

The disturbing effects of vibrations are a well-known and everlasting problem.
The transmission of vibrations from a source can only ever be reduced, not eliminated — that is, not without removing the source entirely from the local region affected.
And in many cases there is no single source; the very ground itself may be a medium through which undesirable vibrations are transmitted.
Earthquakes are an extreme example of this, but on a smaller scale there are continuous time-varying displacements of the `fixed' ground beneath us.
   \note{Not to be too earth-centric, but many of the ideas here have a different relevance in off-planet circumstances.}
The earth should in fact be considered as a distributed vibrating structure, of very great mass in total, that has a range of displacement profiles dependent on the local surrounding impedance conditions.

Whether the source of the disturbance is near or far, or how it propagates through the ground to arrive at the region of interest, these disturbances can cause a variety of problems on equipment that is required to, ideally, remain absolutely still.
A good example is during the electro-lithography performed to construct computer processors, in which nanometre-sized disturbances can affect the overall quality and yield of the silicon wafer produced.
Mitigating the effects of a disturbance through the base on which a structure is supported is known as `vibration isolation' and is the over-arching problem in which the work of this thesis should be put in context.

A contrasting vibration problem occurs when some manner of machine causes its own vibration; a well-known example is a washing machine that exerts an oscillating disturbance force on itself through a mass imbalance.
This type of vibration problem requires a rather different set of design solutions and often its solution acts in opposition to the vibration isolation problem discussed previously.
Reducing the effects of self-induced vibration disturbance will be termed `vibration suppression'
  \note{The descriptor `vibration isolation' is sometimes used to refer to both problems, but in this work it is useful to differentiate between them.}
for the purposes of this thesis and will be revisited on occasion herein.

There are a variety of `classic' solutions for both vibration isolation and vibration suppression.
A particularly simple solution for \emph{both} problems is to mount the equipment on a many-tonne slab of concrete, steel, or granite.
This is not always practical.
Another common approach for vibration isolation is to support the equipment with pneumatic springs.
When in operation, these springs provide a low supporting stiffness and low static deflection; they can typically be used to support hundreds of kilograms with a resonance frequency of less than five hertz.
Research in these various methods is still ongoing \cite{yoshioka2001,chen2007,kawashima2007,hong2010-rsi}.

Other support methods besides pneumatics are able to achieve low stiffness; this is an area that will be further investigated in the literature review.
But one method in particular is interesting in this context: permanent magnets can provide low stiffness support without energy expenditure or air supply.
Their nonlinear forces in both attraction and repulsion allows the possibility of interesting supporting designs, and their non-contact nature allows their use in vacuum and `clean-room' environments.


\subsection{Permanent magnets used for mechanical design}
\seclabel{magnets-summary-intro}

The last twenty years has seen the maturation of the rare earth permanent magnet industry.
These magnets are now widely available in large sizes and strong magnetisation at relatively low prices.
They are now used in a variety of mechanical designs, including bearings, couplings, and \maglev/ trains, all of which take advantage of non-contact attractive and/or repulsive forces.
Magnets can be used to make other magnets move or to keep them in place; there is little limit to the ingenuity of their application.
However, due in part to this complex behaviour between them there are few design guidelines to aid their use:
\begin{quote}
It would be virtually impossible to find more than ten men in the United States who can properly design a permanent magnet for a wide variety of products in which the magnet is a critical operational element. \cite{moskowitz1995}
\end{quote}
We can speak broadly about their integrated use in force design: magnets can be used in conjunction with current-carrying coils to effect time-varying forces (as in shakers and speakers); soft iron can be used to guide the magnetic fields into desired regions or away from unwanted areas (\eg, latches and motors); or magnets can be used alone for unique force--displacement characteristics or simply for applying non-contact forces (\eg, rotational bearings).

These ideas in mechanics and dynamics have application back to the field of vibrations.
A `synergy' between the two fields is seen in areas such as energy harvesting from ambient vibration, the study of vibration in high-speed magnetic bearings, and one of the main themes of this thesis — nonlinear and/or non-contact forces for vibration isolation supports.

Supporting a mass with a non-contact force can also be called `levitation', a topic that deserves its own mention.
In the mid-1800s, \textcite{earnshaw1842} proved that levitation with the force of permanent magnets alone was impossible, although this did not become common knowledge \note{If it can even be said to be `commonly known' today.
Anecdotal evidence suggests otherwise.} until much later.
Exceptions to `Earnshaw's Theorem', those being systems in which non-contact levitation is possible, include the use of diamagnetic materials and actively-controlled electromagnetics, amongst some others.
It is the possibility of overcoming the instability of levitated magnets by active means that is of interest in this thesis.


\subsection{\QZS/ systems}
\seclabel{qzs-summary-intro}

The transition between stable and unstable forces becomes interesting in the context of vibration isolation.
Between positive and negative stiffness in a force \vs\ displacement characteristic, there is an inflexion point of zero stiffness.
This point is termed a `\qzs/' position to emphasise that the dynamic behaviour of the system in this condition can be rather complex and usually unstable.
`True' zero stiffness would imply \emph{no} connection between between the levitated object and the base, as if they were floating in free space — the motion of one would have no effect on the motion of the other.

As systems approach \qzs/, their vibration isolation improves as the resonance frequency decreases.
Operation at the \qzs/ position is not possible as the system is, at best, only marginally stable, and the system must be tuned (based on the applied loading) as close to the \qzs/ position as possible to achieve best results.

Certain magnetic systems are not the only ones to exhibit \qzs/, as will be examined in more detail in \secref{qzs-explore}.
The phenomenon was first proposed using inclined springs to achieve negative stiffness with a `snap-through' effect.
Magnets are more convenient in many ways than inclined or buckling springs in that the negative stiffness can be applied directly without having to exploit the byproduct of a mechanical spring or linkage arrangement, which can be more bulky.

Active control systems can be used with \qzs/ systems to improve their performance in one of three ways:
\begin{enumerate}
  \item Standard active vibration control with velocity feedback;
  \item Remove or limit the instability at the \qzs/ location with a control system;
  \item Online tuning of the system for load-independent operation.
\end{enumerate}
Only the first of these strategies is investigated in this thesis.


\subsection{Project context}

The original seed of the idea for this project was to design and build a vibration isolation table using non-contact magnetic springs.
This goal can be split into two: the design of a non-contact magnetic spring (suitable for a vibration isolation table); and the design of the vibration isolation table itself.

Vibration isolation tables are generally designed to attenuate natural disturbances from the ground to the tabletop.
Current commercial models use pneumatic springs to perform this task, and this project arose out of curiosity: could magnetic springs be used instead?

Using magnets for load bearing brings its own set of challenges.
For completely non-contact support, active control must be used to stabilise \emph{at least} one \dof/.
For the design to be worth investigating, some advantage to using magnets should also be demonstrated.
  \note{Although I took much pleasure in explaining over the years that my \PhD/~project was simply to `build a table that floats on magnets'.}

However, the field of active control has been well-established and the feat of building a stabilising controller for a system with relatively simple dynamics is not worthy of the research for a \PhD/.
The work presented in this thesis is the multiplicity of investigations that arose around the idea of building a `table that floats on magnets', pulling out enough interesting nuggets to prove worthy of the title of `research'.





%%%%%%%%%%%%%%%%%%%%%%%%%%%%%%%%%%%%%%%%%%%%%%%%%%%%%%%
\section{Vibrations}
\seclabel{vibrations-explore}
%%%%%%%%%%%%%%%%%%%%%%%%%%%%%%%%%%%%%%%%%%%%%%%%%%%%%%%

The field of active vibration isolation is a broad topic to cover in review; not everything will be able to be covered here, but it is important to have an overview of what has gone before to place this work in context.
The literature review that follows is strongly biased towards papers that have been recently published.
Tracking their citations backwards will yield a tangled web of prior art in the field of vibration control.

\subsection{Forms of vibration control}

Many descriptions are given to various systems and types of vibration control.
As mentioned in the introduction to this chapter, `vibration isolation' is the main objective of this literature review.
It is instructive to illustrate some of the alternatives and define specifically the terminology used in this thesis.

The most basic vibratory or oscillating system is shown in \figref{simple-suppression}, in which a mass is excited by an external source and behaves with resultant dynamics determined by the stiffness and damping of the connection.
Such a system can also be excited by internal forces such as a rotating imbalance.
The ground to which it is fixed is assumed to have infinite impedance and to have zero displacement.
In this system, motion of the mass can be suppressed by increasing the stiffness of the support, since as the mass becomes more greatly coupled with the ground, the input force has a diminishing influence.
This behaviour is referred to as `vibration suppression' in this thesis.
%A practical example of this sort of vibration problem is a washing machine that causes itself to vibrate through a rotating imbalance; the imbalance causes oscillations in the drum which, when great enough, forces the drum off its axis of rotation. \note{A common problem in our household. Just have to learn to arrange the washing carefully inside the drum.\footnotemark} \footnotetext{Our washing machine has since broken. Imbalance matters!}

\begin{figure}[!b]
  \subbottom[Vibration suppression from disturbance force $\forceDisturb$.\figlabel{simple-suppression}]{\asyinclude{\jobname/vibration-sdof}}
  \hfil
  \subbottom[Vibration isolation from disturbance input $\dispBase$.\figlabel{simple-isolation}]{\quad\asyinclude{\jobname/vibration-base}\quad}
  \lofcaption{Two main varieties of vibration control problem.}
  {
    In both cases, mass $\mass$ is being excited; input control force $\forceIn$ can be generated using feedforward or feedback control to minimise the displacement $\dispMass$ of the mass.
  }

\end{figure}

If the attachment of the mass in \figref{simple-suppression} is assumed \emph{not} to be infinitely massive and stiff, the problem becomes not only to suppress the motion of the mass but also to prevent force transmission from the input disturbance into the base itself.
%A practical example is a piece of vibrating industrial equipment that radiates vibrations through the ground, causing noise and generating ground-borne disturbances for other machinery.
This dual problem is not so easily solved; there is a trade-off in the self-induced displacements of the machinery and the force transmitted to the ground.
By lowering the stiffness of the support, the transmitted force is reduced but the self-induced displacements are increased.
Due to reciprocity, decreasing the transmitted force from the mass to the ground is equivalent to decreasing any disturbances transmitted from the ground to the mass.
Isolating the mass from ground vibration is known in this thesis as `vibration isolation' and is the main vibratory problem considered herein.
For the purposes of this research, there is assumed to be no self-induced vibration in the system.
A schematic of this type of vibration isolation system is shown in \figref{simple-isolation}, for which a practical example is protecting sensitive equipment from ground-based disturbances.

The schematics illustrating vibration suppression and vibration isolation have been shown with active input forces $\forceIn$ that can be used to tune or adjust  oscillations of the systems.
In these examples, it is assumed that the input forces have a negligible effect on the dynamics of the ground or base.
Further complications arise when this assumption no longer holds, such as when the device is mounted on a flexible structure with mass that is not much greater than the mass being supported.
For the sake of this work, this more complex case is not examined in detail.

\subsection{Shock isolation}
\seclabel{shock}

It should be noted that vibration isolation is generally designed to prevent the transmission of broadband noise.
For transient signals, a broadband vibration isolation system may not perform as well as a system designed to reject `shock' inputs.
\textcite{balandin1998} review the field of optimal control as applied to shock and vibration isolation problems.
\note{Their comment that the \enquote{number of papers is so great that there is little incentive to discuss them here} does not bode well for any attempts by me to even summarise their review.}
They differentiate shock and vibration isolation succinctly:
\begin{quote}
The operating quality of shock isolators is usually described in terms of certain characteristics of the transient motion of the body being isolated, whereas the quality of vibration isolators is determined by the characteristics of steady-state forced oscillations.
\end{quote}
That is, optimising for shock will result in minimising, say, the peak displacement of the mass, whereas optimising for vibration will result in a low natural frequency (characterised by a minimum achievable \RMS/ displacement).
Since an optimal controller is based around a cost function that will be dependent both on the vibratory system itself and the mode of disturbance (transient, broadband noise, \etc), such control approaches are heavily case-specific and are best used when a plant is pre-determined and a vibration problem needs to be considered after the fact.
\textcite{bolotnik2001} discusses these methods in more detail; in this thesis only vibration isolation shall be discussed.

\subsection{Fundamentals of active vibration isolation}
\seclabel{fundavibes}

Introduced in the previous section, \figref{simple-isolation} shows the simplest form of vibration isolation in which a mass $\mass$ is mounted with stiffness $\stiffnessRel$ with disturbance input $\dispBase$.
Passive isolation is achieved for zero input control force, $\forceIn=0$.
Feedback control allows the properties of the system to be adjusted according the desired vibration response.

The dynamic response of the isolated mass $\mass$ is
\begin{dmath}[label=simple-isolation]
  \mass\accMass +
  \dampingRel\gp{\velMass-\velBase} +
  \stiffnessRel\gp{\dispMass-\dispBase} = \forceIn
\end{dmath}.
First assume that there is no input force; taking the Laplace transform and rearranging produces the transmissibility $\transmissibility$ of the system in the frequency domain:
\begin{dmath}[compact,label=simple-isolation-freq]
  \transmissibility = \frac{\laplaceMass}{\laplaceBase} =
  \frac{\ii\freq\dampingRel + \stiffnessRel}
  {-\mass\freq^2 + \ii\freq\dampingRel + \stiffnessRel}.
\end{dmath}

For ideal linear state feedback control, the input force can be represented as a linear combination of measured relative displacement $\gp{\dispMass-\dispBase}$, relative velocity $\gp{\velMass-\velBase}$, and acceleration $\accMass$ of the mass.
A force sensor can also be used for feedback purposes, but this yields results for lumped-parameter systems equivalent to acceleration feedback; for vibration isolation of flexible systems, there is evidence to suggest force feedback giving greater stability than acceleration feedback, especially for lightly damped structures \cite{preumont2002-jsv}.
Absolute velocity $\velMass$ of the mass can be estimated by integrating the acceleration or through geophone measurements of the velocity, absolute displacement $\dispMass$ can be estimated through further integration, and for completeness we also consider the relative acceleration between the mass and the base ($\accMass-\accBase$) as a possible signal for feedback control.
The generalised feedback force can then be represented by
\begin{dmath}[label=feedback-force]
 \forceIn =
   \gainAcc \gp{\accMass -\accBase}  +
   \gainVel \gp{\velMass -\velBase}  +
   \gainDisp\gp{\dispMass-\dispBase} +
   \gainSkymass   \accMass +
   \gainSkyhook   \velMass  +
   \gainSkyspring \dispMass
\end{dmath},
where $\gainArbitrary$ are feedback gains chosen appropriately for a given application.
The term involving $\gainSkyhook$ is sometimes known as skyhook damping, which is discussed in more detail in \secref{skyhook-intro}.
Substituting the feedback force \eqref{feedback-force} into the system dynamics \eqref{simple-isolation} yields
\begin{multline}
  \gainAcc \gp{\accMass -\accBase}  +
  \gp{\dampingRel+\gainVel}\gp{\velMass-\velBase} +
  \gp{\stiffnessRel+\gainDisp}\gp{\dispMass-\dispBase} + \\
  \gp{\mass+\gainSkymass}\accMass +
  \gainSkyhook\velMass +
  \gainSkyspring\dispMass
  = 0 \,.
\eqlabel{isolation-feedback}
\end{multline}
The transfer function between base and mass displacement for this generalised feedback case is
\begin{dmath}[label=tf-genfeedback]
  \frac{\laplaceMass\fn{\s}}{\laplaceBase\fn{\s}} =
  \frac{
          \gainAcc \s^2 +
          \gp{\dampingRel+\gainVel}\s +
          \gp{\stiffnessRel+\gainDisp}
       }
       {
          \gp{\mass+\gainAcc+\gainSkymass} \s^2
         +\gp{\dampingRel+\gainSkyhook+\gainVel} \s
         +\gp{\stiffnessRel+\gainDisp+\gainSkyspring}
       }
\end{dmath},
where $\s=\ii\freq$ is the Laplace variable.
From \eqref{isolation-feedback} it can be seen that there is an exact equivalence between the feedback gains of three of the different signals ($\gainDisp$, $\gainVel$, $\gainAcc$) and a corresponding physical parameter of the system ($\stiffnessRel$, $\dampingRel$, $\mass$).
Relative displacement and velocity feedback correspond to a variation in the stiffness and damping, respectively, and absolute acceleration feedback to the mass.

\eqref{tf-genfeedback} describes a system which is stable for characteristic solutions of the denominator (\ie, the poles of the system) which have negative real components. These can be analysed easily with the quadratic equation. Note that the feedback gains for the absolute versus relative signals always appear together in the denominator of the transfer function (\eg, $\gainAcc+\gainSkymass$), and hence affect the stability of the system identically.

To illustrate the effect of various control gains, from transfer function \eqref{tf-genfeedback} the transmissibility magnitude $\Abs{\transmissibility}$ will be drawn in the following sections with parameters $\mass=\SI{1}{kg}$, $\damping=\SI{0.5}{kg/s}$ and $\stiffness=\SI{10}{N/m}$, and each of the six control gains $\gainArbitrary$ varied independently.

\subsubsection{Displacement feedback}
\Figref{disp-vs-sky} shows the effect of varying the control gains for relative displacement and absolute displacement ($\gainDisp$ and $\gainSkyspring$).
In both cases, the resonance frequency is increased with increased feedback gain.
This is usually detrimental to vibration isolation performance.
Relative displacement feedback corresponds to an increased stiffness and a higher resonance frequency, whereas absolute displacement feedback increases the vibration isolation at low frequencies; this scheme is notable for its less than unity response even as the frequency of excitation tends to zero, while the high frequency attenuation is unaffected.

The system with displacement feedback is stable according to the inequality
\begin{dmath}
  \damping>\Real{\sqrt{\damping^2-4 \gp{\gainDisp+\gainSkyspring+\stiffness} \mass}}
\end{dmath},
which holds for all $\gainDisp+\gainSkyspring>-\stiffness$. Therefore, negative feedback gain may be used to lower the resonance frequency of the structure by decreasing the effective stiffness of the system.

This idea presupposes that the absolute displacement is a state that can be measured.
Disregarding slow and inaccurate sensors that can do this directly (such as using the Global Positioning System), the absolute displacement of an object can only be estimated based on other measurements of the system.
The simplest form of this is by double-integrating an accelerometer signal, such as used by \textcite{zhu2006} in combination with other control techniques for vibration isolation in micro-gravity.
The filters required to avoid drift of the signal in this case add low frequency poles to the system; as a result, displacement feedback in practice is highly susceptable to low frequency instabilities and cannot reliably be implemented.
This topic is discussed further in \secref{vibes-feedback}.

\subsubsection{Acceleration feedback}
\Figref{acc-vs-sky} shows the effect of varying the control gains for relative acceleration and absolute acceleration ($\gainAcc$ and $\gainSkymass$).
Absolute acceleration feedback corresponds to an increased system mass, corresponding to a decreased resonance frequency; high frequency vibration isolation is improved.
Relative acceleration feedback is only included for completeness; it has the effect of reducing the resonance peak but effectively eliminating any vibration isolation characteristics at higher frequencies.
Other vibration schemes for reducing a tonal or narrowband disturbance are discussed briefly in \secref{vibneut}.

The system with acceleration feedback is stable according to the inequality
\begin{dmath}
  \Real{\frac{\damping \pm \sqrt{\damping^2-4 \stiffness \gp{\gainAcc+\gainSkymass+\mass}}}{\gainAcc+\gainSkymass+\mass}}>0
\end{dmath},
which is true for all $\gainAcc+\gainSkymass>-\mass$.
It is therefore possible, albeit generally undesirable, to increase the resonance frequency of the system with negative acceleration gain, which corresponds to an effective decrease of the mass of the system.

\subsubsection{Velocity feedback}
\seclabel{skyhook-intro}

Acceleration and displacement feedback both primarily affect the resonance frequency of the system; velocity feedback is different.
The transmissibilities due to the influence of relative velocity and absolute velocity feedback ($\gainVel$ and $\gainSkyhook$) are shown in \figref{vel-vs-sky}.
    \note{In the literature, absolute velocity feedback is often referred to as `skyhook damping'. This term will generally not be used in this thesis as there is potential for confusion with `semi-active skyhook damping', a technique discussed later in this chapter which is used in rather different contexts.}
In both cases the resonance peak is lowered; the relative velocity feedback corresponds to a reduction in attenuation at higher frequencies, while for absolute velocity feedback the high frequency response is unaffected.

\begin{figure}[p]
   \begin{wide}
   \begin{subfigure}
     \psfragfig{\phdpath Simulations/Springs/fig/sdof-relspring}\hfil
     \caption[Relative displacement feedback.]{Relative displacement feedback, $\gainDisp$.}
   \end{subfigure}\hfil
   \begin{subfigure}
     \psfragfig{\phdpath Simulations/Springs/fig/sdof-skyspring}
     \caption[Absolute displacement feedback.]{Absolute displacement feedback, $\gainSkymass$.}
   \end{subfigure}
   \end{wide}
   \caption{Displacement feedback control.}
   \figlabel{disp-vs-sky}
   \begin{wide}
   \begin{subfigure}
     \psfragfig{\phdpath Simulations/Springs/fig/sdof-accrel}
     \caption[Relative acceleration feedback.]{Relative acceleration feedback, $\gainAcc$.}
   \end{subfigure}\hfil
   \begin{subfigure}
     \psfragfig{\phdpath Simulations/Springs/fig/sdof-skymass}
     \caption[Absolute acceleration feedback.]{Absolute acceleration feedback, $\gainSkyspring$.}
   \end{subfigure}
   \end{wide}
   \caption{Acceleration feedback control.}
   \figlabel{acc-vs-sky}
   \begin{wide}
   \begin{subfigure}
     \psfragfig{\phdpath Simulations/Springs/fig/sdof-vel}
     \caption[Relative velocity feedback.]{Relative velocity feedback, $\gainVel$.}
   \end{subfigure}\hfil
   \begin{subfigure}
     \psfragfig{\phdpath Simulations/Springs/fig/sdof-sky}
     \caption[Absolute velocity feedback.]{Absolute velocity feedback, $\gainSkyhook$.\figlabel{skyhook-ideal}}
   \end{subfigure}
   \end{wide}
   \caption{Velocity feedback control.}
   \figlabel{vel-vs-sky}
\end{figure}

Velocity feedback is stable for
\begin{dmath}
  \damping+\gainSkyhook+\gainVel \pm
    \Real{\sqrt{\gp{\damping+\gainSkyhook+\gainVel}^2-4 \stiffness \mass}}>0
\end{dmath},
which is true for $\gainSkyhook+\gainVel>-\damping$.
While increased damping is usually desired, it is possible to reduce the effective damping in the system with negative velocity feedback gain, with the effect of increasing the amplitude of the resonance peak.
Such active damping reduction has been performed to aid the efficacy of `tuned mass dampers' or vibration neutralisers \cite{kidner1998}, which are discussed briefly in \secref{vibneut}.

The absolute and relative velocity feedback results may be compared by calculating the \RMS/ transmissibilities over a frequency range of interest ($\sqrt{\Int{\transmissibility}{\freq,\freq_1,\freq_2}}$) as a function of increasing feedback gain for the two cases.
This is shown in \figref{rms-transmissibility}, where the relative feedback \RMS/ transmissibility has a local minimum whereas the absolute feedback case continuously decreases.
It is clear in the ideal case that absolute velocity feedback is the more effective at reducing the total vibration of a system.
(The maximum frequency in this case was chosen to be much greater than the resonance frequency; $[\freq_1,\freq_2]=[0,\SI{1000}{rad/s}]$.)

\begin{figure}
   \psfragfig{\phdpath Simulations/Springs/fig/rms-transmissibility}
   \caption[\expandafter\MakeUppercase\RMS/ transmissibility versus feedback gain of relative and
   absolute (`skyhook') velocity feedback control.]{
   \expandafter\MakeUppercase\RMS/ transmissibility versus feedback gain of relative and
   absolute (`skyhook') velocity feedback control of the system shown in
   \figref{simple-isolation}.}
   \figlabel{rms-transmissibility}
\end{figure}

Absolute velocity feedback control has been widely implemented in the vibrations literature \cite[for example]{elliott2001,elliott2004,yan2006,kim2008-iecst}.
Instability can occur in cases when the applied control force affects the dynamics of the base, but this is not a problem when considering comparatively massive support structures.


\subsubsection{The analogy with passive control}

Three of the aformentioned active approaches correspond directly to variations for one of the physical parameters of the system: mass, damping, and stiffness.
Active vibration control is usually only attempted when such passive control is inadequate or impractical.
\Eg, adding mass reduces the resonance frequency, which improves vibration isolation; however, it is not always possible to do so \dash consider, say, weight requirements in a vehicle or airborne structure.

Adding additional damping through viscous elements, equivalent to relative velocity feedback control, will reduce the amplitude of the resonance peak, but vibration isolation at higher frequencies will be degraded.
This is a common technique used for ambient energy harvesting, a recently active field of research, accomplished through electromagnetic damping through induced eddy currents in a coil experiencing relative velocity to a magnetic field \parencite{graves2000,stephen2006}.
The technique can be considered as passive damping control with the side-effect of energy generation.
The use of a nonlinear viscous damper provides transmissibility benefits that approach the performance of active absolute velocity feedback  \cite{lang2009}.

Reducing the stiffness between the mass and its supporting base has the same effect as increasing the mass: the resonance frequency is reduced and the bandwidth of vibration isolation is increased.
Passive stiffness reduction can be achieved by adding a negative stiffness element in parallel with the system \cite{lee2007-jsv,xing2005}.
This idea becomes a major theme of the thesis, explored later in this introduction in \secref{qzs-explore}.
Considering variable stiffness elements that can act in a `semi-active' regime, many methods exist to dynamically adjust the stiffness depending on the support being used \cite[for example]{kidner2002,liu2006-jsv,liu2008-jsv}.
Semi-active stiffness modification is used more commonly in \vibneut/ applications, described later in \secref{vibneut}, since for vibration isolation applications, in general, the lower the stiffness the better; there is no need to adjust the stiffness on-line if it is already as low as possible.

\subsection{Semi-active skyhook damping}

It is important to distinguish between active velocity feedback control discussed previously and semi-active skyhook damping, as both are referred to by the term `skyhook' in the literature.

Semi-active skyhook damping control was introduced by \textcite{karnopp1974-jei}, and is often used in vehicle suspension where active systems are impractical due to weight and power constraints.
In this version of skyhook damping, dynamic changes in stiffness or damping are used as a feedback mechanism to approximate absolute velocity feedback control \cite{ahmadian2004,leavitt2007}.
This is achieved by switching on the energy dissipation element when it will resist the motion of the isolated mass and switching it off otherwise.
Semi-active control has the advantages of robustness and low power requirements, especially when large forces would be required for active control.
Its application is largely directed at vehicle vibration control where it is not practical to use purely active control, and thus falls outside the scope of vibration techniques of interest in this thesis; it is important to distinguish between the two different forms of `skyhook damping', however.

In the original and crude method of implementing skyhook control in this manner, the switching mode that is necessary to emulate the `skyhook' in the controlled damping introduces higher order (odd-multiple) harmonics in the frequency response, as shown by \textcite{ahmadian2001}, who later proposed two `jerk-free' skyhook algorithms to avoid this problem \cite{ahmadian2004}.
A number of skyhook-like semi-active damping methods, including smoothing functions to eliminate the problem of jerk, were demonstrated shortly afterwards
\cite{liu2005}.
\textcite{song2007} used the nonlinearities of early semi-active skyhook control as justification for the design of an adaptive controller for vibration isolation, although they did not compare their results with recent work in the skyhook area.

\subsection{Active vibration control in practice}
\seclabel{vibes-feedback}

Velocity feedback control using integrated accelerometer measurements at the location of actuation has been used for some time as an effective vibration isolation mechanism, shown for example by \textcite{kim1999}.
However, the techniques discussed in \secref{fundavibes} do not address the practical implications of measuring and estimating the various states (relative and absolute, displacement/velocity/acceleration feedback) for feedback in a control system.
Such practical implications have a significant effect on the performance and suitability of the various control schemes.
As one example, \textcite{serrand2000} discuss absolute velocity feedback applied to a two \dof/ structure with emphasis on the effects, for their system, due to possible base flexibility.
They showed, amongst other results, that using integrated accelerometer measurements as a velocity feedback term has a low frequency phase shift due to filters in accelerometer charge amplifiers that can induce instability for sufficiently high control gains.

The general limits of vibration control in practical applications of single \dof/ vibration suppression systems have been studied over the last decade \cite{ananthaganeshan2001,brennan2007-jsv}. Time delays and phase lags introduced by digital filters
\note{When absence of all vibrations are required, the side-effects of digital control (which can be as extreme as chaotic effects due to quantisation and time delays \cite{csernak2007}) may prove deleterious for extreme applications.
On the other hand, any sufficiently expensive system should be able to reduce these effects to be negligible.
For the purposes of this thesis, such small effects are of no concern.}
and integrators can have quite significant effects over the ideal case of pure displacement or velocity or acceleration feedback.
As stated by \textcite{williams2009}:
\begin{quote}
In theory integration of accelerometer signals is easily done; however, in practice, inertially referenced velocity proves to be as \mbox{elusive} as it is useful.
\end{quote}
Real integrators (as opposed to ideal integrators) and high pass filters cause instabilities at low frequencies.
Acceleration feedback has a much smaller stability limit than either displacement or velocity feedback, and the effectiveness of displacement feedback is strongly reduced even with small time delays.
Therefore velocity feedback control from integrated accelerometer measurements should be considered the better choice.
Also, the presence of a low pass filter does not significantly affect the efficacy of velocity feedback.

A similar problem was addressed by \textcite{zhu2006}, who examined the use of such feedback for vibration isolation in a highly sensitive micro-gravity environment and show that accelerometer `PD' feedback is unsuitable due to instability at high (\ie, useful) control gains from quantisation and anti-aliasing side-effects.
Rather, integral and double-integral feedback from an accelerometer give better results for their system, which is consistent with later work \parencite{brennan2007-jsv}.

Numerous studies have been published on similar themes, as each vibratory system analysed will have different interactions in closed loop with the filters and time delays inherent in feedback control.
For example, \textcite{zhao2007} examined the vibratory behaviour of a two \dof/ nonlinear system with time-delayed position feedback, emphasising the importance of accounting for time delays in feedback control systems.

This literature on fundamental active vibration control has been presented to suggest that absolute velocity feedback is the more robust and effective method to improve vibration isolation, with several studies successfully using it in practical systems.
It is a control methodology that can be applied to multi-\dof/ systems; \eg, \textcite{engels2008} discuss control gain selection for velocity feedback control systems for the cases of centralised and decentralised control devices.
In centralised control, a global model of the system is used when allocating the feedback signals to each actuator.
In decentralised control, each sensor/actuator pair operates independently at each separate mounting point.
For the simple two-\dof/ vibrating system examined by \textcite{engels2008}, centralised control performed only slightly better; in a system with more degrees of freedom, it is expected that centralised control will show greater benefit.
An analogous result was shown by \textcite{hoque2006} for a three-axis vibration platform supported by a so-called `infinite stiffness' magnet/spring system (also see \secref{infstiff}).

The active and semi-active feedback methods introduced thus far are orthogonal in the sense that more than one scheme may be applied in parallel.
For example, acceleration feedback could be used to reduce the resonance frequency, and absolute velocity feedback used to reduce the height of the resonance peak.
This approach was used by \textcite{savaresi2007} to implement a combination absolute velocity/acceleration-feedback controller in the context of vehicle vibration control for ride smoothness.
Similarly, \textcite{gavin2007-jsv} discuss using negative stiffness and skyhook damping in the context of isolating machinery in a building.
Once the degrees of freedom increase more complex control methodologies become necessary.
\textcite{kerber2007} investigated a plate-plate coupling through four contact points in a six degree of freedom system and applied active vibration isolation in the vertical direction using a range of control methods.
Highlighting the relevance here, an \Hinf/ control method significantly outperformed the use of velocity feedback.


\subsection{Vibration neutralisers and narrow-band vibration control}
\seclabel{narrowband}

Until this point in the introduction, the only focus on the vibration control literature has rested upon the area directly related to vibration isolation through modification (whether active, semi-active, or passive) of the supporting structure for the mass.
The landscape for vibration control is much broader, however.
Here, two minor digressions are made to place the rest of the literature in context with alternate approaches of vibration reduction, and some discussion made on why the techniques herein were deemed unsuitable for the work of this thesis.


\subsubsection{Inertial actuators}

Sometimes it is not possible to integrate the control mechanism into the support of a structure, in which case external actuators need to be added to the device to provide the control forces.
These tend to be inertial electromagnetic actuators, also known as `proof-mass' actuators, where the mass of the moving element provides an external force via coupling to the structure.
This is shown schematically in \figref{vibration-absorb}.

\begin{figure}
   \asyinclude{\jobname/vibration-inertial}
   \lofcaption{An inertial force $\forceAbsorb$ designed to reduce the vibration
   response $\dispMass$ due to disturbance $\dispBase$.
   }{
   The inertial actuator
   has dynamics of its own ($\massAbsorb$, $\stiffnessAbsorb$, $\dampingAbsorb$) that
   influence the overall vibration of the structure.}
   \figlabel{vibration-absorb}
\end{figure}

For inertial actuators, velocity feedback on its own can have problems with stability margins at low frequencies.
\textcite{benassi2002-part2} showed through experiment that stability can be improved by the addition of a phase lag controller with force feedback in conjunction with velocity feedback.
The theory for such `combined-state' feedback cases was developed at the same time \cite{benassi2002-double}.

A number of feedback combinations were analysed by \textcite{diaz2005} focussing on various forms of velocity feedback for \mbox{single-,} \mbox{double-,} and multi-\dof/ vibration isolation systems.
In the two \dof/ system, an inertial actuator is used to provide control force; as well as additional stability constraints due to this arrangement, the resonance at low frequencies of the actuator itself compromised the control performance.
%(A possible solution to this problem might be to select an actuator with a much \emph{greater} resonance frequency than the plant, but this arrangement has been shown to be ineffective at controlling the system \cite[][Appendix~A]{benassi2002-part1}.)

%A comparison of some of these methods, including skyhook damping and semi-active vibration absorber methods, was done by \textcite{huyanan2007}, contrasting the performance and implementation differences between them.

\textcite{paulitsch2003} used an electromechanical actuator that serves as a self-sensing device for vibration control.
The idea of self-sensing for magnetic levitation purposes (in both cases using an electromagnet) has been shown previously \cite{bleuler1992,vischer1993}.
These self-sensing devices do not perform nearly as well as when using a dedicated sensor (perhaps obviously, since the \backemf/ is a relatively noisy signal), but the technique is particularly interesting for low-cost, low-precision devices.

The addition of inertial actuators produces additional resonances into the structure and the resonance of the actuator limits the broadband performance of the control. As such, this approach is best suited for use in \emph{a posteriori} vibration control measures or in narrowband feedforward control.

\subsubsection{Narrow-band vibration isolation}
\seclabel{vibneut}

One method of reducing vibration on a supported mass is to attach a supplementary mass that resonates in concert with the disturbance; this has the effect of adding an anti-resonance to the original system at the frequency of interest.
These systems are the same schematically as shown for inertial actuators, \figref{vibration-absorb}, with zero force input $\forceAbsorb=0$.
They are known under various names, including `tuned mass dampers', `vibration neutralisers', and `dynamic vibration absorbers'.
%  \note{No effort has been made to compile an exhaustive list.}
The descriptions involving such terms as `damper' and `absorber' are not strictly accurate on the grounds that these devices do not primarily act as energy dissipators; rather, they direct energy into a subsystem for which continuous disturbance is not undesirable.
In this thesis, the term `\vibneut/' is used, following \textcite{kidner1998}, and others.

The concept of the \vibneut/ does not lie within the focus of this research, but falls into the category of literature that is often associated with it under the umbrella of `vibration control'.
In the passive application, a \vibneut/ is created by attaching a supplementary mass to the coupled structure for which vibration is to be removed.
The stiffness of the attachment is chosen by matching the natural frequencies of the structure with that of the additional mass.

To show the effect of varying the resonance frequency of the \vibneut/, \figref{tuned-mass-vs-fig} shows the frequency response of a \vibneut/ with a range of neutraliser stiffnesses with other parameters as shown in \tabref{vibneut-param}.
For each different stiffness, an `anti-resonance' is produced in the transmissibility graph at the resonance frequency of the neutraliser.
The neutraliser frequency should be chosen within the narrowband region of desired attenuation; generally, this will be at the resonance frequency of the main structure.

\begin{figure}
   \psfragfig{\phdpath Simulations/Springs/fig/tuned-mass-vs-freq}
   \lofcaption{
     \Vibneut/ with a range of neutraliser stiffnesses $\stiffnessAbsorb$ (labelled).
  }{
     When the neutraliser is tuned to match the natural frequency of the structure the vibration amplitude around that frequency is greatly reduced.}
   \figlabel{tuned-mass-vs-fig}
\end{figure}

\begin{table}
\caption{Simulation parameters for the \vibneut/ results in \figref{tuned-mass-vs-fig}.}
\tablabel{vibneut-param}
\begin{tabular}{cc}
  \toprule
  $\mass$ & \SI{100}{kg} \\
  $\massAbsorb$ & \SI{10}{kg}  \\
  $\damping$ & \SI{20}{kg/s}  \\
  $\dampingAbsorb$ & \SI{2}{kg/s}   \\
  $\stiffness$ & \SI{400}{N/m} \\
  \bottomrule
\end{tabular}
\end{table}

There is a compromise between broadband and narrowband vibration attenuation for \vibneut/s (not unlike that between shock and vibration isolation, \secref*{shock}).
This is illustrated by comparing the transmissibility reduction at a single target frequency versus the broadband reduction in transmissibility due to the same \vibneut/.
\Figref{inertial-trans-delta} illustrates the vibration attenuation at resonance for a system with a \vibneut/ with various damping ratios.
It can be seen that for attenuation at resonance, low absorber damping produces greater vibration attenuation.
The maximum attenuation is achieved when the \vibneut/ resonance matches the frequency of interest.

Conversely, if the \RMS/ transmissibility of an entire frequency band is calculated, as shown in \figref{rms-inertial}, it can be seen that lower absorber damping \emph{decreases} the overall vibration reduction.
It is also interesting to note that in \figref{rms-inertial} the maximum reduction in broadband transmissibility occurs when the neutraliser is tuned slightly below the resonance frequency of the support.
This would be the appropriate response if the vibration neutraliser were to be used against narrowband vibration with a time-varying resonance peak.

\begin{figure}
\begin{wide}
  \begin{subfigure}[0.44]
    \psfragfig{\phdpath Simulations/Springs/fig/inertial-trans-delta}
    \caption{
      Transmissibility reduction at resonance \vs\ stiffness.
      Greater reductions result from \emph{lower} damping.\figlabel{inertial-trans-delta}}
  \end{subfigure}
  \qquad
  \begin{subfigure}[0.44]
    \psfragfig{\phdpath Simulations/Springs/fig/rms-inertial}
    \caption{
      \expandafter\MakeUppercase\RMS/ transmissibility \vs\ stiffness.
      Greater broadband reductions result from \emph{higher} damping.\figlabel{rms-inertial}}
  \end{subfigure}
\end{wide}
\caption[Single-frequency transmissibility reduction and broadband \RMS/ transmissibility versus \vibneut/ stiffness.]{
  Single-frequency transmissibility reduction and broadband \RMS/ transmissibility versus \vibneut/ stiffness for a range of absorber damping ratios.
}
\end{figure}

The efficacy of a \vibneut/ is related to the damping between it and the structure; better results are achieved with lower damping, as shown in \figref{rms-inertial}.
The damping of the absorber can be reduced with an active control system as shown by \textcite{kidner1998}.
This can be understood with the realisation that energy is not dissipated by the vibration `absorber'; rather, motional energy from the vibrating structure is being \emph{transferred} to the supplementary mass, and this process is degraded by the presence of damping.

A \vibneut/ is tuned for a specific resonance frequency, which means that the resonance frequency of interest must be known and largely unvarying for a passive device to achieve useful results.
This is especially true for low-damping neutralisers, since their efficacy decreases rapidly as they become de-tuned (seen in \figref{inertial-trans-delta}).
To avoid the problem of neutraliser de-tuning due to slow variations in the resonance frequency of the structure, semi-active methods can be used to observe or track the frequency of the disturbance and adjust the stiffness of the neutraliser appropriately in order to retain its tuning.
Such neutralisers typically use a variable stiffness element, which can take many forms \cite{ting-kong1999,kidner2002,holdhusen2007}.
\textcite{brennan2006} discusses a wide variety of actuators that may be used to construct a \vibneut/:
\begin{quote}\itshape
There is not a single ``best'' way of making an \textup[adaptive tuned vibration absorber\textup].
It depends upon the required frequency range, the agility (speed of reaction) and cost.
\end{quote}
An interesting addition to this field was shown by
\textcite{ivers2008} with a mechanically self-tuning \vibneut/.
While not as effective as an adaptive \vibneut/ that uses an external power source, the ability to adapt to the excitation frequency using only the energy of the disturbance itself is commendable.

\Vibneut/s have been used to mitigate seismic vibrations in large buildings, but their mass dependence makes their application rather tricky and often impractical.
\textcite{matta2008} proposed a nonlinear rolling structure via which a \vibneut/ can be mounted to good effect despite uncertain masses of either or both of the building and absorber.
Their work focussed on the interesting idea of using a roof-top garden as a \vibneut/ for a building \cite{matta2008a}.

Some researchers have analysed the use of electromagnetic actuators to provide a fully active force with which to cancel system resonances \cite{chen2005a,wu2007,kim2008-iecst}.
The advantage for such a system is the same as for semi-active controllers in general: with a suitable algorithm, changes in the plant can be taken into account in the vibration neutraliser.
However, using a fully active system for this task is not very energy efficient, since all `damping' is achieved artificially with the expenditure of actuator energy.
Nonetheless, good results can be obtained via this method and the potential flexibility of the control system could be a good reason to design such a system.

\Vibneut/s can also be used for modal systems, in which case each neutraliser is designed at the specific resonance frequency of each mode.
In a recent example, \textcite{casciati2007} used a semi-active neutraliser to control the vibrations of a suspended cable; some care was required for their structure as the higher frequency superharmonic behaviour posed an influence even though the targeted (low frequency) mode was damped as desired.
Optimisation techniques can be used, if the mode shapes are known, to place multiple \vibneut/s in a modal system \cite{petit2009-jva}.

When electrical circuits are used to absorb resonant vibrations, the energy absorbed can be redirected to produce a power output \cite{stephen2006}.
Such devices are gaining popularity for ambient vibration--powered applications such as remote sensing \cite{arnold2007}, with practical implementations beginning to appear \cite{ferrari2009-sms}.
Electromagnetic systems tend to be more suitable for larger scale energy harvesting devices, whereas piezoelectric and electrostatic devices are more suitable at the micro-vibration scale \cite{beeby2009}.
Another field of interest for regenerative damping is in vehicle suspensions, in which useable power can be extracted with the same mechanism used to provide greater ride comfort \cite{graves2000thesis}.
Recent work has used self-powered \magnetorh/ dampers as a \vibneut/ \cite{choi2009-jva}.

\textcite{stephen2006} performed a thorough analysis on energy harvesting with micro-actuators.
He considered a single \dof/ mass-spring-damper coupled with a simple electrical circuit.
For best performance, energy should be dissipated as much as possible by the electric components, not the mechanical damping, since in the electrical network the energy is retrievable whereas with viscous damping the energy is dissipated as heat.
This idea has similarities with the concept discussed previously in this section that the effectiveness of the \vibneut/ is reduced with the presence of increased mechanical damping.

Semi-active methods have also been explored to tune energy harvesting devices to the frequency of disturbance.
\textcite{challa2008} investigated a semi-active device that used variable-displacement attractive and repulsive magnets to adjust the resonance frequency of a piezoelectric cantilever.
This is the same mechanism, investigated independently, that is examined in this thesis for `\qzs/' suspensions (see \secref{qzs-explore} and \secref{qzs}).

Finally, to relate the field of \vibneut/s to this thesis, the work by \textcite{tentor2001} analyses in significant detail the static and dynamic nonlinear stiffness and damping terms of a magnetic system used to create a tuneable \vibneut/.
His design is interesting with respect to this thesis in that it uses both permanent magnets for force generation and an electromagnet for active control; the dynamic response of the system can be changed by varying the current in the coil.



\subsection{Summary of the vibrations literature}

The cross-section of introductory concepts and literature in this section have been chosen to illustrate the broad approaches for vibration control of simple systems.
The basic vibration isolation problem was introduced and basic active control solutions presented; of the various feedback models available, absolute velocity feedback is the most effective for low-order systems as considered in this thesis, although there are implementation difficulties in estimating the absolute velocity signal.
Some aspects of other vibration systems were introduced, including semi-active skyhook control, inertial sensors, vibration neutralisers, and energy harvesting, to provide some context on the wider research area and the connections between the approaches taken.


%%%%%%%%%%%%%%%%%%%%%%%%%%%%%%%%%%%%%%%%%%%%%%%%%%%%%%%
\section{Magnetics}
\seclabel{magnets-explore}
%%%%%%%%%%%%%%%%%%%%%%%%%%%%%%%%%%%%%%%%%%%%%%%%%%%%%%%


This section is a general overview of the applications of magnetic fields:
\begin{description}
\item[\Secref{magnetic-apps}]
Introduces the underlying mechanisms and shows (non-exhaustive) examples in the literature of interesting or novel uses of magnets and magnetic fields.
\item[\Secref{magnetsforces-apps}]
Examines some more specific cases; magnetic positioners and movers are discussed (\ie, using magnetic forces to cause things to move): \maglev/ trains, single and multi \dof/ bearings, and other uses that directly use the forces produced from magnetic fields in generally translational \dofs/.
\item[\Secref{earnshaw}] Covers magnetic levitation (\ie, using forces to support objects intended to be stationary); its impossibility with permanent magnets alone, and exceptions to that restriction.
\end{description}



\subsection{The world of magnetic applications}
\seclabel{magnetic-apps}

Magnetic fields can be used for a vast array of scientific uses.
The variety of applications for magnetic fields stems from the different ways in which they can be generated and the different ways they can interact with their environment.
Magnetic fields produce forces on ferromagnetic material, as well as between paramagnetic and diamagnetic material, as well as charged particles and other magnetic field sources. Magnetic field sources can be time varying or constant, and the fields themselves within their zone of interaction can be uniform or non-uniform.
\textcite{coey2002} discusses a broad range of mechanisms that can be exploited for magnetic applications, summarised in \tabref{magnet-applications}.
The work involved in this thesis covers mainly a single line in this table: the interaction force on a permanent magnet due to a non-uniform magnetic field.

\begin{table}
\caption
  [Applications of permanent magnet materials.]
  {Applications of permanent magnet materials, adapted from \textcite{coey2002}.}
\tablabel{magnet-applications}
\begin{wide}
\begin{tabular}{@{}lll@{}}
\toprule
Field & Magnetic effect & Examples \\
\midrule
Uniform & Zeeman splitting   & Magnetic resonance imaging \\
        & Torque             & Alignment of magnetic powder \\
        & Hall effect        & Sensors, read-heads \\
        & Force on conductor & Dynamic Motors, actuators, loudspeakers \\
        & Induced \emf/      & Generators, microphones \\
Nonuniform & Force on charged particles & Beam control, radiation sources  \\
           &                            & (microwave, ultra-violet; X-ray) \\
           & Force on magnet     & Bearings, couplings, Maglev \\
           & Force on paramagnet & Mineral separation \\
           & Force on diamagnet  & Levitation of small objects \\
Time varying & Varying field & Dynamic Magnetometers \\
             & Force on iron & Dynamic Switchable clamps, holding magnets \\
             & Eddy currents & Metal separation, brakes, dampers \\
\bottomrule
\end{tabular}
\end{wide}
\end{table}

The range of application for the magnetic field mechanisms listed in \tabref{magnet-applications} are too numerous to list in detail.
In the case of using magnetic fields for non-contact sensing of material properties that involve variable conductivity, applications include detecting fatigue cracks, defects in printed circuit boards, and plastic landmines \cite{mukhopadhyay2005}.

Magnetic resonance imaging technology is well-known for its non-invasive ability to diagnose a broad range of health issues. Other applications in the medical field include brain imaging \cite{sekino2005,gjini2005,lu2008-ietm,demachi2008}, measurements of the health of the heart \cite{lim2009-ietm}, stimulation of the nervous system \cite{darabant2009}, and studying the effects on tumour growth and immune function \cite{yamaguchi2005-ietm}.
An interesting biomedical application is remote localisation in six \dofs/ within the human body \cite{yang2009-ietm}.

Magnetic fields have been used to great effect within the robotics world, including:
a haptic interface for manipulating small objects with magnetic levitation \cite{vanwest2007};
a wireless motion capture device \cite{hashi2005};
a computer input device built with magnetic sensors placed on the wrist in order to sense single finger-tip motion from the opposite hand \cite{han2008};
using a magnet attached to a cantilever excited by an external field in a water tank as an actuator to propel a robotic fish \cite{tomie2005};
and many others.

For precise wind tunnel measurements, supporting structures can interfere with the fluid flow in the working cross-section.
In such cases, magnetic levitation has been used to suspend objects in a non-contact fashion \cite{higuchi2008}, avoiding this problem.

In the following sections, the literature closer in scope to the research of this thesis will be discussed more detail.


\subsection{Magnets assisting motion}
\seclabel{magnetsforces-apps}

That magnets can apply forces to one another over a distance is quite a novel concept in a mechanical world accustomed to friction.
It has been a short while, relatively speaking, that it has been possible to even produce magnets with enough coercive force to apply useful mechanical forces.
Non-contact magnetics in mechanical systems is advantageous due to high precision and wear-free operation due to lack of friction.
This section broadly examines some of the main applications of the field.


\subsubsection{Maglev transportation}

The largest body of research into magnetic levitation is on so-called `\maglev/' transportation.
Its well-known goal is to use a levitated carriage to provide extremely fast and efficient transportation.
\note{Indeed, in my experience of explaining my thesis work to others, the first thought most people have when one mentions `magnetic levitation' is of maglev trains.}
This field, which is rather diverse in terms of the techniques under investigation, is finally now achieving commercial application in the real world after some thirty years of research \parencite{lee2006-ietm}.
While \maglev/ has some concepts in common with this research (large loads, magnets), the techniques used tend to be rather distanced from those that will be applied for this project because they focus on transportation rather than \emph{elimination} of movement.

Many approaches to the design of \maglev/ systems have been taken, including passively stable designs \cite{musolino2009,hasirci2011-ieps}.
Earnshaw's theorem for stability (\secref*{earnshaw}) is not applicable for \maglev/ systems, since the motion of the vehicle adds a time-varying element to the magnetic system.


\subsubsection{Magnetic actuators}
\seclabel{magnetic-actuators}

In recent years, magnetic levitation has been applied to the field of linear, planar, and multi-\dof/ actuation systems.
Such devices are generally capable of supporting small loads and applying translational forces to effect displacements of up to several hundred millimetres with up to nanometre precision.
\note{
  Note that while these six \dof/ actuators have very high static precision, they do not have the frequency response of the so-called `nano-positioners' using piezoelectric stack actuators.
}
The initial designs allowed travel in a single direction \parencite[\eg,][]{trumper1992} while planar actuators were shown within the decade \parencite[\eg,][]{kim1997-thesis,molenaar2000}.
Soon after, a six \dof/ non-contact actuator using similar principles was demonstrated \cite{verma2004}.

Each of these devices fulfil different design requirements, and are all subject to continuing research; for example, see the recent development of:
\begin{itemize}
\item a single \dof/ linear magnetic bearing \cite{ro2009-preeng},
\item a planar actuator \cite{kim2007} to support \SI{2}{kg} over a travel of $\SI{5}{mm}\times\SI{5}{mm}$ with nanometre precision,
\item a five \dof/ actuator \cite{fulford2009} with \SI{100}{mm} planar travel; and,
\item a six \dof/ actuator \cite{jansen2008}, which supports loads in the order of \SI{10}{kg} with a large air gap (\SI{2}{mm}) and long stroke ($\SI{230}{mm}\times\SI{230}{mm}$).
\end{itemize}
The electromagnetic design principles of these devices are based around actuation rather than load bearing.
The methods used to design the multi-\dof/ actuation may be applied in future research to the actuation stage of a vibration isolator, but
for load bearing there are other magnetic devices to be investigated.


\subsubsection{Magnetic bearings, couplings, and gears}
\seclabel{bearings}

The oldest dynamic mechanical application of magnetics was for rotary
bearings.
The classic magnetic bearing supports a shaft by applying radially centring forces on the spinning rotor.
An example schematic is shown in \figref{radialbearing}, which is axially unstable due to Earnshaw's theorem (\secref*{earnshaw}).
In the most simple of these bearings, this instability is constrained with a physical stop such as a thrust bearing.
Control has been used for many years now to stabilise such systems (\eg, early work by \textcite{shimizu1968}), creating completely non-contact devices.
Such devices are capable of very high speeds due to the absence of mechanical friction and avoid long-term problems associated with wear.
Magnetic bearings can also be used to support axial forces with active control in the radial direction \cite{asami2005}; it is possible to combine radial and axial bearings to improve the passive stability characteristics of these bearings, shown by \textcite{delamare1994-ietm} and analysed for non-rotational systems later in \secref[vref]{rotation-freedom}.

\begin{figure}
  \grf[width=0.75\textwidth]{Figures/Bearings/radialbearing}
  \caption[Radial bearing cross section.]{The cross-section of two radially
magnetised ring magnets in a radial bearing.}
  \figlabel{radialbearing}
\end{figure}

\textcite{backers1961} developed an early active magnetic bearing similar to that shown in \figref{backers-bearing} which used a control system with variable current electromagnets to stabilise the rotor in the unstable axial direction for completely non-contact support.
The multipole nature of this bearing will be more closely examined in \chapref[vref]{multipole}.

\begin{figure}
  \begin{minipage}{0.45\linewidth}\centering
  \includegraphics{PhD/Figures/Bearings/backers-bearing}
  \caption[Multipole bearing cross section.]{Cross-section of a
    multipole radial bearing.}
  \figlabel{backers-bearing}
  \end{minipage}\hfill
  \begin{minipage}{0.45\linewidth}\centering
  \grf[scale = 0.57]{Figures/Bearings/equalbearings}
  \caption[Two equivalent radial magnetic bearings.]{Two equivalent radial magnetic bearings
(with equal forces of repulsion), despite their different directions of
magnetisation.}
  \figlabel{equalbearings}
  \end{minipage}
\end{figure}

In the early decades of magnetics research, it was not feasible to solve the magneto-static forces of these systems completely analytically.
Magnetic bearings could be modelled using a \twoD/ field solution, simplifying the solutions.
In such a way, \textcite{yonnet1978} showed that the forces between axially- and radially-magnetised bearings are equal (\figref{equalbearings}).
However, this equality of forces for both orthogonal- or parallel-magnetised magnets should not be taken as a general result; it comes about due to modelling the two-dimensional model of the geometry \cite{anderson1987-ietm}, and it will be shown later that this is in fact not true for cube magnets (see \secref[vref]{cube-compare-orth}).
\textcite{yonnet1981} described how such axial and radial magnetic bearings may be re-arranged to suit different applications, showing a complete taxonomy of simple magnetic bearing designs.
Varying geometrical parameters of these magnetic bearings can significantly affect the force and stiffness characteristics \cite{bassani2006-trib-int}.


Magnetic bearings are designed to hold two systems apart; when the magnet design is adapted to hold two systems together the device is known as a magnetic couple or coupling.
Rather than isolating components from applied loads, a magnetic coupling serves to transmit the forces and torques to couple two components together without direct contact.
They can be used, for example, to transmit torque between two separated rotating shafts.
\textcite{yonnet1981} highlighted in an early treatment on the topic that, even more so than for magnetic bearings, periodic recurring magnetisation (see \figref{backers-bearing} and \chapref{multipole}) is required for magnetic couplings to transmit torque satisfactorily.
Since then, numerous theories have been developed and applied to the analysis of a number of multipole designs of various geometries \cite{charpentier1999-ietm-mar,charpentier1999-ietm-sep,charpentier2001-compel,chen2003,ravaud2009-coupling-3d,ravaud2010-ietm-coupling}.

Magnetic couplings can also be used in transmission systems as non-contact `gears'; it is by no means a solved research question on how best to design such magnetic gears
\cite{rens2010-ietia}.
While magnetic gears might not always have the torque capacity of a mechanical gear, this is dependent on the device and the design of the gear system.
In some cases, magnetic gears can usefully replace mechanical ones in planetary gear trains avoiding typical problems such as tooth wear and chatter
\cite{gouda2011-ietm}.


\subsection{Magnets opposing motion}

This section covers areas of the literature in which magnetic forces are used to provide support for load bearing or levitation in which the supported object is intended to remain motionless.
Three broad cases are investigated: passive magnetic levitation, which is known to be impossible with regular magnetic fields; diamagnetic levitation, which is not; and actively controlled magnetic suspension.

\subsubsection{The impossible passive magnetic levitation}
\seclabel{earnshaw}

The act of passively levitating a magnet by another is well known as impossible, although popular unlearned opinion is not aware of the fact.
\textcite{earnshaw1842} proved that objects in the influence of fields that apply forces with an inverse-square relation to displacement cannot form configurations of stable levitation.
Approximately one hundred years later, \textcite{tonks1940} wrote a paper reminding his contemporaries of the work of \citeauthor{earnshaw1842} by applying the proof specifically to the field of magnetics:
\begin{quote}
\dots no flexible assemblage of magnetic poles, in which readjustments in
position of the poles in the group can occur, can be stable in either a fixed
field or in the field from another such assemblage\dots
\end{quote}
An interesting retrospective on Earnshaw's theorem related to magnetic levitation is given by \textcite{bassani2006-meccanica}, and an alternative formulation given by \textcite{reusch1994}.
A mathematical demonstration of Earnshaw's theorem is conceptually quite simple.
We start with the equation for the magnetic flux density; when there are no external current terms, it can be shown to be expressed as Laplace's equation:
\begin{dmath}[compact]
\grad\magB = 0 \implies \divgrad\magB = 0
\end{dmath}.
As the potential energy of a magnet is proportional to the magnetic field it is subjected to, $\potentialEnergy = -\magM\bdot\magB$, when the magnetisation is time-invariant (as in the case of a permanent magnet),
\begin{dmath}[compact,label=earnshaw]
\divgrad \potentialEnergy = \divgradxyz{\potentialEnergy} = 0
\end{dmath}.
The double derivatives of the energy are the stiffnesses in each direction.
For a levitating magnet in a state of stable equilibrium, these three terms must be greater than zero.
This requirement cannot satisfy \eqref{earnshaw} and thus levitation cannot occur.

\begin{figure}
  \grf[width=0.6\textwidth]{Figures/Theory/saddle}
  \caption{A ball in unstable equilibrium on a saddle-shaped curve.}
  \figlabel{saddle}
\end{figure}

This situation is easy to visualise by analogy.
\figref{saddle} shows a ball balancing on a saddle-shaped curve, which is a \twoD/ analogy for the condition of \eqref{earnshaw}.
Perturbations on the ball left or right will result in reaction forces keeping it centred and stable, whereas small disturbances into or out from the page will result in increased perturbation as the ball `falls off' the saddle; \ie, a condition of instability.
So it is with any permanent magnet arrangement.


\subsubsection{Exceptions to Earnshaw}

Earnshaw's proof only relates to systems of fixed magnetisation; it does not rule out all forms of `levitation' unconditionally.
\textcite{boerdijk1956a} reviewed the known methods for levitation, covering levitation by gravitation forces, pressure reaction forces, radiation field forces, and finally in detail, various magnetic and electromagnetic forces.
  \note{`With an eye to the practical importance of levitation we feel justified here in disregarding those aspects of it associated with magic, spiritualism, and psychic phenomena\dots' \parencite{boerdijk1956b}.}
\textcite{bassani2006-meccanica} revisited Earnshaw's work, in particular to highlight interesting exceptions to the theory against passive levitation.

Because Earnshaw's theorem examines only the case for static equilibrium, cases when the magnetic field is dynamic are not covered.
This can occur broadly under three circumstances: when the magnetic field is time-varying; when an unstable permanent magnet arrangement is stabilised with an active control system; and when the system itself is composed of elements with some dynamics associated with them.

To achieve levitation using time-varying magnetic fields, \AC/ currents create dynamic magnetic fields that induce eddy currents in the levitated object, and it is the interaction of the magnetic fields of these induced currents that causes the levitation \cite{laithwaite1965}.
The technique uses a large amount of power, as the levitation forces are entirely generated by the current-carrying coils, and for this reason it is not especially suitable for the purposes of this research.

For an actively stabilised levitation system, the weight-bearing forces are created by permanent magnets (which have time-invariant magnetic fields) and the necessary stabilisation is applied with variable current electromagnets — or some other actuator — with a feedback control system.
This system was first implemented by Holmes who levitated a magnetic needle, as cited by \textcite{boerdijk1956a}.
Some more practical examples of these types of system are covered in the next section.
This summary is fairly brief; \textcite{bleuler1992} wrote a more detailed overview.
His paper introduced the `self-sensing active magnetic bearing' \cite{vischer1993}, which uses \backemf/ from the controlling electromagnet to sense the position of the floating element.
This eliminates the need for a more classical position sensor,
but the control system is necessarily more complex and the behaviour not as precise.

Finally, levitation can be achieved in dynamic systems.
\textcite{bassani2007} levitated a ring magnet above another by using continuous base excitation to find a small zone of stability in the nonlinear dynamics of the system.
More well-known, the Levitron toy demonstrates stability of a magnetic spinning top above a ring magnet \cite{berry1997,berry1996,simon1997,denisov2010-japplmech}.


\subsubsection{Diamagnetic levitation}
\seclabel{diamag}

Levitations involving diamagnetic material are also exempt from Earnshaw's theorem.
This was the motivation for the papers of \textcite{boerdijk1956b,boerdijk1956a} in which he cites Braunbek, who derived that magnetic material is governed by Earnshaw's theorem only because it has a relative magnetic permeability ($\permMag$) greater than one — \ie, a permeability greater than that of the surrounding medium.
A separate analysis of magnetic levitation systems provides a more specific measure for testing the stability of magnetic systems with various boundary conditions \cite{reusch1994}.

Material with $\permMag<1$ is \emph{not} covered by the theorem since the magnetic flux from the diamagnetic material becomes dependent on the displacement of the permanent magnet; this violates the condition of Earnshaw's theorem that fixed magnetic fields be used, and so static levitation involving magnets and such diamagnetic material becomes possible.
To demonstrate this, \textcite{boerdijk1956b} levitated a small cylindrical magnet of dimensions \diameter$\,\SI{1}{mm} \times \SI{0.3}{mm}$.
\textcite{simon2000} provide a good background to the area and use modern approaches to levitate a permanent magnet with a variety of magnet/diamagnet geometries, primarily with a vertical diamagnet--magnet arrangement.
\textcite{kustler2012-ietm} examined a horizontal diamagnet configuration (not dissimilar to the `horizontal spring' introduced in \secref{hspring}) which could stably levitate multiple magnets simultaneously.
Other studies on diamagnetic levitation examine the suspension of larger objects including strawberries and frogs \cite{berry1997,geim1998,geim1999,simon2001} and (widely reported in the media) mice \cite{liu2009-spaceresearch} with superconducting electromagnets (on the order of \SI{10}{T}).

Unfortunately, none of these diamagnetism-based approaches are suitable for large load bearing.
Even the element with the strongest diamagnetism in its natural state, bismuth, has a relative permeability $\permMag \approx \num{0.99983}$ — hardly different than that of `free space'.
\note{
  By contrast, water has $\permMag\approx\num{0.999991}$, and since living organisms are mostly water, this is the value typical of frogs and mice and humans as well.
  The strongest diamagnetic material is manufactured pyrolytic graphite, with a permeability of $\permMag\approx\num{0.99955}$.
  Permeability numbers all as cited by \textcite{simon2001}.
}
The forces exchanged via magnetic flux between magnetic and diamagnetic materials, therefore, are incredibly small and not suited at all to the purposes of this research.

Superconducting material, on the other hand, behaves ideally diamagnetic with $\permMag=0$, so the forces produced between a superconductor and a magnet are equal to the forces between two permanent magnets themselves (of equal size to the original magnet and separated by twice the distance between the magnet and the superconductor).
This allows many exciting possibilities for stable levitation.
However, even the so-called `high temperature' superconducting materials must be cooled to very low temperatures in order to achieve superconductivity.
Such a requirement renders this method functionally impractical for this research.
A review of work in the area of superconducting levitation has been published by \textcite{ma2003}.


\subsubsection{Single \dof/ unstable magnetic suspension}
\seclabel{levitation-control}

A simple variety of magnetic levitation or suspension is the counter-acting of gravity with an active electromagnetic force.
The most common form this takes is via an unstable attractive vertical force to directly compensate for gravity; vertically-passive designs are shown in \secref[vref]{hspring} and \secref[vref]{choi-spring} which both require active control in the horizontal directions to maintain stability.

The single degree of freedom magnetic system is very popular as an application for control theory, as it is an unstable system in which both the passive and active magnetic forces are nonlinear with displacement.
As will be seen later, developing models for magnet and coil forces can be quite involved; for the purposes of control, the closed form expressions introduced in \chapref{magnet-theory} are too complex to integrate into a control methodology.
One way to overcome this issue is to develop a low-order empirical model for the system, with unknown parameters that must be identified; an example of this was shown for a coil-iron suspension by \textcite{agamennoni2004}.

%In this thesis, curve-fitting of magnetic force versus displacement characteristics has been used \emph{a priori} to establish such unknown parameters (\secref[vref]{qzs-mag}).
%This method is only possible when sufficient information is known beforehand, and an online method is useful for plant parameters that change over time (such as load force, which can be inherently time-varying).

By contrast, sufficiently advanced nonlinear control can achieve stability and tracking without the use of system identification.
As an example, \textcite{mahmoud2003} used backstepping with a nonlinear model to provide robust control of a magnetic suspension and \textcite{queiroz2007} similarly used nonlinear control to stabilise a magnetic bearing with pull--pull electromagnets with parameter uncertainties, while also minimising power consumption of the system.
\textcite{gentili2003} investigated this system with nonlinear feedback control for robust disturbance suppression, and
\textcite{chang2001} applied nonlinear control to the problem of magnetic levitation, using coupled hybrid magnets (\ie, electromagnets biased with permanent magnet cores) that create a magnetic circuit with the levitated table of~\SI{20}{kg}.

However, such advanced control techniques are not always necessary; \textcite{li2007} report their success in using simple \PID\ control for suspending a magnetic table using a coupled electromechanical model of the system.
\textcite{banerjee2008} used a simple cascaded \PI\ and lead controllers to stabilise an electromagnetic suspension.
An optimisation technique was used to obtain the control gains necessary to achieve adequate performance over a range of displacement gaps; such control is usually only suitable for fixed-gap systems.

Vibration isolation achieved using magnetic suspensions is addressed later in \secref{magnet-platform-isolation}.




\subsection{Magnetic damping}
\seclabel{damping}


Henry Sodano completed his \PhD/{} in 2005 \cite{sodano2005thesis} on
eddy current damping for flexible structures and has published several
papers based on that work.\footnote{I recommend the thesis for the
  additional context and literature review.} His work investigates the
use of non-contact magnetic (permanent \cite{sodano2005,sodano2006,sodano2008-dsmc} or
electric \cite{sodano2007}) elements to passively \cite{sodano2005} or
actively \cite{sodano2006,sodano2007,sodano2008-dsmc} add forces to a structure via
induced eddy currents. His work investigates the potential for use
with flexible structures primarily for use in space
applications.
\note{Note that not all permanent magnets are created
  equal for suitability in space: the cheapest and most common class
  of rare earth magnetic alloy, neodymium-iron-boron, will become
  demagnetised in the influence of radiation due to localised heating
  effects. Samarium-cobalt magnets are less susceptible to this
  problem due to their higher Curie temperature and have been
  previously used in space applications \cite{chen2005}.}

The damping effects of magnetic forces has also been examined by
\textcite{bonisoli2006}. The use of electromagnetic damping can be
advantageous in applications where the absorbed energy is converted to
electrical energy for re-use and storage, increasing the overall
efficiency of the device; a good example is in the automotive industry
\cite{graves2000thesis}.

Another recent investigation of an eddy current damper using an aluminium
plate is shown by \textcite{ebrahimi2008}. Damping of levitated permanent
magnets with a similar technique was shown by \textcite{elbuken2006}. Their
emphasis lay on the problem of micro-levitation, where small stiffnesses (and
damping) results in large amplitudes of disturbance. One of the noted
advantages in this case is the fact that the eddy current damping does not add
other dynamics to the system it is applied to through structural coupling; equilibrium positions
and controller designs are unaffected with this technique.

Eddy currents may generate force on a conductor through two mechanisms: change in magnetic field and/or change in velocity.
Change in velocity leads to a (possibly noticeably nonlinear) viscous damping force that
is dissipative: it can only decrease the energy of the system.
However, a change in the magnetic field can generate forces that can be used to apply work to the system.
This is the same mechanism used by \AC/ current levitating devices \cite{laithwaite1965}.
It is unclear whether the use of conductive material has significant advantages over  ferrous material to generate vibration suppression forces.
For the conductive case, a permanent magnet may be used to increase the field strength of the electromagnet at the expense of added viscous damping (which may or may not be desirable according to the application).
However, constant and low frequency forces will not be able to be generated without significant control design since eddy current forces are more difficult to model than quasi-static magnetic forces.

The calculation of eddy currents is an involved process, and has
not been investigated in detail for this thesis. The eddy current
density $\magJeddy$ induced in a conductive sheet moving through a
magnetic field at velocity $\velocity$ is given by
\begin{dmath}
\magJeddy = \conductivity\gp{\velocity \cross \magflux}.
\end{dmath}
where $\conductivity$ is the conductivity of the sheet. The force
$\forceEddy$ due to these eddy currents is the integral over the conductor volume $\volume$,
\begin{dmath}[compact]
\forceEddy = \Int{\magJeddy \cross \magflux}{\diffvolume,\volume}
           = \conductivity\Int{\gp{\velocity \cross \magflux}\cross \magflux}{\diffvolume,\volume}
\end{dmath},
which is anti-parallel to $\velocity$. Due to the
cross terms, the maximum force is obtained for magnetic fields
perpendicular to the motion of the conductor. Associatively, it is
only the component of magnetic field in the perpendicular direction
that influences the eddy force. This has implications on the
arrangement of eddy current dampers for vibrating structures.
Two configurations of eddy current dampers are shown in \figref{eddy}, which must be designed taking into account the exact shape of the magnetic field and the gap between the magnet and the conductor.
The vertical configuration (\figref{eddy-v}) will exhibit a nonlinear damping force as the gap between the magnet and conductor varies with displacement of the mass.
In practice this may not be a significant problem if the gap is somewhat larger than the motion of the mass.

\begin{figure}
  \begin{subfigure}
    \asyinclude{\jobname/eddy-h}
    \caption{Horizontal configuration; eddy
      currents induced via axial magnetic
      flux.\figlabel{eddy-h}}
  \end{subfigure}
  \hfil
  \begin{subfigure}
    \asyinclude{\jobname/eddy-v}
    \caption{Vertical configuration: eddy
       currents induced via radial magnetic
       flux.\figlabel{eddy-v}}
  \end{subfigure}
  \caption{Orthogonal configurations of eddy current dampers for a vibrating
    (non-magnetic) mass. The shaded section indicates conductive material.}
  \figlabel{eddy}
\end{figure}

Experimental results of a magnetically levitated mass show that eddy current damping between permanent magnets is very low (\secref[vref]{xpmt-ol}), making a passive non-contact eddy current damper of potential importance for vibration suppression in systems that require additional passive damping.
Optimisation in this area could investigate the size and shape of the magnet or electromagnet used to best create the field that impinges on the conductor to generate maximal eddy current forces.
In addition, the formalisation and potential analytical solutions for calculating eddy current forces for a wide range of magnet geometries has not yet been investigated.
This avenue of research is not pursued in this thesis.


\subsection{Summary of magnetics}

From brain imaging to \maglev/ trains, magnetic fields can be used for a very wide range of applications. Of particular interest are those areas in which magnetic fields are used to generate translational forces.

The inherent instability of magnetic suspensions has been introduced and methods shown for overcoming the problems associated with this.
Basic magnetic suspension, as the basis for many `levitation platforms' was addressed.


%%%%%%%%%%%%%%%%%%%%%%%%%%%%%%%%%%%%%%%%%%%%%%%%%%%%%%%
\section{Magnets and vibrations}
\seclabel{magnetsisolation-apps}
%%%%%%%%%%%%%%%%%%%%%%%%%%%%%%%%%%%%%%%%%%%%%%%%%%%%%%%

This section explores the overlap between vibration isolation and magnetic actuation systems.
For the purposes of this thesis, three broad topics are of interest:
\begin{description}
\item[\secref{magnet-platform-isolation}]
Vibration isolation systems designed using magnetic springs and actuators;
\item[\secref{magnet-nonlinear}]
The nonlinear dynamics of magnets in motion; and,
\item[\secref{qzs-explore}]
The field of `\qzs/', to which a contribution is made in \chapref{qzs}.
\end{description}

\subsection{Vibration isolation platforms}
\seclabel{magnet-platform-isolation}

\textcite{puppin2002} demonstrated with a simple system that magnetic springs can be used for vibration isolation. No attempt to achieve contactless suspension was made—the magnets were horizontally constrained in guides.
Furthermore, the springs were only used as passive isolators for vibrations in the vertical direction; no active vibration control was used.

\textcite{nagaya1993} constructed a non-contact vibration isolation table; they report a high-stiffness spring with transmissibility that \enquote{can be controlled to be nearly zero \sic/}.
%\note{Transmissibility is a ratio expressed, usually, in decibels across a frequency spectrum — there is no meaning to the qualitative term `nearly zero' without quantifying it somehow.}
Their table used small magnets in a simple design which could not support large loads.
The authors showed later a better control system for their \enquote{perfect \sic/
%\note{What does `perfect' mean?}
non-contact active vibration isolation table} \cite{nagaya1995a}.

\textcite{watanabe1996} wrote a paper detailing a functional vibration isolator using electromagnetic springs, which could support weights of up to
\SI{200}{kg}.
The control system used a combination of two independent control systems for stable levitation and vibration isolation.
The magnetic actuator design is not described, however.



\subsection{Nonlinear vibration and/or magnetic systems}
\seclabel{magnet-nonlinear}

The field of nonlinear dynamics is very large, and surprising results can arise in applications to vibration suppression.

\subsubsection{Examples of nonlinear vibration systems}

\textcite{oueini1999} considered the response of a nonlinear plant with an additional cubic nonlinear feedback law and established that vibration attenuation was possible and that nonlinear phenomena such as chaos existed, which would generally be undesirable for vibration isolation.

A comprehensive review of nonlinear vibration isolation systems for a broad range of techniques was published by \textcite{ibrahim2008}.
It highlights the importance of nonlinear analysis in this field: in some cases, better results can be achieved using nonlinear spring forces to couple to and absorb vibration energy; in other cases, the behaviour of an isolator cannot be adequately modelled by using linear systems theory.
As an example of the former, \textcite{starosvetsky2008}, with a good review of the literature, introduce the concept of an `energy sink' in which a nonlinear system provides more effective vibration attenuation over a broader frequency range than a linear absorber alone.

Interesting results have been shown using nonlinear springs to attach the vibration absorber to the structure, such as \textcite{jo2008} who use repulsive magnetic springs to produce a tuned absorber with a resonance at double the frequency of the main resonance of the structure.
`Frequency doubling' is a common non-linear effect; it is also seen, for example, in eddy current-based magnetic actuators \cite{sodano2008-dsmc}.

\textcite{mann2008} performed a preliminary investigation into the use of a nonlinear vibration mount for energy harvesting, using a magnetic suspension of repulsive magnets to create a Duffing-like oscillator.
Large damping ensured that the nonlinear regimes were only realised at large excitation amplitudes, but the idea is that highly nonlinear resonances have a much broader resonance peak through the higher branch around the jump phenomenon.

A similar idea is explored by \textcite{shahruz2008} for an energy scavenging cantilever beam that uses an arrangement of attracting magnets to shape the force characteristic of the response.
The aim is to achieve a power spectrum of the response to a random excitation that is greater than the predominantly linear response that is obtained without the magnets present.

\textcite{zhang2008} use a nonlinear damper to excite the structure at harmonic frequencies of the resonance.
This results in less energy at the frequency of vibration, although the resonance peak does remain.
While it does not seem likely that this method can compete with the reductions seen with the approach of a \vibneut/, this nonlinear damping method does have the advantage that it does not require tuning for a particular frequency and its effectiveness will not change with a time-varying resonance frequency.

\textcite{jazar2006} analysed the behaviour of a nonlinear vibration isolation mount in detail, developing analytical models for the jump phenomena of a system with cubic stiffness and quadratic damping.
Critical values were illustrated to avoid the ill effects of the nonlinearities; additional damping had the general effect of decreasing the adverse nonlinear response.

\subsubsection{Nonlinear magnetic systems}

For the purpose of vibration control, augmenting a linear spring system with nonlinear magnetic springs alters the behaviour of the natural frequency of the system to be weakly coupled to the mass of the system.
\textcite{dangola2006} analysed the dynamics of a nonlinear system in which a variation of both stiffness and mass by up to 50\% yields an increase in stiffness of 6\%.
For an equivalent linear system, the natural frequency variation is ten times greater.
This is a very interesting result for loading elements for which the mass to be supported is largely variable, in that the frequency response will vary significantly less than for conventional linear springs.
A system with similar characteristics is proposed in this thesis (\secref[vref]{oblique}).

However, for weakly nonlinear magnetic springs, variation in the mass will still lead to changes in the resonance frequency.
\textcite{todaka2001-ietm} created a mechanical linkage to support two magnets in repulsion such that as their air gap increased (due to less mass being supported), a horizontal offset between them was created to lower the linearised operating stiffness.
This allowed a much smaller variation in resonance frequency than for flush magnets in repulsion.
\textcite{bonisoli2007-mssp} used an experimental apparatus to analyse the nonlinear behaviour of a magnetic and linear spring in parallel, together with further theoretical analysis \cite{bonisoli2007-mrc}.
They showed a configuration of linear and magnetic springs with the notable feature that the resonance frequency exhibits little dependence on mass loading and nonlinear effects can be seen.
A contribution in this area is made in this thesis in \secref[vref]{oblique}, in which a purely magnetic device is designed to achieve constant resonance frequency with variations in mass.




\subsection{\QZS/ systems}
\seclabel{qzs-explore}

In a conventional mass--spring system, the static deflection increases as the stiffness of the support is reduced, and a lower limit on the stiffness is imposed by constraints on the allowable displacement.
Novel approaches are required to reduce the resonance frequency below that possible with a linear spring.
The addition of negative stiffness elements in a design reduces the resonance frequency, which improves vibration isolation.
Early examples of such designs using inclined springs were shown by \textcite{molyneux1957}.
These have an approximately cubic force \vs\ displacement characteristic, which may be tuned to achieve a local region of zero stiffness, which is often termed `\qzs/'.
\textcite{alabuzhev1989} examined the nonlinear characteristics of such systems, as have several others
\cite{carrella2007-jsv,kovacic2008,carrella2009-jsv}.
The dynamic response of these systems has been shown to exhibit prominent nonlinearities that distort the frequency response but that do not decrease the vibration isolation efficacy in general.
Analysis of the nonlinear dynamics of such systems \cite{lee2004-jsv,kovacic2008,kovacic2009} can be quite involved and is outside the scope of this research.

A variety of mechanical linkages and arrangements can be designed for \qzs/ \cite{tarnai2003}.
Mechanical \qzs/ elements, generally using flexible beam supports in a buckling regime, have been used in application for vibration isolation platforms \cite{platus1999}, mounts for seismic noise attenuation \cite{cella2005}, vibration attenuation from hand-held machinery \cite{sokolov2007}, reduction of aircraft cabin noise \cite{baklanov2007-jsv}, and vehicle driver suspension \cite{lee2007-jsv}.
Friction is a limiting factor in these devices \cite{sokolov2007}; this particular problem is obviated when non-contact supports are used.
As an alternative to the classical helical spring for vibration isolation support, the `pinched loop' created by clamping both ends of a slender beam at the same location (the shape resembles a droplet) has a wide range of tuning possibilities and offers isolation in two degrees of freedom \parencite{virgin2008}.
Further detail into the field  of nonlinear passive vibration isolators is given in the recent review by \textcite{ibrahim2008}.

\QZS/ can also be achieved with magnetic systems.
Magnetic configurations with negative stiffness can be used to augment a positive stiffness support (which can be simply a conventional spring) to lower the resonance frequency.
For example, \textcite{beccaria1997} used this technique (under the term `magnetic antisprings') to improve the isolation for gravity wave detectors.
Others have used attractive magnets in parallel with conventional springs to reduce the resonance frequency of the system \parencite{carrella2008-jsv}.
Similarly, the negative stiffness of an electromagnetic actuator has been compensated for by embedding the suspended mass within a membrane to achieve a low overall stiffness \cite{sato2001}.
\textcite{zhou2010-jsv} recently demonstrated an active/passive tuneable \qzs/ system that used a moving permanent magnet in attraction to soft-iron-core electromagnets as the negative stiffness element in series with a clamped--clamped beam for positive stiffness.
A range of stiffness characteristics were demonstrated by varying different system parameters, including the creation of an approximately constant and minimal stiffness over a required displacement range.

Purely non-contact magnetic systems can also be used to similar effect using a repulsive magnet pair in series with an attractive magnet pair both oriented vertically \cite{robertson2006-activeconf,robertson2007-icsv}, analysed in detail in \secref[vref]{qzs}.
More recently, multipole systems (see \chapref{multipole}) have been investigated which use the same general magnet configuration \cite{janssen2009-jsdd}.
An alternate design is shown by \textcite{hol2006}, which uses an axial bearing with \ang{90} rotated magnetisations to bear load in the vertical direction; the force--displacement characteristic is a mirror image of the attraction--repulsion pair.
Systems that use such negative stiffness between attracting magnets cannot be brought to a stable \qzs/ region due to their `softening spring' characteristic.
In these cases the negative stiffness is used to reduce the resonance frequency as much as possible before instability occurs.

\subsubsection{\QZS/ is not zero stiffness}
\seclabel{qzs-not-zerk}

It has been established that the goal of a `zero stiffness' device is to reduce the resonance frequency of the system to as low a value as possible.
In the limiting case, if the system is stable and the nominal force of the spring indeed matches the weight of the mass, then the gradient of the force at the operating point will equal zero.

However, it is necessary to use a nonlinear spring to achieve this zero stiffness condition, and the dynamic behaviour of a nonlinear oscillator varies considerably from that of the classic linear spring.
Most obviously, the shape of the frequency response is not independent of the amplitude of the forcing disturbance.
Consider the stable single \dof/ system
\begin{dmath}[label=duffing]
\mass \ddot {\disp} + \damping \dot {\disp} + \stiffness_3 \gp{\disp+\disturb}^3 = 0,
\end{dmath}
where $\disp$ is displacement and $\disturb$ is an induced displacement disturbance.
At the operating position $\disp=0$, the nonlinear spring stiffness is $3\stiffness_3\disp^2|_{\disp=0}=0$.
For a disturbance $\disturb$, the spring is perturbed and generates a reaction force of $\stiffness_3 \disturb^3$ on the mass.
The stiffness here is $3\stiffness_3\disturb^2$; \ie, dependent on the amplitude of disturbance.
The ramifications of this nonlinear force on the vibratory response of the system are not exactly straightforward.

\textcite{tentor2001} analysed a spring generated by repulsion magnets which behaved as a Duffing oscillator for large amplitude vibrations:
\begin{dmath}
\forceMag_{\mathrm{Duffing}} = \stiffness \disp + \stiffness_3 \disp^3.
\end{dmath}
The nonlinear dynamics only affected the response of the system when the nonlinear term dominated over the linear term.
For a \qzs/ spring $\stiffness=0$ and the nonlinear dynamics are more significant.

\textcite{carrella2009-jsv}, for example, use the `harmonic balance' method to analyse the nonlinear frequency response of a Duffing oscillator model of an inclined spring based \qzs/ system, showing the typical jump-up and jump-down phenomena of such nonlinear systems.
For the purposes of broadband vibration isolation, however, it is instructive to simply examine the power spectra produced with a range of spring stiffnesses and modelling the disturbance input as Gaussian.

\begin{figure}
  \begin{wide}
    \hspace*{-1.3cm}\psfragfig{PhD/Simulations/Zero_stiffness/fig/cubic-resonance-disturb-a}
    \hspace*{+0.5cm}\psfragfig{PhD/Simulations/Zero_stiffness/fig/cubic-resonance-disturb-c}
  \end{wide}
  \caption[Frequency response simulations of a nonlinear dynamic system with cubic stiffness with random noise input.]{Frequency response simulations of nonlinear dynamic system \eqref{duffing} with cubic stiffness with random noise inputs of varying variance $\varianceNoise$.}
  \figlabel{cubic-resonance-disturb}
\end{figure}

\begin{figure}
  \begin{wide}
    \hspace*{-1.3cm}\psfragfig{PhD/Simulations/Zero_stiffness/fig/cubic-resonance-disturb-b}
    \hspace*{+0.5cm}\psfragfig{PhD/Simulations/Zero_stiffness/fig/cubic-resonance-disturb-d}
  \end{wide}
  \lofcaption{Equivalent simulations to \figref{cubic-resonance-disturb} using linearised \eqref{duffing-lin}.}{ Note the two methods of showing the transmissibility are equivalent for the linear system, although significant differences are seen as the frequency approaches the maximum measurable frequency according to the sampling parameters.}
  \figlabel{cubic-resonance-disturb-linear}
\end{figure}

The square root of the ratio of the power spectra, sometimes known as the variance gain, is used to examine the non-linear system in the frequency domain, as the transmissibility removes nonlinear components of the original signals \cite{savaresi2007}.
A comparison between the linear transmissibility and the variance gain of the system for $\disturb$ and $\disp$ is shown in \figref{cubic-resonance-disturb}, where $\disturb$ is a white noise signal of variance $\varianceNoise$.
Parameters used in the simulation are shown in \tabref{duffing-param}.
It can be seen that since the linear transmissibility attempts to reject non-correlated linear signals, as the input amplitude increases its response becomes noisier and much of the shape of the power spectrum is lost due to this.
Note that the resonance peak changes frequency with input amplitude.

\begin{table}
\caption{Parameters used to simulate the dynamics of \eqref{duffing} shown in \figref{cubic-resonance-disturb}.}
\tablabel{duffing-param}
\begin{tabular}{lll}
\toprule
Simulation time   & & \SI{5000}{s} \\
Sample time       & & \SI{0.005}{s} \\
\textsc{fft} size & & $2^{14}$ \\
Mass              & $\mass$ & \SI{1}{kg} \\
Nonlinear stiffness coefficient & $\stiffness_3$ & \SI{1}{kg/m^3} \\
Damping           & $\damping$ & \SI{0.1}{kg/s} \\
\bottomrule
\end{tabular}
\end{table}

The results of the nonlinear simulation can be compared with a linearised equivalent of the same system given by
\begin{dmath}[label=duffing-lin]
\mass \ddot {\disp} + \damping \dot {\disp} + \klin\gp{\disp+\disturb} = 0,
\end{dmath}
which has a linear stiffness $\klin=3\stiffness \varianceNoise^2$ equivalent to the stiffness of the nonlinear spring at the variance displacement.
Frequency response simulations of this linearised system are shown in \figref{cubic-resonance-disturb-linear}.

When comparing against the nonlinear response, the linearised system response is smaller than the \qzs/ response in some cases, but especially as the input amplitude increases and the nonlinearities become stronger.
This potentially restricts the use of nonlinear springs for vibration isolation application to achieve low resonance frequencies: only when it becomes infeasible to decrease the stiffness of a conventional linear system any further should a nonlinear system be chosen instead.
Conversely, the broader `resonance peaks' of the type seen in \figref{cubic-resonance-disturb} have been shown to be useful for energy harvesting purposes, especially for wandering narrowband excitation \cite{ramlan2009-nd}.

\subsubsection{Quasi--infinite stiffness systems}
\seclabel{infstiff}

For completeness, this section covers the opposite of the \qzs/ system: with a different combination of spring elements it is possible to create quasi--\emph{infinite} stiffness, shown in a variety of systems over the last decade starting with single-\dof/ systems
\cite{nijsse2001,mizuno2001,mizuno2002,mizuno2003a,mizuno2003b,mizuno2003c,mizuno2010-jvc} and later being extended to multi-\dof/ \cite{hoque2006,mizuno2007}.
The quasi--infinite stiffness effect is produced with a series combination of a positive stiffness and a negative stiffness spring such that the resultant stiffness is given by
\begin{dmath*}[compact]
  \stiffness_T = \frac{ \stiffness_1 \stiffness_2 }{ \stiffness_1 + \stiffness_2 } = \infty
  \condition{when $ \stiffness_2 = -\stiffness_1 $}
\end{dmath*}.
\textcite{xing2005} formalised the idea of \qzs/ and quasi--infinite stiffness systems.

Like `zero stiffness', some qualification is required on the term `infinite stiffness' in this context.
To achieve `true' infinite stiffness, a control system is required to stabilise the negative stiffness element of the plant.
This control system will impart its own dynamics on the system such that the frequency response of the system will only approach the behaviour desired.
In the case of the aformentioned studies, the static behaviour of the systems does converge to an infinite stiffness property, but \emph{dynamically} there are still dynamics associated with the connection.
These are shown explicitly in the frequency response functions of such systems \parencite{mizuno2010-jvc}, which exhibit significant resonance peaks for both direct and ground-borne disturbance.

\subsection{Summary of vibrations and magnetics literature}

The interesting static and dynamic force characteristics exhibited by magnetic systems has suggested their use in a variety of vibration control applications.
Of particular research interest is the nonlinear regimes that such systems can operate within; in some cases, these nonlinearities can be detrimental to vibration isolation, while in others they can help improve vibration suppression.
For this thesis, the particular use of magnetic systems as softening springs allows them to overcome some limitations seen with linear mechanical springs, the desirable focus in these cases being the `\qzs/' regime to minimise the resonance frequency.

\section{Structure of this thesis}
\seclabel{thesis-structure}

This chapter has introduced the themes of this thesis; at this point it is prudent to suggest where the following chapters will lead.

The first half of this thesis is focused specifically on the analysis and design of purely magnetic systems, beginning in \chapref{magnet-theory} with the theory on calculating forces and torques between magnets of various shape and geometry.
\Chapref{magnet-design} then discusses and analyses load bearing systems composed of various arrangements of singular magnets, culminating in the analysis of a novel magnetic support that uses inclined magnets to achieve a load-invariant resonance frequency.
These studies conclude in \chapref{multipole} with a discussion and analysis of the use of multipole or `Halbach' arrays, which are composed of many variously-oriented magnets, for improving the load bearing capabilities for magnetic supports.

The thread of the second half of the thesis is the realisation of an experimental apparatus to demonstrate the concepts for a magnetic spring design.
Since all forms of unguided magnetic levitation require some degree of active control, \chapref{coil-design} develops theory for calculating quasi-static forces of electromagnets, and performs an optimisation of an electromagnetic actuator suitable for such a task.
In discussing \qzs/ systems for vibration isolation, \chapref{qzs} begins with a planar analysis of a  typical spring-based design from the literature.
It is subsequently shown that a magnetic system can be more flexible for adaptive tuning.
In the analysis of this \qzs/ magnetic system, emphasis is made on the small region in which stable operation is achieved.
Finally, a prototype based on this magnetic system is presented in \chapref{xpmt} with results demonstrating low frequency vibration isolation and the ability of active vibration control to improve the vibration transmissibility.
\chapref{conc} summarises the main findings from the preceding chapters, and suggests a list of possible future directions for subsequent research.

Much of the work in this thesis is publicly available for use in the research community under the principles of reproducible research \parencite{kovacevic2007-icassp}.
This can be found in the code repository $\langle$\url{http://www.github.com/wspr/magcode}$\rangle$, which is a compilation of code for calculating the forces between magnets and magnetic systems.
A discussion of the philosophy behind this effort and a summary of the reproducible work in this thesis are discussed in Appendix~\ref{repro-research}.

\section{Publications arising from this thesis}

The following first-authored articles have been published over the course of this \PhD/{} research.
In one form or another, most of these works have been incorporated into this thesis and are highlighted herein where appropriate.

\def\citejournal#1{
\item[\cite{#1}]
\emph{\citetitle{#1}};
\citefield{#1}{journaltitle}, \citeyear{#1}.
\medskip
}

\def\citeconf#1{
\item[\cite{#1}]
\emph{\citetitle{#1}};
\citefield{#1}{booktitle}, \citeyear{#1}.
\medskip
}

\def\citeconfx#1{
\item[\cite{#1}]
\emph{\citetitle{#1}};
\citefield{#1}{booktitle}.
\medskip
}

\begingroup
\def\enquote#1{#1}
\raggedright
\begin{itemize}
\citejournal{robertson2005-ietm}
\citeconf{robertson2006-activeconf}
\citeconf{robertson2007-icsv}
\citejournal{robertson2009-jsv}
\citejournal{robertson2010-maglett}
\citejournal{robertson2010-maglett-fix}
\citejournal{robertson2011-ietm}
\citejournal{robertson2012-jsv}
\citejournal{robertson2012-ietm}
\citeconfx{robertson2013-aas-zks} \emph{Under review}.
\citeconfx{robertson2013-aas-qzs} \emph{Under review}.
\end{itemize}
\endgroup

\noindent The following conference papers were co-authored:

\begingroup
\def\enquote#1{#1}
\raggedright
\begin{itemize}
\citeconfx{frizenschaf2011-acoustics2011}
\citeconf{zhu2011-icmt}
\end{itemize}
\endgroup

\end{document}
