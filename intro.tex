%!TEX root = thesis.tex

\chapter{Introduction}

\epigraph{In my experience I found that the most effective way to express
something in order to make others understand is to use the simplest language.
Also I learned from teaching that the more rigid the language the less
effective it is.}{\textcite{mahathera1990}}

\section{One paragraph summary of the thesis}

Vibration disturbance is a consistent complaint that will always exist to
negatively influence equipment and processes. The theme of this thesis
investigates using permanent magnets in the design of a vibration isolation
mount. Permanent magnets can be used to either support load or to reduce the
fundamental resonance frequency of a system. In the first half of the thesis,
simple permanent magnet arrangements are examined for load bearing, with
multiple designs suitable for a variety of objectives. In this vein,
electromagnetic coil forces and multipole (or Halbach) arrays of magnets are
also investigated. In the second half of the thesis, more attention is paid on
realising these systems for vibration isolation, namely, by reducing the
resonant frequency with stiffness-reducing elements. Usually these
stiffness-reducing elements are permanent magnets in attraction, but
mechanical springs are analysed briefly as well. Some different active control
techniques are investigated for further improvements to the low-frequency
isolating design. An experimental prototype tying together the major concepts
of work is shown last, demonstrating results of an permanent magnet, active,
low-frequency, vibration-isolating apparatus.

\section{Project exposition}

The purpose of this project is to design and build a vibration
isolation table using non-contact magnetic springs. This goal can be
split into two: the design of a non-contact magnetic spring (suitable
for a vibration isolation table); and the design of the vibration
isolation table itself.

Vibration isolation tables are generally designed to attenuate natural
disturbances from the ground to the tabletop. These are used for
example in optics systems as well as in situations that involve
vibration-sensitive equipment. Current commercial models use pneumatic
springs to perform this task, and this project arose out of curiosity:
could magnetic springs be used instead?

Since the invention of the rare-earth magnet, there have been great
strides in creating wonderful inventions such as the maglev train and
the planar levitator. To date, comparatively little work has been
undertaken on vibration isolation of large loads using predominantly
passive elements. While levitating trains are heavy, they require
little positioning accuracy; meanwhile, planar levitators are
astonishingly accurate, but may only bear small loads. A vibration
isolation table must do both.

Using magnets for load bearing brings its own set of challenges. For
completely non-contact support, active control must be used to
stabilise \emph{at least} one degree of freedom. For the design to be
worth investigating, some advantage to using magnets should also be
demonstrated.\footnote{Although I took much pleasure in explaining
  over the years that my \PhD\ project was to `build a table that
  floats on magnets'.}


\section{Topical applications of magnetic forces}

\textcite{coey2002} discusses a broad range of magnetic applications,
summarised in \tabref{magnet-applications}.

\begin{table}
\begin{tabular}{@{}lll@{}}
\toprule
Field & Magnetic effect & Examples \\
\midrule
Uniform & Zeeman splitting & Magnetic resonance imaging \\
& Torque & Alignment of magnetic powder \\
& Hall effect, magnetoresistance & Sensors, read-heads \\
& Force on conductor & Dynamic Motors, actuators, loudspeakers \\
& Induced emf & Generators, microphones \\
Nonuniform & Force on charged particles & Beam control, 
radiation sources (microwave, uv; X-ray) \\
& Force on magnet & Bearings, couplings, Maglev \\
& Force on paramagnet & Mineral separation \\
Time varying & Varying field & Dynamic Magnetometers \\
& Force on iron & Dynamic Switchable clamps, holding magnets \\
& Eddy currents & Metal separation, brakes \\
\bottomrule
\end{tabular}
\caption{Applications of permanent magnet materials, 
adapted from \textcite{coey2002}.}
\tablabel{magnet-applications}
\end{table}


That magnets can apply forces to one another over a distance is
quite a novel concept in a mechanical world accustomed to
friction. It has been a short while, relatively speaking, that
it has been possible to even \emph{produce} magnets with enough
coercive force to apply useful mechanical forces. Non-contact
magnetics in mechanical systems is advantageous due to high
precision and wear-free operation due to lack of friction. This
section looks broadly at some of the main applications of the
field.

\subsection{Maglev transportation}

The largest body of research into magnetic levitation is on
so-called `maglev' transportation. Its well-known goal is to
use a levitated train or car to provide extremely fast and
efficient transportation. This field, which is rather diverse
in terms of the techniques under investigation, is finally now
achieving commercial application in the real world after some
20 or 30 years of research.  While it has some concepts in
common with this research (large loads, magnets), the
techniques used tend to be rather distanced from those that
will be applied for this project because they focus on
transportation rather than \emph{elimination} of movement.


\subsection{Magnetic actuators}

In more recent years, another application for magnetic
levitation has been investigated, which is the precision
control of a levitated platform.  Commonly cited for use in
the semiconductor industry for photolithography, these
levitators were first researched around 20 years ago.

The first designs allowed travel in a single direction, \eg,
\textcite{trumper1992}, while more recent developments allow
more directions of control.  Such devices are capable of
supporting small loads, and applying horizontal translation
forces to effect displacements of up to around \SI{200}{mm}
with nanometre precision. Two planar devices are invented in
the independent theses of \textcite{kim1997} and
\textcite{molenaar2000}. Most recently, a six degree of
freedom non-contact actuator was demonstrated by
\textcite{verma2004}. See also the device built by
\textcite{kim2007} to support around \SI{2}{kg} over a travel
of $\SI{5}{mm}\times\SI{5}{mm}$ with nanometre precision. The
high performance and mechanical simplicity of their design is
note-worthy.

Here're some more to look at: \textcite{boeij2008,zhang2008a}

The reason these devices are unsuitable for this research is
due to their travelling capability. Rather than using the
primarly magnetic flux for load support, these designs use it
in order to provide positioning control in the horizontal
directions.

Here's something new: \textcite{shameli2008}.

A recent six degree of freedom actuator has been demonstrated
by \textcite{jansen2008}, which supports loads in the order of
\SI{10}{kg} with a relatively large air gap (\SI{1}{mm}--\SI{2}{mm}) 
and large stroke ($\SI{230}{mm}\times\SI{230}{mm}$).

[\textcite{dasilveira2005}] Analytical expression for the normal force
between two magnets on a back-iron plate and a perpendicular coil. The system
is for a planar actuator, and the motivation is to be able to determine the
amount of out-of-plane force generated by a particular design. Could very well
be useful for some of my ideas.

\subsection{Interesting devices}

I like magnetic levitation of objects in a wind tunnel
\cite{higuchi2008}.

Magnetic fields can be used for a vast array of scientific uses. For example,
noncontact sensing of material properties that involve variable conductivity,
including fatigue cracks, defects in printed circuit boards, and even plastic
landmine detection \cite{mukhopadhyay2005}.

Here would be a good place to cite the brain, nerve, stuff, etc
\parencite{lu2008,demachi2008}


Here's a fun application of magnetics where a computer input device is built
with magnetic sensors placed on the wrist in order to sense single finger-tip
motion from the opposite hand \parencite{han2008}.

I don't know if I care, but \textcite{vanwest2007} created a
haptic interface for manuipulating small objects with magnetic
levitation using the so-called `zero power' technique well
known due to Mizuno.

\textcite{park2008} demonstrates a MIMO controller for a flywheel energy
storage mechanism using magnetic bearings while applying active vibration
isolation.

[\textcite{tomie2005}] This paper's a bit of fun; uses a magnet attached
to a cantilever, excited by an external field in a water tank, to propel a
robotic fish. Only left-right oscillations were produced, so the fish was
constrained to move in a plane; two turning methods are investigated.




\section{Digression}

The method for literature review taken in this thesis may be considered rather
eccentric. During the course of my research, I couldn't help but notice the
flood of literature to be an almost unstoppable flow of information that was
rather difficult to remain `on top of'. Certainly required some expenditure of
effort on my part. And perhaps to my detriment, the bibliography in this
thesis is rather large — although certainly not exceeding that of a single
literature review in certain specific fields.

This thesis is a drop of water in the ocean of this literature that is
increasing every month. It will be a defining problem in the next few decades,
I believe, to attempt a unification and consolidation of the amount of
information that is currently available and, also, and more importantly, the
way that new work is published.

\section{Notation}

The mathematics in this thesis is typeset consistently but somewhat
differently than might be expected. Here's an arbitrary example:
\begin{dmath*}
f\fn{x} = \half\gp{x+\Sin{x}}.
\end{dmath*}


Here's a nice big glossary section that will take me fucking ages to get
right. Maybe I should just scrap it.

\section{Typesetting}
The typesetting of this thesis has been carefully designed with the following main objectives:
\begin{enumerate}
\item Ease of reading content, and
\item Ease of finding information.
\end{enumerate}
For the first point, a highly legible font has been chosen and the size of the
text calculated for optimal reading properties. For the second point, the
outer margins of the page have been utilised for holding `markers' such as
equation numbers and section numbers; this places such information close the
edge of the paper to aid `flicking through' the thesis. It also moves them
away from the main text where their encroachment of the material in the main
text is less useful.

In the \PDF\ version of this document, clickable hyperlinks have been inserted in
all relevant cross-referencing situations to aid navigation.
