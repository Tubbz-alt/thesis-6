\documentclass[11pt,a4paper]{memoir}
\edef\asydir{\jobname}
\usepackage{thesis-preamble}
\EndPreamble
\begin{document}

\chapter{Magnetic springs}
\chaplabel{magnet-design}

\epigraph {
  As the weaver elaborated his pattern for no end but the pleasure of his aesthetic sense, so might a man live his life, or if one was forced to believe that his actions were outside his choosing, so might a man look at his life, that it made a pattern.
  There was as little need to do this as there was use.
  It was merely something he did for his own pleasure.
}
{\citetitle{maugham1915}\\\textcite{maugham1915}}

\section{Introduction}

The force created between permanent magnets can be used in various ways for supporting load.
This chapter investigates a number of different configurations, some of which having been used to various degrees in the literature.
Varieties in design lend themselves for different purposes, such as optimising for static load force, number of directions of instability, and low variability in resonance frequency.

\referpaper{The material presented in \secref{oblique} is based on material that has been published as a journal paper~\cite{robertson2012-jsv}.}

\section{Magnet properties and selection}

Permanent magnets are available in a range of materials; for the purposes of this work we will consider only neodymium-boron-iron alloys, which are cheap and widely used nowadays.
These magnets are available in a variety of grades corresponding to their strength of magnetisation.
Generally, the larger the magnet the lower the grade.

Magnet price is approximately linear with the volume of magnetic material, as shown in the cost graph from a typical magnet supplier shown in \figref{mag-price}.
Prices for permanent magnets have increased dramatically in recent years with greater demand for rare-earth metals in the global market \cite{coey2011-ietm}.

\begin{figure}
  \psfragfig{\jobname/mag-price}
  \lofcaption{Magnet price versus magnet volume for rare earth magnets.}{
    Data obtained from \KJMagnetics/ for cube magnets of magnetisation grade \acro{N42}.
    Late 2012 prices are slightly lower than those shown for 2011.}
  \figlabel{mag-price}
\end{figure}

Permanent magnets are graded according to their maximum energy product (\eqref[vref]{bhmax}); \eg, a magnet with grade N42 has $\BHmax=\SI{42}{MG.Oe}=\SI{334}{kJ/m^3}$,
  \note{The conversion to metric for the units of $\BHmax$ is $\SI{1}{MG.Oe}=\num{100}/(4\pi)$ \si{kJ/m^3}.}
from which the remanence can be calculated using
\begin{dmath}
  \remanence = 2\sqrt{\permvac\BHmax}
\end{dmath},
which corresponds to $\remanence=\SI{1.3}{T}$ for an N42 grade magnet.

Magnets with grade N52 have an approximately 12\% increase in magnetic flux density over N42 grade, which corresponds to a force increase of 25\%, for an average cost increase of 38\%.
As an aside, the highest-known theoretically achievable remanence magnetisation for a permanent magnet is around \SI{2.4}{T} \cite{sellmyer2002}, corresponding to a force increase of 236\% over N42 grade magnets.
No permanent magnetic material has yet been discovered to achieve anywhere near this magnetisation strength.

The analyses of  performed in this thesis is generally normalised by magnet strength; that is, the remanence magnetisation for any theoretical calculations is assumed to be $\remanence=\SI{1}{T}$.
For reference, grade N42 magnets have a remanence of around \SI{1.3}{T} (varying by manufacturer) and grade N52 magnets of around \SI{1.5}{T}.

\subsection{Homogeneity of magnetisation}

In \secref{magnet-forces} it was assumed for deriving the equations for the forces between magnets that each magnet had a constant and uniform magnetisation.
To be fabricated, a homogeneous permanent magnet requires a very strong magnetic field of constant direction and magnitude, and the larger the cross-section of the magnet the harder such a field is to obtain.
For larger permanent magnets, care must be taken when using the force equations presented in the previous chapter.
For example, the magnetic field of a grade \textsc{n35} cylindrical magnet with diameter \SI{100}{mm} and thickness \SI{30}{mm} was compared against theoretical calculations (performed with ANSYS) \cite{bruno2011-honoursthesis}.
While the comparison across axial displacement (\figref{gauss-vertical}) only shows that the measured magnetic field is lower than expected, the lateral displacement results (\figref{gauss-radial}) indicates why: the magnetic field is lower than expected closer to the centre of the magnet.

\begin{figure}
\begin{wide}
\begin{subfigure}
\psfragfig{PhD/Experiments/Gauss/fig/gauss-vertical}
\caption{Axial displacement down the centreline.\figlabel{gauss-vertical}}
\end{subfigure}\hfil
\begin{subfigure}
\psfragfig{PhD/Experiments/Gauss/fig/gauss-radial}
\caption{Radial displacement with a \SI{5}{mm} offset.\figlabel{gauss-radial}}
\end{subfigure}
\lofcaption{
  Magnetic field measurements for a large cylindrical permanent magnet.
}{
  The origin of the measurements is \SI{5}{mm} from the face of the magnet.
  Measured results are not consistent with the theoretical results that assume constant magnetisation.
}
\end{wide}

\end{figure}

\section{Magnets in repulsion}
\seclabel{unstable-vertical}

Two magnets with faces aligned in repulsion create a simple spring with a nonlinear `stiffening' force versus displacement characteristic; as the magnets become closer together the change in force becomes larger.
When bearing load at equilibrium, for small displacements such an arrangement closely approximates a classic spring--mass--damper system which can be used for vibration isolation \parencite{puppin2002}.

An example of such a magnetic spring can be seen in \figref{magratio-schematic}, with one fixed and one floating magnet arranged vertically.
With like poles facing, the two magnets repel each other and will hold the system at equilibrium with an air gap between them.
Displacement towards each other is restored by the repulsive magnetic force, and displacement away is restored by gravity.
The floating magnet must be constrained in both horizontal directions by the shaft.
If the contraint is removed, it will be naturally unstable horizontally due to Earnshaw's theorem (see \secref{earnshaw}).

A natural question for magnet selection in such cases is `what size should the magnets be?'.
As magnet price scales with volume (\figref{mag-price}), it is often desirable to minimise the magnet volume for a required force, and the question then becomes what magnet \emph{shape} to choose.
Permanent magnets in two different shapes are examined here: square-face cuboid and cylindrical.

\subsection{Square-face cuboid magnets}
\seclabel{square-cuboids}

Consider the basic magnetic spring shown in \Figref{magratio-schematic}, consisting of two magnets separated by a displacement, $\displ$ (measured between the near faces), and generating a repulsive force, $\forceMag$, between them. The magnets have square facing sides, a height-to-width ratio of $\ratioCub=\lengthCub/\faceCub$, and a fixed volume $\volMag$; the height of each magnet and the face size width (and length into the page) is, \resp,
\begin{align}
\lengthCub &= \gp{\volMag\ratioCub^2}^{1/3}, &
\faceCub   &= \gp{\frac\volMag\ratioCub}^{1/3}.
\end{align}

\begin{figure}
\centering
\asyinclude[height=4cm]{\jobname/magratio.asy}
\lofcaption{Schematic of a repulsive magnetic spring.}{ The magnets have square facing sides and extend a distance $\faceCub$ in the direction toward the reader. When $\displ=0$ the magnet faces are touching.}
\figlabel{magratio-schematic}
\end{figure}

The magnetic forces between the magnets can be calculated by applying the theory of \textcite{akoun1984}, where the force $\forceMag=F_m\fn{\volMag,\ratioCub,\displ}$ is a function of magnet volume $\volMag$, size ratio $\ratioCub$, and displacement $\displ$.

Such forces were calculated between these magnets for a magnet volume $\volMag=(\SI{10}{mm})^3$ over a displacement $\displ$ from \SIrange{0}{10}{mm} and a magnet size ratio $\ratioCub$ from $0.1$ to $1$. Note that the forces were calculated with a magnetisation of \SI{1}{T} for both magnets, essentially normalising the output forces by the magnetisation strength.

In order to compare the force versus displacement characteristics for a range of magnet size ratios, the forces are normalised by the  $\forceMag_s=F_m(\volMag,1,\displ)$; that is, the force for unity magnet size ratio $\ratioCub=1$. \Figref{magratios} shows the normalised force $\normforceMag=\forceMag/\forceMag_s$ as a function of displacement $\displ$ over a range of magnet size ratios $\ratioCub$. The figures are drawn as separate graphs in order to avoid overlap of the curves; size ratio $\ratioCub$ varies from $0$ to $0.4$ in \Figref{magratio1} and from $0.4$ to $0.8$ in \Figref{magratio2}. It can be seen from the two graphs that a magnet size ratio $\ratioCub$ of around $0.4$ produces the greatest forces; for values both less and greater than $0.4$, the normalised force curves decrease.

\begin{figure}
\begin{wide}
\subfloat
  [Magnet ratios $\ratioCub=\{0.1,0.2,0.3,0.4\}$
   \figlabel{magratio1}]
  {\psfragfig{magcode/examples/fig/magratio1}}
\subfloat
  [Magnet ratios $\ratioCub=\{0.4,0.5,0.6,0.7,0.8\}$
   \figlabel{magratio2}]
  {\psfragfig{magcode/examples/fig/magratio2}}
\end{wide}
\caption{
  Normalised force $\normforceMag=\forceMag/\forceMag_s$ for square-faced cuboid magnets as a function of displacement $\displ$.
}
\figlabel{magratios}
\end{figure}

Some small overlap in the force curves for $\ratioCub=0.4$ and $\ratioCub=0.5$ is seen in \Figref{magratio2}. This indicates that the optimum magnet size ratio (to maximise the force) is dependent on the displacement between the magnets. \Figref{magratio} shows the magnet force varying as a function of magnet size ratio $\ratioCub$ with a set of curves corresponding to fixed displacements from \SIrange{1}{10}{mm}. For each curve, there is a local maxima in the force; this corresponds to the magnet size ratio that produces the greatest force at that displacement. While the magnet size ratio that produces the greatest forces varies somewhat with displacement, the graph shows that the optimum magnet size ratio remains around $\ratioCub\approx0.4$.
A similar result was shown by \textcite{cooper1973-ietm} to optimise the forces between magnetic cylinders, using a numerical integration method for the calculations.

\subsection{Cube magnets in two orientations}
\seclabel{cube-compare-orth}

As stated in the introduction in \secref{bearings}, by taking a \twoD/ solution for analysing the forces between ring magnets (\ie, assuming the curvature of the magnets has negligible effect), \textcite{yonnet1978} showed that axially- and radially-magnetised ring magnets produce the same reaction forces.
This is an important result for ring magnets, as it is much more difficult to magnetise a ring in the radial direction than in the axial direction.
There might be considered to be an analogous result for cube magnets; after all, their cross-sections are the same.
To investigate this possibility, the force versus displacement characteristic between two cube magnets was calculated for displacement perpendicular and orthogonal to the magnetisation directions, as shown in \figref{xz-magforce-schem}.
The results shown in \figref{xz-magforce-plot} show that these forces are not equal; in fact, for cube magnets it can be seen that the forces for displacement orthogonal to the magnetisation are exactly half those for magnets which are parallel.

\begin{figure}
\begin{wide}
\begin{subfigure}
  \asyinclude{\jobname/twomagdir}
  \caption{
    Schematic of the two magnet orientations.
    \figlabel{xz-magforce-schem}
  }
\end{subfigure}
\begin{subfigure}
  \psfragfig{magcode/examples/fig/xz-magforce}
  \caption{
    Forces in two directions versus displacement.
    \figlabel{xz-magforce-plot}
  }
\end{subfigure}
\end{wide}
\caption[Force/displacement curves between two cube magnets in two orientations.]{
  Force/displacement curves between two cube magnets in two orientations with side lengths of \SI{10}{mm} and a magnetisation of \SI{1}{T}.
}
\figlabel{xz-magforce}
\end{figure}


\subsection{Cylindrical magnets}

A similar analysis can be performed with cylindrical magnets using a cylindrical magnet ratio defined as the ratio between magnet length and radius $\ratioMag=\lengthMag/\oradiusMag$.
Accordingly, the magnet length and radius are calculated using, \resp,
\begin{align}
\lengthMag &= \gp{ \frac{\volMag}{\pi \ratioMag^2} }^{1/3},  & \radiusMag &= \gp{ \frac{\ratioMag \volMag}{\pi} }^{1/3},
\end{align}
with a fixed magnet volume $\volMag$ over a range of magnet ratios $\ratioMag$.
The force is calculated using \eqref{simpl4} a magnet volume of $\volMag=\gp{\SI{10}{mm}}^3$ with force \vs\ magnet ratio over a range of displacements shown in \figref{magratio-cyl}.
Interestingly, the behaviour for cylindrical magnets shows a different trend than for cuboid shaped magnets; additionally, these results suggest that the force per unit volume is slightly larger for cylindrical magnets due to the closer proximity of the elemental volumes.


\begin{figure}
\begin{wide}
\subfloat
  [Square-faced cuboid magnets.\figlabel{magratio}]
  {\psfragfig{magcode/examples/fig/magratios}}
\subfloat
  [Cylindrical magnets. Note the greater force peaks compared to the cuboid magnets.\figlabel{magratio-cyl}]
  {\psfragfig{magcode/examples/fig/magratios-cyl}}
\end{wide}
\lofcaption{Force between two $\volMag=\gp{\SI{10}{mm}}^3$ magnets as a function of magnet ratio for a set of fixed displacements from \SIrange{1}{10}{mm}.}{ Dots mark the positions of maximum force.}
\end{figure}

\subsection{Comparing cuboid and cylindrical magnets}

In the previous section it was shown that cylindrical magnets appear to achieve a slightly larger force per unit volume in their optimum magnet ratio.
By reformulating the magnet ratios to incorporate face surface area, it is possible to directly analyse the two magnet shapes over a range of alternative aspect ratios to make a definitive comparison.

An alternative magnet ratio $\nu$ is defined to be the ratio of magnet length squared to face area.
For square-face cuboid magnets and cylindrical magnets the magnet ratio is therefore, \resp.,
\begin{align}
\nu_{\mathrm{cub}} &= \lengthCub^2/\faceCub^2 = \ratioCub^2,
&
\nu_{\mathrm{cyl}} &= \frac{\lengthMag^2}{\pi\radiusMag^2} = \ratioMag^2/\pi.
\end{align}
In other words, for equal $\nu$ a cylindrical and cuboid magnet will have the same face area as shown in \figref{squarecircle}.
  \note{Relating to a geometrical problem of historical interest known as `squaring the circle' \cite{hobson1913-squircle}.}
Calculating the force versus displacement characteristics across magnet ratio $\nu$ between cylindrical and cuboid magnets ($F_{\mathrm{cyl}}$ and $F_{\mathrm{cub}}$, \resp) shows that the cylindrical magnets produce greater forces across almost the entire ranges of displacement and magnet ratio (\figref{cub-cyl-ratios-plot}).

This particular method for demonstrating the force comparison between cylindrical and cuboid magnets is new work; however, \textcite{nagaraj1988} performed a more extensive comparison between cuboid and cylindrical magnets including an analysis of eccentric displacements.

\begin{figure}
\begin{wide}
\begin{subfigure}
\asyinclude{\jobname/equal-area}
\caption{
  Equal face area for the cuboid and cylindrical magnets.
  \figlabel{squarecircle}
}
\end{subfigure}\hfil
\begin{subfigure}
\psfragfig{magcode/examples/fig/cub-cyl-ratios}
\caption{
  Ratio of cylindrical force to cuboid force versus displacement for a range of magnet ratios $\nu$.
  \figlabel{cub-cyl-ratios-plot}
}
\end{subfigure}
\end{wide}
\caption{For magnets with equal volume and face area to length ratios, cylindrical magnets produce greater force.}
\figlabel{cub-cyl-ratios}
\end{figure}


\subsection{Force coupling between degrees of freedom}
\seclabel{coupling}

A magnetic suspension has inherent dynamic coupling between its degree of
freedom because the magnetic force is a function of displacement in both
horizontal and vertical displacements.

This is essential to consider for the purpose of vibration isolation because
disturbances from the ground can be transmitted via all six degrees of
freedom. Lateral vibrations will cause vertical vibrations; due to symmetry of
the forces in the horizontal directions, vertical disturbances will only couple
to horizontal forces if the magnets are not vertically aligned.

An example of this is shown in \figref{coupling-oscillations} for a
suspended magnet that is stabilised in the horizontal direction with the
addition of active positive stiffness and constrained in the other horizontal direction (\ie, only a planar case is considered).
This simulation was performed on a magnetic system with \SI{10}{mm} cubes in vertical repulsion, with a floating mass of \SI{0.5}{kg} and damping ratio of \SI{5}{\%}.
The base magnet was excited with a sinusoidal input $p$ at the vertical resonance of the system with a magnitude of \SI{1}{mm}.
The dynamics were defined in the horizontal and vertical directions, \resp,~as
\begin{dmath}
\mass \ddot y + \damping \gp{\dot y - \dot p} - F_y\fn{0,y-p,z} + \gainDisp \gp{y-p} = 0
\end{dmath},
\begin{dmath}
\mass \ddot z + \damping \dot z - F_z\fn{0,y-p,z} + \mass g = 0
\end{dmath},
where $F_y$ and $F_z$ are the magnetic forces calculated with \eqref{akoun}, and $\gainDisp$ is the feedback gain on the relative horizontal displacement term.
The (unstable) horizontal stiffness for this system was calculated as around \SI{-200}{N/m} and the displacement feedback gain was chosen as $\gainDisp=250$ to overcome the instability due to this.

\begin{figure}
  \subbottom[Schematic.]{\asyinclude{\jobname/coupling}}\hfill
  \subbottom[Displacement trace.]{\psfragfig{\phdpath Simulations/Coupling/fig/coupling-oscillations}}
  \lofcaption{Oscillations over time of the displacement of a suspended magnet that is excited through horizontal vibrations of the base magnet.}{ The figure-eight trajectory shows that the horizontal vibration couples to the vertical forces.}
  \figlabel{coupling-oscillations}
\end{figure}

In the equilibrium position, the vertical forces are strongest and the
horizontal forces are weakest. The vertical force varies little with
horizontal displacement, as shown in \figref{mag-coupling-equil}, implying
that the coupling between horizontal displacement and vertical force will be
small.

\begin{figure}
\begin{wide}
  \begin{subfigure}
  \psfragfig{\phdpath Simulations/Coupling/fig/mag-coupling-equil}
  \caption{Equilibrium positions of a suspended magnet bearing a range of
  masses varying with horizontal displacement.}
  \figlabel{mag-coupling-equil}
  \end{subfigure}\hfil
  \begin{subfigure}
  \psfragfig{\phdpath Simulations/Coupling/fig/mag-couplingratio-equil}
  \caption{Ratio of vertical to horizontal displacement (`coupling ratio') of
  a suspended magnet bearing a range of masses varying with horizontal
  displacement.}
  \figlabel{mag-couplingratio-equil}
  \end{subfigure}
\end{wide}
  \lofcaption{
    Evaluating the coupling between horizontal offset and vertical forces.}{
    The range of displacement is one quarter of the magnet width.
  }
\end{figure}

The coupling can be quantified by looking at the ratio between horizontal
disturbance displacement range and vertical reaction displacement range. The
larger the horizontal displacement range, the larger the coupling ratio as
shown in \figref{mag-couplingratio-equil}.

While not investigated further in this work, the coupling ratio requires careful attention in the design on a magnetic isolation device with three or more \dof/.
Depending on the ratios of vertical to horizontal resonance frequency (and this will be dependent on the control used, as well), improving the horizontal isolation performance may come as a detriment to the vertical isolation.
As shown in \figref{mag-couplingratio-equil}, however, for small displacement regimes any coupling problems may be negligible compared to the other sources of disturbance.
Regardless, this is a factor that should be considered for all practical levitation devices.
The coupling between vertical and horizontal directions can be amplified in magnetic systems composed of multipole or Halbach arrangements, which is discussed further in \secref[vref]{multipole-coupling}.


\section{Simple magnetic springs using cuboid magnets}

The aim of this section is to introduce a variety of simple magnetic springs, to show some of the possibilities in force behaviour that can be achieved.
It is implicit in Earnshaw's theorem \cite{earnshaw1842} (elucidated by Tonks \cite{tonks1940}) that all permanent magnet suspensions must be unstable, so the one of the main factors involved in analysing the efficacy of a magnetic spring is its type and degree of stability.

Earnshaw established the relation for purely magnetic systems that the translational stiffesses in each direction sum to zero
\begin{dmath}[label=stiffness-relation]
  K_x+K_y+K_z=0
\end{dmath},
although, as discussed in \secref{rotation-freedom}, no such relation exists between the rotational stiffnesses.
Introducing soft iron, which only attracts permanent magnets, into the system changes \eqref{stiffness-relation} to $K_x+K_y+K_z<0$, reducing the stability of the system \cite[App.\,A]{nijsse2001}.
For this reason, it is rare to see ferrous material featured as part of the design for magnetic springs.
This is in contrast with magnetic circuit design (which includes magnetic fasteners in application) which uses ferrous material to guide and constrain the magnetic flux.

Magnetic spring systems can be created with different arrangements of magnets and combinations thereof, influencing the stiffness characteristics in various ways.
In the following sections (\secref*{unstable-spring}--\secref*{combination-spring}), six different magnetic spring configurations are introduced based around cuboid magnets.
These are compared with each other in \secref{isoforces} with \figrangeref{vattr-forces}{zhv-forces} showing for the different designs their contours of force \vs\ displacement in each direction.
These plots allow a qualitative assessment of the complexity of the force fields of each spring design.

\subsection{Unstable vertical spring}
\seclabel{unstable-spring}

The first three spring designs, \figref{spring-zk3}, are variations on using vertically-aligned magnets for load bearing.
The unstable arrangement consists of a fixed upper magnet which applies a gravity-cancelling force on a lower magnet in attraction.
This configuration is discussed in the literature review in the context of active control methods (\secref{levitation-control}), and is the most common magnetic suspension for demonstrating nonlinear control design.

\begin{figure}
  \asyinclude{\jobname/spring-zk3}
  \caption[Three magnetic springs for load bearing in the vertical $\mathbf z$ direction.]{Three magnetic springs for load bearing in the vertical $\mathbf z$ direction with magnetic force indicated with the solid arrowhead.}
  \figlabel{spring-zk3}
\end{figure}

The stability criteria for this system is
\begin{dmath}[compact]
  K_z<0,\quad K_x=K_y=-\half K_z>0
\end{dmath},
and although it is unstable in only a single \dof/, it is inconvenient to bear loads with a negative stiffness spring.


\subsection{Stable vertical spring}
\seclabel{spring-repl}

The opposite of the unstable vertical spring is the stable vertical spring, consisting of a fixed lower magnet supporting a floating upper magnet in repulsion.
Its stability criteria is
\begin{dmath}[compact]
K_z>0,\quad K_x=K_y=-\half K_z<0
\end{dmath}.
The vertical force optimisation for this arrangement has already been analysed in detail in \secref{unstable-vertical}.
Since this configuration experiences instability in both horizontal directions, its stabilisation requires multiple-\dof/ control.

\subsection{\QZS/ spring}
\seclabel{qzs-basics}

First introduced by \textcite{nijsse2001}, the \qzs/ spring uses equal and opposite magnet pairs to create a force--displacement relationship that has a region of zero slope, shown in \figref{zspring-vary-gap} for half-inch magnet cubes.
With a positive vertical stiffness in series with a negative stiffness, the local minimum in the force/displacement curve creates a marginally stable point of \qzs/ in all three translational \dof/.
If the magnet positions are chosen such that the operating point is midway between the fixed magnets, the stability criteria for this spring is
\begin{dmath}[compact]
  K_x=K_y=K_z=0
\end{dmath}.
While the \qzs/ magnetic spring cannot be passively operated at this centred position of zero stiffness (as it has only marginal stability there), the upper attractive magnet can be considered to simply \emph{reduce} the natural frequency of the system without driving it to zero.
In this case, the operating position of the floating magnet is situated closer to the lower magnet, as shown in \figref{spring-zk3}.
This improves vibration isolation characteristics without altering the stability of the system (given bounded displacements), and this system is analysed in detail in \secref{qzs}.
This configuration has also been used for a tunable vibration energy harvesting device \cite{challa2008}, since the resonance frequency can be adjusted by varying the positions of the outer magnets.

\begin{figure}
  \psfragfig{PhD/Simulations/Single_magnets/z_spring/fig/qzs-sum}
  \lofcaption{`\QZS/' vertical spring forces \vs\ vertical displacement from the centred position.}{ The magnets are \SI{20}{mm} cubes with \SI{1}{T} magnetisation, and there is \SI{80}{mm} between the top and bottom magnet centres. Dotted region for the summation of forces corresponds to negative (unstable) stiffness.}
  \figlabel{zspring-vary-gap}
\end{figure}


\subsection{Horizontal spring}
\seclabel{hspring}

Unlike the vertical magnetic springs discussed above, a horizontal arrangement of magnets in attraction, \figref{hspring}, provides stability in the vertical direction with only a single degree of horizontal instability.
The stability criteria here is
\begin{dmath}[compact]
K_x<0,\quad K_y>0, \quad K_z>0
\end{dmath}.
Note this configuration of magnets adds stability in the out-of-plane ($\mathbf y$) direction.

\begin{figure}
  \asyinclude{\jobname/spring-ha}
  \caption[`Horizontal' attraction magnetic spring.]{A horizontal spring with attracting magnets to create positive vertical stiffness.}
  \figlabel{hspring}
\end{figure}

The force--displacement behaviour for the horizontal spring, \figref{h+v-forces-v}, is more complex than for the vertical springs as the vertical forces are dependent on the horizontal placement of the magnets.
\Figref{h+v-forces-v} includes the force curve of the vertical spring for comparison; magnets for the simulation were \SI{20}{mm} cubes with \SI{1}{T} magnetisation.
The vertical spring force occurs as the magnet displaces vertically downwards, which spring force will increase with distance only for a certain range.
Beyond this range, the force will begin to decrease as the floating magnet becomes further from the fixed magnets — until the spring stiffness turns negative and the system becomes unstable.

The effect of varying the gap between the fixed and floating magnets for the horizontal spring also varies this behaviour; the further away the fixed magnets are, the weaker the forces are but the larger the displacement range before instability.
The inflection point where the horizontal spring force reaches its peak is a \qzs/ point; careful tuning of such a spring could yield similar advantages to the vertical \qzs/ spring introduced in \secref{qzs-basics}.

\begin{figure}
  \psfragfig{PhD/Simulations/Single_magnets/4_magnet_spring/fig/hvspring}
  \caption[Spring forces of the repulsion \& horizontal springs.]{
    Vertical forces of the repulsion (dashed) and horizontal (solid) springs with cube magnets of size $m=\SI{20}{mm}$. The horizontal spring has a magnet face gap of $l/m\in\{0.1,0.5,1\}$.
  }
  \figlabel{h+v-forces-v}
\end{figure}


\subsection{Combination spring}
\seclabel{spring-hv}

There is a degree of compromise with the horizontal spring on its load bearing ability.
It is possible to offset this by augmenting the horizontal spring with an additional vertical magnet in repulsion to increase its load bearing capacity (\figref{hvspring}).
There are now three fixed magnets: one below, which provides the majority of the stiffness of the spring, and two aside, for stabilising one \dof/ in the out-of-plane direction, creating a stability criteria of
\begin{dmath}[compact]
K_z>0,\quad K_x>0,\quad K_y<0
\end{dmath}.
An alternative double-stable spring is discussed in \secref{choi-spring}.

\begin{figure}
   \asyinclude{\jobname/spring-hv}
   \caption[Combination vertical/horizontal magnetic spring.]{Combination magnetic spring, with the strength of the vertical spring and the stability of the horizontal spring. Solid arrows indicate magnetic forces.}
   \figlabel{hvspring}
\end{figure}


\subsection{Combination \qzs/ spring}
\seclabel{spring-zhv}

Lastly, one additional attractive magnet can be added to this system to produce a \qzs/ effect (\figref{zhv-spring}).
This system can approach zero stiffness in the vertical direction with a small positive stiffness in one horizontal direction and an equal negative stiffness in the other:
\begin{dmath}[compact]
K_z=0,K_y=-K_x>0
\end{dmath}.
A demonstration of the force--displacement curves is shown in \figref{zhvspring-move-z}, for a system with \SI{20}{mm} cubes magnets with \SI{1}{T} magnetisation and a nominal gap of \SI{30}{mm} between the centres of the floating magnet and the horizontal magnets and \SI{40}{mm} nominal gap for the vertical magnet centres.
The addition of the horizontal magnets produces an asymmetry in the total vertical force characteristic of the system, yet it still retains a \qzs/ inflection point.

\begin{figure}
   \asyinclude{\jobname/spring-zhv}
   \caption{Geometry of the combination \qzs/ magnetic spring with horizontal magnets for stabilisation.}
   \figlabel{zhv-spring}
\end{figure}

\begin{figure}
   \centering
   \psfragfig{PhD/Simulations/Single_magnets/5_magnet_spring/fig/zhv}
   \lofcaption{Individual and total vertical forces on the combined spring for
   displacement in the \z\ direction.}{ Dashed line for the total forces shows the region of instability.}
   \figlabel{zhvspring-move-z}
\end{figure}


\newcommand\isoforces[2]{%
  \begin{figure}[p]
    \begin{wide}
      \vspace*{-0.5cm}
      \centerline{
      \subfloat[$x$-forces.\figlabel{iso-#1-x}]{%
        \psfragfig
          {PhD/Simulations/Single_magnets/all_springs/frag/#1-xforce}%
      }\hspace*{-0.5cm}%
      \subfloat[$y$-forces.\figlabel{iso-#1-y}]{%
        \psfragfig
          {PhD/Simulations/Single_magnets/all_springs/frag/#1-yforce}%
      }\hspace*{-0.5cm}%
      \subfloat[$z$-forces.\figlabel{iso-#1-z}]{%
        \psfragfig
          {PhD/Simulations/Single_magnets/all_springs/frag/#1-zforce}%
      }%
      }%
    \end{wide}
    \caption{#2}
    \figlabel{#1-forces}
  \end{figure}
}

\isoforces{vattr}{Isoforces of the unstable vertical spring (\secref*{unstable-spring}).}
\isoforces{vrepl}{Isoforces of the stable vertical spring (\secref*{spring-repl}).}
\isoforces{z}{Isoforces of the \qzs/ spring (\secref*{qzs-basics}).}
\isoforces{h}{Isoforces of the horizontal spring (\secref*{hspring}).}
\isoforces{hv}{Isoforces of the combination spring (\secref*{spring-hv}).}
\isoforces{zhv}{Isoforces of the combination \qzs/ spring (\secref*{spring-zhv}).}

\subsection{Isoforces for magnetic springs}
\seclabel{isoforces}

As the magnetic springs become composed of more elements, analysing their force--displacement characteristics becomes quite challenging.
As the \threeD/ forces vary with displacement in three \dof/, it is not possible to directly visualise the behaviour these systems will have under arbitrary displacement.
An attempt to do so here highlights the difficulties involved.
In this case, representative force component contours are plotted against displacement in each direction separately; these contours are termed `isoforces' as they trace contours of equal force (\figrangeref{vattr-forces}{zhv-forces}).

The simple magnetic systems of a pair of magnets in either attraction (\secref*{unstable-spring}) or repulsion (\secref*{spring-repl}) will be treated first as their qualitative analysis is straightforward; consider their isoforces shown in \figref{vattr-forces,vrepl-forces}, \resp.
Each figure shows three plots which correspond to each component of force: $F_x$, $F_y$, $F_z$.
The plots are three dimensional, and partially transparent contours represent the magnitude of the respective force component under three dimensional displacement.
The colour scale goes through the spectrum from orange to purple as the forces increase.
The vertical components are easiest to comprehend.
\Figref{iso-vattr-z} shows that as the floating magnet is displaced upwards, the magnitude of the vertical force increases (more blue).
Contrariwise, \figref{iso-vrepl-z} shows an increase in vertical force with negative displacement for the spring with magnets in repulsion.
For symmetrical cases like the \x- and \y-direction forces, this makes pale green-blue the colour of zero force, with orange and purple the negative and positive extremes, respectively.
Thus, for the vertical attracting spring, \figref{iso-vattr-x,iso-vattr-y} demonstrate horizontal stability while for the vertical repelling spring,  \figref{iso-vrepl-x,iso-vrepl-y} demonstrate horizontal instability.

The vertical \qzs/ spring (\secref*{qzs-basics}) retains this simplicity of representation in \figref{z-forces}, noting particularly the `switching-stability' horizontal behaviour with vertical displacement.
Above the centred position, the spring dynamically behaves as the vertically attracting spring, and below it behaves as the repulsive spring.

The isoforces of the horizontal spring (\figref{h-forces}, \secref*{hspring}) show similar shapes to the vertical \qzs/ spring (\figref{z-forces}), albeit rotated ninety degrees.
Note, however, that due to the differences in magnetisation direction between the magnets that different a stability pattern is seen; increases in $y$- and $z$- displacement respectively both result in restoring forces.

The final two `combination' springs, \figref{hv-forces,zhv-forces}, involve superposition of isoforces seen previously.
As more magnets are used, the more complex the isoforce patterns become.
For the first combination spring (\secref*{spring-hv}), in which the stable vertical spring is combined with the horizontal spring, the isoforces are very similar to the horizontal spring (\figref{h-forces}) with the exception in the vertical forces where the increased load bearing ability can be seen in the centred blue region.

The second combination spring involves the superposition of the \qzs/ and horizontal spring, with the intention of stabilising one \dof/.
As this stability regime is the same as for the previous combination spring, the isoforces are similar in the horizontal directions.
In the vertical $z$-direction, the \qzs/ behaviour with positive displacements is more clearly seen; the secondary vertical forces added by the horizontal spring can be seen to add a `bulge' for negative displacements.

In conclusion, combining simple magnets with a variety of geometries can lead to a wide selection of magnet spring types.
Only a preliminary analysis of these has been conducted here; in \secref{oblique} a different spring type using inclined magnets is covered in detail, and later in \secref{qzs} the design characteristics of the zero stiffness spring is examined in more detail.


\subsection{Stability in two \dof/}
\seclabel{choi-spring}

\begin{figure}
\begin{wide}
\begin{subfigure}[0.35]
\asyinclude{\jobname/choi-spring}
\caption{
  Schematic of the magnetic spring.
  \figlabel{choi-schem}
}
\end{subfigure}\hfil
\begin{subfigure}[0.55]
\psfragfig{PhD/Simulations/Single_magnets/3_magnet_spring/fig/choi-stiffness}
\caption{
  Stiffnesses in each direction as a function of vertical displacement for a fixed magnet gap $\choiGap=\SI{20}{mm}$.
  \figlabel{choi-stiffness}
}
\end{subfigure}
\end{wide}
\lofcaption{
  The singularly-unstable magnetic spring proposed by \textcite{choi2003}.}{
  \SI{20}{mm}, \SI{1}{T} cube magnets are used to demonstrate the concept.
}
\figlabel{choi}
\end{figure}

In the previous section a number of magnetic designs have been proposed; here the arrangement of \textcite{choi2003} is examined.
It is interesting as it has a single unstable degree of freedom both translationally and rotationally.
This design consists of two fixed vertical magnets with some horizontal offset between them, with a floating magnet that levitates halfway between and above the fixed magnets (\figref{choi-schem}).
The work by \textcite{choi2003} was focussed on active control of this system in one degree of freedom to achieve stability; with the theory developed for modelling the forces between magnets in \chapref{magnet-theory}, this design can be analysed in more detail.
The stability and force characteristics of this design are strongly related to the geometry of the system; the sizes and aspect ratios of each magnet can be varied, along with the distance separating the fixed base magnets.
Verification of the translational stability of the system is shown in \figref{choi-stiffness}, which shows the stiffness in each direction as a function of vertical displacement for an example system using \SI{20}{mm} \SI{1}{T} cube magnets and $\choiGap=\SI{20}{mm}$ separation between the fixed magnets.
It can be seen that for a certain range of vertical displacement, both the $x$- and $z$-stiffnesses are positive, indicating a single degree of instability.
When four such devices (or any other symmetric pair configuration) are arranged in a plane, the rotational degrees of freedom can all be stabilised \cite{choi2003}.

The design of such a device has not been analysed until now; while \textcite{choi2003} demonstrated the concept, their work did not investigate the influence of varying the design parameters.
An example of the variability of characteristics seen is shown here for the simple case of using cube magnets with side length \SI{20}{mm} and magnetisation strength \SI{1}{T}.
The variable parameter in this case is the gap $\choiGap$ between the fixed magnets;
\Figref{choi-xz} shows the effect on the vertical and horizontal stiffnesses due varying the magnet gap; the out-of-plane stiffness is not shown as it is always negative.
It can be seen from the figure that the magnet gap has a large effect on the stiffness characteristics of the system, and across vertical displacement there are regions of both positive and negative stiffness in each direction.

\begin{figure}[p]
\begin{wide}
\begin{subfigure}
\psfragfig{PhD/Simulations/Single_magnets/3_magnet_spring/fig/choi-xstiff}
\caption{
  $x$-direction.
  \figlabel{choi-x}
}
\end{subfigure}\hfil
\begin{subfigure}
\psfragfig{PhD/Simulations/Single_magnets/3_magnet_spring/fig/choi-zstiff}
\caption{
  $z$-direction.
  \figlabel{choi-z}
}
\end{subfigure}
\end{wide}
\lofcaption{
  Stiffnesses as a function of vertical displacement across a range of fixed magnet gaps.
}{
  Both can be seen to have positive and negative regions.
  Out-of-plane stiffness is always negative.
}
\figlabel{choi-xz}
\end{figure}

The intention of the spring is to load the system such that positive stiffness is achieved in both directions; this displacement region can be found by taking only the positive regions of the curves shown in \figref{choi-xz}.
Finding these regions and plotting the vertical force/displacement curves against magnet gap (\figref{choi-force}) demonstrates the useable ranges of the device.
By adjusting the magnet gap in the design phase, the system can be tuned to achieve either large force and large stiffness or low force and low stiffness; the lower the load bearing capacity of the system, the larger the permissable displacement range becomes.

\begin{figure}[p]
\psfragfig{PhD/Simulations/Single_magnets/3_magnet_spring/fig/choi-zforce}
\lofcaption{
  Force versus vertical displacement across a range of magnet gaps.
}{
  Regions of positive stability are indicated with thick solid lines;
  dashed lines indicate regions of negative stiffnes in the vertical and/or horizontal directions.
}
\figlabel{choi-force}
\end{figure}

This idea of tuning the magnet characteristics to achieve certain stiffness properties is explored in more detail for a variation of this magnet design using inclined magnets in \secref{oblique}.





\subsection{Rotational degrees of freedom}
\seclabel{rotation-freedom}

It becomes more difficult to examine the behaviour of these systems in six degrees of freedom.
Equilibria of a system that have rotational instabilities can be stabilised by coupling together identical systems.
An example is the vertically stable spring, which is unstable around both horizontal directions.
By rigidly connecting a multiple of these springs, the system can be stabilised due to the addition of lever arm moments.
This is depicted for a planar case in \figref{rot-couple}.

\begin{figure}
   \asyinclude{\jobname/rot-couple}
   \lofcaption{An example of stabilising rotationally unstable springs.}{ Forces due to magnets and gravity, $F_m$ and $F_g$ are depicted. The rotationally unstable case is shown on the left; the right schematic shows the stablised coupling.}
   \figlabel{rot-couple}
\end{figure}

\textcite{delamare1994-ietm} demonstrated this concept for a radial magnetic bearing by adding a weaker, axial bearing to the system with a larger radius.
The coupling of the axial and radial bearings eliminated the rotational instability, while the strength of the radial bearing bearing was such that the effects of the axial bearing were minimised for normal operation.
This design is simpler for a bearing system due to the rotational symmetric of the ring magnets and the assumption that the bearing will enable free rotation around one axis.
In this section, a design with similar principles is shown to demonstrate that it is possible to eliminate rotational instability for a magnetic spring application (also see \secref{choi-spring}).

As with the bearing system of \textcite{delamare1994-ietm}, the method of eliminating one of these rotational instabilities involves adding supplementary weak magnets that apply small translational forces to the structure with large lever arms, such that the added translation stiffness is negligible but the added rotational stiffness is significant and stabilising.

A rotationally-stable magnet spring based on the `horizontal spring' (\secref*{hspring}) is presented in \figref{example-8-mag-spring}, which shows a top-down view.
The translational stability of the system can be assessed using the theory between parallel cuboid magnets (\eqref{akoun}).
In order to model the rotational stability, a measure must be calculated of the torque on the spring design due to rotations around each axis.
Forces in three directions can only be calculated for parallel magnets, and while there are expressions in the literature to calculate the torques between cuboid magnets \cite{janssen2010-ietm}, these expressions also only apply for parallel magnets (\ie, magnets that have no relative rotation between them).
Furthermore, the moments produced by the magnets at their lever arms around the centre of gravity of the spring are likely to dominate over the torque induced between each pair of magnets, since the rotations will be small but the lever arms will be (relatively) large.
Finally, if small rotations of the spring are assumed, then the effect of this rotation on the forces and torques between the magnets can be neglected; the assumption is that for the purposes of force calculation, the magnets will remain parallel to each other.

\begin{figure}
  \begin{wide}
    \begin{subfigure}[0.4]
      \asyinclude{PhD/Figures/Systems/magrotate2.asy}
      \caption{After rotation around $\az$.\figlabel{example-8-mag-spring}}
    \end{subfigure}\hfil
    \begin{subfigure}[0.4]
      \asyinclude{PhD/Figures/Systems/magrotate3.asy}
      \caption{Small-angle approximation for calculating forces and moments.\figlabel{magrotate}}
    \end{subfigure}
  \end{wide}
  \caption{Top-view schematic of a single unstable \dof/ concept.}
\end{figure}

The total torque on the system is therefore calculated by neglecting the torques between the magnets and by neglecting the effects of rotation on the forces between the magnets (\figref{magrotate}).
A static analysis of forces and torques created in this magnetic spring was performed using the geometry shown in \figref{magrotate-schem}.
For this analysis, cube magnets were used for simplicity and the magnet gaps in the centred position are defined to be equal between the strong and the stabilising magnets.

\begin{figure}[t]
  \begin{subfigure}[0.4]
  \asyinclude{\jobname/magrotate-schem.asy}
  \end{subfigure}\hfil
  \begin{subfigure}[0.4]
  \raisebox{2.5cm}{
  \begin{tabular}{cr}
    \toprule
      $w_1$ &  \SI{100}{mm}  \\
      $w_2$ &  \SI{300}{mm}  \\
      $d_1$ &  \SI{100}{mm}  \\
      $g_1$ &  \SI {10}{mm}  \\
      $s_1$ &  \SI {10}{mm}  \\
      $s_2$ &  \SI  {5}{mm}  \\
    \midrule
      $z_{\text{eq}}$ & \SI{-4}{mm} \\
    \bottomrule
  \end{tabular}
  }
  \end{subfigure}
  \caption{Geometry used for analysing the magnetic spring with stable rotation.}{ Magnet remanence was normalised at \SI{1}{T}.}
  \figlabel{magrotate-schem}
\end{figure}

The spring is chosen to have an equilibrium position at $z_{\text{eq}}$ below the height of the fixed magnets due to the weight being supported.
Spring parameters (shown in \figref{magrotate-schem}) are chosen to illustrate the concept of having a single degree of instability, and the forces and moments calculated with results shown in \figref{demon-stable-rotation}.
In this figure, the stabilities for displacement and rotation in each degree of freedom are illustrated with the associated forces and torques due to each perturbation.
The spring is in equilibrium at some displacement below the fixed outer magnets such that the force from the magnets balances the load force on the spring; forces in the vertical $z$ direction are given in relative difference terms to this equilibrium force.
In \figref{demon-stable-rotation} negative gradient indicates stability, as the force or torque acts in opposition to the displacement or rotation that caused it.
The figure shows that only the translatory $y$ direction is unstable, and thus in theory only a single actuator would be required to control this system in a non-contact levitating state.

\begin{figure}
  \psfragfig{\phdpath magdof/fig/stablerot-xyzall}
  \lofcaption{
    Demonstration of single-degree-of-freedom instability.
  }{
    The gradients of all forces and torques are negative (infering stability)
    except for the force along the $y$ direction.
  }
  \figlabel{demon-stable-rotation}
\end{figure}

The stability is created by the lever arm of the smaller outer magnets.
The influence of this lever arm on the rotational stiffness of the spring is shown in \figref{stablerot-length}, where the lever arm varies from twice
to four times the centre distance of the strong magnets.
In this range, the rotation stiffness begins negative (as it is without any stabilising magnets) and as the lever arm is increased the stability is increased until it becomes positive between \num{2.5} and~\num{3} times the lever arm of the centre magnets.

\begin{figure}
  \begin{wide}
    \begin{subfigure}
      \psfragfig{\phdpath magdof/fig/stablerot-length-ry}
      \caption{Around the $y$ axis. Stability decreases with increased lever arm but does not become unstable in the range shown.}
    \end{subfigure}
    \hfil
    \begin{subfigure}
      \psfragfig{\phdpath magdof/fig/stablerot-length-rz}
      \caption{Around the $z$ axis. Stability ranges from unstable to stable as the lever arm increases.}
    \end{subfigure}
  \end{wide}
  \lofcaption{
    Moment of the magnet spring as it rotates around the $\ay$ and $\az$ axes with a varying lever arm of the stabilising magnets.%
  }{
    (The moment around the $\ax$ axis does not vary with lever arm and remains stable.)
    Plots are labelled in terms of the ratio $w_1/w_2$ between lever arm of the outer stabilising magnets and the inner strong magnets.
  }
  \figlabel{stablerot-length}
\end{figure}

There is a trade-off between the size of the stabilising magnets and the effectiveness of the main force-providing magnets.
The large lever arm of the stabilising magnets will also affect the stability of the spring in the rotational direction around the $\ay$ axis; if the stabilising magnets are too large then an added instability will be created, counteracting the added stability around the $\az$ axis.
This is illustrated in \figref{stablerot-sizes-all} with spring parameters as shown in \figref{magrotate-schem} but with a stabilising magnet size varying from \numrange{0.4}{0.8} of the size of the main magnet size.

\begin{figure}
\begin{wide}
  \psfragfig{\phdpath magdof/fig/stablerot-sizes-all}
\end{wide}
  \lofcaption{
    Varying the size of the stabilising magnets for the conceptual single-degree-of-freedom magnet spring.
  }{
    Magnet size ratios $s_2/s_1$ are labelled on each plot or indicated in increasing directions with an arrow.
    The lever arm remains constant, in contrast to \figref{stablerot-length}.
    The relative sizes of the magnets is critical to achieving passive stability around the $\ay$ and $\az$ directions.
  }
  \figlabel{stablerot-sizes-all}
\end{figure}

Finally, the depth of the spring (size in $\ay$) also affects the stability of this design.
Again using the illustrative parameters of \figref{magrotate-schem}, but with spring depth $d_1$ varying from \SIrange{100}{400}{mm}, the forces and moments are calculated and shown in \figref{stablerot-depth}.
As the spring depth increases, the lever arms in both $\ay$ and $\az$ directions increase, thus increasing the magnitude of the moments produced by a given rotation.
This result indicates that increasing the spring depth does not affect the stability of the spring.

\begin{figure}
    \begin{subfigure}
      \psfragfig{\phdpath magdof/fig/stablerot-depth-rx}
      \caption{Around the $\ax$ axis.}
    \end{subfigure}
    \hfil
    \begin{subfigure}
      \psfragfig{\phdpath magdof/fig/stablerot-depth-rz}
      \caption{Around the $\az$ axis.}
    \end{subfigure}
  \caption
  [
    Moment of the magnet spring as it rotates around the $\ax$ and $\az$ axes with a varying spring depth.
  ]
  {
    Moment of the magnet spring as it rotates around the $\ax$ and $\az$ axes with a varying spring depth $d_1$, labelled in metres.
    In both cases, the stability increases with greater spring depths.
    (The moment around the $y$ axis does not vary with spring depth and remains stable.)
  }
  \figlabel{stablerot-depth}
\end{figure}

Variation of the magnetic spring parameters in this case indicates that is possible to increase the load-bearing ability of such a design without compromising the stability of the system.
In this section we have shown one magnetic system that has a single degree of translational instability, and the literature has a small number of other examples \cite{delamare1994-ietm,choi2003}.
To conclude, there are a plurality of magnet designs that can take advantage of mechanical lever arms to stabilise rotational degrees of freedom even if low-order magnet systems exhibit rotational instability.
Optimisation of such systems is an open question that will require a large amount of ingenuity; it is likely that different requirements will yield varying solutions in this area.









\section{Oblique magnetic spring design}
\seclabel{oblique}

In comparison to using springs with a linear force--displacement relationship for vibration isolation, using permanent magnets for load bearing can be advantageous due to the smaller variation in resonance frequency seen with increased load as a result of a corresponding increased stiffness.
However, two permanent magnets in direct repulsion will not completely eliminate the variability in resonance frequency due to load, only reduce it.

Often, vibration isolation systems are tuned to a narrow-band frequency range and are only effective for a given mass being supported (\secref*{narrowband}).
A resonance frequency that varies little with load force is desirable due to the resulting predicability of the vibratory behaviour; for example, changes in load force over time will not affect the resonance frequency of the support, which simplifies the system modelling and possible control scenarios.
In this sectoin, an arrangement of magnets is investigated with the aim to design a nonlinear spring such that varying the applied load $F=\mass\gravity$ results in a change in stiffness $\stiffness$ such that the natural frequency $\natfreq=\sqrt{\stiffness/\mass}$ remains approximately constant.

A similar idea using permanent magnets has been mentioned previously by \textcite{todaka2001-ietm}, who suggested using a mechanical linkage with two vertically-oriented magnets such that the floating magnet moved in an arc around a fixed magnet due to the effects of the linkage.
However, the parameters governing this design were not investigated at that time; their paper primarily investigated the relationship between resonance frequency and horizontal/vertical displacement between the two permanent magnets.
Other work proposed coupling a magnetic spring with a linear elastic spring, for which a nonlinear analysis and experimental results were shown \cite{bonisoli2007-mssp, bonisoli2007-mrc}.
Such coupled elastic--magnetic systems have been investigated by several authors to various degrees \parencite{trimboli1994, beccaria1997,carrella2008-jsv,zhou2010-jsv}, especially in the design of \qzs/ devices (\secref*{qzs-explore}).

In contrast to the \qzs/ systems introduced in \secref{qzs-basics} and analysed in more detail later in \chapref{qzs} which attempt to reduce the stiffness as much as possible, in this section magnetic forces are used in such a way to yield a larger region of low stiffness.
As with all magnetic springs, positive stiffness in the vertical direction infers negative stiffness or instability in at least one horizontal direction (\secref*{earnshaw}).
This instability may be countered with a linear bearing (or some other physical constraint) or with an active control system.

This section consists of three main parts: \secref{geom} defines the geometry of the system and presents the theory for analysing its behaviour; \secref{design} uses this theory to demonstrate the advantages of this magnet design, specifically in terms of its natural frequency versus applied load; and \secref{stabl-3dof} extends the model to analyse rotations and torques to investigate the planar stability of the system.

\subsection{Oblique spring geometry and theory}
\seclabel{geom}

A schematic of the oblique magnetic spring is shown in \Figref{mbq-schematic}.
Cuboid magnets are used that extend a distance $b$ into the page such that their facing sides are square.
The magnet angle $\mbqmagangle$ can range from \SIrange{0}{90}{\degree}, where $\mbqmagangle=\SI{0}{\degree}$ has horizontally-oriented magnets and $\mbqmagangle=\SI{90}{\degree}$ has vertically-oriented magnets.
The spring is composed of two symmetric pairs of oblique magnets; this ensures the horizontal forces cancel when the spring is centred and force is produced in the vertical direction only.

\begin{figure}
\centering
\asyinclude[width=0.8\linewidth]{\jobname/oblique.asy}
\lofcaption {
  Schematic of the oblique-magnet spring.
}{
  When magnet offset $\mbqoffset=0$ and displacement $\mbqvdisp=0$, the magnet faces are aligned and touching.
  Displacements $\mbqhdisp$ and $\mbqpdisp$ (not shown) are in the horizontal and out-of-plane directions, respectively.
}
\figlabel{mbq-schematic}
\end{figure}

Note that opposing magnets have parallel sides and anti-parallel magnetisations; hence, the force calculations by \textcite{akoun1984} (\eqref{akoun}) may be applied to this system.

Two dimensions are used to describe the relative displacement between adjacent magnet pairs.
The magnet offset $\mbqoffset$, fixed during operation, is the horizontal face gap in the centred position, and the displacement $\mbqvdisp$ can be considered as the vertical face gap in the centred position, designed to vary as the load on the spring changes.
With displacement $\mbqvdisp=0$, the facing magnets are horizontally aligned, and with magnet offset $\mbqoffset=0$ also, the magnet faces are touching.
The force and stiffness characteristics of the spring can be affected by adjusting the magnet angle $\mbqmagangle$ and the magnet offset $\mbqoffset$.

We assume that there are no magnetic interactions between magnets from one side of the spring to magnets on the other side.
This can be ensured in practice with a large enough separation between the pairs on opposite sides.
Accordingly, the total force of the spring is given by the superposition of forces for each magnet pair:
\begin{equation}
\mbqforce=\mbqforce_1+\mbqforce_2.
\eqlabel{mbq-force}
\end{equation}
To calculate $\mbqforce_{1}$ and $\mbqforce_{2}$ a local coordinate system  is defined for each magnet pair aligned in each direction of magnetisation.
Then $\mbqforce_1=\mbqrot{\theta} \mbqmforce_1$ and $\mbqforce_2=\mbqrot{\phi}\mbqmforce_2$, where $\phi=\pi-\theta$, $\mbqmforce_{1}$ and $\mbqmforce_{2}$ are the forces between the magnet pairs in the local coordinate systems of the base magnets, and $\mbqrot{\cdot}$ is the planar rotation matrix
\begin{equation}
\def\t{t}
\mbqrot{\t} = \bmatrix
 \cos \t & -\sin \t & 0 \\
 \sin \t &  \cos \t & 0 \\
0 & 0 & 1 \\
\endbmatrix
.
\end{equation}

These forces $\mbqmforce_{1}$ and $\mbqmforce_{2}$ are calculated with $\mbqmforce_{i}=\vect F_m\fn{\mbqmdispl_{i}}$ where $\vect F_m\fn{\cdot}$ given in \eqref{akoun} is the force between parallel cuboid magnets \parencite{akoun1984} and $\mbqmdispl_{1}$ and $\mbqmdispl_{2}$ are the displacement vectors between the magnet centres in the local coordinate system of the magnets given by
\begin{align}
  \mbqmdispl_1 &=
    \mbqrot{-\theta}
    \begin{bmatrix}\mbqoffset + \mbqhdisp \\ \mbqvdisp \\ \mbqpdisp \end{bmatrix} +
    \begin{bmatrix}\mbqmagh\\0\\0\end{bmatrix} , &
  \mbqmdispl_2 &=
    \mbqrot{-\phi}
    \begin{bmatrix}-\mbqoffset + \mbqhdisp \\ \mbqvdisp \\ \mbqpdisp \end{bmatrix} +
    \begin{bmatrix}\mbqmagh \\ 0 \\ 0\end{bmatrix} ,
\end{align}
where $\mbqmagh$ and $\mbqoffset$ are geometric parameters defined in \Figref{mbq-schematic}, and $[\mbqhdisp,\mbqvdisp,\mbqpdisp]\T$ are displacements in the horizontal, vertical, and out-of-plane directions, respectively. In \secref{stabl-3dof} this model will be extended with a small angle approximation to calculate forces and torques due to rotation around the $\mbqpdir$ axis.


\subsection{Influence of design parameters}
\seclabel{design}

From \secref{geom}, it is possible to calculate total force $\mbqforce$ in terms of displacement.
This section will outline the influence of the various design parameters on the force, stiffness, and natural frequency characteristics of the system.
To begin, vertical force as a function of vertical displacement $\mbqvforce(\mbqvdisp)=\mbqvforce(0,\mbqvdisp,0)$ will be considered (with other displacements $\mbqhdisp=\mbqpdisp=0$).

\subsubsection{Magnet shape}

For this entire analysis, a magnet size ratio of $\mbqmagratio=\mbqmagh/\mbqmagw=\num{0.4}$ is used.
Depending on the exact desired displacement range, values around this magnet ratio produce the maximum force between two opposing cuboid magnets for a fixed magnet volume (\secref*{square-cuboids}).
For the analysis to follow directly, the magnet volume is fixed at $\mbqvolume=\mbqmagh\mbqmagw^2=(\SI{10}{mm})^3$.
We define a `unit length' $\mbqunit=\sqrt[3]{V}=\SI{10}{mm}$ and refer in the subsequent analysis to the `magnet offset ratio' defined as $\mbqoffset/\mbqunit$.
The effects of increasing the magnet volume are addressed subsequently.


\subsubsection{Magnet angle}

Having chosen the magnet size ratio, there are two parameters that influence the force and stiffness characteristics of the spring; these are the magnet angle~$\mbqmagangle$ and the magnet offset~$\mbqoffset$.
Variations in the magnet angle affect the force characteristics to a greater extent and will be examined first.

The theory outlined in \secref{geom} was used to calculate force versus displacement curves over a range of magnet angles from \SIrange{0}{90}{\degree}.
These are shown in \Figref{mbq-fvx-angle}, which shows a dramatic effect on the force and stiffness characteristics due to changes in the inclination angle of the magnets.
Of particular interest are the low-stiffness regions in the force curves in \Figref{mbq-fvx-angle}; these are potential areas for improved vibration isolation.

\begin{figure}
\centering
\pregen{\psfragfig{magcode/examples/oblique/fig/mbq-fvx-angle}}
\caption
[
Force versus displacement of the inclined magnet spring for a range of magnet angles.
]
{Force versus displacement $\mbqvdisp$ of the inclined magnet spring for magnet angles from \SIrange{0}{90}{\degree} in \SI{5}{\degree} increments.
Offset $\mbqoffset$ between the magnets is zero.
Light gray lines indicate negative stiffness (instability) and markers show the position of quasi--zero stiffness.}
\figlabel{mbq-fvx-angle}
\end{figure}

\Figref{mbq-fvx-angle} is difficult to use for design purposes because the required load force will affect the dynamic stiffness as the system sits in equilibrium at a given displacement.
However, this equilibrium displacement is not a parameter of particular interest provided the magnetic spring is still levitating.
Therefore, for interpreting the operating conditions of the system it is more useful to consider the relationship between load force and natural frequency.

The vertical stiffness $\mbqvstiff$ can be obtained by numerical differentiation of the vertical force $\mbqvforce$:
\begin{equation}
  \mbqvstiff(\mbqvdisp) \approx - \tfrac{1}{2}\bigl[\mbqvforce(\mbqvdisp+\delta)-\mbqvforce(\mbqvdisp-\delta)\bigr]/\delta ,
\end{equation}
where $\delta$ is a small displacement increment.
The natural frequency $\omega_n(\mbqvdisp)$ as a function of displacement was calculated in terms of this vertical stiffness $\mbqvstiff$ with
\begin{equation}
  \omega_n(\mbqvdisp) = \sqrt{\frac{\mbqvstiff(\mbqvdisp)}{\mbqmasseq}} = \sqrt{\frac{\mbqvstiff(\mbqvdisp)}{\mbqvforce(\mbqvdisp)/g}}
\end{equation}
where the equivalent mass $\mbqmasseq=\mbqvforce(\mbqvdisp)/g$ is the mass required to load the spring such that its equilibrium position lies at the displacement $\mbqvdisp$.
The force corresponding to this equivalent mass is referred to as the `load force'.

By plotting natural frequency as a function of load force in \Figref{mbq-wvf-angle}, we can choose a magnet angle based on a certain load to satisfy a desired natural frequency.
Specifically, for the case of zero offset between the magnets $\mbqoffset=0$ (\Figref{mbq-wvf-angle-1}), it can be seen that at a magnet angle of $\mbqmagangle=\SI{35}{\degree}$ the natural frequency is almost independent of force for a large range of applied load (approximately $\SI{30}{N}\pm\SI{10}{N}$).

\begin{figure}
\begin{wide}
\subfloat
  [Zero offset $\mbqoffset$ between the magnets.
   At \SI{35}{\degree} the natural frequency is near-constant for a wide range of load forces.
  \figlabel{mbq-wvf-angle-1}]
  {\pregen{\psfragfig{magcode/examples/oblique/fig/mbq-wvf-angle}}}
\hfill
\subfloat
  [Magnet offset ratio $\mbqoffset/\mbqunit=\num{0.25}$.
   Near-constant natural frequency occurs at \SI{70}{\degree}.
   \figlabel{mbq-wvf-angle-2}]
  {\pregen{\psfragfig{magcode/examples/oblique/fig/mbq-wvf-angle2}}}%
\end{wide}
\caption[Natural frequency versus load force for a range of magnet angles.]{Natural frequency versus load force for magnet angles from \SIrange{0}{90}{\degree} in \SI{5}{\degree} increments.}
\figlabel{mbq-wvf-angle}
\end{figure}


\subsubsection{Magnet offset}

\Figref{mbq-wvf-angle} shows the natural frequency versus load curve for a magnet offset ratio $\mbqoffset/\mbqunit$ of zero.
Increasing the magnet offset $\mbqoffset$ changes the force and stiffness relationships of the spring; \Figref{mbq-wvf-angle-2} shows the same plot with a magnet offset ratio $\mbqoffset/\mbqunit = \num{0.25}$.
The difference in the shape of the curves is not great, but \Figref {mbq-wvf-angle-2} shows that a greater magnet offset results in smaller load forces and a smaller range in load force.
Also, the angle which corresponds to the almost-flat natural frequency curve has changed to $\mbqmagangle=\SI{70}{\degree}$.

The natural frequency versus load force is redrawn in \Figref{mbq-kvf-gaps} for a fixed magnet angle of $\mbqmagangle=\SI{45}{\degree}$ over a range of magnet offset ratios $\mbqoffset$ from zero to \num{0.5}.
At this angle, it can be seen that the region of mostly-flat natural frequency occurs at a offset ratio of $\mbqoffset/\mbqunit=\num{0.05}$.
This indicates that the magnet angle should be chosen only after the tolerances of magnet displacement are decided and a minimum offset ratio established.

\begin{figure}
\centering
\pregen{\psfragfig{magcode/examples/oblique/fig/mbq-wvf-gaps}}
\caption[Natural frequency versus load force for a range of magnet offset ratios.]{Natural frequency versus load force for magnet offset ratios from zero to \num{0.5} in increments of \num{0.05} and a magnet angle of \SI{45}{\degree}.}
\figlabel{mbq-kvf-gaps}
\end{figure}



\subsubsection{Horizontal and out-of-plane stability due to vertical displacement}
\seclabel{stabl-v}

In \Figref{mbq-wvf-angle,mbq-kvf-gaps}, design curves were presented under the assumption that the vertical stiffness only was under consideration.
Due to the inclination of the magnets, however, the horizontal and out-of-plane stiffness will also vary as the magnet spring parameters are changed.
If active control is used to constrain the floating magnets, it may be desirable to minimise the horizontal instability of the magnet spring in order to reduce the number of sensors and actuators required to stabilise the system.

The horizontal stiffness is calculated with a numerical gradient of the forces when the magnets are centred and when a small horizontal displacement~$\mbqhdisp$ is applied.
In this case, the horizontal force $\mbqhforce$ will be considered as a function of vertical displacement $\mbqvdisp$, with horizontal stiffness calculated as
\begin{equation}
\mbqhstiff(\mbqvdisp) = -\frac{1}{\delta}\biggl[\mbqhforce(\delta,\mbqvdisp,0)-\mbqhforce(0,\mbqvdisp,0)\biggr] = -\frac{1}{\delta}\mbqhforce(\delta,\mbqvdisp,0),
\end{equation}
where $\delta$ is a small displacement increment.
An equivalent formulation can be used to calculate the out-of-plane stiffness due to a vertical displacement based on the out-of-plane force $\mbqpforce$:
\begin{equation}
\mbqpstiff(\mbqvdisp) = -\frac{1}{\delta}\biggl[\mbqpforce(0,\mbqvdisp,\delta)-\mbqpforce(0,\mbqvdisp,0)\biggr] = -\frac{1}{\delta}\mbqpforce(0,\mbqvdisp,\delta).
\end{equation}

An example of spring parameters that achieve positive stability in both the vertical and horizontal directions is shown in \Figref{mbq-kvxyz-gaps-v,mbq-kvxyz-gaps-h}.
This is possible as the stiffness in the out-of-the-page direction of \Figref{mbq-schematic} is always negative (\Figref{mbq-kvxyz-gaps-p}), and as a consequence of Earnshaw's theorem \parencite{bassani2006-meccanica} the stiffnesses in each direction must sum to zero; that is, $\mbqhstiff(\mbqvdisp)+\mbqvstiff(\mbqvdisp)+\mbqpstiff(\mbqvdisp)=0$.

\begin{figure}
\begin{wide}
\hspace{-1cm}
\subfloat[Vertical stiffness.\figlabel{mbq-kvxyz-gaps-v}]
  {\pregen{\psfragfig[crop=preview]
    {magcode/examples/oblique/fig/mbq-kvx-gaps}}}%
\subfloat[Horizontal stiffness.\figlabel{mbq-kvxyz-gaps-h}]
  {\pregen{\psfragfig[crop=preview]
    {magcode/examples/oblique/fig/mbq-kvy-gaps}}}%
\subfloat[Out-of-plane stiffness.\figlabel{mbq-kvxyz-gaps-p}]
  {\pregen{\psfragfig[crop=preview]
    {magcode/examples/oblique/fig/mbq-kvz-gaps}}}%
\end{wide}
\caption
[Stiffness in three directions versus vertical displacement for a fixed magnet offset ratio.]
{Stiffness in three directions versus vertical displacement for a magnet offset ratio of $\mbqoffset/\mbqunit=\num{0.2}$ and magnet angles from \SIrange{0}{90}{\degree} in \SI{5}{\degree} increments (arrows indicate increasing magnet angle).
For the horizontal and vertical stiffness plots (a)~and (b), regions of positive stiffness for both directions are coloured; regions of gray indicate that either the vertical and/or horizontal stiffness is negative in that position for that magnet angle.
The out-of-plane stiffness plot~(c) shows instability over the entire displacement range.
}
\figlabel{mbq-kvxyz-gaps}
\end{figure}

The drawback of achieving minimal instability is a reduction in the achievable low-stiffness regions of the spring.
\Figref{mbq-wvf-angle-stabl} shows a plot of natural frequency versus load force for a magnet angle of \SI{40}{\degree} and for a variety of magnet offsets.
In this graph, regions of negative horizontal stiffness have been de-emphasised by drawing those sections of the curves in light grey.
It can be seen here that the `flat' sections of the curve (that correspond to configurations of largely-flat natural frequency against load force) occur largely in the regions of horizontal instability.
\Figref{mbq-wvf-angle-stabl} also demonstrates that when designing the system for horizontal stiffness, a larger magnet offset increases the displacement range of the magnetic spring, albeit with a decrease in possible load force.

\begin{figure}
\centering
\pregen{\psfragfig{magcode/examples/oblique/fig/mbq-wvf-angle-stabl}}
\caption
[Natural frequency versus load force for a fixed magnet angle.]
{Natural frequency versus load force for offset ratios from \num{0.05} to \num{0.5} in \num{0.05} increments and a magnet angle of \SI{40}{\degree}.
Regions of negative horizontal stiffness are drawn in light gray, and displacements are labelled with dotted lines for every change in displacement of \SI{1}{mm}.}
\figlabel{mbq-wvf-angle-stabl}
\end{figure}

A more detailed investigation on the planar stability of the system is performed in \secref{stabl-3dof}.


\subsubsection{Magnet volume}
\seclabel{mbq-vol}

Having examined the influence of magnet angle and magnet offset on the natural frequency and load force characteristics, it is essential to confirm that this arrangement is scalable for arbitrary loads by increasing the magnet volumes.
With fixed magnet offset ratio of $\mbqoffset/\mbqunit=\num{0.2}$ and magnet angle of $\mbqmagangle=\SI{40}{\degree}$, the natural frequency/force characteristic with volumes from $\mbqvolume=(\SI{10}{mm})^3$ to $\mbqvolume=(\SI{50}{mm})^3$ is shown in \Figref{mbq-wvf-vol}, which shows that larger magnet sizes permit larger load forces while also retaining a low natural frequency.
In fact, the natural frequency decreases with larger magnet sizes.
This shows that the oblique magnet spring system is suitable for bearing large loads with low stiffness, and fits into the category of springs that exhibit `high-static--low-dynamic' stiffness \parencite[e.g.,][]{carrella2008-jsv}.

\begin{figure}
\centering
\pregen{\psfragfig{magcode/examples/oblique/fig/mbq-wvf-vol}}
\caption
[Natural frequency versus load force for a fixed magnet offset ratio and angle.]
{Natural frequency versus load force for a magnet offset ratio of \num{0.2} and a magnet angle of \SI{40}{\degree} over a range of magnet volumes from $(\SI{10}{mm})^3$ to $(\SI{50}{mm})^3$.
The displacement ranges are proportional to the magnet size such that the system with magnet volume $(\SI{10}{mm})^3$ undergoes displacement from \SIrange{0}{10}{mm} and the system with volume $(\SI{50}{mm})^3$ moves over \SIrange{0}{50}{mm}.
Regions of negative horizontal stiffness are drawn in light gray.}
\figlabel{mbq-wvf-vol}
\end{figure}

\subsubsection{Design based on these results}

Clearly there is a large space of design possibilities for such a magnet arrangement.
Using these results requires an iterative approach based around the following constraints:
\begin{enumerate}
\item Magnets are large enough to bear the required load variance, which will inform a maximum and minimum magnet clearance;
\item Stiffness at the equilibrium point is satisfied by varying the magnet offset and angle;
\item Load variation is modelled and natural frequency remains within acceptable limits.
\end{enumerate}
Generally, a larger magnet size will permit a larger range of approximate natural frequency invariance (\Figref{mbq-wvf-vol}).
Only by evaluating a number of trial solutions for magnet angle and magnet offset can an acceptable design be found to satisfy a specified amount of load variability.



\subsection{Investigation into planar stability}
\seclabel{stabl-3dof}

In \secref{stabl-v}, the translational stiffness of the system in three directions was discussed in terms of a change in the vertical equilibrium position of the spring (corresponding to a variation in applied load, say).
However, this is not enough to establish the global stability of the system due to cross-axis coupling and rotational affects that were not included as part of the model.
Here, the planar stability of the system will be investigated to attempt to provide some picture of the complex kinetics seen due to planar translation and rotation; the system is assumed to be constrained in a single plane for this analysis with geometry shown in \figref{mbq-angle-schem}.

An analytical formulation for calculating the torques between two cuboid parallel magnets has recently been presented by \textcite{janssen2010-ietm}.
The torque equations will not be reproduced here but they follow a similar (albeit more complex) form than that of \eqref{akoun} for force.
Note that, with reference to \figref{mbq-angle-schem}, the torques are \emph{not} calculated by using the already-calculated force terms (the vectors $\mbqforce_1$ and~$\mbqforce_2$ in that figure); the torque is calculated using a separate integral equation that takes the lever arm into account.

Note, however, that the force and torque equations do not permit a relative rotation between the two interacting magnets (their sides must remain parallel).
Therefore, in order to analyse the rotational stability of the magnetic system a small angle approximation must be made, which is illustrated in \figref{mbq-angle-schem-approx}:
due to overall rotation $\mbqrotz$ of the spring the moving magnets will translate around their lever arms $\mbqlever$ (the centre of rotation is here assumed to be the mid-point between the magnet centres) but their angle to the horizontal remains fixed.
Calculating the force and torque in this way is only valid for small rotations, but is sufficient to establish relationships regarding rotational stability and cross-coupling with translational forces.

\begin{figure}
\centering
\pregen{\psfragfig{magcode/examples/oblique/fig/mbq-angle-schem}}
\lofcaption{Geometry of the planar system in which forces and torques due to rotation $\mbqrotz$ are calculated.}{
The system is shown with $\mbqrotz=\SI{15}{\degree}$, lever arm ratio $\mbqlever/\mbqunit=\num{2}$, magnet angle $\mbqmagangle=\SI{30}{\degree}$ and magnet offset ratio $\mbqoffset/\mbqunit=\num{0.5}$.
\figlabel{mbq-angle-schem}}
\end{figure}

\begin{figure}
\begin{wide}
\subfloat[Without rotation.]
  {\pregen{\includegraphics{magcode/examples/oblique/fig/mbq-angle-schem-1}}}\qquad
\subfloat[With rotation shown in black; the unrotated position, as in (a), is shown in light grey.]
  {\pregen{\includegraphics{magcode/examples/oblique/fig/mbq-angle-schem-2}}}\qquad
\subfloat[With small angle approximation of zero magnet rotation shown in colour; the rotated magnets, as in (b), are shown in black.]
  {\pregen{\includegraphics{magcode/examples/oblique/fig/mbq-angle-schem-3}}}
\end{wide}
\caption[Visual representation of the small angle approximation for magnet rotations.]{Visual representation of the small angle approximation in which the magnet structure rotates but the magnets themselves are assumed to remain parallel to their respective partner.}
\figlabel{mbq-angle-schem-approx}
\end{figure}

\subsubsection{Theory for planar force and torque calculations}
\seclabel{oblique-planar-theory}

The model developed for this system in \Secref{geom} is here extended to calculate torques and allow (small) rotations, both around the $\mbqpdir$ axis only.

The vector equations for this new geometry require an additional term to accomodate rotation.
First define two lever arm vectors for each magnet with respect to a centre of rotation denoted by
$
\vect l_1 = [-\mbqlever, 0, 0]\T
$ and
$
\vect l_2 = [ \mbqlever, 0, 0]\T
$
in the local coordinate system of the spring (although other centres of rotation are certainly possible).
These lever arms define additional translations of the magnets $\vect p_1$ and $\vect p_2$ due to rotation of the system:
\begin{align}
\eqlabel{vec-levers}
  \vect p_1 &= \mbqrot{\mbqrotz}\vect l_1 - \vect l_1,
&
  \vect p_2 &= \mbqrot{\mbqrotz}\vect l_2 - \vect l_2.
\end{align}
The displacement vectors (again in the coordinate system of the magnets) between the magnet pairs are then given by
\begin{align}
\eqlabel{vec-displ}
  \mbqmdispl_1 &=
    \mbqrot{-\theta}
    \left(
    \vect p_1 +
    \begin{bmatrix}\mbqoffset + \mbqhdisp \\ \mbqvdisp \\ \mbqpdisp \end{bmatrix}
    \right) +
    \begin{bmatrix}a\\0\\0\end{bmatrix} , &
  \mbqmdispl_2 &=
    \mbqrot{-\phi}
    \left(
    \vect p_2 +
    \begin{bmatrix}-\mbqoffset+ \mbqhdisp \\ \mbqvdisp \\ \mbqpdisp \end{bmatrix}
    \right) +
    \begin{bmatrix}a \\ 0 \\ 0\end{bmatrix} .
\end{align}
Also, the displacement vectors in the coordinate system of the magnets from the spring magnet centres to the centre of rotation (required for torque calculation) are given by
\begin{align}
\eqlabel{vec-displ-local}
  \vect t_1 &= \mbqrot{-\theta}\left( -\mbqrot{\mbqrotz}\vect l_1 \right),
&
  \vect t_2 &= \mbqrot{-\phi}\left( -\mbqrot{\mbqrotz}\vect l_2 \right).
\end{align}
\Eqrangeref{vec-levers}{vec-displ-local} are kept in a more general transformation matrix form to accomodate extensions into more rotational degrees of freedom.

As before, the total force is
\begin{equation}
\mbqforce=\mbqforce_1+\mbqforce_2=\mbqrot{\theta} \vect F_m(\mbqmdispl_1)+\mbqrot{\phi}\vect F_m(\mbqmdispl_2),
\eqlabel{mbq-force3}
\end{equation}
where $\vect F_m(\cdot)$ is the magnet force equation given in \eqref{akoun}. The torque is not affected by the rotation transformations (recall it is around the $\mbqpdir$ axis only) and is given by the sum of torques between the magnet pairs
\begin{equation}
\mbqptorque= T_{m_\mbqpdir}(\mbqmdispl_1,\vect t_1)+ T_{m_\mbqpdir}(\mbqmdispl_2,\vect t_2),
\eqlabel{mbq-torque}
\end{equation}
where $T_{m_\mbqpdir}$ is the appropriate component of the magnetic torque equation given by \textcite{janssen2010-ietm}.
(To be precise, the equations of Janssen et al.\ are written for magnets with $z$ direction magnetisation, so they require a coordinate transformation as the analysis here casts the magnetisations into the $x$ direction.)

\subsubsection{Planar stability results}

The system is not expected to be completely stable due to cross-axis coupling. For example, after horizontal translation the magnetic force will become asymmetric and a torque will result. Similarly, after a rotation the reverse will occur and a horizontal force will be produced, which can be seen from the resultant vectors in \figref{mbq-angle-schem,mbq-rot-ex-diag}. Due to the large number of possible magnet parameter combinations, only a select number of cases will be analysed in detail here.

The torsional stability due to rotation is affected by the geometric parameters of the system as shown by example in \figref{mbq-rot-ex}, for which each geometry is drawn to relative scale in \figref{mbq-rot-ex-diag}.
Further torque variations can be effected by varying the lever arm and the position of the centre of rotation.
The validity of the torque calculations can be assessed by comparing the torques calculated with the magnet forces only using the equation
\begin{equation}
\eqlabel{mbq-fake-torque}
\mbqptorque \approx \mbqlever (-\mbqforce_1+\mbqforce_2) \bm\cdot \begin{bmatrix} -\sin\mbqrotz \\\phantom{-}\cos\mbqrotz \\ 0\end{bmatrix}
\end{equation}
where the dot product extracts the component of force perpendicular to the lever arm.
Torques calculated in this manner are shown in \figref{mbq-rot-ex} as dashed lines and it can be seen they match closely for small angles of rotation.

\begin{figure}
\centering
\pregen{\psfragfig{magcode/examples/oblique/fig/mbq-rot-ex}}
\caption
[Torque versus rotation for a certain spring configuration.]
{Torque versus rotation for a certain spring configuration with parameters
$\mbqunit = \mbqvdisp = \SI{10}{mm}$,
$\mbqmagangle = \SI{30}{\degree}$, and
$\mbqlever/\mbqunit = \num{2}$.
Dashed lines show verification torques calculated by using the magnetic forces around their lever arms only (\eqref{mbq-fake-torque}). Notice that varying the magnet offset ratio (shown) can vary the rotational stiffness from stable ($\mbqoffset/\mbqunit=\{0.25, 0.5\}$) to unstable ($\mbqoffset/\mbqunit=1$). Geometries for these three configurations are shown in \Figref{mbq-rot-ex-diag}.}
\figlabel{mbq-rot-ex}
\end{figure}

\begin{figure}
\begin{wide}
\subfloat[$\mbqoffset/\mbqunit=0.25$]
  {\pregen{\psfragfig{magcode/examples/oblique/fig/mbq-rot-ex-diag-1}}}\qquad\qquad
\subfloat[$\mbqoffset/\mbqunit=0.5$]
  {\pregen{\psfragfig{magcode/examples/oblique/fig/mbq-rot-ex-diag-2}}}\qquad\qquad
\subfloat[$\mbqoffset/\mbqunit=1$]
  {\pregen{\psfragfig{magcode/examples/oblique/fig/mbq-rot-ex-diag-3}}}
\end{wide}
\caption
[Visual representation of the forces and torques at a certain rotation.]
{Visual representation of the forces and torques at a rotation of $\mbqrotz=\SI{10}{\degree}$ corresponding to the stability results shown in \Figref{mbq-rot-ex}. Force vector lengths are proportional to their magnitude, but torque arc lengths are not.}
\figlabel{mbq-rot-ex-diag}
\end{figure}

Stability results will be shown using perturbations of a dynamic simulation of the system in a small number of variations of design parameters.
The equations of motion are defined as
\begin{dgroup}[label=mbq-dyn]
\begin{dmath}
\mbqmass \mbqhacc = \mbqhforce\fn{\mbqhdisp,\mbqvdisp,\mbqrotz} - \mbqhdamp \mbqhvel
\end{dmath},
\begin{dmath}
\mbqmass \mbqvacc = -\mbqmass \gravity + \mbqvforce\fn{\mbqhdisp,\mbqvdisp,\mbqrotz} - \mbqvdamp \mbqvvel
\end{dmath},
\begin{dmath}
\mbqmomentofinertia \mbqaccrotz = \mbqptorque\fn{\mbqhdisp,\mbqvdisp,\mbqrotz} - \mbqzrotdamp \mbqvelrotz
\end{dmath},
\end{dgroup}
for which a time-domain solution was produced numerically with a Runge-Kutta technique (Matlab's \texttt{ode45} function). Viscous damping terms $\mbqhdamp$, $\mbqvdamp$, and $\mbqzrotdamp$ account for energy loss in the system. The force and torque terms are those defined in \eqref{mbq-force3,mbq-torque} respectively.

The parameters used in \Tabref{mbq-dyn-param} were used for the dynamic simulations. The equilibrium displacement $y_0$ is found by numerically inverting a static analysis of the magnet forces $\mbqvforce(0,\mbqvdisp_0,0)=\mbqmass g$; a damping ratio of \SI{20}{\%} is assumed to account for eddy current damping and any other energy losses; and the moment of inertia is approximated with $\mbqmomentofinertia=\tfrac13\mbqmass\mbqlever^2$. The parameters have been selected such that the vertical, horizontal, and rotational direct stiffnesses are all positive for this equilibrium displacement.

\begin{table}
\caption{Parameters used for the dynamic simulations.}
\tablabel{mbq-dyn-param}
\centering
\begin{tabular}{@{}l >{$}c<{$} c l >{$}c<{$} c@{}}
\toprule
\multicolumn{3}{c}{Explicit parameters} & \multicolumn{3}{c}{Implicit parameters} \\
\cmidrule(r){1-3}
\cmidrule(l){4-6}
 Mass             & \mbqmass & \SI{3}{kg}            & Equilibrium position & {\mbqvdisp}_0 & \SI{14.04}{mm} \\
 Damping ratio    & \zeta  & \num{0.2}               & Moment of inertia& \mbqmomentofinertia & \SI{1.60}{g/m^2} \\
 Magnetisation    & J_1, J_2  & \SI{1}{T}            & Horizontal stiffness & \mbqhstiff & \SI{15.43}{N/m} \\
 Unit length      & \mbqunit & \SI{20}{mm}           & Vertical stiffness   & \mbqvstiff & \SI{170.5}{N/m} \\
 Magnet angle     & \mbqmagangle & \SI{45}{\degree}  & Rotational stiffness & \mbqzrotstiff & \SI{31.3}{mN.m/rad{.}} \\
 Offset ratio     & \mbqoffset/\mbqunit  & \num{0.4} & Horizontal damping & \mbqhdamp & \SI{9.05}{kg/s} \\
 Magnet ratio     & \mbqmagratio  & \num{0.4}        & Vertical damping   & \mbqvdamp & \SI{2.72}{kg/s} \\
 Lever ratio      & \mbqlever/\mbqunit  & \num{2}    & Rotational damping & \mbqzrotdamp & \SI{2.83}{mN.m.s/rad{.}} \\
\bottomrule
\end{tabular}\end{table}

\begin{figure}
\begin{wide}
\subfloat[Horizontal and vertical displacement from the equilibrium position.]
  {\pregen{\psfragfig{magcode/examples/oblique/fig/mbq-dyn-xy}}}\qquad
\subfloat[Displacement map with colour progressing with time from dark to light.]
  {\pregen{\psfragfig{magcode/examples/oblique/fig/mbq-dyn-xy-map}}}
\end{wide}
\lofcaption
{Dynamic simulation of the rotationally-constrained system.}
{Dynamic simulation of the system defined by \eqref{mbq-dyn} with perturbation of $\Delta x=\Delta y=\SI{1.5}{mm}$ and constraint on rotational $\mbqrotz$.}
\figlabel{mbq-dyn-xy}
\end{figure}

Assuming that the device is always designed to move freely in the vertical direction to accommodate changing load, there are three regimes in which stability could be assessed:
\begin{enumerate}
\item constraining rotation;
\item constraining horizontal displacement;
\item unconstrained.
\end{enumerate}
It is evident that the case of constraining both rotation and horizontal displacement will be stable provided the vertical stiffness is positive.
The first of the dynamic simulations presented is displacement in the $x$--$y$ plane with constrained rotation.
Given the system described in \Eqref{mbq-dyn} and a perturbation of $\Delta x=\Delta y=\SI{1.5}{mm}$, the resultant dynamics are shown in \Figref{mbq-dyn-xy} as displacements from the equilibrium position of the spring.
While this is close to the maximum perturbation for this system before instability, this example illustrates that there is a region around the equilibrium position within which stability is achieved.

\begin{figure}
\begin{wide}
\subfloat[Relative vertical displacement $\mbqvdisp-\mbqvdisp_0$.]
  {\pregen{\psfragfig{magcode/examples/oblique/fig/mbq-dyn-yr-y}}}\qquad
\subfloat[Rotation $\mbqrotz$.]
  {\pregen{\psfragfig{magcode/examples/oblique/fig/mbq-dyn-yr-r}}}
\end{wide}
\caption
[Dynamic simulation of the horizontally-constrained system.]
{Dynamic simulation with perturbation of $\Delta \mbqvdisp=\SI{1.5}{mm}$ and $\Delta \mbqrotz=\SI{3}{\degree}$ with constraint in horizontal displacement $\mbqhdir$.}
\figlabel{mbq-dyn-yr}
\end{figure}

The second stability example constrains horizontal displacement while allowing free rotation of the system. A perturbation of $\Delta y=\SI{1.5}{mm}$ and $\Delta \varphi=\SI{3}{\degree}$ is modelled with dynamic results shown in \Figref{mbq-dyn-yr}.
Again, with one constraint on the system there is a stable region around the equilibrium point.
In fact, this arrangement is more stable than the previous as there is less cross-coupling between the vertical and rotational degrees of freedom.

\begin{figure}
\begin{wide}
\subfloat[Horizontal displacement $\mbqhdisp$.]
  {\pregen{\psfragfig[scale=0.9]{magcode/examples/oblique/fig/mbq-dyn-xyr-x}}}\qquad
\subfloat[Relative vertical displacement $\mbqvdisp-\mbqvdisp_0$.]
  {\pregen{\psfragfig[scale=0.9]{magcode/examples/oblique/fig/mbq-dyn-xyr-y}}}\qquad
\subfloat[Rotation $\mbqrotz$.]
  {\pregen{\psfragfig[scale=0.9]{magcode/examples/oblique/fig/mbq-dyn-xyr-r}}}
\end{wide}
\caption
[Dynamic simulation of the unconstrained system.]
{Dynamic simulation without constraint and vertical perturbation only. Despite `stable' stiffnesses in each direction (seen in \figref{mbq-dyn-xy,mbq-dyn-yr}), the unconstrained system is unstable due to cross-axis coupling.}
\figlabel{mbq-dyn-xyr}
\end{figure}

Finally, it might now be expected that since stability was achieved in both $x$--$y$ and $y$--$\varphi$ regimes, an unconstrained system might be similarly stable.
Unfortunately this is not the case, as cross-coupling influences are too great and even an incremental perturbation eventually leads to instability as shown in \Figref{mbq-dyn-xyr}.
(There is a macroscopic perturbation of $\Delta\mbqvdisp=\SI{-1}{mm}$ and incremental perturbations of $\Delta\mbqhdisp=\SI{1e-9}{m}$ and $\Delta\mbqrotz=\SI{1e-9}{deg}$.)
Despite achieving positive direct stabilities in all three degrees of freedom, some form of control over this cross-coupling instability is required for stable operation; this could take the form of passive bearings or non-contact electromagnetic actuators.

\subsection{Conclusion}

In this section, a particular magnet geometry was investigated for the purposes of developing a spring for vibration isolation with the goal of a load-invariant natural frequency.
The resonance--load relationship was found to have significant flat areas, indicating this goal could be achieved for certain geometries.
The load-bearing capacity could be largely increased by scaling the volumes of the magnets; this was shown to have small effect on the natural frequency of the system.

Since the system uses magnetic levitation to achieve its force characteristic, there are various instabilities inherent in its dynamics.
Some of these instabilities are due to coupling between horizontal and rotational degrees of freedom have been highlighted, but a complete six degree of freedom analysis must await future developments in magnetic torque modelling.
A physical realisation of this system has been constructed, which is discussed in \secref{mbq-prototype}.


\section{Prototype inclined magnet system}
\seclabel{mbq-prototype}

To explore the possibilities raised by the theoretical analysis of the inclined magnetic spring (\secref*{oblique}), an honours project supervised by the author was undertaken in 2011 to design and build a prototype based on these ideas \cite{frizenschaf2011-honoursthesis,frizenschaf2011-acoustics2011}.
\note{The members of the project were Mr Yann Frizenschaf, Ms Siobhan Giles, Mr Jack Miller, Mr Thomas Pitman, and Mr Christopher Stapleton.}

The prototype they built used Maytec \note{Maytec Australia Pty Ltd, \url{http://www.maytec.com.au/}} aluminium extrusions to allow modular placement of the inclined magnets with variable magnet angle.
A photo of the prototype is shown in \figref{oblique-rig}, which shows the main features of the system.
Six identical magnet pairs were used; four on the long side of the device and two on the short side.
The rare earth magnets of volume \SI{25x25x12.5}{mm^3} were chosen to generate enough repulsive force to bear a weight of \SI{10}{kg} with a vertical displacement of around \SI{20}{mm} depending on the magnet angle.
Four electromagnetic actuators were placed at the corners of the device for active vibration isolation.
For their project, passive levitation control was achieved using a linear bearing to constrain the device to vertical motion, coupled with a rotational bearing to permit rotation around the long horizontal axis.
The device was quite stable in this configuration, but when analysed for active stabilisation a strong instability around the $y$-axis (yaw rotation) was found.

\begin{figure}
\begin{wide}
\includegraphics{PhD/Figures/Oblique/oblique-rig}
\end{wide}
\caption
[Prototype inclined magnet isolation device.]
{Prototype inclined magnet isolation device constrained with a linear bearing to translate only in the $y$-direction and rotate solely around the $x$-axis.}
\figlabel{oblique-rig}
\end{figure}

The six degree of freedom system was modelled using an extension of the theory developed in \secref{oblique-planar-theory} for the planar three \dof/ system.
The model predicted the behaviour of the prototype well; \tabref{oblique-rig-resonances} shows the measured and modelled resonance frequency for two magnet angles, which vary by around 5\%.
The resonance frequencies as a function of load mass are shown in \figref{oblique-resonance}.
Due to limitations of the resonance frequency measurements, the measured data is quite noisy but the correspondence with the expected modelled results is can still be seen.

\begin{table}
\caption{Modelled and measured vibration results for the prototype isolator.}
\tablabel{oblique-rig-resonances}
\begin{tabular}{@{}lll@{}}
\toprule
Magnet angle & \SI{45}{\degree} & \SI{60}{\degree} \\
Resonance frequency, analytical & \SI{4.9}{Hz} & \SI{4.0}{Hz} \\
Resonance frequency, measured & \SI{5.1}{Hz} & \SI{3.8}{Hz} \\
Measured damping ratio & \SI{4.4}{\%} & \SI{9.0}{\%} \\
\bottomrule
\end{tabular}
\end{table}

\begin{figure}
\centering
\includegraphics{PhD/Figures/Oblique/resonance}
\caption{Measured and modelled resonance frequencies of the inclined magnet isolation prototype as a function of load mass.}
\figlabel{oblique-resonance}
\end{figure}

It is interesting to compare the significant difference in damping ratio for these two magnet angles.
The strongest contributor to the damping ratio in this system is from induced eddy currents in each permanent magnet by the magnetic field of the other (\secref*{damping}), as viscous damping in the bearings and due to air resistance can be assumed to be negligible.
There are two main factors that would cause changes in the magnitude of the induced eddy currents.
For a varying magnet angle, the vertical displacement between the magnet pairs would be affected.
Secondly, as the angle between the direction of displacement (vertical, in this case) and the direction of magnetisation increases, the gradient of magnetisation decreases over a certain displacement.
The effect on the damping ratio with a variety of load masses is shown in \figref{oblique-damping}, where it can be seen that as the magnets become closer together, the magnitude of the induced eddy currents are increased and so does the damping ratio.
These values were taken in a different configuration than the data shown in \tabref{oblique-rig-resonances}, and the damping ratios are not comparable between these two cases.

\begin{figure}
\centering
\includegraphics{PhD/Figures/Oblique/damping}
\caption{Measured damping ratios of the inclined magnet isolation prototype.}
\figlabel{oblique-damping}
\end{figure}

The theoretical model was also used in a dynamic simulation which predicated the transmissibility of the isolator against ground disturbance.
The measured and modelled transmissibility is shown in \figref{oblique-frf}, which shows good agreement at low frequencies around the resonance peak.
The measured results do not roll off at high frequencies, however, and this is an important point: due to the inherent friction in the linear bearing at high frequencies, the performance of the isolator is compromised in the passive design.
For a high-performance magnetically levitated isolation platform, these experimental results indicate that active control must be used to achieve good high-frequency performance.

\begin{figure}
\centering
\includegraphics{PhD/Figures/Oblique/frf}
\caption
[Measured and modelled transmissibility of the inclined magnet isolation prototype.]
{Measured and modelled transmissibility of the inclined magnet isolation prototype with a magnet angle of \SI{60}{\degree} and a magnet offset ratio of \num{0.5}. Behaviour from \SI{15}{Hz} is due to bearing friction.}
\figlabel{oblique-frf}
\end{figure}

The sharp peak in \figref{oblique-frf} is due to the low damping inherent in magnetically levitated systems.
This peak can be reduced with the active vibration control using skyhook damping (\secref*{skyhook-intro}).
Measured transmissibility results with and without active vibration control are shown in \figref{oblique-skyhook}.
The skyhook damping method achieves a clear reduction in the resonance peak; around \SI{9}{dB} in this case.
Behaviour at higher frequencies is unaffected, as is expected with this method of vibration control (compare \figref{skyhook-ideal}).

\begin{figure}
\centering
\includegraphics{PhD/Figures/Oblique/skyhook}
\caption{Measured transmissibility of the inclined magnet isolation prototype with and without active vibration control.}
\figlabel{oblique-skyhook}
\end{figure}

\subsection{Limitations of the prototype}

This project was unable to achieve active stabilisation of the inclined magnet prototype; a strong instability was found in yaw rotation.
No attempt in the prototype design was made to investigate variation in horizontal stability, but it is unlikely that the yaw rotation would be stabilised by removing the short-side magnet pairs.
Indeed, due to the consequences of Earnshaw's theorem it is expected that there will be significant instability in at least one rotational direction unless specific design effort is expended to attempt stability in all rotational degrees of freedom (\secref*{rotation-freedom}), and even if so this attempt could have negatively affect the performance of the translational behaviour of the design.

Due to the lightly damped nature of the system and the coupling between different axes for each magnet pair, a significant degree of cross-coupling was observed in the overall vibration behaviour of the system, requiring careful adjustment to excite the structure in the vertical direction only.
Active control on each corner compounded this problem, as imperfections in the coil manufacture led to variations in coil impedance and therefore differences in force--displacement behaviour for each coil.
As a result, control signal were difficult to tune such that their resultant forces on the system were in a single \dof/ only.
When not perfectly tuned, these unbalanced coils also had the capability to induce cross-coupling motion (generally rotations around the horizontal axes) unless carefully calibrated and monitored.

\section{Summary}

A number of designs for load bearing using individual magnets in a variety of configurations were introduced and discussed in this chapter.
It was shown that by carefully introducing addition magnets in a design the degree of instability can be minimised.
A particular focus was placed on the inclined magnet spring and this design was experimentally tested; its results tracked well with theoretical calculations and its utility for vibration isolation was explored.


\end{document}
