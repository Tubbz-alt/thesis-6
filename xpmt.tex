\documentclass[11pt,a4paper]{memoir}
\def\asydir{\jobname}
\usepackage{thesis-preamble}
\EndPreamble
\begin{document}

\chapter{Prototype low-stiffness magnetic spring}
\chaplabel{xpmt}

\epigraph{While they uncovered the sheaves he stood apathetic beside his
portable repository of force, round whose hot blackness the morning
air quivered.  He had nothing to do with preparatory labour.  His
fire was waiting incandescent, his steam was at high pressure, in
a few seconds he could make the long strap move at an invisible
velocity.  Beyond its extent the environment might be corn, straw,
or chaos; it was all the same to him.  If any of the autochthonous
idlers asked him what he called himself, he replied shortly, `an
engineer.'}{\citetitle{hardy1981-tess}\\\textcite{hardy1981-tess}}

\noindent
This chapter covers the implementation details, experiments performed, and results obtained from the experimental apparatus built for this project used to validate some of the concepts and theory discussed in previous chapters.

\section{Design of the experimental apparatus}

\subsection{Overview of the experimental rig}

The experimental apparatus that was designed and built as part of this project
is shown as a schematic in \figref{rig} and as a photograph in
\figref{rig-photo}. Physical parameters of the design are shown in
\tabref{rigprop}. The experimental rig was built to serve as a platform for
investigating the dynamics of a simple system supported by magnets with
varying stiffnesses; specifically, for demonstrating a specific case of
the ideas behind the \qzs/ arrangement analysed in detail in \secref{qzs}.

\begin{figure}
  \begin{wide}
    \asyinclude{\jobname/rig-schematic}
  \end{wide}
  \caption[Schematic of the experimental rig.]{
    Schematic of the experimental rig (not to scale). Position shown is
    the marginally stable configuration with equal gap between the lower
    and upper pairs of magnets.}
  \figlabel{rig}
\end{figure}

\begin{figure}
  \includegraphics{PhD/Experiments/Rig/photo-labelled.pdf}
  \lofcaption{Photo of the experimental apparatus.}{ Base shaker is not shown.}
  \figlabel{rig-photo}
\end{figure}


\begin{table}
\caption{Physical properties of the experimental rig.}
\tablabel{rigprop}
\begin{minipage}{\textwidth}
\def\footnoterule{}
\begin{tabularx}{\textwidth}{@{}Xll@{}}
\toprule
                Rig height  & $\HeightRig$ & \SI{209}{mm} \\
                  Beam mass &         $\MassBeam$ & \SI{266}{g}\thinspace
\footnote{Mass of the accelerometer is accounted for in this value.}  \\
                Beam length &       $\LengthBeam$ & \SI{320}{mm} \\
                Beam height &       $\HeightBeam$ & \SI{ 25}{mm} \\
                 Beam width &        $\WidthBeam$ & \SI{ 40}{mm} \\
             Beam thickness &    $\ThicknessBeam$ & \SI{  2}{mm} \\
       Beam vertical offset &       $\OriginBeam$ & \SI{ 82}{mm} \\
\midrule
      Magnet support height & $\HeightRigMagnets$ & \SI{105  }{mm} \\
   Magnet support lever arm & $\LengthRigMagnets$ & \SI{300  }{mm} \\
        Magnet support mass &   $\MassRigMagnets$ & \SI{ 87  }{g}  \\
             Magnets height &        $\HeightMag$ & \SI{  9.5}{mm} \\
           Magnets diameter &          $\DiamMag$ & \SI{ 12.7}{mm} \\
  Lower fixed magnet origin &     $\OriginMagLow$ & \SI{ 44  }{mm} \\
  Upper fixed magnet origin &    $\OriginMagHigh$ & \SI{ 30  }{mm} plus offset\thinspace
  \footnote{Offset varied to adjust the amount of added negative stiffness.} \\
\midrule
  Sensor height & $\HeightSensor$ & \SI{85}{mm} \\
  Sensor horizontal offset & $\LengthSensor$ & \SI{252}{mm} \\
  Sensor displacement measurement & $\DisplSensor$ & \SI{43}{mm}--\SI{53}{mm} \\
\bottomrule
\end{tabularx}
\end{minipage}
\end{table}

The rig therefore consists of an arrangement of magnets, of which two are
fixed to the frame; `floating' magnets are supported by these of which one
is situated below to apply a repulsive force (positive stiffness) and the
other is situated above to apply an attractive force (negative stiffness). The
physical location of the fixed magnets may be moved vertically in order to
vary the respective amounts of positive and negative stiffness.

The magnet system is designed to investigate the dynamics in the vertical displacement
direction; having stability in this direction implies instability in the
horizontal direction (as discussed in \secref{earnshaw}). In order to remain
stable, the magnets require a physical constraint, achieved by placing the
floating magnets at the end of a long pinned rigid beam. Small rotations of
this beam can be assumed to correspond to largely vertical displacements of
the end magnets. This is ensured in the design, which is addressed in
\secref{rotating-beam}.

The beam itself was chosen as a hollow rectangular section in order to
minimise weight and maximise stiffness; it is assumed to be a rigid body
for the purposes of these experiments (especially at the low vibrational
frequencies under investigation).

The pin support was constructed by clamping the beam to a thin piece of flexible plastic which was clamped to the frame.
The use of a flexural element was chosen to avoid stiction that would be present in a bearing or hinge joint.
The stiffness of this plastic can be assumed to be negligible as it played no part in the load bearing of the beam.
Having a physical constraint on the motion of the floating magnets compromises
the non-contact idea of using magnets for vibration isolation. This
experimental rig has been designed to analyse the behaviour of the system in
terms of vertical translational disturbances. For a completely non-contact
system, a control system with electromagnets may be used to apply the
horizontal constraint (also see \secref{coupling}).

\subsection{Magnets}

The magnets used for the experimental apparatus were chosen based on ease of
availability and construction. Cylindrical magnets were used of dimensions
$\mbox{\diameter}\,\SI{12.7}{mm}\times\SI{9.5}{mm}$ and material properties as
shown in \tabref{magnet-prop}.
The magnets were held in place within a hollow brass cylinder of outer diameter \SI{13.5}{mm}.

\begin{table}
  \lofcaption{Material properties for the magnets in the experimental apparatus.}
    { Magnet properties taken as minimum grade values from \KJMagnetics/.}
  \tablabel{magnet-prop}
  \begin{tabular}{@{}ll@{}}
    \toprule
    Diameter & \SI{12.7}{mm} \\
    Height   & \SI{9.5}{mm}  \\
    Grade    & \acro{N42} \\
    Remanence & \SI{1.3}{T} \\
    Intrinsic coercivity & \SI{960}{kA/m} \\
    Maximum energy product & \SI{325}{kJ/m^3} \\
    \bottomrule
  \end{tabular}
\end{table}

Cylindrical magnets were chosen for their ease of integration into the
construction of the rig; \eg, it is much easier to fix a cylindrical magnet
into an object by boring a round hole as opposed to milling a square one.

\subsection{Actuators}
\seclabel{actuators}

To highlight the dynamics of the magnetic support as much as possible, any
additional actuators for applying control forces should affect the dynamics of
the system as little as possible. Fitting with the theme of non-contact
support, a non-contact actuator should be chosen if possible. An
electromagnetic force was the obvious choice to satisfy these requirements.

Electromagnetic actuators are usually designed in push-pull configurations,
with fixed non-biased coils applying force on a moving ferrous material. Since
only attractive forces can be generated in this configuration, such
electromagnets must be used in pairs to achieve forces in two directions.

An alternative is to use biased electromagnets, such that either the coil
itself contains a permanent magnet, or the coil acts against a permanent
magnet, or both. For a biased coil, instability results due to the presence of
the negative stiffness attractive forces between the permanent magnet and the
ferrous moving material. In these cases, pairs of electromagnets are no longer
strictly necessary since the current in the coil may be reversed and a
repulsive force generated.

For example, \textcite{nandi2009} demonstrate a single-sided design in which a
bias current is used to keep the structure in permanently `sprung' position;
relaxing the current is the equivalent of a repulsive force, while increasing
the current is the usual `pull' or attractive force of a regular
electromagnetic actuator. This design is relatively inefficient since it
requires a continuous energy expenditure to remain in equilibrium.

Using a non-ferrous coil to apply forces to a moving permanent magnet,
however, is passively stable and requires no steady current for operation.
However, the forces applied by a coil to an external magnet are relatively
small; larger forces are achieved when the magnet can be inserted within the
coil itself. This suits a secondary requirement for the electromagnetic
actuator to keep it inline with the load-bearing magnets, and is suitable for a
future application of such an arrangement for an apparatus built to support
multiple degrees of freedom. The ferrous-free coil and moving permanent magnet
approach was adopted for the choice of electromagnetic actuator for the rig,
as has already been shown in the schematic \figref[vref]{rig}.

The dual-coil electromagnetic arrangement described in \secref{dualcoil} was
custom-built for the actuator for the experimental apparatus. A schematic of
the dual-coil system is shown in \figref{rig-dual-coil} with parameters in
\tabref{coil-prop}. The coil was designed to have an impedance of
\SI{8}{\ohm}, from which the outer radius of the coil was calculated given a
certain wire diameter and resistance.

\begin{figure}
  \begin{wide}
  \begin{sidefigure}
  \asyinclude{\jobname/rig-coil}
  \lofcaption{Schematic of the dual-coil electromagnet built for the experimental apparatus.}{ (Not to scale.)}
  \figlabel{rig-dual-coil}
  \end{sidefigure}\hfil
  \begin{sidefigure}
  \psfragfig{\phdpath Simulations/Coil/fig/2coils-normdesign}
  \lofcaption{Normalised force \vs\  displacement curve for the dual-coil electromagnet.}
  { Force is normalised to the maximum value at zero displacement; the force imparted by the coil remains within 10\% of maximum over a displacement range of around \SI{\pm2}{mm}}
\end{sidefigure}
  \end{wide}
\end{figure}

\begin{table}
  \caption{Dual-coil electromagnet parameters.}
  \tablabel{coil-prop}
  \begin{tabular}{@{}lll@{}}
    \toprule
    Magnet radius & $\RadiusMag$ & \SI{6.4}{mm} \\
    Magnet height & $\HeightMag$ & \SI{9.5}{mm}  \\
    \midrule
    Coil height & $\HeightCoil$ & \SI{7}{mm} \\
    Coil inner radius & $\RadiusICoil$ & \SI{10}{mm} \\
    Coil outer radius & $\RadiusOCoil$ & \SI{10.7}{mm} \\
    Coil gap & $\GapCoil$ & \SI{7}{mm} \\
    Turns (approx.) & & \num{105} \\
    \midrule
    Former inner radius & & \SI{14.2}{mm} \\
    \bottomrule
  \end{tabular}
\end{table}

The dimensions of the coil were chosen based on the trends shown in the
analytical calculations from \secref{dualcoil} to ensure sufficient force over the displacement range expected from the system.
A force \vs\  displacement curve of this particuar coil is shown in \figref{2coils-normdesign}; from this
graph, it can be seen that around the centred position (displacement of zero
in the curve), the force imparted by the coil remains within 10\% of maximum
over a displacement range of around \SI{\pm2}{mm}, which is acceptable for the purposes of the design.

\subsection{Sensors}
\seclabel{xpmt-sensors}

The requirements for the sensing equipment is similar to that of the actuators; noncontact spatial measurement is desired to remove any possibility of adding further dynamics to the system.
A displacement sensor is required to measure the position of the fixed and moving magnets in the system.
Velocity sensors can be used for feedback control to add damping to the system (see \secref[vref]{skyhook}) in order to aid the vibration isolation characteristic of the system.
Accelerometers are used to measure the vibration response on the beam and on the base and thus determine the transmissibility of the system.

\subsubsection{Displacement sensors}

There are five main choices for non-contact distance sensing,
enumerated below. See \textcite{boehm1993} for a more detailed overview.
\begin{description}

\item[Ultrasonic]

    These sensors work by sending out modulated
    ultra-sonic pulses to be reflected off the target. The time spent
    in the round trip gives a linear indicator of the
    distance. Of the sensors described here, ultrasonic is the only one
    that is unsuitable due to low accuracy and frequency response.

\item[Inductive]

    An inductive, or Hall effect, sensor works by exciting a coil with a high
    frequency sinusoidal current which induces eddy currents in the target.
%    \note{An example of a commercial inductive sensor is \url{http://www.microstrain.com/ncdvrt.aspx}}
    These eddy currents may be measured very accurately, but the whole effect
    is very dependent on a lack of magnetic noise. This makes these type of
    sensors difficult to use in magnetic applications.

\item[Capacitive]

    A capacitive sensor measures the capacitance between a plate and the
    target.
%    \note{An example of a commerical capacitive sensor is \url{http://www.micro-epsilon.com/products/displacement-position-sensors/capacitive-sensor/capaNCDT_6019/index.html}.}
    To measure large distances, however, a large capacitive head is
    required (approx.\ $1$\,cm diameter for every $1$\,mm of range
    \cite{boehm1993}).

\item[Linear Variable Differential Transformer]

    `\acro{lvdt}' sensors use a moving probe attached to an electrical
    circuit that exhibits a measureable effect which varies with displacement
    of the probe.
%    \note{An example of a commercial \acro{lvdt} sensor is \url{http://www.microstrain.com/dvrt.aspx}.}
    Differential Variable Reluctance Transducers
    (`\textsc{dvrt}') are a variation on this idea. The probe must either be
    fixed to the measured object or be pressed against it with a positive
    spring force. Non-contact operation is only possible for linear
    displacements in a single translational direction.

\item[Laser]

    A laser sensor uses interferometry of a reflected laser signal to
    calculate the absolute displacement of the target from the sensor.
%    \note{An example of a commerical laser displacement sensor is \url{http://tinyurl.com/wenglor}.}
    These sensors have the advantage of being unaffected by electrical or
    magnetic noise and robust working environment against any surface (albeit
    with a suitable finish).

\end{description}
The sensor purchased for the experimental apparatus was a Wenglor 05\,MGV\,80
opto-electronic sensor, which uses a laser to measure distance over a
range of \SI{10}{mm}. It was selected for its convenient availability,
flexibility of application, and good performance characteristics.
Relevant operating properties are listed in \tabref{wenglor}.

\begin{table}
  \caption{Relevant properties of the Wenglor~05\,MGV\,80 laser distance sensor.}
  \tablabel{wenglor}
  \begin{tabular}{@{}ll@{}}
    \toprule
    Measurement offset & \SI{43}{mm} \\
    Measurement range & \SI{10}{mm} \\
    \midrule
    Working range & \SI{43}{mm}--\SI{53}{mm} \\
    Output range & \SI{10}{V}--\SI{0}{V} \\
    \midrule
    Resolution & $<\SI{10}{\micro m}$ \\
    Response time & \SI{0.5}{ms} \\
    \bottomrule
  \end{tabular}
\end{table}

\subsubsection{Velocity sensors}

As velocity is defined in terms of a change in displacement from some reference point, it is only meaningful to refer to the velocity of an object with respect to an inertial reference frame.
With a sensor attached to another body, separate from the object, relative velocity can be measured between the disturbance and the resulting excitation ($\velMass-\velBase$ in terms of \figref{simple-isolation}).
Laser sensors
    \note{Also known as vibrometers, such as sold by Polytec GmbH: \url{http://www.polytec.com/}}
and capacitive sensors \cite{nijsse2001} may be used to measure relative velocity directly, while relative velocity may be estimated by differentiating the output of a displacement sensor.
Such estimation is prone to high frequency noise that must be filtered away.

Absolute velocity sensors must use more complex physical processes to obtain a measure of an object with respect to the `fixed' earth.
Absolute angular velocity can be infered by measuring the coriolis force induced on a vibrating element due to the rotational energy of the earth \cite{konno1996} and is the basis of solid state gyroscope sensors.
A geophone uses a moving magnet or moving coil arrangement to generate \backemf/ from the motion of an inertial mass \cite{oome2009-saa} from which absolute velocity can be calculated.
This process is limited in the low frequency by the resonance frequency of the inertial mass and in the high frequency by the roll-off of the frequency response.
Absolute translational velocity can be estimated by integrating a measured acceleration; this technique (which is used in the experimental apparatus) is discussed in the next section.

\subsubsection{Acceleration sensors}

For the beam, a \BnK/ 4367 accelerometer was used to measure the `output' signal; for the base, a 4332 accelerometer measured the `input' signal of the system (\tabref{rig-apparatus}).
The accelerometers were used with \BnK/ 2635 charge amplifiers set to appropriate gain values for the input signals.
The signals were low-pass filtered at \SI{50}{Hz} to avoid aliasing effects using a Krohn-Hite Model 3362 digital filter (using a 4-pole Butterworth filter).

\begin{table}
  \caption{Relevant properties of the accelerometers used in the experimentation.}
  \tablabel{rig-apparatus}
  \begin{tabular}{@{}lllll@{}}
    \toprule
           & Model & Charge sensitivity & Weight \\
    \cmidrule{2-4}
      Base & 4367 & \SI{20.1}{pC/g} & \SI{13}{g} \\
      Beam & 4332 & \SI{75.4}{pC/g} & \SI{30}{g} \\
    \bottomrule
  \end{tabular}
\end{table}

For open loop measurements, these sensors were used to measure acceleration directly; for closed loop control, the charge amplifier was used to integrate the measured signals to estimate the velocities.
Standard methods for interpreting the accelerometer involve high-pass filtering the measured signal to avoid drift as integration errors accumulate \cite{brennan2007-jsv}.
The corner frequency for the integration must be reasonably high, as it is not possible to distinguish between drift and low frequency components in the  signal; in this case the charge amplifiers used a \SI{1}{Hz} cut-on frequency.
An alternative is the use of `drift-free' integrators \cite{gavin1998}, in which the drift is compensated for using alternative signal processing methods, but this approach is only advantageous for measuring low-frequency periodic signals; for transient or wide-band signals, the typical approach is better suited.

Low frequency velocity signals can also be estimated by differentiating a displacement measurement.
If the displacement sensor is fixed to a fixed inertial reference frame, then this estimate may be combined with an integrated accelerometer measurement to obtain a more accurate estimate of the velocity \cite{bennett2007}.
However, if relative displacement is measured between two coupled systems (as often the case for vibration isolation) then this technique does not capture the absolute velocity. (\Eg, with reference to \figref{simple-isolation}, the signal $\gp{\velMass-\velBase}$ is being estimated, not $\velMass$.)

Recently, \textcite{williams2009} showed a filter system designed to estimate absolute velocity from accelerometer measurements that can mitigate the the low-frequency resonance issues seen due the high-pass filters used in the charge amplifiers used (seen in the experimental results in \secref*{acc-resonance}).
Unfortunately, this research was published after experimental measurements for this thesis had been completed and their ideas were not able to be tested in the apparatus discussed in this section.


\subsection{Translational effects of the rotating beam}
\seclabel{rotating-beam}

Three moving magnets were required in the system: one at the main end of the
motional beam to be repelled for positive stiffness by the fixed lower magnet;
a second to be used with electromagnetic coil for control forces; and the
third to be attracted for negative stiffness by the fixed upper magnet.

The beam added a horizontal constraint to the system for stability. As the
beam rotates, the magnets move predominantly in the vertical direction; there
is still some horizontal motion, however, and the area restricted by the
electromagnetic coil requires attention to ensure that there is no contact
between the moving magnet and the fixed coil.
However, the smaller the air gap between the coil and the magnet, and thus the smaller the inner
radius of the coil, the greater the forces imparted by the coil on the magnet
(see \figref[vref]{magcoil-ecc-vary}), so the smaller the tolerance the better.

A simplified geometry of the moving magnet arrangement is used to calculate
the minimum tolerance required to avoid contact with the coil as a function
of the vertical motion of the magnets. This geometry is shown in
\figref{horiz-tolerance}, where the L-shaped magnet support is shown
in the `rest' position $\overline{OAB}$ and in a rotated position $\overline{OA'B'}$.

\begin{figure}
  \asyinclude{PhD/Figures/Rig/horiz-tolerance}
  \caption
  [Geometry for calculating the minimum horizontal tolerance of the inner
           dimensions of the coil due to vertical displacement of the magnet arrangement.]
  {Geometry for calculating the minimum horizontal tolerance $x$ of the inner
           dimensions of the coil due to vertical displacement $z$ of the magnet arrangement pinned at point $O$.}
  \figlabel{horiz-tolerance}
\end{figure}

The horizontal displacement of the highest point after rotation, $x$, can be
calculated as a function of rotation angle $\theta$
\begin{dgroup}
\begin{dmath}
  x = h \Sin{\theta} + f
\end{dmath},
\begin{dmath}
  f = l - l \Cos{\theta}
\end{dmath},
\end{dgroup}
where $l$ is the horizontal length of the magnet arrangement, and $h$ is the
vertical height. This can be expressed in terms of the vertical displacement
of the main beam, $z$, with $\Sin{\theta}=z/l$ and using the trigonometric
relationship $\Cos{\ArcSin{a}}=\sqrt{1-a^2}$:
\begin{dmath}
  x = h z / l + l - \sqrt{l^2-z^2}
\end{dmath}.

From the dimensions chosen for the rig, $l=\SI{300}{mm}$ and
$h\approx\SI{50}{mm}$, the minimum horizontal tolerance for a vertical
displacement $z=\SI{1}{mm}$ is $x=\SI{0.168}{mm}$ and for $z=\SI{2}{mm}$ is
$x=\SI{0.34}{mm}$. (The relationship between $x$ and $z$ being fairly linear for
most values of $l$ and $h$ with small ranges of $z$.)
The actual clearance between the outer radius of the brass cylinder holding the magnets and the inner radius of the coil former was \SI{0.35}{mm}.
This tolerance was judged to be small enough to allow a surrounding coil without having a
significantly diminished force characteristic from the air gap required to
avoid contact.
It should be noted that this tolerance caused a degree of inconvenience since the attachment and positioning of the coil required careful alignment in order to allow free movement of the cylinder holding the magnets.

\section{Experimental results}

A number of measurements were performed using the experimental apparatus;
in the sections following, data recorded is presented for
\begin{enumerate}
\item Magnet gap versus beam displacement;
\item Open loop frequency responses for a range of magnet gaps; and,
\item Velocity feedback in a single configuration.
\end{enumerate}

\subsection{Static displacement measurements}
\seclabel{xpmt-displ}

The upper magnet's position was varied until the limit of stability was
reached. The lower fixed magnet was kept fixed, which means that the position
of \qzs/ was changing; with counter-threaded mounts for the upper and lower
magnets, they could be adjusted in parallel to achieve a fixed \qzs/ location.

As the upper magnet placement was lowered, the rest position of the beam moved
closer to the \qzs/ position (as more force was supported by the upper magnet).
This relationship is shown in \figref{turns-qzs}.
The normalised magnet gap $\rigNormGap$ is used in the following sections to
represent the varied configuration of the spring in the experiments, defined as
$\rigNormGap=\OriginMagGap/\HeightMag$, where
$\OriginMagGap$ is the magnet gap at \qzs/ and $\HeightMag$ is the height of the magnets.
From the geometry of the rig,
the position of the upper magnet was used to calculate the
gap between the magnets at \qzs/:
\begin{dmath}
  \OriginMagGap\fn{\OriginMagHigh} =
    \half\gp{\HeightRig-\OriginMagLow-\OriginMagHigh-\HeightRigMagnets}+\HeightMag-\HeightBuffer
\end{dmath},
where geometrical properties are described in \tabref{rigprop} and
$\HeightBuffer=\SI{2.5}{mm}$ is a little extra height to account for extra
space taken up by the thicknesses of the magnet mounting.
The height of the \qzs/ location itself, $\OriginQZS$, is given by
\begin{dmath}
  \OriginQZS\fn{\OriginMagHigh} = \half\gp{\HeightRig-\OriginMagLow+\OriginMagHigh}
\end{dmath}.

\begin{figure}
  \asyinclude{\jobname/spring-zk}
  \hfil
  \psfragfig{PhD/Experiments/XPMT/latex/turns-qzs}
  \lofcaption{Rest position of the system as the magnet position varies.}{
    For simplicity, only a single floating magnet is shown here.}
  \figlabel{turns-qzs}
\end{figure}

The measured output of the sensor $\DisplSensor$ was used infer a magnet displacement, $\DispMag$, (referenced from the \qzs/ position) with the following relationships.
Firstly, the rotational origin of the beam was used as a vertical reference point, and the displacement of the beam $\DisplSensorBeam$ at the laser sensor location calculated as
\begin{dmath}
  \DisplSensorBeam\fn{\DisplSensor} = \HeightRig - \OriginBeam - \HeightSensor - \DisplSensor
\end{dmath},
which can be extrapolated using the respective lever arms to calculate the resultant vertical displacement of the moving magnets with respect to the beam origin with
\begin{dmath}
  \DisplSensorBeamProj\fn{\DisplSensorBeam} = \DisplSensorBeam \frac \LengthRigMagnets \LengthSensor
\end{dmath}.
This moving magnet displacement can be written with respect to the origin of the frame of the rig with
\begin{dmath}
  \DisplMag\fn{\DisplSensorBeamProj} = \OriginBeam + \DisplSensorBeamProj - \HeightBeam + \ThicknessBeam + \half \HeightRigMagnets
\end{dmath}.
Accordingly, the displacement of the system $\DispMag$ from the \qzs/ location is given by
\begin{dmath}
  \DispMag\fn{\DisplSensor,\OriginMagHigh} = \DisplMag - \OriginQZS
\end{dmath}.

The displacements between the centres of each magnet in the interacting pairs can be calculated similarly based on the displacement of the moving magnet assembly $\DisplMag$.
With respect to the frame origin, the magnet centres for the base magnet $m_1$, upper magnet $m_2$, lower moving magnet $m_3$, and upper moving magnet $m_4$ are 
\begin{align}
m_1 &= \OriginMagLow - \HeightBuffer - \half\HeightMag  \,, \\% base magnet
m_2 &= \HeightRig - \gp{\OriginMagHigh - \HeightBuffer - \half\HeightMag}  \,, \\% upper magnet
m_3 &= \DisplMag - \half\HeightRigMagnets + \half\HeightMag  \,, \\% B moving magnet
m_4 &= \DisplMag + \half\HeightRigMagnets - \half\HeightMag  \,. % U moving magnet
\end{align}

\subsection{Predicted resonance frequencies}

From the displacement results shown in \secref{xpmt-displ}, predicted resonance frequencies can be calculated for this system as a function of magnet gap.
The expected magnetic forces $F$ due to the measured displacements were be calculated using the theory for coaxial cylindrical magnets (\secref[vref]{cyl-forces}) based on magnet centre displacements $m_3-m_1$ and $m_2-m_4$.
Numerical differentiation was used to calculate the stiffnesses $k$ at these displacements, and the natural frequency at each location calculated with $\natfreq=\sqrt{\stiffness/m_{eq}}$ where $m_{eq}=F/g$ is the equivalent mass borne by the static force $F$.
With parameters as specified, the expected natural frequency versus magnet gap results are shown in \figref{xpmt-magres}.

\begin{figure}
  \psfragfig{PhD/Experiments/XPMT/fig/xpmt-magres}
  \lofcaption{Expected natural frequency as the magnet position varies.}{}
  \figlabel{xpmt-magres}
\end{figure}

\subsection{Open loop dynamic measurements}
\seclabel{xpmt-ol}

As the position of the upper magnet is varied, the amount of negative
stiffness added to the system changes. This predominantly affects the rest
position and the resonance frequency, along with small changes in
damping. Frequency response measurements were taken at a number of discrete
locations of the upper fixed magnet to observe the changes in dynamics as the
rest position of the system approached the \qzs/ position.

The parameters used to perform the spectral analysis for each measurement are
shown in \tabref{rig-ol-spect}. Due to the low damping and low resonance
frequency of the system, very long sample times were required to achieve
results with enough frequency resolution and sufficient coherence to
characterise the response. A high sample rate (\SI{1000}{Hz}) was chosen to
reduce the possibility of controller time delays influencing the feedback
control, as discussed in \secref{vibes-feedback}.

\begin{table}
  \caption{Parameters used in the signal and spectrum analysis for the
   experimental measurements.}
  \tablabel{rig-ol-spect}
  \begin{tabular}{@{}ll@{}}
    \toprule
      Sample rate        & \SI{1000}{Hz}           \\
      \FFT\ points       & $2^{16}$                \\
      Sample time        & $\approx$\SI{17.5}{min} \\
      Average overlap    & $0.75$                  \\
      Number of non-overlapping averages & $16$    \\
    \bottomrule
  \end{tabular}
\end{table}

Open loop measurements were taken both with and without the electromagnetic
actuator connected (wired up as both a short circuit and an open circuit). In
the closed circuit configuration, the coil adds damping via induced eddy
currents from the moving magnet. As an open circuit, the coil has no effect on
the dynamics of the system.

Open loop measurements without the coil connected are shown in
\figref{ol-undamped-frflin}. and measurements taken with the coil (\ie, with
added damping) are shown in \figref{ol-damped-frflin}.
For both of these results, the transmissibility $\transmissibility{}$ shown is the transfer function between the accelerometer measurements of the base and magnet-supported beam:
\note{Calculated with \Matlab/'s \texttt{tfestimate} command.}
\begin{dmath}[label=Tbm,compact]
  \transmissibility{} = \frac{\crossSpectMagnetBase}{\powerSpectBase}
\end{dmath},
% TODO: bigger better definition
where $\powerSpectMagnet$ is the power spectral density of the accelerometer
measurements at the moving magnet, ($\powerSpectMagnet$ is the power spectral density
of the accelerometer measurements of the base,) and $\crossSpectMagnetBase$ is the cross power spectral density of the two.
The variance gain (also see \secref*{qzs-not-zerk}) of the measurements is analysed in \secref{xpmt-nonlin}.

\begin{figure}
  \begin{wide}
  \begin{subfigure}
  \centerline{\psfragfig{PhD/Experiments/XPMT/latex/ol-undamped-frflin}}
  \caption{Open circuit coil; no additional damping is added to the system.\figlabel{ol-undamped-frflin}}
  \end{subfigure}
  \hfil
  \begin{subfigure}
  \centerline{\psfragfig{PhD/Experiments/XPMT/latex/ol-damped-frflin}}
  \caption
  {Closed circuit coil.
           The coil adds damping to the system, which can
           be seen by the reduction in height of the resonant peaks in
           comparison to \figref{ol-undamped-frflin}.
  \figlabel{ol-damped-frflin}}
  \end{subfigure}
  \end{wide}
  \caption
  [Open loop measurements with the electromagnetic coil connected in an open and closed circuit.]
  {Open loop measurements with the electromagnetic coil connected in an open and closed circuit as a function of normalised gap~$\rigNormGap$.}
\end{figure}


\subsection{Analysis of the open loop data}

From the measurements shown in the previous section, data fitting of the frequency response functions was used to extract a linear model of the system in each configuration.
While more sophisticated techniques are possible \cite{chen2009}, fitting the data to a known exact frequency response function yielded acceptable results in this case since the linear model is quite simple.

% doublecheck how I want to say this
The model used to fit the data was a single \dof/ vibration isolation system
given by \eqref{simple-isolation-freq} in terms of the natural frequency, $\natfreq =
\sqrt{\stiffness/\mass}$, and damping ratio $\dampingratio = \damping /
(2\sqrt{\stiffness\mass})$. The resulting expression is
\begin{dmath}[label=fit-model]
  \transmissibility = \frac
    {2\ii\dampingratio\freq\natfreq + \natfreq^2}
    {-\freq^2 + 2\ii\dampingratio\freq\natfreq + \natfreq^2}
\end{dmath},
where $\freq$ is the frequency at which to calculate the transmissibility
$\transmissibility$. The data was fit
    \note{Using \Matlab/'s \code{fminsearch}.}% check
to \eqref{fit-model} between $\half\natfreq \le \freq
\le 2\natfreq$, returning good estimates of the natural frequency and
damping ratio of the system corresponding to each measurement taken. Plots of
the curve fit models shown against the measured data are shown in
\figref{ol-undamped-frflin-fit,ol-damped-frflin-fit}. Due to the influence of
higher-order dynamics in the system, the model starts to deviate from the data
at higher frequencies. The natural frequencies and damping ratios are shown
in \figref{ol-results}.

\begin{figure}[p]
  \psfragfig{PhD/Experiments/XPMT/latex/ol-undamped-frflin-fit}
  \lofcaption{Curve fit model of \figref{ol-undamped-frflin}.}{ Grey lines show the original data.}
  \figlabel{ol-undamped-frflin-fit}
\end{figure}

\begin{figure}[p]
  \psfragfig{PhD/Experiments/XPMT/latex/ol-damped-frflin-fit}
  \lofcaption{Curve fit model of \figref{ol-damped-frflin}.}{ Grey lines show the original data.}
  \figlabel{ol-damped-frflin-fit}
\end{figure}

\begin{figure}
  \begin{wide}
  \begin{subfigure}
    \psfragfig{PhD/Experiments/XPMT/latex/ol-reson}
    \caption{Model-derived resonance frequencies shown with predicted values.\figlabel{ol-reson}}
  \end{subfigure}
  \hfil
  \begin{subfigure}
    \psfragfig{PhD/Experiments/XPMT/latex/ol-dampi}
    \caption{Model-derived damping ratios. Closing the coil circuit increased the damping.\figlabel{ol-dampi}}
  \end{subfigure}
  \end{wide}
  \lofcaption{Analysed results from fitting the open loop measurements to the isolator model of \eqref{fit-model}}{. With coil circuit open and closed, the resonance frequencies were constant but the damping ratios changed.}
  \figlabel{ol-results}
\end{figure}

The curve of the measured resonance frequencies (\figref{ol-reson}) follows the same trend as the predicted  natural frequencies (redrawn from \figref{xpmt-magres}).
However, their magnitudes are not well matched.
As the system approaches the \qzs/ position, even small changes in alignment and physical tolerances have significant effects on the calculated instability region; as seen in the low end of the quasi-static measurements, \figref{xpmt-magres}, a fraction of a millimetre change in the position of the top magnet can change the natural frequency by 25\%.
Associatively, with larger magnet gaps the discrepancy becomes lower.
Therefore, the discrepancy seen between theoretical and measured results should be expected.

The resonance frequency of the system was unaffected by the presence of the electromagnetic coil (which can also be observed by comparing \figref{ol-undamped-frflin,ol-damped-frflin}) but the damping was increased significantly when the coil circuit was closed.


\subsection{Observed nonlinear behaviour}
\seclabel{xpmt-nonlin}

The dynamics shown in the undamped case are more nonlinear than the damped
case; this is due to the greater displacements experienced by the beam moving
the magnets through greater ranges of stiffness variation. When the
transmissibility is calculated as the transfer function between the input and
output signals, the system is assumed to be linear and nonlinearities are
rejected by the ratio of the cross-spectrum and power-spectrum terms in
\eqref{Tbm}.
A different result can be shown by calculating the transmissibility instead with a ratio of the individual power spectra of the base and magnet:
\begin{dmath}[label=frfnl]
  \transmissibility{} = \sqrt{\powerSpectMagnet/\powerSpectBase},
\end{dmath}
In this case, any nonlinearities in the signals are retained in the final result.
(Recall \secref{qzs-not-zerk} for a comparison between the transmissibility and the variance gain for a cubic stiffness nonlinearity.)

The transmissibility calculated with this method is shown in
\figref{ol-undamped-frfnl} for the coil in an open circuit (\ie, low damping).
In this case, there are significant nonlinearities present in the data, seen by a clear peak at close to twice the `linear' resonance frequency.
This superharmonic behaviour is similar to that seen in simulations of such a \qzs/ system (\figref[vref]{variance-gaps}).
When the coil is connected and the damping present in the system increased, these nonlinearities are no longer seen (the results are indistinguishable to those shown in \figref{ol-damped-frflin}).
The reduced nonlinearity with increased damping is consistent with the work of other researchers \cite{jazar2006}.


\begin{figure}
  \psfragfig{PhD/Experiments/XPMT/latex/ol-undamped-frfnl}
  \caption
    [Open loop transmissibility measurements without the coil connected.]
    {Open loop measurements without the coil connected; the transmissibility is calculated with \eqref{frfnl}. Black markers show resonance frequencies and the points at twice each resonance frequency, indicating the nonlinear behaviour.}
  \figlabel{ol-undamped-frfnl}
\end{figure}

\subsection{Closed loop velocity feedback dynamic measurements}

In this section, results are shown using closed loop feedback control to improve the vibration isolation characteristics of the system.
For this experiment, the rest position of the spring was chosen to achieve an arbitrary low
resonance frequency (approximately \SI{3.5}{Hz}).
In this position, the electromagnetic coil was used in a simple absolute velocity
feedback controller in an attempt to reduce the magnitude of the resonance peak.
The gain of the feedback control was increased until the system became close
to instability.
Frequency response measurements over this range of feedback gains are shown in \figref{cl-results}.
Sampling parameters were as shown previously in \tabref{rig-ol-spect}.

\begin{figure}
  \psfragfig{PhD/Experiments/XPMT/latex/cl-results}
  \caption{Closed loop frequency response measurements for a range of velocity feedback gains.}
  \figlabel{cl-results}
\end{figure}

To estimate the velocity of the moving beam, the accelerometer measurement was
integrated by the charge amplifier with a cut-on frequency of \SI{1}{Hz}.
Because the accelerometers were being used to measure velocity (via integrators in the charge amplifiers used), the transmissibility curves shown in \figref{cl-results} were calculated from the
ratio of the velocity measurements instead of acceleration measurements.

Due to the presence of higher-order dynamics in the structure, a low pass
filter was used to reject signals above \SI{50}{Hz}. This also ensured that
aliasing was avoided when taking the frequency response measurements and when
feeding back the velocity signal for the controller.

The overall improvement to the vibration isolation can be shown by calculating
the root-mean-square of the transmissibility over a certain frequency range.
\begin{dmath}
  \RMSof{\transmissibility{}} =
  \sqrt{\Sum{\transmissibility{\freq}^2}{\freq,\freq_1,\freq_2}}
\end{dmath}
The lower frequency limit is defined as $\freq_1=\SI{1}{Hz}$, as the results
become very noisy below this frequency. The upper frequency limit
$\freq_2=\SI{11}{Hz}$ was chosen as the higher-order dynamics (not shown in
\figref{cl-results}) have little impact yet at this frequency.

\Figref{cl-reduction} shows the reduction in root-mean-square transmissibility
$\RMSof{\transmissibility{}}$ as the negative feedback gain increases.
The overal transmissibility reduction is calculated as $1-\RMSof{\transmissibility{}}/\transmissibility{}_0$, where $T_0$ is the root-mean-square transmissibility of the open loop system.
The resonance at \SI{1}{Hz} of \figref{cl-results} as the feedback gain increases causes the overall transmissibility reduction to have a local maximum.

\begin{figure}
  \psfragfig{PhD/Experiments/XPMT/latex/cl-reduction-2}
  \caption[{Reduction on overall transmissiblity as velocity feedback gain was increased.}]{Affect on overall transmissiblity $\RMSof{\transmissibility{}}$ between \SI{1}{Hz} and \SI{11}{Hz} as velocity feedback gain was increased.}
  \figlabel{cl-reduction}
\end{figure}

\subsubsection{On the feedback gain magnitude}

\Figref{cl-results,cl-reduction}
plot results in terms of a `feedback gain' given in dimensionless units. This
gain is the multiplier from the controller on the velocity signal that also
incorporates the effects of the amplifiers for both sensors and actuators;
therefore, the absolute magnitude of this number is unimportant and is used
purely to demonstrate the range of behaviours seen as the gain is increased
until instability appears.

\subsection{Analysis on the gain-induced resonance}
\seclabel{acc-resonance}

The appearance of a low frequency peak at \SI{1}{Hz} in \figref{cl-results} as the feedback
gain is increased is explained by the presence of the high-pass filter
incorporated in the accelerometer charge amplifier, which also has a pole at \SI{1}{Hz}.
This behaviour has been shown previously for single \dof/ structures with velocity feedback \cite{ananthaganeshan2001,brennan2007-jsv}.
Here, the same type of analysis will be used to investigate the response of a linear two \dof/ isolator
system (shown in \figref{vibration-base-feedback}) with integrated accelerometer measurements used for velocity feedback control.

\begin{figure}[t]
\centerline{
\begin{minipage}[b]{0.5\textwidth}
  \asyinclude{PhD/Figures/Systems/vibration-base-feedback}
  \caption{Vibration isolation schematic with active feedback.}
  \figlabel{vibration-base-feedback}
\end{minipage}\qquad
\begin{minipage}[b]{0.8\textwidth}
  \asyinclude{PhD/Figures/Control/simple-isolator-block}
  \caption{Block diagram of \eqref{simple-isolator-laplace} representing
  the system shown in \figref{vibration-base-feedback}.}
  \figlabel{simple-isolator-block}
\end{minipage}
}
\end{figure}


In the time domain, the response of this linear system is given by
\begin{dmath}
\mass \accMass =
  \forceIn - \damping\gp{\velMass-\velBase} - \stiffness\gp{\dispMass-\dispBase}
\end{dmath},
which can be re-written in the Laplace domain as
\begin{dmath}[label=simple-isolator-laplace]
s^2 \laplaceMass \underbrace{\gp{\mass+\damping/s+\stiffness/s^2}}_{\gp{\Block[2]}^{-1}} =
  \laplaceForce + s^2 \laplaceBase \underbrace{\gp{\damping/s+\stiffness/s^2}}_{\Block[1]}
\end{dmath}.
This is shown as a block diagram in \figref{simple-isolator-block}. If the
control force is written as a function of the acceleration of the mass,
$\laplaceForce=s^2\laplaceMass\BlockController$, the transmissibility of the system is
\begin{dmath}[label=cl-generic]
  \frac\laplaceMass\laplaceBase = \frac{\Block[1]\Block[2]}{1+\BlockController\Block[2]}
\end{dmath}.

When the controller is some gain $\gainSkyhook$ in series with an ideal integration of the accelerometer signal, $\BlockController = \gainSkyhook/s$, \eqref{cl-generic} simplifies to the idealised absolute velocity feedback expression introduced in \secref[vref]{skyhook}:
\begin{dmath}
\frac{\gainDisp+\stiffness+s \gp{\damping+\gainVel+\gainAcc s}}{\gainDisp+\gainSkyspring+\stiffness+s \gp{\damping+\gainSkyhook+\gainVel+\gp{\gainAcc+\gainSkymass+\mass} s}}
\end{dmath}
This result is plotted in \figref{cl-ideal} for equivalent values of stiffness and damping as the experimental setup.

A more complex model for the controller block is required to account for the
signal processing involved with amplifying and filtering the accelerometer
signal to measure the velocity in reality. Assuming that the entire process
between acceleration measurement and velocity output from the charge amplifier
can be approximated as an ideal integrator in series with two high pass
filters\footnote{One filter for the integration of the acceleration signal,
another for the conditioning electronics in the amplifier; assume for
simplicity that they have the same cut-on frequency.} \cite{brennan2007-jsv},
the controller block is defined as
\begin{dmath}[label=cl-filter-controller]
  \BlockController = \frac \gainSkyhook s \gp{\frac{s}{s+\freqHPfilter}}^2
\end{dmath},
where $\gainSkyhook$ is the absolute velocity feedback gain and $\freqHPfilter$ is the corner frequency
of the two high pass filters in the charge amplifier.

Using \eqref{cl-filter-controller} in \eqref{cl-generic} gives
the final transfer function between the mass and base states,
\begin{dmath}[label=cl-filter]
  \frac\laplaceMass\laplaceBase =
    \frac{ \gp{\damping s+\stiffness} \gp{s+\freqHPfilter}^2 }
         {
           \gainSkyhook s^3 +
           \gp{ \mass s^2 + \damping s + \stiffness }
           \gp{ s+\freqHPfilter }^2
         }
\end{dmath}.
This is plotted versus frequency in \figref{cl-filter}, where the resonance
induced by the high pass filter becomes apparent as the feedback gain is
increased to 99\% of the gain margin, which is the gain when the system
becomes unstable.
The simulation uses linear parameters $\freq=\SI{3.5}{Hz}$
and $\dampingratio=\num{0.023}$ in order to show results at similar behaviour to \figref{cl-results}.

In order to calculate the gain margin of \eqref{cl-filter},
$\BlockController\Block[2]$ was evaluated equal to $-1$, where the system
response of \eqref{cl-generic} becomes unbounded. To do this, first the
critical frequency was found as the frequency at which
$\Imag{\BlockController\Block[2]}=0$; this frequency was then substituted into
$\Real{\BlockController\Block[2]}=-1$, which was solved for $\gainSkyhook$ to find
the gain margin.

\begin{figure}[!htbp]
  \begin{wide}
  \begin{subfigure}
    \psfragfig{\phdpath Simulations/Springs/fig/cl-ideal}
    \caption{Ideal velocity signal.\figlabel{cl-ideal}}
  \end{subfigure}
  \hfil
  \begin{subfigure}
    \psfragfig{\phdpath Simulations/Springs/fig/cl-filter}
    \caption{Integrated accelerometer feedback with a second order high pass filter.\figlabel{cl-filter}}
  \end{subfigure}
  \end{wide}
  \caption
  [Closed loop simulation with increasing gain before instability.]
  {Closed loop simulation with gains $\{0, 5, 10, 20, 50, 99\}$ percent of the gain margin in \figref{cl-filter}.}
\end{figure}

Comparing \figref{cl-filter} with the experimental results of \figref{cl-results} shows a clear similarity.
The corner frequency of the high pass filter at \SI{1}{Hz} in the
charge amplifier to measure velocity is an impediment to the vibration
isolation properties of the feedback-controlled system.
This is a fundamental limitation in the use of accelerometers for estimating velocity for vibration control.
To avoid this issue, the use of geophone sensors which measure velocity directly (such as used by \textcite{hong2010-rsi}) are a promising alternative to integrated accelerometer measurements.
However, geophones are not a panacea since their response rolls off at low frequencies, limiting their performance at the frequency range of interest in this case.

\section{Conclusion on the experimental results}

In this chapter, the experimental results from a magnetic device have been presented.
The magnetic device was designed to demonstrate the ability of variable stiffness through position changes of the load-bearing magnets.
The magnetic system exhibited expected static and dynamic behaviour.

As the system was brought closer to \qzs/, the resonance frequency reduced until the operating point became too close to the position of marginal stability where even slight disturbances would yield instability.
The minimum resonance frequency that could be achieved passively with this system was around \SI{2}{Hz}.
A lower resonance frequency than this could be achieved with larger magnets with larger equilibrium magnet gaps, which would result in a larger physical region of stable operation near the \qzs/ position.

Without feedback control and without the actuator coil connected, the system showed very small damping ratios around \num{0.005}; these were dependent on the distance between the magnets at equilibrium. 
Connecting the non-contact electromagnetic actuator increased this damping ratio to around \num{0.03}--\num{0.04} due to eddy currents induced in the coils from the permanent magnet.

Dynamically, the system showed transmissibilities which could be predicted well by standard single degree of freedom models.
With the very low damping of the open loop system, superharmonics were clearly visible in the power spectra but the resonance peaks remained linear-like.
Once the actuator was connected, the additional damping suppressed these nonlinearities.

Absolute velocity feedback control was successful in reducing the transmissibility peaks, but as the gain was increased the additional poles added by the integration filters caused an additional, lower frequency peak to appear as the closed loop system approached the gain margin.
At best, the resonance peak was reduced by over a decade and the root mean square transmissibility was reduced by around 65\%.
Improving the performance of the closed loop system may require high fidelity geophones; there are inherent difficulties associated with low frequency feedback control for vibration systems.

\end{document}
