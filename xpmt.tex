%!TEX root = thesis.tex

\chapter{Prototype low-stiffness magnetic spring}
\chaplabel{xpmt}

\chapterprecis{
  And finally, some experiments that use magnets and measure vibrations. First, I
  describe the rig and other implementation details, including various tangents
  on appropriate sensor types and future designs that could be implemented
  in the future. Experimental results from the rig that was commissioned are
  presented, demonstrating the `\qzs' design where the resonant frequency
  of a passive system is reduced with attractive magnets, and further
  improvements gained from applying standard velocity feedback control.
}

This chapter covers the implementation details and results of the
experimental apparatus.

\section{Magnets}

K\,\&\,J~Magnetics\note{\url{http://www.kjmagnetics.com/}}
is an American supplier of surplus rare earth magnets in
various geometrical shapes.
They offer a large range of sizes, \eg, from
one-sixteenth inch cubes (100 for \textsc{us\$}6.00) to one inch
cubes, (\textsc{us\$}14.85 each), and even larger.
\note{Prices correct as of April 2009.} Price has an
approximately linear relationship with magnet volume,
as shown in the graph of magnet volume versus price in
\figref{mag-price}.

\begin{figure}
  \psfragfig{\phdpath Rig/latex/mag-price}
  \caption{Magnet price versus magnet volume for rare earth magnets.
    Data obtained for cube magnets of magnetisation grade \acro{N42}.}
  \figlabel{mag-price}
\end{figure}

The magnets used for the experimental apparatus were chosen
based on ease of availability and construction. Cylindrical
magnets were used of dimensions $\mbox{\diameter}\,\SI{12.7}{mm}\times\SI{9.5}{mm}$
and material properties as shown in \tabref{magnet-prop}.

\begin{table}
  \caption{Material properties for the magnets in the experimental apparatus.}
  \tablabel{magnet-prop}
  \begin{tabular}{lc}
    \toprule
    Diameter & \SI{12.7}{mm} \\
    Height   & \SI{9.5}{mm}  \\
    Grade    & \acro{N42} \\
    \bottomrule
  \end{tabular}
\end{table}

\section{Actuators}
\seclabel{actuators}

While electromagnetic actuators are usually designed in push-pull configurations,
\textcite{nandi2009} demonstrate a single-sided design in which a bias current
is used to keep the structure in permanently `sprung' position; relaxing the
current is the equivalent of a repulsive force, while increasing the current
is the usual `pull' or attractive force of a regular electromagnetic actuator.
This design is relatively inefficient since it requires a continuous energy
expenditure to remain in equilibrium.

An alternative single-sided design that could be investigated is described
here. Rather than a coil and iron setup, the floating element could be a
permanent magnet; changing the direction of the current flowing in the coil
would then produce an alternating force on the magnet.

The dual-coil electromagnetic arrangement described in \secref{dualcoil} was
custom-built for the actuator for the experimental apparatus. I could add
a schematic and some dimensions here but I don't know necessary that would be.

\section{Sensors}

There are four main choices for non-contact distance sensing.
\begin{description}
\item[Ultra-sonic] 
    These sensors work by sending out modulated
    ultra-sonic pulses to be reflected off the target. The time spent
    in the round trip gives a linear indicator of the
    distance. Ultra-sonic sensors are cheap, but slow and
    inaccurate. They are unsuited for use in this project.
\item[Inductive] 
    An inductive, or Hall effect, sensor works by
    exciting a coil with a high frequency sinusoidal current which
    induces eddy currents in the target.  These eddy currents may be
    measured very accurately, but the whole effect is very dependent
    on a lack of magnetic noise. This makes these type of sensors
    difficult to use in magnetic applications.
\item[Capacitive] 
    A capacitive sensor measures the capacitance
    between a plate and the target.  It can be very accurate and quite
    fast. To measure large distances, however, a large capacitive head
    is required (approx.\ $1$\,cm diameter for every $1$\,mm of
    range.) They are also very expensive due to the conditioning
    electronics required.
\item[Laser] 
    A laser sensor uses interferometry to calculate position
    of the target. For their price, they offer very good accuracy and
    speed; also, they do not suffer from electrical noise.
\end{description}

See \textcite{boehm1993} for a more detailed overview.

The sensor purchased for the experimental apparatus is the Wenglor~05\,MGV\,80
opto-electronic sensor, which uses a laser to measure distance over a
range of \SI{10}{mm}. This sensor was purchased from a local supplier at a reduced
price. It was selected for its convenient availability, flexibility of application,
and good performance characteristics. Relevant operating properties are listed in \tabref{wenglor}.

\begin{table}
  \caption{Some properties of the Wenglor~05\,MGV\,80 laser distance sensor.}
  \tablabel{wenglor}
  \begin{tabular}{@{}lc@{}}
    \toprule
    Measurement offset & \SI{43}{mm} \\
    Measurement range & \SI{10}{mm} \\
    \midrule
    Working range & \SI{43}{mm}--\SI{53}{mm} \\
    Output range & \SI{10}{V}--\SI{0}{V} \\
    \midrule
    Resolution & $<\SI{10}{\micro m}$ \\
    Response time & \SI{0.5}{ms} \\
    \bottomrule
  \end{tabular}
\end{table}

\section{Design of the experimental apparatus}

A schematic of the experimental apparatus that was designed and built as part of
this project is shown in \figref{rig}. Physical parameters of the design
are shown in \tabref{rigprop}.

\begin{figure}
  \asyfig{Rig/rig-schematic}
  \caption{Schematic of the experimental rig.}
  \figlabel{rig}
\end{figure}

\begin{table}
\caption{Physical properties of the experimental rig.}
\tablabel{rigprop}
\begin{tabular}{@{}lcc@{}}
\toprule
                   Property &              Symbol &         Value \\
\midrule
                  Beam mass &         $\MassBeam$ &     \SI{0}{kg} \\
                Beam length &       $\LengthBeam$ &   \SI{320}{mm} \\
                Beam height &       $\HeightBeam$ &    \SI{25}{mm} \\
                 Beam width &        $\WidthBeam$ &    \SI{40}{mm} \\
             Beam thickness &    $\ThicknessBeam$ &     \SI{2}{mm} \\
\midrule
      Magnet support height & $\HeightRigMagnets$ &    \SI{78.5}{mm} \\
   Magnet support lever arm & $\LengthRigMagnets$ &   \SI{310}{mm} \\
        Magnet support mass &   $\MassRigMagnets$ &    \SI{87}{g} \\
             Magnets height &        $\HeightMag$ &    \SI{10}{mm} \\
           Magnets diameter &          $\DiamMag$ &    \SI{12.7}{mm} \\
\bottomrule
\end{tabular}
\end{table}

\subsection{Physical/geometric design}

The main mass of the vibrationary component of the rig is composed of a
rigid square cross-section beam. This beam is used to constrain the system
of moving magnets to motion in a single degree of freedom. After taking a small
approximation, this motion can be assumed to be largely linear in the vertical
direction. This assumption is verified in \secref{coil-design}.

Having established that the rotational system imposes a linear constraint on
the system under small enough displacements, it is better to examine the
response of the system as if it were a linear system. This involves writing
the dynamics of the system in terms of rotational motion and coercing the
variables into linear ones using a small angle approximation.

The rotational equation of motion for the beam--magnet system, neglecting
damping, is given by
\begin{dmath}[label=rig-moment-dynamics]
  \InertiaMomentTotal \ddot\AngleMag - \MomentMag = 0
\end{dmath}.
But the system has been designed such that the rotation of the system is
small and the motion at the magnets-end of the beam can be considered to be 
constrained in the vertical translational direction alone. The equation of motion
is re-written in terms of vertical displacement, $\DispMag$, arising from the
small angle approximation of $\AngleMag$. 
That is, 
\begin{dmath}
\AngleMag\approx\Sin{\AngleMag}=\DispMag/\LengthRigMagnets
\end{dmath}.
The moment on the beam due to the magnet force is 
$\MomentMag=\LengthRigMagnets\ForceRigMagnets$; \eqref{rig-moment-dynamics} can
be re-written as
\begin{dmath}[label=rig-force-dynamics]
  \frac{\InertiaMomentTotal}{\LengthRigMagnets^2}\ddot\DispMag 
    - \ForceRigMagnets = 0
\end{dmath}.

\subsubsection{Moment of inertia calculation}

A rectangular cross-section with a centre of rotation around a point offset in
one direction is shown in \figref{moment-of-inertia}. The moment of inertia in
the $x$--$y$ plane of such a cross section, of length $\BlockInertiaLength$
and height $\BlockInertiaHeight$, and offset $\BlockInertiaOffset$ from the
midpoint in the $y$ direction is
\begin{dmath}[label=InertiaMoment]
  \InertiaMoment\fn{
    \BlockInertiaMass,
    \BlockInertiaLength,
    \BlockInertiaHeight,
    \BlockInertiaOffset} 
  = \frac{1}{12} 
    \BlockInertiaMass 
    \gp{ 4\BlockInertiaLength^2 + \BlockInertiaHeight^2 } 
    + \BlockInertiaMass \BlockInertiaOffset^2
\end{dmath}.

\begin{figure}
  \begin{subfigure}
    \asyfig{Rig/moment-of-inertia}
    \caption{Schematic for calculating the moment of inertia of a block that
    is not rotating around the midpoint of an edge.\figlabel{moment-of-inertia}}
  \end{subfigure}
  \begin{subfigure}
    \asyfig{Rig/moment-beam}
    \caption{Model of a square-cross section beam for calculating the moment 
      of inertia around the centre of rotation shown.
      \figlabel{moment-beam}}
  \end{subfigure}
  \caption{Diagrams for calculating moments of inertia.}
\end{figure}

The moment of inertia for a square-section beam, $\InertiaMomentBeam$, can be
calculated by applying \eqref{InertiaMoment} to the four sides of the beam and
summing the results.
\begin{dgroup*}
\begin{dmath}
  \InertiaMomentBeam = \InertiaMoment_{\text{top}} 
    + 2\InertiaMoment_{\text{side}} + \InertiaMoment_{\text{bottom}}
\end{dmath},
\intertext{where}
\begin{dmath}
  \InertiaMoment_{\text{top}} = 
    \InertiaMoment\fn{\MassBeamTop,\LengthBeam,\ThicknessBeam,\half\ThicknessBeam}
\end{dmath},
\begin{dmath}
  \InertiaMoment_{\text{side}} = 
    \InertiaMoment\fn{\MassBeamSide,\LengthBeam,\HeightBeam,\half\HeightBeam}
\end{dmath},
\begin{dmath}
  \InertiaMoment_{\text{bottom}} = 
    \InertiaMoment\fn{\MassBeamTop,\LengthBeam,\ThicknessBeam,\HeightBeam+\half\ThicknessBeam}
\end{dmath},
\intertext{and}
\begin{dmath}
  \MassBeamTop = \half \MassBeam \frac{\WidthBeam}{\WidthBeam+\HeightBeam}
\end{dmath},
\begin{dmath}
  \MassBeamSide = \half \MassBeam \frac{\HeightBeam}{\WidthBeam+\HeightBeam}
\end{dmath}.
\end{dgroup*}

The total moment of inertia $\InertiaMomentTotal$ of the motional element of
the experimental rig is given by
\begin{dmath}
  \InertiaMomentTotal = \InertiaMomentBeam + \InertiaMomentMagnets
\end{dmath}.
The moment of inertia of the magnets and their supports,
$\InertiaMomentMagnets$, is assumed to contribute to total moment of inertia
as a point mass, since they act at a large distance from the axis of
revolution over a small and little-changing range of displacements from the
axis.
\begin{dmath}
  \InertiaMomentMagnets = \MassRigMagnets \LengthRigMagnets^2
\end{dmath}

From the numbers tabulated in \tabref{rigprop}, the total moment of inertia
of the motional beam is calculated to be
\begin{dmath}[label=InertiaMomentTotal]
  \InertiaMomentTotal = \SI{0.0156}{kg.m^2}
\end{dmath}.

\subsubsection{Equivalent mass calculation}

This produces the `equivalent mass' of the system
  $\EquivMassRig\equiv\inlinefrac{\InertiaMomentTotal}{\LengthRigMagnets^2}$ 
as if the arrangement of magnets was self-contained. From \eqref{InertiaMomentTotal},
this equivalent mass is 
\begin{dmath}[label=EquivMassRig]
  \EquivMassRig = \SI{0.173}{kg}
\end{dmath}.
This equivalent mass figure will be used later \fixme{crossref} to estimate
the equivalent stiffness of the system in different configurations.

\subsection{Coil design}
\seclabel{coil-design}

The geometry of the coil was designed from theory developed in \secref{emforces}.
The separation required between the permanent magnets in order to incorporate
an inline coil for actuation lead the design of the minimum dimensions of the coil.
\note{Generally, the author feels that for anyone else this sort of design 
  would be acceptable to determine empirically — \ie, guess — but a  
  combination between understanding one's limitations of practical design and 
  a tendency of over-analysing insignificant details produced the work in this 
  section.}

A long, rigid square-section beam was used to constrain the magnets to motion
in the predominantly vertical direction. Due to vertical motion from the 
ground, the magnets moving on the arc of beam experience some small horizontal
motion. Since the magnet section is held inside the coil, enough
tolerance is necessary to avoid contact between the moving magnet and the fixed coil.
On the other hand, the smaller the inner radius of the coil, the greater the
forces imparted by the coil on the magnet.

A simplified geometry of the moving magnet arrangement is used to calculate
the minimum tolerance required to avoid contact with the coil as a function
of the vertical motion of the magnets. This geometry is shown in 
\figref{horiz-tolerance}, where the L-shaped magnet support is shown
in the `rest' position $\overline{OAB}$ and in a rotated position $\overline{OA'B'}$.

\begin{figure}
  \asyfig{Rig/horiz-tolerance}
  \caption{Geometry for calculating the horizontal tolerance of the inner
           dimensions of the coil.}
  \figlabel{horiz-tolerance}
\end{figure}

The horizontal displacement of the highest point after rotation, $x$, can be
calculated as a function of rotation angle $\theta$
\begin{dgroup}
\begin{dmath}
  x = h \Sin{\theta} + f
\end{dmath},
\begin{dmath}
  f = l - l \Cos{\theta}
\end{dmath},
\end{dgroup}
where $l$ is the horizontal length of the magnet arrangement, and $h$ is the 
vertical height.
This can be expressed in terms of the vertical displacement of the main beam,
$z$, with $\Sin{\theta}=z/l$ and using the trigonometric relationship $\Cos{\ArcSin{a}}=\sqrt{1-a^2}$:
\begin{dmath}
  x = h z / l + l - \sqrt{l^2-z^2}
\end{dmath}.

From the dimensions chosen for the rig, $l=\SI{300}{mm}$ and $h=\SI{78.5}{mm}$,
the minimum horizontal tolerance for a vertical displacement $z=\SI{1}{mm}$ is
\SI{0.26}{mm} and for $z=\SI{2}{mm}$ is \SI{0.53}{mm}. 
(The relationship between $x$ and $z$ being fairly linear for most values
of $l$ and $h$ with small ranges of $z$.)
This tolerance was judged to be small enough to wrap a coil around.

\section{Experimental results}

A number of measurements were performed using the experimental apparatus;
in the sections following, data recorded is presented for 
\begin{enumerate}
	\item Magnet gap versus beam displacement;
	\item Open loop frequency responses for a range of magnet gaps; and,
	\item Velocity feedback in a single configuration.
\end{enumerate}

\subsection{Displacement calibration}

The upper magnet's position was varied until the limit of stability was
reached. The lower fixed magnet was kept fixed, which means that the position
of \qzs\ was changing; with counter-threaded mounts for the upper and lower magnets,
they could be adjusted in parallel to achieve a fixed \qzs\ location.

As the upper magnet placement was lowered, the rest position of the beam moved
closer to the \qzs\ position (as more force was supported by the upper magnet).
This relationship is shown in \figref{turns-qzs}. From the geometry of the rig,
the position of the upper magnet was used to calculate the normalised magnet
gap between the magnets at \qzs; this parameter is used here and below to
represent the varied configuration of the spring.

\begin{figure}
  \psfragfig{\phdpath Experiments/Zero_Stiffness/latex/turns-qzs}
  \caption{Rest position of the system as the magnet position varies.}
  \figlabel{turns-qzs}
\end{figure}

\subsection{Open loop measurements}

As the position of the upper magnet is varied, the amount of negative
stiffness added to the system changes. This predominantly affects the rest
position and the resonance frequency, along with with small changes in damping.

Frequency response measurements were taken at a number of discrete locations
to observe the changes in dynamics as the rest position of the system approached
the \qzs\ position.

Due to the low damping and low resonance frequency of the system, very long
sample times were required to achieve results with enough frequency resolution
and sufficient coherence to characterise the response. The spectrum analysis
parameters are shown in \tabref{rig-ol-spect}.

\begin{table}
  \begin{tabular}{lc}
    \toprule
      Sample rate        & \SI{1000}{Hz}           \\
      \FFT\ points       & $2^{16}$                \\
      Sample time        & $\approx$\SI{17.5}{min} \\
      Average overlap    & $0.75$                  \\
      Number of non-overlapping averages & $16$    \\ 
    \bottomrule
  \end{tabular}
  \caption{Parameters used in the signal and spectrum analysis for the
   experimental measurements.}
  \tablabel{rig-ol-spect}
\end{table}

Open loop measurements were taken both with and without the electromagnetic
actuator connected (wired up as both a short circuit and an open circuit). In
the closed circuit configuration, the coil adds damping via induced eddy
currents from the moving magnet. As an open circuit, the coil has no effect on
the dynamics of the system.

Open loop measurements without the coil connected are shown in \figref{ol-undamped-frflin}.
and measurements taken with the coil (\ie, with added damping) shown in \figref{ol-damped-frflin}.
For both of these results, the transmissibility, \transmissibility, shown is the transfer function
between the accelerometer measurements of the base and magnet-supported beam:
\note{Calculated with \Matlab's \texttt{tfestimate} command.}
\begin{dmath}[label=Tbm]
	\transmissibility = \frac{\crossSpectMagnetBase}{\powerSpectBase}
\end{dmath}.

\begin{figure}[p]
  \psfragfig{\phdpath Experiments/Zero_Stiffness/latex/ol-undamped-frflin}
  \caption{Open loop measurements with the electromagnetic coil connected in
           an open circuit; no additional damping is added to the system.}
  \figlabel{ol-undamped-frflin}
\end{figure}

\begin{figure}[p]
  \psfragfig{\phdpath Experiments/Zero_Stiffness/latex/ol-damped-frflin}
  \caption{Open loop measurements, with the electromagnetic coil connected
           in a closed circuit. The coil adds damping to the system, which can
           be seen by the reduction in height of the resonant peaks in comparison 
           to \figref{ol-undamped-frflin}.}
  \figlabel{ol-damped-frflin}
\end{figure}

\subsection{Analysis of the open loop data}

From the measurements shown in the previous section, curve fitting was
used to extract information from the raw data.

The model used to fit the data to is the simple system of vibration isolation,
\eqref{simple-isolation-freq}
in terms of the resonant frequency, $\natfreq=\sqrt{\stiffness/\mass}$, and damping ratio $\dampingratio=\damping/(2\sqrt{\stiffness\mass})$. The resulting expression is
\begin{dmath}[label=fit-model]
  \transmissibility\fn{\freq} = \frac
    {2\ii\dampingratio\freq\natfreq + \natfreq^2}
    {-\freq^2 + 2\ii\dampingratio\freq\natfreq + \natfreq^2},
\end{dmath}
where $\freq$ is the frequency at which to calculate the transmissibility
$\transmissibility$. The data is fit \note{Using \Matlab's
\texttt{fminsearch}.} to \eqref{fit-model} between
$\half\natfreq\le\freq\le2\natfreq$, returning good estimates of the resonance
frequency and damping ratio of the system corresponding to each measurement
taken.
Plots of the curve fit models shown against the measured data are shown in
\figref{ol-undamped-frflin-fit,ol-damped-frflin-fit}. Due to the influence
of higher-order dynamics in the system, the model starts to deviate from the
data at higher frequencies. The resonance frequencies and damping ratios
are shown in \figref{ol-results}.

\begin{figure}[p]
  \psfragfig{\phdpath Experiments/Zero_Stiffness/latex/ol-undamped-frflin-fit}
  \caption{Curve fit model of \figref{ol-undamped-frflin}. Grey lines show the original data.}
  \figlabel{ol-undamped-frflin-fit}
\end{figure}

\begin{figure}[p]
  \psfragfig{\phdpath Experiments/Zero_Stiffness/latex/ol-damped-frflin-fit}
  \caption{Curve fit model of \figref{ol-damped-frflin}.  Grey lines show the original data.}
  \figlabel{ol-damped-frflin-fit}
\end{figure}

\begin{figure}
  \begin{wide}
  \begin{subfigure}
    \psfragfig{\phdpath Experiments/Zero_Stiffness/latex/ol-reson}
    \caption{Model-derived resonant frequencies\figlabel{ol-reson}}
  \end{subfigure}
  \begin{subfigure}
    \psfragfig{\phdpath Experiments/Zero_Stiffness/latex/ol-dampi}
    \caption{Model-derived damping ratio\figlabel{ol-dampi}}
  \end{subfigure}
  \end{wide}
  \caption{Analysed results from fitting the open loop measurements to 
  the isolator model of \eqref{fit-model}.}
  \figlabel{ol-results}
\end{figure}

From \figref{ol-reson} it can be seen that the resonant frequency remains
unaffected by the presence of the electromagnetic coil (which can also be
observed by comparing \figref{ol-undamped-frflin,ol-damped-frflin}) but the
damping is increased significantly.


\subsection{Observed nonlinear behaviour}

The dynamics shown in the undamped case are more nonlinear than the damped
case; this is due to the greater displacements experienced by the beam moving
the magnets through greater ranges of stiffness variation. When the
transmissibility is calculated as the transfer function between the input and
output signals, the system is assumed to be linear and nonlinearities are
rejected by the ratio of the cross-spectrum and power-spectrum terms in \eqref{Tbm}. 
A different result can be shown by calculating the transmissibility instead with a ratio of the
individual power spectra of the base and magnet:
\begin{dmath}[label=frfnl]
  \transmissibility\fn{\freq} = \sqrt{\powerSpectMagnet/\powerSpectBase},
\end{dmath}
where $\powerSpectMagnet$ is the power spectra of the accelerometer measurements at the
moving magnet and $\powerSpectMagnet$ is the power spectra of the accelerometer measurements
of the base. In this case, any nonlinearities in the signals are retained in the final result.

The transmissibility calculated with this method is shown in
\figref{ol-undamped-frfnl}. In this case, the nonlinearities present in the
system are more prominent, and a clear peak is shown at close to twice the `natural'
resonance frequency.

\begin{figure}
  \psfragfig{\phdpath Experiments/Zero_Stiffness/latex/ol-undamped-frfnl}
  \caption{Open loop measurements without the coil connected; the transmissibility
  is calculated with \eqref{frfnl}. Black markers show resonant frequencies and
  the points at twice each resonant frequency, indicating the higher-order dynamics.}
  \figlabel{ol-undamped-frfnl}
\end{figure}

\subsection{Velocity feedback results}

For this experiment, the rest position of the spring was chosen to some low
resonant frequency (approximately \SI{3.5}{Hz}) and velocity feedback applied
to the system. Sampling parameters are as shown previously in
\tabref{rig-ol-spect}.

In this position, the electromagnetic coil was used in a simple velocity
feedback controller in an attempt to reduce the height of the resonance peak.
The gain of the feedback control was increased until the system became close
to instability. Frequency response measurements over this range of feedback
gains are shown in \figref{cl-results}.

To measure the velocity of the moving beam, the accelerometer measurement is
integrated by the charge amplifier with a cut-on frequency was \SI{0.5}{Hz}. Because the
accelerometers were now being used to measure velocity, the transmissibility
curves shown below were calculated from the ratio of the velocity measurements
instead.

Due to the presence of higher-order dynamics in the structure, a low pass filter
was used to reject signals above \SI{50}{Hz} \fixme{check!}.

The appearance of a low frequency peak in \figref{cl-results} as the feedback
gain is increased is explained by the presence of the high-pass filter incorporated
in the accelerometer charge amplifier. This exact behaviour is shown
by \textcite{brennan2007} for single degree of freedom structures with velocity
feedback.

\begin{figure}
  \psfragfig{\phdpath Experiments/Zero_Stiffness/latex/cl-results}
  \caption{Closed loop measurements.}
  \figlabel{cl-results}
\end{figure}

The overall improvement to the vibration isolation can be shown by calculating
the root-mean-square of the transmissibility over a certain frequency range.
\begin{dmath}
  \transmissibility_{\text{RMS}} = 
  \sqrt{\Sum{\transmissibility\fn{\freq}^2}{\freq,\freq_1,\freq_2}}
\end{dmath}
The results become very noisy in \figref{cl-results} below
$\freq_1=\SI{1}{Hz}$, so this is defined as our lower frequency
limit. The upper frequency limit $\freq_2=\SI{11}{Hz}$ is chosen
as the higher-order dynamics (not shown in \figref{cl-results}) have little
impact yet.

\Figref{cl-reduction} shows the reduction in `overall transmissibility'
$\transmissibility_{\text{RMS}}$ as the negative feedback gain increases. The
resonance appearing in the lower frequencies of \figref{cl-results} causes the
overall transmissibility reduction to have a local minimum.

\begin{figure}
  \psfragfig{\phdpath Experiments/Zero_Stiffness/latex/cl-reduction}
  \caption{Reduction in overall transmissiblity between \SI{1}{Hz} and \SI{11}{Hz}.}
  \figlabel{cl-reduction}
\end{figure}

\paragraph{The feedback gain magnitude} \Figref{cl-results,cl-reduction} plot
results in terms of a `feedback gain' given in dimensionless units. This gain
is the multiplier from the controller on the velocity signal that also
incorporates the effects of the amplifiers for both sensors and actuators;
therefore, the absolute magnitude of this number is unimportant and is used
purely to demonstrate the range of behaviours seen as the gain is increased
until instability appears.

