\documentclass[10pt,a4paper]{memoir}
\def\asydir{\jobname}
\usepackage{thesis-preamble}
\usepackage{geometry}
\EndPreamble
\pagestyle{empty}
\def\standaloneabstract{}
\begin{document}

\ifx\standaloneabstract\undefined

  \section*{Abstract}

\else

  \noindent {\Large Modelling and design of magnetic levitation systems for vibration isolation}

  \noindent {\scshape Will Robertson}

  \bigskip
  
  \expandafter\noindent
\fi
\small
Vibration disturbance has a consistent negative impact on equipment and processes.
The central theme of this thesis is the investigation of using permanent magnets in the design of a system for vibration isolation.

The thesis begins with a comprehensive literature review on the subjects of passive and active vibration isolation, permanent magnetic systems, and the common area between these on nonlinear vibration systems using magnetic forces.
The use of cylindrical and cuboid magnets is the primary focus of this work for which analytical solutions are known for calculating forces and torques.
Subsequently, the state of the art in analytical modelling of permanent magnet systems is covered, including a contribution in this area for calculating the forces between cylindrical~magnets.

A range of load bearing designs using simple permanent magnet arrangements are examined, with multiple designs suitable for a variety of objectives.
A particular emphasis is placed on a system using inclined magnets, which can exhibit a load-independent resonance frequency.
Load bearing using multipole magnet arrays is also discussed, in which a large number of magnets are used to generate more complex magnetic fields.
A variety of multipole arrays are compared against each other, including linear and planar magnetisation patterns, and an optimisation is performed on a linear array with some resulting guidelines for designing such systems for load bearing.

Permanent magnet levitation requires either passive or active stabilisation; therefore, the design of electromagnetic actuators for active control is covered with a new efficient method for calculating the forces between a cylindrical magnet and a solenoid.
The optimisation of a solenoid actuator is performed and geometric parameters are found which are near-optimal for a range of operating~conditions.

Two \qzs/ systems are introduced and analysed next.
These systems are designed with a nonlinearity such that low stiffnesses are achieved while bearing large loads.
The first system analysed is a purely mechanical device using linear springs; unlike most analyses of this design, the horizontal forces are also considered and it is shown that \qzs/ is capable in all translational directions simultaneously.
However, a notable disadvantage of such spring systems is their difficulty in online tuning to adapt to changing operating conditions.
A magnetic \qzs/ system is then analysed in detail and design criteria are introduced, providing a design framework for such systems and showing how the complex interaction of variables affects the resulting dynamic behaviour.
Although the system is nonlinear, the effects of the nonlinearities on the vibration response are shown to be generally negligible.

The thesis concludes with some experimental results of the same \qzs/ system, constructed as a single degree of freedom prototype.
The quasi-static and dynamic behaviour of the system matches the theory well, and active vibration control is performed to improve the vibration isolation characteristics of the device.

\normalsize
\end{document}

