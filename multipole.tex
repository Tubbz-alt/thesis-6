%!TEX root = thesis.tex

\chapter{Multipole}
\chaplabel{multipole}

\chapterprecis{
  This chapter investigates techniques for increasing the forces between arrays
  of magnets, and looks at how these techniques may be integrated into a
  magnetic spring design. This is largely unfinished because my results are
  a little funny, and inconclusive. I use the theory introduced in \chapref{magnet-theory}
  to investigate the forces between multiple arrays which are composed of
  magnets with rotating polarisations; this approximates a magnet with sinusoidal
  polarisation. The amount of discretisation and the spatial wavelength of
  magnetisation of the arrays both influence the resulting forces.
}

\section{actuators}

\textcite{trumper1996} developed the electromagnetic analogue to the
Halbach array: a triangular coil-winding arrangement that produces
sinusoidal and predominantly single-sided flux. This technique could
prove especially well suited to generating forces against permanent
magnet arrays.


[\textcite{dasilveira2005}] Analytical expression for the normal force between two magnets on a back-iron plate and a perpendicular coil.
The system is for a planar actuator, and the motivation is to be able to determine the amount of out-of-plane force generated by a particular design.
Could very well be useful for some of my ideas.

\section{Code}

A multipole array can be uniquely defined in terms of several sets of variables.   The simplest such description is:
\begin{itemize}
\item Size of each magnet $[\mupmaglength,d,h]\T$,
\item Number of magnets $T$, and
\item Magnetisation direction of the first magnet $\vartheta_0$ and rotation between successive magnets $\vartheta_i$.
\end{itemize}
Other variables that can also be used to describe the array are: (assuming an array aligned with the $x$-axis)
\begin{itemize}
\item Length of the array $\muplength = \mupmaglength T$,
\item Number of magnets per wavelength $M=2\pi/\vartheta_i$,
\item Wavelength of magnetisation $\mupwavelength= \mupmaglength M$, and
\item Number of wavelengths $N=[T-1]/M$.
\end{itemize}
Figure~\ref{multipole-variables} illustrates the relationship between magnet length, array length, and wavelength for an example linear array. The wavelength of magnetisation is the length required to achieve, with successive magnets, a full rotation of magnetisation direction.

Note the presence in Figures \ref{halbach} and~\ref{multipole-variables} and in general of an `end magnet' that adds symmetry to the discretisation of the magnetisation. This extra magnet is necessary to balance the forces in the horizontal direction.

Provided that enough information is specified and it is internally consistent, the Matlab implementation for calculating array forces can accept any combination of the variables listed above when defining the geometry of each array. Rather than explicitly enumerating the location of each magnet and the direction of its magnetisation, the implementation requires just an axis with which to align the array and the facing direction of its `strong' side.

\begin{figure}
\centering
\asyfig{Multipole/multipole-variables}
\caption{Geometry of a linear multipole array with 90\textdegree\ magnetisation rotations. This array contains two wavelengths of magnetisation with an end magnet for symmetry.}
\label{multipole-variables}
\end{figure}


\section{Forces between multipole magnet arrays}

A classical multipole or Halbach array is a linear array of magnets stacked to approximate a single magnet with sinusoidal magnetisation, first analysed in the '70s \parencite{halbach1981,shute2000}. Multipole arrays have been analysed for a variety of force-producing applications; only a small selection are included in the bibliography here \parencite{lee2004-mx,robertson2005,rovers2009-ietm}. One objective behind such a design is to focus the magnetic field on one side of the array, such to increase the forces exerted by which magnetic field on one side of the array and to reduce or eliminate any need for magnetic shielding on the reverse side. The magnetic field produced by one such multipole array is shown in Figure~\ref{halbach} in which the single-sided nature of the magnetic field can clearly be seen.

\begin{figure}
\centering
\includegraphics[width=\linewidth]{PhD/Figures/Multipole/halbach.pdf}
\caption{Magnetic field lines for a multipole array with 45\textdegree\ magnetisation rotations. The single-sided nature of the magnet field is evident.}
\label{halbach}
\end{figure}

Having expressed the forces between two magnets with arbitrary magnetisation, it becomes simple to use this expression iteratively over an array of magnets with varying magnetisation strengths or directions. The force between two arrays is simply the superposition of every combination of forces between the individual magnets in each array.

Some abstractions to the manner in which the iteration over each combination of magnets is performed allows us to simplify the code necessary to express a variety of configurations of the multiple arrays. This facilitates easy comparisons between different designs.

\begin{figure}
\centering
\asyfig{Multipole/multipole-variables}
\caption{Geometry of a linear multipole array with 90\textdegree\ magnetisation rotations. This array contains two wavelengths of magnetisation with an end magnet for symmetry.}
\label{multipole-variables}
\end{figure}

A multipole array can be uniquely defined in terms of several sets of variables.   The simplest such description is:
\begin{itemize}
\item Size of each magnet $[\mupmaglength,d,h]\T$,
\item Number of magnets $T$, and
\item Magnetisation direction of the first magnet $\vartheta_0$ and rotation between successive magnets $\vartheta_i$.
\end{itemize}
Other variables that can also be used to describe the array are: (assuming an array aligned with the $x$-axis)
\begin{itemize}
\item Length of the array $\muplength = \mupmaglength T$,
\item Number of magnets per wavelength $M=2\pi/\vartheta_i$,
\item Wavelength of magnetisation $\mupwavelength= \mupmaglength M$, and
\item Number of wavelengths $N=[T-1]/M$.
\end{itemize}
Figure~\ref{multipole-variables} illustrates the relationship between magnet length, array length, and wavelength for an example linear array. The wavelength of magnetisation is the length required to achieve, with successive magnets, a full rotation of magnetisation direction.

Note the presence in Figures \ref{halbach} and~\ref{multipole-variables} and in general of an `end magnet' that adds symmetry to the discretisation of the magnetisation. This extra magnet is necessary to balance the forces in the horizontal direction.

Provided that enough information is specified and it is internally consistent, the Matlab implementation for calculating array forces can accept any combination of the variables listed above when defining the geometry of each array. Rather than explicitly enumerating the location of each magnet and the direction of its magnetisation, the implementation requires just an axis with which to align the array and the facing direction of its `strong' side. This implementation is contained within the Matlab function `multipoleforces'.

\subsection{Example}

Allag et al.~\textcite{allag2009-electromotion} calculated the forces between two five-magnet multipole arrays with 90\textdegree\ rotation between successive magnets. Their results are reproduced in Figure~\ref{allag}, using the following parameters:
\begin{itemize}
\item Arrays aligned along $y$ and facing vertically $\pm z$.
\item $[m,d,h]\T = [\SI{0.01}{m},\SI{0.01}{m},\SI{0.01}{m}]\T$.
\item Total number of magnets $T = 5$ and number of magnets per wavelength of magnetisation $M=4$.
\item Displacement between arrays $\bm d=[0,\delta,\SI{0.015}{m}]\T$.
\end{itemize}
A schematic of this system is shown in Figure~\ref{allag-system}.

\begin{figure}
\centering
\asyfig{Multipole/allag-system}
\caption{Multipole system composed of \SI{10}{mm} cube magnets for $\delta=0$. Forces in the $y$- and $z$-directions as $\delta$ varies are shown in Figure~\ref{allag}.}
\label{allag-system}
\end{figure}

The Matlab code to generate Figure~\ref{allag} may be found in `examples/multipole\_example.m'. This is a good example of the convenience of using this framework for calculating results of this kind. Of the around 70 lines of code in `multipole\_example.m', only some 20 lines of code are necessary to set up the system parameters and calculate the forces; the rest of the file consists of comments, whitespace, and producing the actual figure itself.

\begin{figure}
\centering
\psfragfig{magcode/examples/fig/allag-repro}
\caption{Reproduction of Allag et al.'s results~\textcite{allag2009-electromotion}. The forces in the $x$-direction are zero.}
\label{allag}
\end{figure}

\subsection{Planar multipole arrays}

It is also possible to calculate the forces between magnet arrays with magnetisation as a function of position in two directions. For two multipole arrays of the structure shown in Figure~\ref{trumper-system} that have magnetisation directions based on the superposition of orthogonal linear multipole arrays, results are shown in Figure~\ref{trumper} for horizontal displacement with arrays composed of \SI{10}{mm} cube magnets and \SI{15}{mm} vertical displacement between the array centres. Each magnet of each array (except those with zero magnetisation) has a magnetisation of \SI{1}{T}. The Matlab code to generate Figure~\ref{trumper} is located in the file `examples/planar\_multipole\_example.m'.

Note the differences in the shape of the force curves between Figures \ref{allag} and~\ref{trumper}; in the planar multipole arrangement, the vertical force is largely positive even as the arrays shift into their attractive zone, whereas the linear multipole forces are more symmetrical and become significantly negative under a similar displacement. A more detailed comparison between various configurations of multipole arrays is currently under investigation and will be reported at a later date.

\begin{figure}
\centering
\includegraphics{PhD/Figures/Multipole/pa-trumper}
\caption{Planar multipole array, facing up, with magnetisation directions as the superposition of two orthogonal linear Halbach arrays. Non-filled arrowheads denote diagonal magnetisation and empty magnets have zero magnetisation.}
\label{trumper-system}
\end{figure}

\begin{figure}
\centering
\psfragfig{magcode/examples/fig/planar-halbach}
\caption{Force vs.\ horizontal displacement results for two planar multipole arrays such as shown in Figure~\ref{trumper-system}. The $x$-forces are all zero.}
\label{trumper}
\end{figure}







\section{Forces between arrays}

It has been previously mentioned that the purpose of the arrays is to focus
the flux into the areas where it will be generating spring force. This has the
beneficial side-effect of reducing the forces created between an array and any
ferrous or magnetic object that is nearby in any other direction.
\Figref{halbach-lion} shows three Halbach arrays. They are arranged all in
repulsion, but the top two have their strong sides facing whereas the bottom
two have their weak sides facing. In this static arrangement, the forces
between the strong pair are about 200 times stronger than the weak pair. This
demonstrates the advantages of using Halbach arrays in this manner.

\begin{figure}[htbp]
   \centering
   \grf[height=0.4\textheight]{Figures/Multipole/halbach-lion-fieldstr}
   \caption{Opposing Halbach arrays in the strong and weak fields.}
   \figlabel{halbach-lion}
\end{figure}

As previously mentioned, \cite{cho2001} derived analytic equations for the
flux above several planar arrays; namely, the patchwork and extended linear
designs, and Kim's Halbach superposition. An analytical solution for their own
array is not attempted due to the complexity of the model.

\begin{figure}
   \grf{Figures/Multipole/pa-arb}
   \caption[Arbitrary planar array with cuboid magnets.]{Arbitrary planar
array with cuboid magnets.
$i$ and $j$ are the magnet numbers in the $x$ and $y$ directions.}
   \figlabel{pa-arb}
\end{figure}

However, it is as complex again to derive the forces between two such
facing arrays. Therefore, analysis is performed with finite element
analysis in \ANSYS/. For this purpose, then, the only relations required
are equations deriving the magnetisations of each individual magnet.
\figref{pa-arb} shows an arbitrary planar array, created from a number
of equal-sized cuboid magnets stacked
perpendicular to the $\z$ direction, for which we define the magnetisation,
$\magM$, of \emph{each} magnet as:
\begin{dmath}
  \magM(i,j) = M_x\x + M_y\y + M_z\z
\end{dmath}
where $i$ and $j$ are the magnet numbers in the $\x$ and $\y$ directions,
respectively. This equation is redefined to remove the constant magnetisation
magnitude, $M$, from each term:
\begin{align}
  \magM(i,j) & = M\cdot\hat\magM(i,j) \\
\text{where}\qquad
  \hat\magM(i,j) & = \hat M_x\x + \hat M_y\y + \hat M_z\z
\end{align}
Note that this is the magnetisation of only the bottom array; the
magnetisation of the opposite, repulsive, array can be generated by inverting
the magnetisation in the facing direction: ${\hat M}_{z_2} = -\hat M_z$.

\subsection{Simple arrays}

To begin, the simple arrays are examined and the vertical forces between two
facing arrays solved statically for a range of displacements.

For opposing homogeneous arrays, equivalent to a single magnet, the
magnetisations do not vary:
\begin{dmath}[compact]
\hat M_x = \hat M_y = 0 \condition*{\hat M_z = 1}
\end{dmath}.
For simple `patchwork' magnets as in \figref{pa-ns}, the magnetisation
directions are:
\begin{align}
  \hat M_x & = 0,\\
  \hat M_y & = 0,\\
  \hat M_z & = \Cos{(i\pi)}\Cos{(j\pi)}.
\end{align}
For the case when the linear Halbach array (with
\ang{90} rotations) is extended into the third dimension as in
\figref{pa-x2d}, the magnetisation of the array is:
\begin{align}
  \hat M_x & = \Cos{(\half i \pi)},\\
  \hat M_y & = 0,\\
  \hat M_z & = \Sin{(\half i \pi)}.
\end{align}

The forces between facing pairs of these three arrays is shown in
\figref{pa-simple-forces}. For this graph, and subsequent ones,
$5\times5$ arrays of half-inch rare-earth cube magnets have been used
. Surprisingly, the patchwork array did not perform better than the
homogeneous array except in close proximity. The finite element
analysis in \ANSYS/ was capable of generating solutions for distances
equal to $0.3$ magnet widths between the arrays; closer than this and
the number of nodes required for the solution surpassed the capacity
of the department's license for the software.

\begin{figure}[htbp]
   \centering
   \grf{Figures/Multipole/pa-simple-forces}
   \caption{Forces between three simple facing arrays.}
   \figlabel{pa-simple-forces}
\end{figure}


\fixme{Compare simple forces}

\subsection{Halbach superposition}

The array developed by \citeauthor{kim1997} is shown in \figref{pa-trumper}.
The magnetisation can be given in two ways. A strict superposition would
simply add the magnetisations of two orthogonal Halbach arrays, yielding:
\begin{align}
    \hat M_x & = -\Sin{(\half i \pi)},\\
    \hat M_y & = -\Sin{(\half j \pi)},\\
    \hat M_z & =  \Cos{(\half i \pi)} + \Cos{(\half j \pi)}.
\end{align}
However, an inspection of the magnetisations shows that the magnets are now
significantly stronger than for the other arrays that have been looked at.
To make the comparison worthwhile, some care must be taken to ensure the
normalised magnetisations of each block
are equal and within the same range ($0\leq\hat M\leq1$) as the other arrays
already shown:
\begin{align}
    \hat M_x & = -\tfrac{1}{\sqrt{2}}\Sin{(\half i \pi)},\\
    \hat M_y & = -\tfrac{1}{\sqrt{2}}\Sin{(\half j \pi)},\\
    \hat M_z & =  \tfrac{1}{\sqrt{2}}\sqrt{\cos^2{(\half i \pi)} + \cos^2{(\half j
\pi)}}.
\end{align}

The first array will be referred to as a `Halbach addition', whereas
the second will be called a `Halbach
superposition'. \Figref{pa-kim-forces} demonstrates the significant
differences between the two, showing the forces with the extended
Halbach array (refer to \eqref{pa-x2d}) for comparison. The simple
addition of two orthogonal arrays shows approximately double the
spring stiffness---consistent with the method of generation. However,
facing superposition arrays creates approximately \emph{half} the
force as between the simple extended Halbach arrays. This is
surprising, and yet promising, since the magnetisations for this
planar array are awkward to obtain.

\begin{figure}[htbp]
   \centering
   \grf{Figures/Multipole/pa-kim-forces}
   \caption[Forces between two planar Halbach arrays.]
   {Forces between planar arrays: Halbach addition and Halbach superposition.}.
   \figlabel{pa-kim-forces}
\end{figure}

\fixme{revise previous section}

\subsection{Cho's array}

To look at the array by \citeauthor{cho2001}, a more complex model is
required than the arbitrary array composed of simple block magnets.
For this task, each block magnet may be decomposed into
triangular cross-section pieces, as shown in
\figref{cho-triangles}. The magnetisations can then be expressed in
terms of $\magM_H$, those square blocks containing multipole
magnetisations in the $\x$-$\y$ plane, and $\magM_V$, those in the $\z$
directions, as shown in the figure: (note that $\half(1-(-1)^i) =
0,1,0,1,...$)
\begin{align}
  \begin{split}
  \hat \magM_{i,j} & = \tfrac{1}{4}(1-(-1)^{j})
                \gp{
                    \hat \magM_V (1-(-1)^{i})
                  - \hat \magM_H (1-(-1)^{i+1})
                } \\
          & \quad
            + \tfrac{1}{4}(1-(-1)^{j+1})
                \gp{
                    \hat \magM_H (1-(-1)^{i})
                  - \hat \magM_V (1-(-1)^{i+1})
                }
  \end{split}\\
\intertext{where}
  \hat \magM_V & = (0,0,1), \qquad \text{(for $k=0,\dots,3$)}\\
  \begin{split}
    \hat M_{H_x} & = \Cos{(\half k \pi)},\\
    \hat M_{H_y} & = \Sin{(\half k \pi)},\\
    \hat M_{H_z} & = 0.
  \end{split}
\end{align}

\begin{figure}
   \centering
   \grf{Figures/Multipole/pa-cho-triangles}
   \caption{Cho's array, decomposed into repeating elements.}
   \figlabel{cho-triangles}
\end{figure}

Optimal forces for arrays have been shown by \textcite{marble2008}. Shapes for
achieving 80\% magnetic field strength while using simpler magnet geometries
are shown.


\section{New work}

Trying to consolidate some earlier work, \figref{ansys} shows are some
ANSYS results for 2D multipole arrays in repulsion over a range of
distances. These show trends as expected: the more magnets used for
discretisation, the greater the forces are, with 45° rotations pretty
close to the asymptote. And of course, the smaller the wavelength of
magnetisation is, the greater the spring stiffness, but which acts
over a smaller range.

\begin{figure}
   \centering
   \grf{Simulations/Magnet_arrays/Halbach_pitch/eps/vary-magnets-3}
   \grf{Simulations/Magnet_arrays/Halbach_pitch/eps/vary-wavelength-3}
   \caption{Old ANSYS plots. First has wavelength as twice the height.
   Second has four magnets per wavelength.}
   \figlabel{ansys}
\end{figure}

For these graphs, 2D analyses were taken for cross sections
$4''\times0.5''$ (a length of eight thicknesses). The coercive force
of the magnets was $900000$, and the magnetic permeability was
$1.1$. The resulting force per metre was multiplied by the thickness
of the array for an overall magnet array dimension of
$4''\times0.5''\times0.5''$.


Taking another look at Paden's paper, I realised I had the Fourier expansion
of the magnetisation all wrong in my MATLAB simulations for such square wave
magnetisation in multipole arrays (the results of which were funny and which
hadn't been used for anything). Fixing this up (see page~112 of my workbook),
the graphs now match up with FEA pretty well.
\begin{dmath*} P_y =
\frac{4B_r}{\pi^2\permVac}\sum^\infty_{n=1,3,5,\dots} \frac{1}{n^2}\exp(-k_n
g)\gp{1-\exp(-k_n d)}^2
\end{dmath*}
The results are shown in
\figref{square-magnetisation-comparison}, but are not compared to a magnetic
nodes solution because at this stage I'm happy enough with this graph.

\begin{figure}
   \centering
   \grf{Simulations/Magnet_arrays/Halbach_pitch/eps/square-magnetisation-comparison}
   \caption{Comparison between analytical and FEA solutions to two repulsive
   eight magnet multipole arrays with half-inch cube magnets alternating
   in magnetisation.}
   \figlabel{square-magnetisation-comparison}
\end{figure}

For eight magnets per wavelength, we get approximately the same
results as for sinusoidally rotating magnetisation. Some old ANSYS
results showing this can be seen in \figref{vw4}. However, these
numbers are quite bigger than results obtained using Backers' formula,
given by \begin{dmath*} P_y = \frac{{B_r}^2}{4\permVac}\exp(-k g)\gp{1-\exp(-k
d)}^2 ; \quad k = \frac{2\pi}{\lambda}.  \end{dmath*}

The reason for this is because Backers' formula uses only sinusoidal
magnetisation in the vertical direction. Campbell, however, derives
some similar equations using rotating
magnetisation~\cite{campbell2002}. However, trying to use this gives
me numbers that are much much too small, so I've obviously done
something terribly wrong there.

\begin{figure}
   \centering
   \grf{Simulations/Magnet_arrays/Halbach_pitch/eps/vary-wavelength-4}
   \caption{Eight magnets per wavelength approximately equals sinusoidal magnetisation.}
   \figlabel{vw4}
\end{figure}

Some notes on Backers' derivation of the forces between two
sinusoidally magnetised plates.  Main features of the work:
\begin{enumerate}
\item infinite plates in \x\ and \y, separated by a gap in \z;
\item magnetisation of the first plate given by
  \begin{dmath}
    \magM_1 = (0, 0, \overline M\Cos{ky}),
  \end{dmath}
  that is, varying sinusoidally with \z, magnetised in the same
  direction as the gap;
\item magnetisation of the second plate is phase shifted, possibly,
  with an offset:
  \begin{dmath}
    \magM_2 = (0, 0, \overline M\Cos{k(y-y_0)}).
  \end{dmath}
\end{enumerate}

To get the forces between them, we start off by looking at the
magnetic potential created due to the first array at any point
$(r_x,r_y,r_z)$. This is given by a volume integral over the infinite
plate, in which the point vector is given by $\vect{r} =
(X-x,Y-y,Z-z)$:
\begin{align}
U(x,y,z) & = \frac{1}{4\pi\permVac}\intd{\limits_{V_1} \frac{\magM_1\bdot\vect{r}}{r^3}}{V_1} \\
         & = \frac{1}{4\pi\permVac}
             \int\limits^0_{-d}
             \int\limits^\infty_{-\infty}
             \int\limits^\infty_{-\infty}
                \frac{(Z-z)\overline M \Cos{ky}}
                     {((X-x)^2+(Y-y)^2+(Z-z)^2)^{3/2}}
             \,\dee X\dee Y\dee Z                      \\
         & = \frac{\overline M \Cos{ky}}{4\pi\permVac}
             \iiint
                \frac{(Z-z)}
                     {\gp{(X-x)^2+(Y-y)^2+(Z-z)^2}^{3/2}}
             \,\dee X\dee Y\dee Z
\end{align}
Here's where Backers comes in with his solution:
\begin{dmath}
U(x,y,z) = \frac{\overline M\Cos{kx}}{2k\permVac}\gp{\e^{-ky}}\gp{1-\e^{-kd}}
\end{dmath}
from which we eventually calculate the forces between the plates with
the following steps:
\begin{align}
\magH & = \nabla U \\
E  & = - \int\limits_{V_2}\magM_2\bdot\magH\,\dee{V_2} \\
\vect F  & = -\nabla E
\end{align}

The next step for me is to perform these steps using rotating
magnetisation, that is
\begin{dmath}
  \magM_1 = (0, \overline M\Sin{ky}, \overline M\Cos{ky}),
\end{dmath}
and similarly for $\magM_2$. We end up with
\begin{align}
\begin{split}
U(x,y,z)={}& \frac{\overline M \Cos{ky}}{4\pi\permVac}
             \int\limits^0_{-d}
             \int\limits^\infty_{-\infty}
             \int\limits^\infty_{-\infty}
                \frac{(Z-z)}
                     {\gp{(X-x)^2+(Y-y)^2+(Z-z)^2}^{3/2}}
             \,\dee X\dee Y\dee Z                                  \\
         & + \frac{\overline M \Sin{ky}}{4\pi\permVac}
             \int\limits^0_{-d}
             \int\limits^\infty_{-\infty}
             \int\limits^\infty_{-\infty}
                \frac{(Y-y)}
                     {\gp{(X-x)^2+(Y-y)^2+(Z-z)^2}^{3/2}}
             \,\dee X\dee Y\dee Z,
\end{split}
\end{align}
whose two terms are not equivalent due to different limits of
integration. So here we are.



I've some analyses performed with my (now efficiently and correctly
working) implementation of Charpentier \etal's technique of the
forces between two multipole arrays; similarly to the previous such
analyses, the wavelength is varied
(\figref{french-array-zforces-vary-wavelength}) and number of
magnets used per wavelength is varied
(\figref{french-array-zforces-vary-nummagnets}) for two fixed
multipole arrays at varying distances. Only forces in the
\z-direction are considered at this point.

\begin{figure}
\begin{wide}
  \def\figpath{\phdpath Simulations/Magnet_arrays/French_array/eps-20051109/}
  \subfloat[Two magnets per wavelength]
    {\includegraphics{\figpath french-array-zforces-vary-wavelength-nummagnets=2}}
  \hfill
  \subfloat[Four magnets per wavelength]
    {\includegraphics{\figpath french-array-zforces-vary-wavelength-nummagnets=4}}
  \\
  \subfloat[Eight magnets per wavelength]
    {\includegraphics{\figpath french-array-zforces-vary-wavelength-nummagnets=8}}
  \hfill
  \subfloat[Sixteen magnets per wavelength]
    {\includegraphics{\figpath french-array-zforces-vary-wavelength-nummagnets=16}}
\end{wide}
  \caption{\z-forces between two multipole arrays with
           varying wavelength of magnetisation.}
  \figlabel{french-array-zforces-vary-wavelength}
\end{figure}

\begin{figure}
\begin{wide}
  \def\figpath{\phdpath Simulations/Magnet_arrays/French_array/eps-20051109/}
  \subfloat[Normalised wavelength of one]
    {\includegraphics{\figpath french-array-zforces-vary-nummagnets-wavelength=1}}
  \hfill
  \subfloat[Normalised wavelength of two]
    {\includegraphics{\figpath french-array-zforces-vary-nummagnets-wavelength=2}}
  \\
  \subfloat[Normalised wavelength of four]
    {\includegraphics{\figpath french-array-zforces-vary-nummagnets-wavelength=4}}
  \hfill
  \subfloat[Normalised wavelength of eight]
    {\includegraphics{\figpath french-array-zforces-vary-nummagnets-wavelength=8}}
\end{wide}
  \caption{\z-forces between two multipole arrays with
           varying number of magnets per wavelength.}
  \figlabel{french-array-zforces-vary-nummagnets}
\end{figure}

\begin{figure}
   \centering
   \grf{Figures/Multipole/magnets-per-wavelength}
   \caption{Increasing approximation of sinusoidal magnetisation.}
   \figlabel{magnets-per-wavelength}
\end{figure}

\subsection{Experimental measurements}

Mike Devine at `Dexter Magnetic Technologies, Inc.' kindly sent us a
nine-pole Halbach array for testing purposes.  The properties of the
array are summarised in \tabref{multipole-sample-properties}.  The
multipole array is composed of eights magnets with \ang{45}
magnetisation rotations (a single wavelength of magnetisation) plus
one magnet for symmetry (\ie, the first and last magnet both have
$\ang{+90}$ magnetisations).

The magnet dimensions are $\SI{12.7}{mm} \times \SI{12.7}{mm} \times
\SI{21.3}{mm}$, with the square faces in the plane of the rotation of
magnetisation. They are set on an aluminium plate, which is \SI{1.6}{mm}
thick, that is two magnet widths wider than the array itself.

\begin{table}
  \caption{Multipole array properties.}
  \tablabel{multipole-sample-properties}
  \begin{tabular}{@{} ll @{}}
    \toprule
    Number of magnets      & Nine             \\
    Magnetisation rotation & \ang{45}         \\
    Nominal energy product & \SI{382}{kJ/m^3} \\
    Intrinsic coercivity   & \SI{875}{kA/m}   \\
    Array height           & \SI{12.7}{mm}    \\
    Array width            & \SI{114.3}{mm}
                             ($9\times\SI{12.7}{mm}$) \\
    Array depth            & \SI{21.3}{mm}    \\
    Former depth           & \SI{1.6}{mm}     \\
    Magnet coating         & Everlube 9800    \\
    \bottomrule
  \end{tabular}
\end{table}

The remanence magnetisation is calculated from the nominal energy
product of the magnets \cite{campbell1994}. The nominal energy product
$\BHmax$ is related to the saturation magnetisation by
\begin{dmath}
\BHmax = \permVac\gp{\half \Msat}^2,
\end{dmath}
which can be used to calculate the remanence with the relation
$\remanence=\permVac\Msat$:
\begin{dmath}
\remanence=\sqrt{4\permVac\BHmax}=\SI{1.39}{T}.
\end{dmath}


\section{Future work}

The code presented here currently only contains algorithms for calculating forces and stiffnesses between cuboid magnets without rotation. \textcite{charpentier1999-ietm-sep} present expressions for two components of the force between cuboid magnets with rotation around one axis (also see their related publications around the same time), but there is no general solution known without using semi-numerical methods \parencite{charpentier2001}.

\textcite{agashe2008-applphys} and \textcite{nagaraj1988} have also calculated forces between cuboid magnets as well as between cylindrical magnets, which latter have more complex solutions due to the presence of elliptic integrals. \textcite{janssen2009-ietm} have derived the analytical forces between pyramidal-frustrum shaped magnets, which have not quite yet been published at time of writing. \textcite{lee2006-mx} and \textcite{cho2001} present different planar arrays using block magnets with trapezoidal and triangular cross-sections, and \textcite{yan2006-iemx} calculate torque for spherical multipole arrays with `dihedral cone'-shaped magnets. And there are a wide variety of cylindrical multipole arrays in use for bearing and brushless motor applications \parencite{zhu2001-ipep}.

The work introduced here is a very humble beginning to incorporating as much of the cited work above (and this is certainly an incomplete list) into a unified framework for calculating the dynamics of permanent magnet--based systems.

