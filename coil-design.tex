%!TEX root = thesis.tex
\chapter{Magnet--coil actuators}

\section{Forces due to an electromagnetic field}
\seclabel{emforces}

This section describes a theoretical, parametric study that was undertaken to
describe the forces between a cylindrical magnet and cylindrical
current-carrying coil. While approximate formulations are known in the general
literature (\fxfatal{reference}), the approach taken here was a complete
analytical study using the exact geometry of both coil and magnet. The
forces were numerically evaluated from a series of symbolic integrals in
Mathematica.

The purpose of these calculations was to design a coil to drive a non-contact
vibrating platform; a non-contact coil, rather than a classical shaker, say,
was chosen for its zero stiffness properties; without driving current, the
coil does not add stiffness to the structure (although it does add damping via
velocity-induced eddy currents). Additionally, the actuator does not add mass
to the vibratory system. \fxfatal{discuss `zero impendance' instead of
zero stiffness; added `mass' via induced eddy currents, etc.}

The study also investigates the possibility of building large, low-cost
inertial shakers by using an arrangement of smaller magnets instead of one
large, expensive permanent magnet. (\fxfatal{reference Chris Fuller's work with
George.}) However, this study does \emph{not} investigate the dynamics
associated with driving the coil with an alternating current; only the static
force due to a constant current is investigated.

\fxfatal{Figure for dual coil system}

Finally, a symmetric arrangement of two coils in series driving a central
magnet is examined. Since a coil/magnet pair must be located asymmetrically to
generate maximum force, generally electromagnets of this type are built with
`push-pull' coils, where a second coil is added to restore symmetry to the
force \vs\  displacement response of the actuator.

The study is split into three parts, as described above:
\begin{enumerate}
\item Force on a magnet due to a single coil;
\item Force from a single coil on a multitude of smaller magnets; and,
\item Force on a magnet due to symmetric dual-coils.
\end{enumerate}

\subsection{Derivation of the force equations}

In this section, the equations of the force between a single cylindrical coil and a
cylindrical magnet
are established; this forms the mathematical basis for the three parts of the
study. The theory in this section is based on the theory given by
\textcite{furlani2001}. Conceptually, deriving the force between a coil and
magnet can be summarised in two steps:
\begin{itemize}
\item Integrate a function of the current over the volume of the coil
  to find the magnetic field produced at a specific point in space.
\item Integrate a function of the magnetic field over the volume of a
  magnet to find the force produced by the interaction of the magnetic
  field of the coil and the magnet itself.
\end{itemize}

A schematic of the general system is shown in two dimensions in
\figref{int-geometry}, in which the \compound{co}{ordinate} systems Cartesian
$(\coordca{1},\coordca{2},\coordca{3})$ and cylindrical
$(\coordra{1},\coordra{2},\coordra{3})$ are defined for the geometry of the
coil; and the \compound{co}{ordinate} systems Cartesian
$(\coordcb{1},\coordcb{2},\coordcb{3})$ and cylindrical
$(\coordrb{1},\coordrb{2},\coordrb{3})$ are defined for the geometry of the
magnet. Axial \compound{co}{ordinates} $\coordca{3}$ and $\coordcb{3}$ are oriented in
the `out-of-page' direction.

\begin{figure}
\small
\asyfig{Coil/integral}
\caption{
  Cross section schematic of the system with coil cross section
  $\surfCoil$ and magnet cross section $\surfMag$.
}
\figlabel{int-geometry}
\end{figure}

Start by looking at the magnetic field produced by the electromagnet. Define
in cylindrical \compound{co}{ordinates} $\distra=\distra{1,2,3}$ as the displacement
vector for the point in space at which the magnetic field is being calculated
(inside the volume of the magnet), and $\distrb=\distrb{1,2,3}$ as the
displacement vector to a differential volume of the coil. The magnetic
field $\magB$ from a current source $\magJ$ is given by:
\cite[][\S3.3]{furlani2001}
\begin{dmath}
\magB\fn{\distra} = \magconst\Int{\frac{\magJ\fn{\distrb}\cross\gp{\distra -
\distrb}}{\Abs{\distra-\distrb}^3}}{\diffvolCoil,\volCoil}
\end{dmath},
where $\permVac$ is the permeability of free space and $\cross$ is the vector
cross product.

In \figref{int-geometry} the volumes $\dee\diffvolCoil$ and $\volCoil$ are
shown as cross-sectional surfaces $\dee\diffsurfCoil$ and $\surfCoil$. In
cylindrical \compound{co}{ordinates},
$\dee\diffvolCoil=\distrb{1}\dee{\distrb{1}}\dee{\distrb{2}}\dee{\distrb{3}}$,
and
\begin{dmath}
\Abs{\distra-\distrb} = \sqrt{\distra{1}^2+\distrb{1}^2-2\distra{1}\distrb{1}\Cos{\distra{2}-\distrb{2}}+\gp{\distra{3}-\distrb{3}}^2},
\end{dmath}
which results in the intermediate expression
\begin{dmath}
\magB(\distra{1},\distra{2},\distra{3}) =
\magconst
  \Int{%
    \gp{\magJ(\distrb{1},\distrb{2},\distrb{3})\cross
          \gp{\distra-\distrb}}
         \gp{\distra{1}^2+\distrb{1}^2-2\distra{1}\distrb{1}\Cos{\distra{2}
          -\distrb{2}}+\gp{\distra{3}-\distrb{3}}^2}^{-3/2}\distrb{1}}
  {\distrb{1},\offsetCoilMag-\lengthCoil/2,\offsetCoilMag+\lengthCoil/2}
  {\distrb{2},0,2\pi}
  {\distrb{3},\iradiusCoil,\oradiusCoil}
\end{dmath}
for inner and outer radii of the coil $\iradiusCoil$ and $\oradiusCoil$
respectively, axial coil length $\lengthCoil$, and axial distance between the
magnet and coil centres $\offsetCoilMag$.

The current density of the coil is
$\magJ(\distrb{1},\distrb{2},\distrb{3})=NI/A\coordra{2}$, where $N$ is the
number of turns of wire, $I$ is the current flowing in the coil, and
$A=\lengthCoil\gp{\iradiusCoil-\oradiusCoil}$ is the cross sectional
area of the electromagnet. Therefore,
\begin{dmath}
\magJ\fn{\distrb{1},\distrb{2},\distrb{3}}\cross\gp{\distra-\distrb} =
  NI/A\cdot\inlinevect{\distra{3}-\distrb{3},0,\distra{1}-\distrb{1}}
\end{dmath},
resulting in the final magnetic field equation
\begin{dmath}[label=coil-B]
\magB(\distra{1},\distra{2},\distra{3}) =
\magconst\cdot\frac{NI}{A}
  \Int{
      \distrb{1}
      \inlinevect{\distra{3}-\distrb{3},0,\distra{1}-\distrb{1}}
      \cdot\\
      \gp{\distra{1}^2+\distrb{1}^2-2\distra{1}\distrb{1}\Cos{\distra{2}-
          \distrb{2}}+\gp{\distra{3}-\distrb{3}}^2}^{-3/2}
  }
  {\distrb{1},\offsetCoilMag-\lengthCoil/2,\offsetCoilMag+\lengthCoil/2}
  {\distrb{2},0,2\pi}
  {\distrb{3},\iradiusCoil,\oradiusCoil}
\end{dmath}
or more applicably here (since it is the axial force that is of interest),
\begin{dmath}
\magB{z}(\distra{1},\distra{2},\distra{3}) =
\magconst\cdot\frac{NI}{A}
\Int{
  \distrb{1}\gp{\distra{1}-\distrb{1}}
  \cdot
  \gp{\distra{1}^2+\distrb{1}^2
      -2\distra{1}\distrb{1}\Cos{\distra{2}-\distrb{2}}
      +\gp{\distra{3}-\distrb{3}}^2
     }^{-3/2}
  }
  {\distrb{1},\iradiusCoil,\oradiusCoil}
  {\distrb{2},0,2\pi}
  {\distrb{3},\offsetCoilMag-\lengthCoil/2,\offsetCoilMag+\lengthCoil/2}
\eqlabel{coil-Bz}
\end{dmath}

To obtain force, the magnetic field from the electromagnet is
integrated over the region of the permanent magnet. This is the
summation of a volume and surface integral:
\begin{dmath}
\force = \Int{\gp{\Div{\magM}}\magB}{\diffvolMag,\volMag}+\oint_{\surfMag}
\gp{\magM\dotprod\normn}\magB\,\dee{\diffsurfMag}
\end{dmath}
The magnet is magnetised along its axis, and the magnetisation is
assumed to be constant (a reasonable assumption for rare earth
magnets \fxfatal{cite}). The magnetisation is $\magM=M\hat{\vect{z}}$, which implies
$\Div{\magM}=0$. Therefore,
\begin{dmath}
\force = \oint_{S_m} \gp{\magM\dotprod\normn}\cdot\magB(\distra{1},\distra{2},\distra{3})\,\dee{s_m} ,
\end{dmath}
where $\normn$ is the normal vector of the magnet surface $\surfMag$. For a
cylindrical magnet,
\begin{dmath}
\normn = \left\{
  \begin{array}{@{}c@{\quad}l}
    \pm\hat{\vect{z}} &
        \distra{3} = \pm \half\lengthMag,
        \quad \distra{1}\leq \oradiusMag \\
    \hat{\vect{r}} &
        -\half\lengthMag < \distra{3} < \half\lengthMag,
        \quad \distra{1}=\oradiusMag
  \end{array}\right.
\end{dmath}
for axial magnet length $\lengthMag$. Since $\magM$ is orthogonal to
$\hat{\vect{r}}$ their dot product is zero, and
\begin{dseries}
\begin{math}
  \magM\dotprod\normn = \pm M
\end{math},
\begin{math}
  \distra{3}=\pm \half\lengthMag
\end{math}
\end{dseries}
which leads to the final expression
\begin{dmath}
\eqlabel{Fds}
\force =
  \left.
  \Int{
       M \magB\fn{\distra{1},\distra{2},\distra{3}}
      }
      {\diffsurfMag,\surfMag}
  \right|_{\distra{3}=\lengthMag/2}
 +\left.
  \Int{
   -M \magB\fn{\distra{1},\distra{2},\distra{3}}
      }
      {\diffsurfMag,\surfMag}
  \right|_{\distra{3}=-\lengthMag/2}
=
  \Int{M\magB\fn{\distra{1},\distra{2},\lengthMag/2} \distra{1}}
      {\distra{1},\iradiusMag,\oradiusMag}
      {\distra{2},0,2\pi}
 -\Int{M\magB\fn{\distra{1},\distra{2},-\lengthMag/2} \distra{1}}
      {\distra{1},\iradiusMag,\oradiusMag}
      {\distra{2},0,2\pi}
\end{dmath},
and more applicably,
\begin{dmath}[label=coilFzindirect]
F_z =
  \Int{M\magB{z}\fn{\distra{1},\distra{2},\half\lengthMag} \distra{1}}
      {\distra{1},\iradiusMag,\oradiusMag}
      {\distra{2},0,2\pi}
 -
  \Int{M\magB{z}\fn{\distra{1},\distra{2},-\half\lengthMag} \distra{1}}
      {\distra{1},\iradiusMag,\oradiusMag}
      {\distra{2},0,2\pi} ,
\end{dmath}

Substitution of \eqref{coil-B} into \eqref{coilFz} yields a quintuple integral,
which can be evaluated to obtain the force on the magnet due to the current carrying coil.
A complete analytical solution for this integral is not known to be analytically tractable at this time;
the usual approach is to symbolically integrate as many terms as possible and then
numerically integrate the rest. Here, the angles $\distra{2}$ and $\distrb{2}$
are symbolically integrated in \eqref{coilFzindirect} to produce the final
equation
\begin{dmath}[label=coilFz]
  F_z = B_r \tfrac{N I}{A} \Int{
      \sumset{e_1,e_2}{\{1,-1\}^2}
      e_1 e_2 f_z\fn{e_1\lengthMag/2,\offsetCoilMag+e_2\lengthCoil/2}
    }
    {\distra{1},\iradiusMag,\oradiusMag}
    {\distrb{1},\iradiusCoil,\oradiusCoil}
\end{dmath}
where the intermediate term $f_z$ is
\def\l#1{l_{#1}}
\begin{dmath}
  f_z\fn{\l1,\l2} =
   \frac
    {
     2 \distra{1} \distrb{1}
     \gp{ \l1 - \l2 }
    }
    {
     \gp{\distra{1}-\distrb{1}}
     \sqrt{
          \gp{\distra{1}-\distrb{1}}^2
         +\gp{\l1-\l2}^2
     }
    }\times\\
    \Pi\fn{
      -\frac
        { 4 \distra{1} \distrb{1} }
        { \gp{\distra{1}-\distrb{1}}^2 }
      \middle|
      -\frac
        { 4 \distra{1} \distrb{1} }
        {
          \gp{\distra{1}-\distrb{1}}^2
         +\gp{\l1-\l2}^2
        }
     }
\end{dmath}.
The function $\Pi\fn{n|m}$ is the complete elliptic integral of the third kind.

Varying centre displacement $\offsetCoilMag$ produces force versus displacement results for a given magnet/coil geometry.
Varying $\lengthCoil$, $\lengthMag$ and the other parameters in the integration limits, $\iradiusMag$ ($=0$ for cylindrical magnets), $\oradiusMag$, $\iradiusCoil$, $\oradiusCoil$, allows comparisons to be made between differently sized magnets and coils.

\subsubsection{Including axial offset}

It is also desirable to calculate forces for asymmetric magnet positions, for
reasons of either evaluating the effects of unbalanced load, or for building
more complex geometries. It is the latter application that will be
investigated here, with an analysis of the forces produced with a ring of a
number of smaller magnets compared to that of using a solid ring magnet.

The integrals used in \eqref{Fds} to calculate the forces have variables of
integration $\dee\diffsurfMag=\distra{1}\dee{\distra{1}}\dee{\distra{2}}$ and
integrands containing the the magnetic field $\magB$ in terms of those
variables $\distra{1}$ and $\distra{2}$. However, when the magnet is no longer
centred inside the coil, the integration must take place in the shifted
coördinate system shown in \figref{int-offset}.

\begin{figure}
\asyfig{Coil/circle-offset}
\caption{Region of integration for axially displaced magnet
  surfaces. The magnet is shifted by $(\xi,\eta)$ from the original
  coördinate system.}
\figlabel{int-offset}
\end{figure}

In order to perform this integration, terms in $\magB$ of $\distra{1}$
and $\distra{2}$ must be expressed instead in terms of $\distra{1}'$
and $\distra{2}'$. That is, to use a slightly general example, we wish
to analyse the integral of generic function $A$:
\begin{dmath}
\Int{A\fn{\distra{1},\distra{2}}}
    {a} =
\Int{A\fn{\distra{1},\distra{2}} \distra{1}'}
    {\distra{1}'}
    {\distra{2}'} .
\end{dmath}
The following transformation can be used to perform this:
\begin{align}
\distra{1} &= \sqrt{\distca{1}^2+\distca{2}^2} &
\distca{1} &= \distca{1}' - \xi                 &
\distca{1}' &= \distra{1}'\Cos{\distra{2}'}
\\
\distra{2} &= \ArcTan{\distca{2},\distca{1}}  &
\distca{2} &= \distca{2}' - \eta               &
\distca{2}' &= \distra{1}'\Sin{\distra{2}'}
\end{align}
Accordingly, the force integral for the magnet--coil system with radial eccentricity as well as axial displacement is given by
\begin{dmath}[label=coileccfz]
  \frac{NI\remanence}{4A\pi}
  \Int{
    \sumset{e_1,e_2}{\{1,-1\}^2}
      e_1 e_2 f_z\fn{e_1\lengthMag/2,\offsetCoilMag+e_2\lengthCoil/2}
    }
    {\distra{1}',\iradiusMag,\oradiusMag}
    {\distra{2}',0,2\pi}
    {\distrb{1},\iradiusCoil,\oradiusCoil}
\end{dmath},
where the intermediate function $f_z$ is
\begin{dmath}
  f_z\fn{l,L} =
  f_2\fn{l,L}
  \frac{
    \distra{1}' \distrb{1} \gp{ l-L }
    \sqrt{
      1-\frac{2 \distra{1} \distrb{1} \gp{\cos\fn{\phi_1}-1}}
             {\gp{l-L}^2+\gp{\distrb{1}-\distra{1}}^2}
    }
  }
  {
   \gp{\distrb{1}-\distra{1}}^2
   \sqrt{
         \distra{1}^2+\distrb{1}^2
         - 2 \cos\fn{\phi_1} \distra{1} \distrb{1}
         + \gp{l-L}^2
        }
  }
\end{dmath},
and the second intermediate function $f_2$ is given by
\begin{dmath}
  f_2\fn{l,L} =
     \Pi\fn{
       \frac{ - 4 \distra{1} \distrb{1}}
             {\gp{\distrb{1}-\distra{1}}^2}
       ;
       \frac{\phi_1}{2}
       \middle|
       \frac{ - 4 \distra{1} \distrb{1}}
             {\gp{l-L}^2+\gp{\distrb{1}-\distra{1}}^2}
     }
     -
     \Pi\fn{
       \frac{ - 4 \distra{1} \distrb{1}}
             {\gp{\distrb{1}-\distra{1}}^2}
       ;
       \frac{\phi_1}{2}-\pi
       \middle|
       \frac{ - 4 \distra{1} \distrb{1}}
             {\gp{l-L}^2+\gp{\distrb{1}-\distra{1}}^2}
     }
\end{dmath}.
In \eqref{coileccfz}, two symbolic integrations have been performed (first over $\distrb{3}$ and subsequently over $\distrb{2}$) and the remaining three
integration variables must be evaluated numerically.


\subsubsection{Integration accuracy}

The precision of the numerical integration needs to be selected for an
appropriate compromise between accuracy and computation time. Varying
Mathematica's \verb|PrecisionGoal| from 1--4 (which is equivalent to the
number of `correct' significant figures), a typical force curve \vs\
displacement calculation ranges in processing time as shown in
\tabref{numerical-integration-times}. Eleven integrations were performed over
a range of displacements from zero to \SI{20}{mm} in steps of \SI{2}{mm}. The
physical parameters used are shown in \tabref{prec-param}.

\begin{table}
\caption{Processing time of numerical integration for increasing
precision (in significant figures accurate) of the results.}
\tablabel{numerical-integration-times}
\begin{tabular}{lcccc}
\toprule
Precision & \num{1}    & \num{2}   & \num{3}  & \num{4}   \\
Time (s)  & \num{0.25} & \num{1.6} & \num{18} & \num{187} \\
\bottomrule
\end{tabular}
\end{table}

\begin{table}
\caption{Parameters used in the model for evaluating the numerical precision of integration.}
\tablabel{prec-param}
\begin{tabular}{@{}llcc@{}}
\toprule
\multicolumn{2}{@{}l}
  {Parameter}       & Symbol       & Value         \\
\midrule
Magnet & Remanence  & $\remanence  $ & \SI{1.2}{T}   \\
       & Radius     & $\oradiusMag $ & \SI{6.35}{mm} \\
       & Length     & $\lengthMag  $ & \SI{6.35}{mm} \\
\midrule
Coil & Turns        & $\turnsWire  $ & \num{400}     \\
     & Current      & $\current    $ & \SI{1}{A}     \\
     & Inner radius & $\iradiusCoil$ & \SI{7}{mm}    \\
     & Outer radius & $\oradiusCoil$ & \SI{15}{mm}   \\
     & Length       & $\lengthCoil $ & \SI{7.5}{mm}  \\
\bottomrule
\end{tabular}
\end{table}

The forces calculated with precisions of one, two, and three significant
figures are shown in
\figref{voice-coil-integration-precision} accompanied by the error of the
curves with respect to the values calculated with a precision of four significant
figures. The
error was calculated with
\begin{dmath*}
    100 \times \Abs{\frac{F_4 - F_i}{F_4}}
\end{dmath*}
where $F_i$ is the force calculated with a precision of $i$. Reasonable results are
obtained with a precision of two significant figures (error less than 5\%), and very accurate
results (less than 0.1\% error) are obtained with a precision of three.

\begin{figure}
\begin{wide}
\begin{subfigure}
  \psfragfig{\coilpath voice-coil-integration-precision}
  \caption{Forces with labelled precisions of 1--3.}
\end{subfigure}
\begin{subfigure}
  \psfragfig{\coilpath voice-coil-integration-precision-error}
  \caption{Percentage error with respect to the forces calculated with a precision of four.}
\end{subfigure}
\end{wide}
\caption{Precision of numerical integration for calculating the force \vs\  distance
values on a magnet from an electromagnet.}
\figlabel{voice-coil-integration-precision}
\end{figure}

\subsection{Parameters}

Now that a mathematical model is available to perform the calculations,
thought must be given to the variables used in the forthcoming parametric
study. The parameters must be varied in such a way that comparisons between
cases are meaningful. Changing the coil's physical parameters (length, radii,
wire thickness) affect the current flowing through it for a fixed input
voltage; these parameters are interrelated and cannot be varied independently
while making direct comparisons between the change in the forces due to these
variations. Careful consideration on which parameters to vary directly, and
how, is required.

\paragraph{Electromagnet parameters}

The magnetic field (which is directly responsible for inducing force
on a magnet) produced by a coil varies with a number of parameters:
axial length, inner radius, outer radius, current, number of turns, wire
diameter, and so on. However, these are not all independent.

The starting point for the design of a coil is to choose its
resistance, which is performed by matching the impedance of the
amplifier used to drive the coil. In this case, resistance was taken
as $R=\SI{8}{\ohm}$. This requirement ensures that comparisons can be
made between differently sized coils that consume the same amount of
power.

The resistance of the coil directly infers the length of the wire
winding, $\lengthWire$, through the relation
\begin{dmath}
\lengthWire = \resistanceCoil \areaWire / \resistivity,
\end{dmath}
for wire of cross sectional area $\areaWire$, and resistivity $\resistivity$.
Copper wire with a diameter $\diamWire=\SI{0.2}{mm}$ was selected with an
approximate resistivity $\resistivity=\SI{1.7e-8}{\ohm.m}$.

The fixed parameters of the coil are driving voltage, resistance, wire
thickness and material, which in turn fix the total length of wire. The
results of the parametric study are specific to the values chosen above. Based
on these values, a relationship between the macroscopic parameters of the coil
can be derived.

\begin{figure}
  \asyfig{Coil/turns-cross-section}
  \caption{Schematic representation of a crude wire-wound coil with
    labelled dimensions.}
\figlabel{coil-turns}
\end{figure}

The dimensions of a cylindrical coil are depicted in \figref{coil-turns},
showing three turns of wire in each direction in an
$\coordca{1}$--$\coordca{3}$ cross-section. Given a total length of wire to
use (determined by the total resistance allowed for the coil), it is possible
to derive a relation between the coil length, $\lengthCoil$, and the coil
radii, $\iradiusCoil$ and $\oradiusCoil$ ($\iradiusCoil<\oradiusCoil$).
Assuming that each turn of wire sits directly above or adjacent to its
neighbours, an approximate expression for the total wire length is given by
\begin{dmath}[label=coil-lengthwire]
\lengthWire = \turnsZWire\Sum{2\pi\gp{\iradiusCoil+n\diamWire}}{n,0,N-1}
  = 2\pi \turnsRWire \turnsZWire \gp{\iradiusCoil+\diamWire/2},
\end{dmath}
where $\turnsZWire=\gp{\oradiusCoil-\iradiusCoil}/\diamWire$ and
$\turnsRWire=\lengthCoil/\diamWire$ are the number of turns in the axial and
radial directions. While this does not accurately model the packing effect of
how tightly-packed coils will sit (since ideally \eqref{coil-lengthwire} will
under-represent the number of turns possible in a given cross-sectional area),
this equation is simple and allows some conservatism in the quality of the
construction of the electromagnet.

Therefore, an expression for the outer radius of the coil that is defined for
a coil of fixed inner radius and fixed total wire length is:
\begin{dmath}
  \oradiusCoil = \iradiusCoil +
    \dfrac{\lengthWire\diamWire^2}
          {\pi\lengthCoil\gp{2\iradiusCoil+\diamWire}}.
\end{dmath}

\paragraph{Overall parameters varied}

In the parametric study to follow, a number of independent parameters have
been found to vary (assuming fixed impedance and wire diameter): axial distance
between the magnet and coil centres, $\offsetCoilMag$; axial lengths of the
magnet and coil, $\lengthMag$ and $\lengthCoil$; inner and outer radius of the
magnet, $\iradiusMag$ and $\oradiusMag$; and, inner radius, $\iradiusCoil$, of
the coil.

The remanence of the permanent magnet, $\remanence$, is taken to be \SI{1.2}{T}
for this section; this approximates using a low-strength rare earth permanent
magnet. The magnetisation, $\magM$, then, is given by $\remanence/\permVac$.

Later, when dual coils are used to drive the magnet, the effect of the
adjusting the gap between the two coils on the induced force on the
magnet is examined.

\fxfatal{Tabulate coil simulation parameters/}

\paragraph{On comparisons}

The various graphs presented later are generated with different geometries of
the coil(s) and magnet(s). This makes comparisons between separate results
difficult or inconclusive. This has been done with a mind to try and show
trends in the behaviours of varying various parameters separately; no single
set of results can illustrate the behaviour of the system under such a large
number of parameter variations.

Take the results as shown, then, as indicative of methods that \emph{can} be
used for optimising a certain electromagnetic design. Without a cost function
to optimise, however, such `optimisation' here would be premature. \fxfatal{reword}

\subsection{Single coil/single magnet simulations}

In this section, several results are shown demonstrating the effect on the
force/displacement curves of varying different parameters in a single
magnet--single coil system. Diagrams, which are drawn to relative scale,
accompany each force \vs\  displacement curve to illuminate the systems under
analysis. For axial (side-on) diagrams, the magnet is located at the position
of maximum force.

\paragraph{Equal length magnet and coil}

For coils and magnets of equal lengths,
shown in \figref{voicecoil3-vary-length-2d}, the peak force is
obtained for magnets with an approximately equal diameter and
length. Despite the greater volume of magnetic material, the peak force actually
decreases with increasing magnet lengths.

\begin{figure}
  \begin{subfigure}
    \grf{Figures/Coil/single-coil-2d_1_0}
  \end{subfigure}\par
  \begin{subfigure}
    \psfragfig{\coilpath voicecoil3-vary-length-2d}
  \end{subfigure}
  \lofcaption{Forces between various coil and magnet pairs of equal but
    varying length.}{ The magnet radius is $\oradiusMag=\SI{6.4}{mm}$ and the
    inner coil radius is $\iradiusCoil=\SI{7}{mm}$. Note the change in
    outer coil radius as the coil length varies in order to retain a constant
    resistance.}
  \figlabel{voicecoil3-vary-length-2d}
\end{figure}

While this effect is also governed by the length of the coil (which will be
examined subsequently), \figref{vary-radii} shows that increasing the magnet
\emph{radius} (and the coil radius in tandem) does increase the
total force from the electromagnet.

In both cases, the total magnet volume is varied while the coil power is kept
constant. While it may be intuitive that increasing the size of the magnet
increases the force due to the coil, this is only true in the second case.

\begin{figure}
  \begin{subfigure}
    \grf{Figures/Coil/single-coil-2d_2_0}
  \end{subfigure}\par
  \begin{subfigure}
    \psfragfig{\coilpath voicecoil3-vary-radii-2d}
  \end{subfigure}
  \lofcaption{Forces between various coil and magnet pairs of fixed length
    but varying diameters.}{ The coil and magnet lengths are
    $\lengthCoil=\SI{7}{mm}$ and $\lengthMag=\SI{12.7}{mm}$,
    and the inner radius of the coil is $\iradiusCoil=\oradiusMag+\SI{1}{mm}$.
    The `glitch' at \SI{0.25}{mm} for the $\oradiusMag=\SI{20}{mm}$ case
    is due to numerical imprecision in the computation of the integral.}
  \figlabel{vary-radii}
\end{figure}

\paragraph{Length/radius ratio between the coil and magnet}

The ratio between the lengths of the coil and magnet has several
effects on the force between them. \Figref{vary-mag-length} shows the
effect of increasing the magnet length with a fixed-size coil, for
which the maximum force produced asymptotes to a maximum as the
length of the magnet increases.

\begin{figure}
  \begin{subfigure}
    \grf{Figures/Coil/single-coil-2d_3_0}
  \end{subfigure}\par
  \begin{subfigure}
    \small\psfragfig{\coilpath voicecoil3-vary-lengthratio-2d}
  \end{subfigure}
  \lofcaption{
Forces between varius coil and magnet pairs with fixed coil length.}{
The coil length is $\lengthCoil=\SI{20}{mm}$,
the magnet radius is $\oradiusMag=\SI{6.4}{mm}$, and
the coil inner radius is $\iradiusCoil=\SI{7}{mm}$.}
  \figlabel{vary-mag-length}
\end{figure}

Conversely, \figref{vary-coil-length} shows that \emph{decreasing} the
length of the coil increases the force created on a fixed-size
magnet. Recall that as the coil geometry is chosen to ensure a constant
resistance and hence power consumption. This is shown in the schematics of
\figref{vary-coil-length}: as the coil length increases, its height decreases.

It is interesting to compare the positions of maximum force
between the cases shown in \figref{vary-mag-length,vary-coil-length};
which is depicted in the schematics above each graph.
It can be seen that the maximum force is obtained, approximately, when
the shorter of the magnet or coil is axially centred at an edge of the longer.

\begin{figure}
  \begin{subfigure}
    \grf{Figures/Coil/single-coil-2d_4_0}
  \end{subfigure}\par
  \begin{subfigure}
    \psfragfig{\coilpath voicecoil3-vary-lengthratio2-2d}
  \end{subfigure}
  \lofcaption{
Forces between varius coil and magnet pairs with fixed magnet length.}{
The magnet width is $\lengthMag=\SI{20}{mm}$,
the magnet radius is $\oradiusMag=\SI{6.4}{mm}$, and
the coil inner radius is $\iradiusCoil=\SI{7}{mm}$.}
  \figlabel{vary-coil-length}
\end{figure}

Finally, it is worth noting the width of the peaks in the
force/displacement curves for these various systems. The shorter coil
gives greater forces, but at the expense of a smaller `width' of force
close to the maximum. When used as shakers, these coils are assumed to
have a proportional relationship between current and force, which is only valid
around the region of maximum force for a certain displacement range.
Practically, this limits the lower frequency (which has larger displacements)
at which such a device could impart vibrations into a structure.

\paragraph{Air gap}

A study of the effect of the size of the air gap between the coil and the
magnet, with results in \figref{vary-airgap}, shows that the smaller the air
gap, the significantly greater the forces generated. The slight movement of
the position of maximum force is interesting but negligible.

\begin{figure}
  \begin{subfigure}
    \grf{Figures/Coil/single-coil-2d_5_0}
  \end{subfigure}\par
  \begin{subfigure}
    \psfragfig{\coilpath voicecoil-vary-coilradius-2d}
  \end{subfigure}
  \lofcaption{
    Forces between varius coil and magnet pairs with varying
    coil inner radius.}{ The peak force is indicated by the labelled points.
    The magnet radius is $\oradiusMag=\SI{5}{mm}$, and
    the magnet and coil lengths are $\lengthMag=\lengthCoil=\SI{7.5}{mm}$.
  }
  \figlabel{vary-airgap}
\end{figure}

\paragraph{Eccentricity}

In \secref{manymag}, the superposition of eccentric magnets will be used to
analyse some multi-magnet cases. The effects of eccentricity of the inner
magnet are shown in \figref{vary-eccentric}, where it can be seen that
displacements close to the centre have only a small effect on the force.
Closer to the inner radius of the coil, however, large increases in force
occur as the magnet draws near, which follows from results seen previously in
\figref{vary-airgap}.

\begin{figure}
  \begin{subfigure}
    \asyfig{Coil/eccentric}
  \end{subfigure}\par
  \begin{subfigure}
    \psfragfig{\coilpath coils-offsetTest}
  \end{subfigure}
  \lofcaption{
    Forces between a coil and a magnet with varying radial eccentricity,
    $\offsetCoilMag$\periodifnocomma}{, shown in \si{mm}.
    The inner radius of the coil is $\iradiusCoil=\SI{21}{mm}$, and
    the radius of the magnet is $\oradiusMag=\SI{10}{mm}$
    Note that while the offset
    is in the radial direction, the force shown is still in the axial
    direction.}
  \figlabel{vary-eccentric}
\end{figure}


\subsection{Approximation of a large magnet by the superposition of others smaller}
\seclabel{manymag}

As has been shown in \figref{vary-radii}, increasing the radii of the
magnet/coil system increases the forces generated between them, but it
can be difficult or expensive to procure such large magnets. A
collection of smaller magnets can be used instead, with improved cost
and handling convenience, albeit with some reduction in the maximum
force achievable.

Referring to \figref{circles}, the radius of a smaller circle, $r$,
that fits \COILnmag\ times around inside the circumference of a larger circle
of radius $R$, with a gap between adjacent circles of $d$ is given by
\begin{dmath}
\eqlabel{circles}
r = \frac{R\Sin{\pi/\COILnmag}-d/2}{\Sin{\pi/\COILnmag}+1}.
\end{dmath}

\begin{figure}
  \begin{subfigure}
    \asyfig{Coil/circles}
  \end{subfigure}
  \lofcaption{Geometry of smaller circles tangential to a larger
    circle.}{ $\theta=2\pi/\COILnmag$ for \COILnmag\ smaller circles in total.}
  \figlabel{circles}
\end{figure}

This expression can be used to calculate the radius of \COILnmag\ smaller
magnets that can be used to fill up the volume of a larger one. In the
first case, the larger magnet is approximated with a concentric
arrangement of disc magnets, approximating ring magnetisation.
Such a configuration can be modelled with the same expressions derived
for the simple cases in the previous section. Superposition can be
used to extrapolate the force from a single small eccentric magnet to
a ring of them.

The results for such arrangements are shown in \figref{coils-manyMag},
where it can be seen that increasing numbers of magnets initially
increases the force, until the inner radius caused by the shrinking
small magnets causes the force to reduce.

\fxfatal{Wouldn't comparison with equal magnetic volume be more relevant in some ways?}

\begin{figure}
  \begin{subfigure}
    \asyfig{Coil/multimagnet}
  \end{subfigure}\par
  \begin{subfigure}
    \psfragfig{\coilpath coils-manyMag}
  \end{subfigure}
  \lofcaption{Forces due to a coil on circular arrangements of
    magnets.}{ The outer radius of the magnets is \SI{20}{mm}, and the
    depths of the magnets and coil are \SI{10}{mm} and \SI{10}{mm}
    respectively.}
  \figlabel{coils-manyMag}
\end{figure}

These results are compared to their equivalent ring magnet geometries,
for equivalent inner and outer magnet radius with solid composition. In
\figref{coils-manyMagVerify}, the effect of increasing the inner
radius of a ring magnet is seen to decrease the total force.

\begin{figure}
  \begin{subfigure}
    \asyfig{Coil/ringmagnet}
  \end{subfigure}\par
  \begin{subfigure}
    \psfragfig{\coilpath coils-manyMagVerify}
  \end{subfigure}
  \lofcaption{Forces due to a coil on various ring magnets.}{ The outer
    radius of the magnets is \SI{20}{mm}, and the depths of the
    magnets and coil are both \SI{10}{mm} respectively.}
  \figlabel{coils-manyMagVerify}
\end{figure}

Greater forces can be achieved by filling in the gaps with smaller
magnets. While great effort could be expended solving an optimal fill
problem, \note{\Eg, ``Kissing Circles: A French romance in MetaPost''
  by Denis
  Roegel:\\\qquad\url{http://www.tug.org/TUGboat/Articles/tb26-1/}} it
is possible to predict the results of asymptotically filling in the
empty space with smaller and smaller magnets: the force would converge
to that achieved with the filled magnet. The impracticalities of
manufactoring such a magnet would also probably increase the expense
prohibitively.

Crudely looking at the problem, application of \eqref{circles} shows
that with six equally spaced magnets in a circle and some small gap
between them, there is just enough space to fit another equally-sized
magnet in the centre. That is, the smallest $n$ for which $3r<R$ is six.

For the double-layer arrangement, the outer magnets need to be small
enough to fit another ring of six equally-sized magnets within them,
which can in turn fit a centre magnet of the same size. For this
problem, with some small gap between the magnets, the smallest $n$ is
twelve.\footnote{\emph{Without} a gap between the magnets, this just
  bumps up to thirteen, but twelve fill the space a little better in
  the centre.} Twelve outer magnets, six inner magnets, and one centre
magnet make nineteen in total.

These filled large-magnet approximations are shown with their force
curves in \figref{coils-gaps}. In the case of six outer magnets, it
can be seen that the inner magnet has little influence on the total
force produced. This echoes the results shown in the eccentricity
forces of \figref{vary-eccentric}, where the forces increased dramatically
when the magnet approached close to the coil. The same would be true
for the nineteen magnet case.

\begin{figure}
  \begin{subfigure}
    \asyfig{Coil/manymagnet}
  \end{subfigure}\par
  \begin{subfigure}
    \psfragfig{\coilpath coils-gaps}
  \end{subfigure}
  \lofcaption{Forces due to a coil on `filled in' circular arrangements of
    magnets.}{ The outer radius of the magnets is \SI{20}{mm}, and the
    depths of the magnets and coil are both \SI{10}{mm}.}
  \figlabel{coils-gaps}
\end{figure}

\paragraph{Cost considerations}

While greater forces can be obtained from using a single large magnet,
the costs of magnetic material is not proportional to volume. As
example, consider arrangements similar to that shown in
\figref{coils-gaps} for the seven and nineteen magnet constructions,
but omit the centre magnet from each (as it has been shown that the
force due to that one is small compared to the rest).

Appropriate magnets to build these systems have been chosen from K\&J
Magnetics and their approximate prices (without volume discount)
tabulated in \tabref{prices}, based around a nominal large magnet of
diameter \SI{50.8}{mm} and thickness \SI{6.4}{mm}. While exact radii
to fill up the gaps could not be chosen, they are close enough to make
a valid comparison. Significant cost savings on magnetic material can
be made by using these multi-magnet configurations instead of a single
magnet, albeit with a decrease in maximum force and an increase in
construction complexity.

\begin{table}
\caption[Approximate magnet prices.]
{Approximate prices for magnets of thickness \SI{6.4}{mm} that could be used
to construct magnet systems
as shown in \figref{coils-gaps}. Data obtained from
\url{http://www.kjmagnetics.com}, 2007/\textsc{may}/08.}
\tablabel{prices}
\begin{tabular}{@{}cccc@{}}
\toprule
Diameter & Price & Number & Total Price \\
\midrule
\SI{50.8}{mm} & \price{17}   & \num{1}  & \price{17} \\
\SI{15.9}{mm} & \price{1.75} & \num{6}  & \price{10.5} \\
\SI{ 9.5}{mm} & \price{0.70} & \num{18} & \price{12.6} \\
\bottomrule
\end{tabular}
\end{table}

\subsection{Dual coil simulations}
\seclabel{dualcoil}

The $NI/A$ term in the force equation between a magnet and coil is the cause
of some subtle considerations when designing coil--magnet actuators.
The force created by a single layer coil
(that is, $N=1$) cannot be improved upon by adding more layers radially,
because the total length of wire (and hence the resistance, and the
inverse of current) will grow faster than the linear gains achieved in
the force.

But extra turns of wire are required to achieve a large enough
resistance to be able to drive the coil without overheating it due to
excessive current. It has been shown in the previous section that
increasing the axial length of a coil is also detrimental to the maximum force
created on a magnet.

A different method of increasing the number of turns is shown here
that does not have the side-effects of effectively reducing the force
that is being generated. By wiring two coils in series and in opposite
winding directions, with the permanent magnet in the gap between them,
greater forces can be achieved than with a single coil of the same
resistance. This design is also better at dissipating heat, as the effective
surface area is greater.

\begin{figure}
  \asyfig{Coil/dual-coil-single}
  \caption{Dual coil system.}
  \figlabel{dual-coil-schematic}
\end{figure}

For two coaxial coils with a distance $G$ between their
centres, superposition can be used to calculate the forces $F_1$ and
$F_2$ generated between each and a permanent magnet. A schematic of this
system is shown in \figref{dual-coil-schematic}. The axial
distance between the permanent magnet and the two coils is denoted by
$d_1$ and $d_2$. The total force on the magnet is given by
\begin{dmath}
F_{\mathrm{T}}(d_1,d_2)
  =F_1(d_1)+F_2(d_2)
  =F_1(d_1)+F_2(G-d_1)
  =F_1(d_1)-F_1(G-d_1)
\end{dmath}.
The two coils are assumed to be the same size.
For a fixed dual-coil geometry, this equation allows an
investigation of the force \vs\ displacement curves while varying the
distance between the coils and the position of the permanent magnet.

Less comprehensive parameter variations are performed to analyse the dual coil
arrangement, as many of the results carry over from the single coil
cases. \Figref{dual-coil} shows a dual-coil design with increasing coil
gaps. The results are evident: if the coils are too far apart, they
act separately and their benefit is not realised; conversely, if they
are too close their effects cancel.

When the separation distance is larger than that for the maximum force, a long stroke actuator can also be designed with such a setup by sacrificing maximum force to achieve a wider but lower force peak (\eg, separation gap of \SI{12.5}{mm} in \Figref{dual-coil}).

\fxfatal{SHOW THAT THE DUAL COIL IS BETTER!!}

These results follow directly from the single coil simulations. For
optimal coil spacing in the dual coil case, the maximum force position
of a single coil should be used (as in \figref{vary-airgap}) for each
side. \fxfatal{expand}

\begin{figure}
  \begin{subfigure}
    \grf{Figures/Coil/dual-coil-2d_1_0}
  \end{subfigure}\par
  \begin{subfigure}
    \psfragfig{\coilpath 2coils-trial}
  \end{subfigure}
  \caption{Forces between dual-coil/magnet pairs with varying
    inter-coil gap distances. \fxfatal{parameters for the dual coil}}
  \figlabel{dual-coil}
\end{figure}
